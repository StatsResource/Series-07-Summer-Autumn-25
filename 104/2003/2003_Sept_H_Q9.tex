\documentclass[a4paper,12pt]{article}

%%%%%%%%%%%%%%%%%%%%%%%%%%%%%%%%%%%%%%%%%%%%%%%%%%%%%%%%%%%%%%%%%%%%%%%%%%%%%%%%%%%%%%%%%%%%%%%%%%%%%%%%%%%%%%%%%%%%%%%%%%%%%%%%%%%%%%%%%%%%%%%%%%%%%%%%%%%%%%%%%%%%%%%%%%%%%%%%%%%%%%%%%%%%%%%%%%%%%%%%%%%%%%%%%%%%%%%%%%%%%%%%%%%%%%%%%%%%%%%%%%%%%%%%%%%%

\usepackage{eurosym}
\usepackage{vmargin}
\usepackage{amsmath}
\usepackage{graphics}
\usepackage{epsfig}
\usepackage{enumerate}
\usepackage{multicol}
\usepackage{subfigure}
\usepackage{fancyhdr}
\usepackage{listings}
\usepackage{framed}
\usepackage{graphicx}
\usepackage{amsmath}
\usepackage{chngpage}

%\usepackage{bigints}
\usepackage{vmargin}

% left top textwidth textheight headheight

% headsep footheight footskip

\setmargins{2.0cm}{2.5cm}{16 cm}{22cm}{0.5cm}{0cm}{1cm}{1cm}

\renewcommand{\baselinestretch}{1.3}

\setcounter{MaxMatrixCols}{10}

\begin{document}


9
On 1 June 2001 an insurer issued a 20 year level temporary annuity to a life then aged
exactly 60. The single premium was paid on 1 June 2001.
Let V t denote the prospective policy value held after t years (t = 0, 1, ..., 20)
calculated using a constant force of interest d and assuming the member is subject to
force of mortality m 60+t .
(i) If the benefit is \$1 paid annually in arrears (so the first payment was made on
31 May 2002), write down and explain a recurrence relation between V t and

V t+1 for t = 0, 1, ..., 19.
(ii) Suppose the benefit is \$h payable in arrears every h years, where h < 1. Write

down a recurrence relation between V t and V t+h for t = 0, h, 2h, 3h...
(iii) By considering the limit as h ® 0 show that Thiele’s equation for the policy
value if the benefit is paid continuously is
¶ V t
= ( m 60 + t + d ) V t - 1
¶ t
and state the boundary condition for V 20 .
(iv)
Assuming that m 60+t is a constant m , solve Thiele’s equation for the policy
value in (iii). [Hint: use an integrating factor.]

Basis: Expenses are ignored


%%%%%%%%%%%%%%%%%%%%%%%%%%%%%%%%%%%%%%%%%%%%%%%%%%%
9
(i)
e \delta \; (V t + P) = p 60 +t ( V t +1 + 1)
e \delta \; V t = p 60 +t ( V t +1 + 1)
t = 0
t > 0
Explanation: LHS = reserve at time t plus one year’s interest. This must equal
the reserve needed at time t + 1, plus the benefit payment then due, allowing
for the probability that the policyholder survives from time t to time t + 1,
which is the RHS.
(ii) e \delta \;h V t = h p 60+ t ( V t + h + h ).
(iii) Using e \delta \;h = 1 + \delta \; h + o ( h )
%% ---  13%%%%%%%%%%%%%%%%%%%%%%%%%%%%%%%%%%%%%%%%%%%%— September 2003 — %%%%%%%%%%%%%%%%%%%%%%%%%%%%%%%%%%%%%%%%%%%%
and h p 60+ t = 1 − h q 60+ t = 1 − h \mu 60+ t + o ( h )
Rewriting the above gives
[1 + \delta \; h + o ( h )] V t = (1 − h \mu 60+ t + o ( h )) ( V t + h + h )
so V t + h − V t = \delta \; hV t − h + h \mu 60+ t V t + h + o ( h )
so V t + h − V t
= \delta \; V t + \mu 60 + t V t + h − 1 + o ( h ) / h
h
Taking the limit as h → 0 gives
\frac{\partial}{\partial } V t
= (\delta \; + \mu 60+ t ) V t − 1
\frac{\partial}{\partial } t
as required. The boundary condition is
V 20 = 0.
(iv)
We have
\frac{\partial}{\partial } V t
= (\delta \; + \mu) V t − 1
\frac{\partial}{\partial } t
so \frac{\partial}{\partial } V t
− (\delta \; + \mu) V t = −1
\frac{\partial}{\partial } t
so e − ( \delta \;+\mu ) t
\frac{\partial}{\partial } V t
− ( \delta \; + \mu ) e − ( \delta \;+\mu ) t V t = − e − ( \delta \;+\mu ) t .
\frac{\partial}{\partial } t
This means that
\frac{\partial}{\partial } − ( \delta \;+\mu ) t
[ e
V t ] = − e − ( \delta \;+\mu ) t
\frac{\partial}{\partial } t
so
e − ( \delta \;+\mu ) t V t = ∫ − e − ( \delta \;+\mu ) t dt
=
1 − ( \delta \;+\mu ) t
e
+ C
\delta \;+\mu
Hence
V t =
%% ---  14
1
+ Ce ( \delta \;+\mu ) t .
\delta \;+\mu
(1)
(2)%%%%%%%%%%%%%%%%%%%%%%%%%%%%%%%%%%%%%%%%%%%%— September 2003 — %%%%%%%%%%%%%%%%%%%%%%%%%%%%%%%%%%%%%%%%%%%%
Using the boundary condition V 20 = 0 we have
0 =
1
+ Ce ( \delta \;+\mu )20
\delta \;+\mu
e − ( \delta \;+\mu )20
C = −
.
so
\delta \;+\mu
So the solution is
V t =
=
1
[1 − e − ( \delta \;+\mu )20 e ( \delta \;+\mu ) t ] (3)
\delta \;+\mu
1
[1 − e − ( \delta \;+\mu )(20 − t ) ]
\delta \;+\mu
= a 20 − t
(4)
(5)
where the annuity is calculated using an effective force of interest of \delta \; + \mu.
%% ---  15%%%%%%%%%%%%%%%%%%%%%%%%%%%%%%%%%%%%%%%%%%%%— September 2003 — %%%%%%%%%%%%%%%%%%%%%%%%%%%%%%%%%%%%%%%%%%%%
