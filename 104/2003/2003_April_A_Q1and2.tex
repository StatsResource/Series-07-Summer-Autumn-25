\documentclass[a4paper,12pt]{article}

%%%%%%%%%%%%%%%%%%%%%%%%%%%%%%%%%%%%%%%%%%%%%%%%%%%%%%%%%%%%%%%%%%%%%%%%%%%%%%%%%%%%%%%%%%%%%%%%%%%%%%%%%%%%%%%%%%%%%%%%%%%%%%%%%%%%%%%%%%%%%%%%%%%%%%%%%%%%%%%%%%%%%%%%%%%%%%%%%%%%%%%%%%%%%%%%%%%%%%%%%%%%%%%%%%%%%%%%%%%%%%%%%%%%%%%%%%%%%%%%%%%%%%%%%%%%

\usepackage{eurosym}
\usepackage{vmargin}
\usepackage{amsmath}
\usepackage{graphics}
\usepackage{epsfig}
\usepackage{enumerate}
\usepackage{multicol}
\usepackage{subfigure}
\usepackage{fancyhdr}
\usepackage{listings}
\usepackage{framed}
\usepackage{graphicx}
\usepackage{amsmath}
\usepackage{chngpage}

%\usepackage{bigints}
\usepackage{vmargin}

% left top textwidth textheight headheight

% headsep footheight footskip

\setmargins{2.0cm}{2.5cm}{16 cm}{22cm}{0.5cm}{0cm}{1cm}{1cm}

\renewcommand{\baselinestretch}{1.3}

\setcounter{MaxMatrixCols}{10}

\begin{document}

1 List three advantages of the two-state model over the Binomial model for the
estimation of transition intensities in a case where exact dates of entry into and exit
from observation are known.

2 (i)
3
4
State what is meant by the assumption of a uniform distribution of deaths
between integer ages x and x + 1.

(ii) Using the assumption of a uniform distribution of deaths and ELT15(Males)
mortality table calculate 0.25 p 75 and 0.25 p 75.5 .

[Total 5]


%% ---  3%%%%%%%%%%%%%%%%%%%%%%%%%%%%%%%%%%%%%%%%%%%%— April 2003 — %%%%%%%%%%%%%%%%%%%%%%%%%%%%%%%%%%%%%%%%%%%%
1
If data on exact times of death are available, the two-state model uses all the
information available (ie the times of death), whereas the Binomial model does not (it
uses the fact that death has occurred).
The Binomial model requires estimation of q and some assumption about the
distribution of deaths with age in order to calculate \mu; the two-state model does not.
The two-state model is extended very simply to processes with more than one
decrement (i.e. to a multiple-state model), and to processes with increments and
decrements. The Binomial model is not.
2
(i)
The assumption of a uniform distribution of deaths between integer ages x and
x + 1 means that, for 0 „ t „ 1, the function t p x \mu x+t is a constant.
The assumption implies that, for 0 „ t „ 1, t q x = tq x .
Alternatively,
The assumption of a uniform distribution of deaths (UDD) between integer
ages x and x + 1 means that the exact ages at death of persons dying within
this age range are evenly spaced along the age axis.
This, for example, if there are j deaths between exact ages x and x + 1, UDD
would be achieved if the exact ages at death are x + 1/( j + 1), x + 2/( j + 1), ...,
x + j /( j + 1).
(ii)
= 1 - 0.25 q 75 = 1 - 0.25  ́ q 75 under the assumption of a uniform
distribution of deaths (UDD) between ages 75 and 76.
0.25 p 75
From ELT 15, q 75 = 0.06197, so
0.25 p 75
= 1 - 0.25  ́ 0.06197 = 0.98451
Either
Under UDD we have, for 0 „ s < t „ 1,
t - s q x + s
=
( t - s ) q x
.
1 - sq x
Putting t = 0.75, s = 0.5 and x = 75, therefore,
0.75 - 0.5 q 75 + 0.5
=
0.25 q 75
, and so
1 - 0.5 q 75
%% ---  4%%%%%%%%%%%%%%%%%%%%%%%%%%%%%%%%%%%%%%%%%%%%— April 2003 — %%%%%%%%%%%%%%%%%%%%%%%%%%%%%%%%%%%%%%%%%%%%
0.25 p 75.5
= 1 -
0.25 q 75
.
1 - 0.5 q 75
Using ELT15, this is evaluated as
1 -
0.25 ( 0.06197 )
1 - 0.5 ( 0.06197 )
= 1 -
0.0154925
= 1 - 0.0159879 = 0.98401
0.969015
Or
Using t p x
0.75 p 75
= s p x \int t - s p x + s ,
=
0.5 p 75 \int 0.25 p 75.5
Using an assumption of UDD between 75 and 76, we have
So,
0.5 p 75 = 1 - 0.5  ́ 0.06197 = 0.969015
0.75 p 75 = 1 - 0.75  ́ 0.06197 = 0.9535225
0.25 p 75.5 =
0.75 p 75
0.5 p 75
=
0.9535225
= 0.98401
0.969015
%% ---  5%%%%%%%%%%%%%%%%%%%%%%%%%%%%%%%%%%%%%%%%%%%%— April 2003 — %%%%%%%%%%%%%%%%%%%%%%%%%%%%%%%%%%%%%%%%%%%%
