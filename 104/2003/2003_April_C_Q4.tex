\documentclass[a4paper,12pt]{article}

%%%%%%%%%%%%%%%%%%%%%%%%%%%%%%%%%%%%%%%%%%%%%%%%%%%%%%%%%%%%%%%%%%%%%%%%%%%%%%%%%%%%%%%%%%%%%%%%%%%%%%%%%%%%%%%%%%%%%%%%%%%%%%%%%%%%%%%%%%%%%%%%%%%%%%%%%%%%%%%%%%%%%%%%%%%%%%%%%%%%%%%%%%%%%%%%%%%%%%%%%%%%%%%%%%%%%%%%%%%%%%%%%%%%%%%%%%%%%%%%%%%%%%%%%%%%

\usepackage{eurosym}
\usepackage{vmargin}
\usepackage{amsmath}
\usepackage{graphics}
\usepackage{epsfig}
\usepackage{enumerate}
\usepackage{multicol}
\usepackage{subfigure}
\usepackage{fancyhdr}
\usepackage{listings}
\usepackage{framed}
\usepackage{graphicx}
\usepackage{amsmath}
\usepackage{chngpage}

%\usepackage{bigints}
\usepackage{vmargin}

% left top textwidth textheight headheight

% headsep footheight footskip

\setmargins{2.0cm}{2.5cm}{16 cm}{22cm}{0.5cm}{0cm}{1cm}{1cm}

\renewcommand{\baselinestretch}{1.3}

\setcounter{MaxMatrixCols}{10}

\begin{document}
%%%%%%%%%%%%%%%%%%%
(i) Explain why crude mortality rates will be graduated before being used for
premium calculations.

(ii) (a)
Describe how you would graduate crude rates by reference to a
standard mortality table.
(b)
Comment on any further considerations that an insurance company
would take into account before using these graduated rates for
premium calculations for annuity policies.

[Total 6]

%%%%%%%%%%%%%%%%%%%
4
(i)
It is generally assumed that the true mortality of the population progresses
smoothly with age.
The crude rates at each age will be estimated independently of the data at other
ages. The data will not be spread evenly across ages and at some ages may be
sparse. For these reasons the rates will not necessarily progress smoothly.
Graduation allows us to use the information from adjacent ages to smooth the
rates.
Insurance companies prefer to use smooth rates for its premium calculations
— any irregularities in the premium rates will be hard to justify to customers
or would leave the company open to the risk of anti-selection or lapse/re-entry.
%% ---  6%%%%%%%%%%%%%%%%%%%%%%%%%%%%%%%%%%%%%%%%%%%%— April 2003 — %%%%%%%%%%%%%%%%%%%%%%%%%%%%%%%%%%%%%%%%%%%%
(ii)
(a)
The first step is to select a suitable standard table, based on a similar
group of lives.
We then plot the crude rates q x against q x s from the standard table to
identify any simple relationship (logs may be plotted rather than q x ).
Having selected a suitable relationship, we find the best-fit parameters,
for example by maximum likelihood or least squares estimates.
We then test the graduation for goodness of fit using statistical tests; if
the fit is not adequate the process is repeated.
(b)
The rates will be used for annuity premiums. It is important for the
company not to overestimate mortality (as the premiums charged
would then be inadequate).
In addition, as the company is using the rates as an estimate of future
mortality, it should take into account any trends — particularly any
likely reduction in future mortality rates.
The company should also consider premium rates charged by
competitors — if rates are out of line either too little business or too
much (unprofitable) business may be attracted.
