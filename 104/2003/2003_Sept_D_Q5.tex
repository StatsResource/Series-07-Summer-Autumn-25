

\documentclass[a4paper,12pt]{article}

%%%%%%%%%%%%%%%%%%%%%%%%%%%%%%%%%%%%%%%%%%%%%%%%%%%%%%%%%%%%%%%%%%%%%%%%%%%%%%%%%%%%%%%%%%%%%%%%%%%%%%%%%%%%%%%%%%%%%%%%%%%%%%%%%%%%%%%%%%%%%%%%%%%%%%%%%%%%%%%%%%%%%%%%%%%%%%%%%%%%%%%%%%%%%%%%%%%%%%%%%%%%%%%%%%%%%%%%%%%%%%%%%%%%%%%%%%%%%%%%%%%%%%%%%%%%

\usepackage{eurosym}
\usepackage{vmargin}
\usepackage{amsmath}
\usepackage{graphics}
\usepackage{epsfig}
\usepackage{enumerate}
\usepackage{multicol}
\usepackage{subfigure}
\usepackage{fancyhdr}
\usepackage{listings}
\usepackage{framed}
\usepackage{graphicx}
\usepackage{amsmath}
\usepackage{chngpage}

%\usepackage{bigints}
\usepackage{vmargin}

% left top textwidth textheight headheight

% headsep footheight footskip

\setmargins{2.0cm}{2.5cm}{16 cm}{22cm}{0.5cm}{0cm}{1cm}{1cm}

\renewcommand{\baselinestretch}{1.3}

\setcounter{MaxMatrixCols}{10}

\begin{document}


%%%%%%%%%%%%%%%%%%%%%%%%—25
The Cox proportional hazards model is to be used to model the rate at which students
leave a certain profession before qualification. Assuming they stay in the profession,
students will qualify three years after joining the profession. In the fitted model, the
hazard depends on the time, t, since joining the profession and three covariates. The
covariates, their categories and the fitted parameters for each category are shown in
the table below:
Covariate
6
Possibility
Parameter
Size of employer large
small
0
0.4
Degree studied none
Science
Arts
other 0.3
-0.1
0.2
0
Location London
other UK
overseas 0
-0.3
0.4
(i) Defining clearly all the terms you use, write down an expression for the hazard
function in this model.

(ii) State the class of students which is most likely to proceed to qualification
under this model, and that which is least likely.

(iii) A student who has been in the profession for one year moves from a “small”
employer to a “large” employer. Express the probability that he will qualify
with the “large” employer P L in terms of the probability that he would have
qualified if he had stayed with the “small” employer P S , all other factors being
equal.

[Total 7]%%%%%%%%%%%%%%%%%%%%%%%%%%%%%%%%%%%%%%%%%%%%%5

5
l ( t ; z i ) = l 0 ( t ) exp( b z i T )
(i)
where l ( t ; z i ) is the hazard at duration t ,
b = (0.4, 0.3, - 0.1, 0.2, - 0.3, 0.4)
and
z i = ( z 1 , z 2 , z 3 , z 4 , z 5 , z 6 )
where
ì 1 : small employer
z 1 = í
î 0 : otherwise
ì 1 : no degree
z 2 = í
î 0 : otherwise
%% ---  7%%%%%%%%%%%%%%%%%%%%%%%%%%%%%%%%%%%%%%%%%%%%— April 2003 — %%%%%%%%%%%%%%%%%%%%%%%%%%%%%%%%%%%%%%%%%%%%
ì 1 : Science degree
z 3 = í
î 0 : otherwise
ì 1 : Arts degree
z 4 = í
î 0 : otherwise
ì 1 : other UK location
z 5 = í
î 0 : otherwise
ì 1 : overseas location
z 6 = í
î 0 : otherwise
l 0 ( t ) = baseline hazard at duration t
Alternatively
l ( t ; z i ) = l 0 ( t ) exp(0.4 z 1 + 0.3 z 2 - 0.1 z 3 + 0.2 z 4 - 0.3 z 5 + 0.4 z 6 )
where l ( t ; z i ) is the hazard at duration t ,
l 0 ( t ) = baseline hazard at duration t
and
where
ì 1 : small employer
z 1 = í
î 0 : otherwise
ì 1 : no degree
z 2 = í
î 0 : otherwise
ì 1 : Science degree
z 3 = í
î 0 : otherwise
ì 1 : Arts degree
z 4 = í
î 0 : otherwise
ì 1 : other UK location
z 5 = í
î 0 : otherwise
ì 1 : overseas location
z 6 = í
î 0 : otherwise
%% ---  8%%%%%%%%%%%%%%%%%%%%%%%%%%%%%%%%%%%%%%%%%%%%— April 2003 — %%%%%%%%%%%%%%%%%%%%%%%%%%%%%%%%%%%%%%%%%%%%
Alternative solutions for β and z were acceptable, but the important point was that the
vector β is a constant. One possible alternative given was:
β = (0.4, 0.1, 0.1)
z1 = 0 (large); 1(small)
z2 = 3 (none); -1 (science); 2 (arts); 0 (other)
z3 = -3 (other UK); 0 (London); 4 (overseas)
(ii)
Most likely corresponds to lowest hazard, so large employer, Science degree,
other UK location.
Least likely is highest hazard, so small employer, no degree, overseas location.
(iii)
We have
P L = e
- ò 1 3 l L ( t ; z i ) dt
where l L ( t ; z i ) is the hazard rate for a large employer
We know that
l L ( t ; z i )
e 0
= 0.4 = 0.6703
l s ( t ; z i )
e
3
- l ( t ; z ) dt
so, P L = e ò 1 L i
3
- 0.6703 l s ( t ; z i ) dt
= e ò 1
æ - 3 l ( t ; z ) dt ö
= ç e ò 1 s i ÷
è
ø
0.6703
= (P S ) 0.6703 .
%% ---  9%%%%%%%%%%%%%%%%%%%%%%%%%%%%%%%%%%%%%%%%%%%%— April 2003 — %%%%%%%%%%%%%%%%%%%%%%%%%%%%%%%%%%%%%%%%%%%%
