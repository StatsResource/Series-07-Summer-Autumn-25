
\documentclass[a4paper,12pt]{article}

%%%%%%%%%%%%%%%%%%%%%%%%%%%%%%%%%%%%%%%%%%%%%%%%%%%%%%%%%%%%%%%%%%%%%%%%%%%%%%%%%%%%%%%%%%%%%%%%%%%%%%%%%%%%%%%%%%%%%%%%%%%%%%%%%%%%%%%%%%%%%%%%%%%%%%%%%%%%%%%%%%%%%%%%%%%%%%%%%%%%%%%%%%%%%%%%%%%%%%%%%%%%%%%%%%%%%%%%%%%%%%%%%%%%%%%%%%%%%%%%%%%%%%%%%%%%

\usepackage{eurosym}
\usepackage{vmargin}
\usepackage{amsmath}
\usepackage{graphics}
\usepackage{epsfig}
\usepackage{enumerate}
\usepackage{multicol}
\usepackage{subfigure}
\usepackage{fancyhdr}
\usepackage{listings}
\usepackage{framed}
\usepackage{graphicx}
\usepackage{amsmath}
\usepackage{chngpage}

%\usepackage{bigints}
\usepackage{vmargin}

% left top textwidth textheight headheight

% headsep footheight footskip

\setmargins{2.0cm}{2.5cm}{16 cm}{22cm}{0.5cm}{0cm}{1cm}{1cm}

\renewcommand{\baselinestretch}{1.3}

\setcounter{MaxMatrixCols}{10}

\begin{document}


%%%%%%%%%%%%%%%%%%%%%%%%%%%%%%%%%%%%%%%%%%%%%%%%%%%
[Total 8]5
A life insurance company has carried out an investigation, over N years, of the
mortality of its term assurance policyholders. Premiums for these contracts are based
on the policyholder’s age last birthday at the date of issue.
The following data are available from the investigation:
d x
= number of deaths during the investigation aged x
P x , t = number of lives under observation aged x at time t (t = 0, 1, ... N)
where x is the policyholder’s age last birthday at date of issue plus the number of
policy anniversaries passed.
(i)
(ii)
(a) State the type of rate interval.
(b) Write down an expression that may be used to evaluate the central
exposed to risk using the available data for P x , t . State any assumptions
made.

The following is an extract from the data:
x d x P x , t P x , t +1
59
60
61 75
67
80 6,276
6,551
6,689 6,824
6,340
6,750
Calculate an estimate of \mu 60 , stating any further assumptions made.
(iii)
104 S2003—3
Give an example of why the assumptions you have made may not be
appropriate in this particular investigation.

%%%%%%%%%%%%%%%%%%%%%%%%%%%%%%%%%%%%%%%%

%% ---  7%%%%%%%%%%%%%%%%%%%%%%%%%%%%%%%%%%%%%%%%%%%%— September 2003 — %%%%%%%%%%%%%%%%%%%%%%%%%%%%%%%%%%%%%%%%%%%%
5
(i)
(a)
The age definition of the death data is:
x = age last birthday at date of issue plus number of complete years
since issue
= age last birthday at previous policy anniversary
Age changes from x to x + 1 on a policy anniversary.
This is therefore a policy year rate interval.
(b)
E x c =
N N − 1
0 t = 0
∫ P x , t dt =
1
\sum  2 ( P x , t + P x , t + 1 )
Assuming policy anniversaries are uniformly spread over the calendar
year.
(ii)
\hat{\mu} x =
d x
estimates \mu at the mean age halfway through the interval.
E x c
Assuming birthdays are uniformly distributed over policy years,
the mean age at the start of the interval is x + 1⁄2.
The mean age halfway though is then x + 1.
So, \hat{\mu} x estimates \mu x +1
So, \hat{\mu} 59 =
(iii)
d 59
c
E 59
=
75
1
( 6276 + 6824 )
2
= 0.01145 estimates \mu 60 .
Premiums are based on age last birthday. There could be a tendency for
policyholders to take out policies just before their birthday to benefit from
lower premiums. The assumption that birthdays are uniformly distributed
would then be invalid.
Other plausible reasons why assumptions made may not be valid were given
credit.
