\documentclass[a4paper,12pt]{article}

%%%%%%%%%%%%%%%%%%%%%%%%%%%%%%%%%%%%%%%%%%%%%%%%%%%%%%%%%%%%%%%%%%%%%%%%%%%%%%%%%%%%%%%%%%%%%%%%%%%%%%%%%%%%%%%%%%%%%%%%%%%%%%%%%%%%%%%%%%%%%%%%%%%%%%%%%%%%%%%%%%%%%%%%%%%%%%%%%%%%%%%%%%%%%%%%%%%%%%%%%%%%%%%%%%%%%%%%%%%%%%%%%%%%%%%%%%%%%%%%%%%%%%%%%%%%

\usepackage{eurosym}
\usepackage{vmargin}
\usepackage{amsmath}
\usepackage{graphics}
\usepackage{epsfig}
\usepackage{enumerate}
\usepackage{multicol}
\usepackage{subfigure}
\usepackage{fancyhdr}
\usepackage{listings}
\usepackage{framed}
\usepackage{graphicx}
\usepackage{amsmath}
\usepackage{chngpage}

%\usepackage{bigints}
\usepackage{vmargin}

% left top textwidth textheight headheight

% headsep footheight footskip

\setmargins{2.0cm}{2.5cm}{16 cm}{22cm}{0.5cm}{0cm}{1cm}{1cm}

\renewcommand{\baselinestretch}{1.3}

\setcounter{MaxMatrixCols}{10}

\begin{document}



%%%%%%%%%%%%%%%%%%%%%%%%%%%%%%%%%%%%%%%%%%%%%%%%%%%
7
A very large employer is considering the simple three state Markov model of its
pension scheme shown below:
Employed (E)
v x
r x
Retired (R)
m x
Dead (D)
(i) Define t p x RD .
(ii) Suppose we observe a large number N of employees aged x. The ith life is
observed from age x+ a i to age x + b i where 0 \$ a i < b i \$ 1. Suppose that of
the employees observed, K retire and M die.
104 S2003—4

(a) Assuming the transition intensities are constant between ages x and
x + 1, write down the likelihood function for this data.
(b) Derive an expression for the maximum likelihood estimate r ˆ x of r x .
(iii)
The employer has observed that retired people are living longer than
anticipated and therefore increasing the cost of the pension scheme. Annual
records are kept of numbers of retired people at each age, and a summary is
given below.
Age last
birthday Number
of retired people
at 1 January 2001 Number
of retired people
at 1 January 2002 Number
of deaths
in 2001
70
71
72
73 8,143
7,592
6,811
7,249 8,741
8,062
7,493
6,693 103
76
107
94
Use this data to estimate q 71 and q 72 stating clearly any assumptions you
make.


7
(i) p x RD is the probability that a life aged x and in the retired state at time 0 will
be in the dead state at time t.
(ii) (a)
t
Omitting the x subscripts for clarity, the likelihood function is given by
L = Const \int
N
∏ e − ( v +ρ )( b − a ) \int v M ρ K .
i
i
i = 1
This solution assumes (as was intended by the question) that
employees ceased to be observed when they retired. Credit was also
given if candidates assumed observation continued into retirement, so
that deaths after retirement were taken into account.
(b)
The log-likelihood is then given by
N
\sum  − ( v + ρ )( b i − a i ) + M log v + K log ρ .
logL = const +
i = 1
Differentiating to find the maximum gives
N
d
K
log L = − \sum  ( b i − a i ) + .
d ρ
ρ
i = 1
To find the maximum likelihood estimate, we set this equal to zero so
N
K
= \sum  ( b i − a i )
ρ ˆ i = 1
so
ρ ˆ =
K
N
\sum  ( b i − a i )
.
i = 1
We check this is a maximum by finding the second derivative:
d 2
d ρ
2
log L = −
K
ρ 2
< 0
so indeed we have a maximum.
%% ---  10%%%%%%%%%%%%%%%%%%%%%%%%%%%%%%%%%%%%%%%%%%%%— September 2003 — %%%%%%%%%%%%%%%%%%%%%%%%%%%%%%%%%%%%%%%%%%%%
(iii)
c
E 71
 0.5 \int ( P 71 (2001) + P 71 (2002))
≈ 0.5 \int (7592 + 8062) = 7827
so \hat{\mu} 71 =
d 71
c
E 71
=
76
= 0.009710
7827
and q 71 ≈ 1 − e −\hat{\mu} 71
= 0.009663.
Similarly
c
E 72
≈ 0.5 \int (6811 + 7493) = 7152
and
\hat{\mu} 72 =
107
= 0.01496.
7152
and hence
q 72 ≈ 1 − e −\hat{\mu} 72 = 0.01485.
Assumptions
- \mu is constant over year of age
- population varies linearly between census dates.
ALTERNATIVELY
(iii)
C
E 71
 0.5 \int (7592 + 8062) = 7827 as above
E 71
c
= E 71
+
d 71
2
= 7827 +
so q 71 =
=
76
= 7865
2
d 71
E 71
76
= 0.009663
7865
%% ---  11%%%%%%%%%%%%%%%%%%%%%%%%%%%%%%%%%%%%%%%%%%%%— September 2003 — %%%%%%%%%%%%%%%%%%%%%%%%%%%%%%%%%%%%%%%%%%%%
Similarly,
c
E 72
 0.5 \int (6811 + 7493) = 7152
E 72 = 7152 +
q 72 =
107
2
107
7205.5
= 7205.5
= 0.01485
Assumptions
- birthdays are distributed uniformly over calendar years
- deaths are distributed uniformly over calendar years
