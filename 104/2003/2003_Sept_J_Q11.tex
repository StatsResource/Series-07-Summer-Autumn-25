\documentclass[a4paper,12pt]{article}

%%%%%%%%%%%%%%%%%%%%%%%%%%%%%%%%%%%%%%%%%%%%%%%%%%%%%%%%%%%%%%%%%%%%%%%%%%%%%%%%%%%%%%%%%%%%%%%%%%%%%%%%%%%%%%%%%%%%%%%%%%%%%%%%%%%%%%%%%%%%%%%%%%%%%%%%%%%%%%%%%%%%%%%%%%%%%%%%%%%%%%%%%%%%%%%%%%%%%%%%%%%%%%%%%%%%%%%%%%%%%%%%%%%%%%%%%%%%%%%%%%%%%%%%%%%%

\usepackage{eurosym}
\usepackage{vmargin}
\usepackage{amsmath}
\usepackage{graphics}
\usepackage{epsfig}
\usepackage{enumerate}
\usepackage{multicol}
\usepackage{subfigure}
\usepackage{fancyhdr}
\usepackage{listings}
\usepackage{framed}
\usepackage{graphicx}
\usepackage{amsmath}
\usepackage{chngpage}

%\usepackage{bigints}
\usepackage{vmargin}

% left top textwidth textheight headheight

% headsep footheight footskip

\setmargins{2.0cm}{2.5cm}{16 cm}{22cm}{0.5cm}{0cm}{1cm}{1cm}

\renewcommand{\baselinestretch}{1.3}

\setcounter{MaxMatrixCols}{10}

\begin{document}


%%%%%%%%%%%%%%%%%%%%%%%%%%%%%%%%%%%%%%%%%%%%%%%%%%%
[Total 15]11
A large investigation has been carried out into mortality among people of working
age. It is suggested that the underlying mortality rates among the lives in the
investigation are the same as those tabulated in a well-known standard table.
(i) Describe three possible features of the mortality data in the investigation that
might lead you to reject the suggestion.

(ii) The actual number of deaths of people in the investigation, together with the
exposed to risk, are tabulated below. The expected number of deaths which
would arise if we applied the rates in the standard table to the exposed to risk
in the investigation, and the standardised deviations, are also shown below.
Age
Exposed to risk Actual deaths E x d x Expected
deaths from
rates in
standard table
E x q x s
35,000
33,000
30,000
30,000
31,000
28,000
25,000
23,000
20,000 35
30
31
45
84
138
229
360
522 34
29
35
52
80
130
213
348
505
20–24
25–29
30–34
35–39
40–44
45–49
50–54
55–59
60–64
Standardised
deviations
z x
0.17150
0.18569
- 0.67612
- 0.97072
0.44721
0.70165
1.09630
0.64327
0.75649
Using 3 appropriate statistical tests, compare the observed mortality rates with
those in the standard table.
For each test you perform, state the possible differences between the data and
the standard table that it is designed to detect, perform the test and clearly state
your conclusions.
Note: the standardised deviations, z x , are given by the formula:
z x =
d x - E x q x s
E x q x s
using the approximation E x q x s ; E x q x s (1 - q x s ) . You should use this
approximation in your calculations.
(iii)
104 S2003—7
[10]
Comment on your results. Include in your comments an observation on the
consequences for a life office which used the standard table mortality rates
when calculating premiums for a class of lives that experienced the mortality
rates underlying those observed in the investigation.

[Total 16]
%%%%%%%%%%%%%%%%%%%%%%%%%%%%%%%%%%%%%%%%%%%%%%%%
\newpage

11
(i)
On average, the underlying mortality of the lives in the investigation could be
systematically heavier or lighter than that represented by the standard table.
Even if, overall, the mortality in the investigation is not significantly different
from that in the standard table, there could be individual age-groups where
large differences exist.
Even if, overall, the mortality in the investigation is not significantly different
from that in the standard table, there could be significant sections of the age
range (i.e. runs of consecutive age groups) over which it is heavier or lighter.
Reasons as to why these points might be the case (for example discussions
about homogeneity) were given credit if valid in the context of the question.
(ii)
The null hypothesis for all the tests is that the underlying mortality of the lives
in the investigation is that of the standard table.
CHI SQUARED TEST
The Chi-squared test is based on z x 2 , the calculation of which is shown below
Age group Actual
deaths Expected
deaths z x z x2
20–24
25–29
30–34
35–39
40–44
45–49
50–54
55–59
60–64 35
30
31
45
84
138
229
360
522 34
29
35
52
80
130
213
348
505 0.17150
0.18569
−0.67612
−0.97072
0.44721
0.70165
1.09630
0.64327
0.75649 0.02941
0.03448
0.45714
0.94231
0.20000
0.49231
1.20188
0.41379
0.57228
Using the data in the table above,
\sum  z x 2 = 4.34360 .
x
This is a test of overall adherence of the data to the standard table.
The test statistic is
\sum  z x 2 ~ \chi^2 2 m ,
x
where m is the number of age groups ( m = 9 in our case), because we are
comparing an experience with a standard table.
%% ---  19%%%%%%%%%%%%%%%%%%%%%%%%%%%%%%%%%%%%%%%%%%%%— September 2003 — %%%%%%%%%%%%%%%%%%%%%%%%%%%%%%%%%%%%%%%%%%%%
This is less than the critical value of the \chi^2 9 2 distribution at the 5% level.
We accept the null hypothesis.
STANDARDISED DEVIATIONS TEST
This tests for the possibility that there are a small number of age groups with
large differences between the mortality rates in the investigation and the
standard table.
The z x s comprise m independent samples from a Normal (0,1) distribution.
We can compare the expected and actual number of deviations in the
following ranges:
Range
(-3,-2) (2,-1) (-1,0) (0,1) (1,2) (2,3)
Expected
0.18 1.26 3.06 3.06 1.26 0.18
Actual
0
0
2
6
1
0
Therefore under the null hypothesis we should expect fewer than 1 in 20 to be
> 2 in absolute magnitude. In this case none of the z x s exceeds 2 in absolute
value, so we accept the null hypothesis.
Also, under the null hypothesis about half the deviations will lie between -2/3
and +2/3. In this case 4 out of 9 do, which is consistent with the null
hypothesis.
SIGNS TEST
This tests for the possibility of the mortality rates in the investigation being
systematically lower or higher than those in the standard table.
Let P be the number of z x s that are positive.
Then under the null hypothesis, P ~ Binomial(9, 0.5).
We have 7 positive signs. The probability of getting 7 or more positive signs
if the null hypothesis is true is (also available from tables in Gold book):
⎛ 9 ⎞ 1 ⎛ 9 ⎞ 1 ⎛ 9 ⎞ 1
⎜ ⎟ 9 + ⎜ ⎟ 9 + ⎜ ⎟ 9
⎝ 7 ⎠ 2 ⎝ 8 ⎠ 2 ⎝ 9 ⎠ 2
9! 1
9! 1
1
=
+
+ 9
9
9
7!2! 2 8!1! 2
2
= (36 + 9 + 1)0.001953125
= 0.08984375.
%% ---  20%%%%%%%%%%%%%%%%%%%%%%%%%%%%%%%%%%%%%%%%%%%%— September 2003 — %%%%%%%%%%%%%%%%%%%%%%%%%%%%%%%%%%%%%%%%%%%%
Or in this case, it is sufficient to look at the probability of getting just 7 signs
(= 0.0703).
This is greater than 0.025 (2-tailed test)
We accept the null hypothesis.
CUMULATIVE DEVIATIONS TEST
When using the whole age range, this tests for the possibility of the mortality
rates in the investigation being systematically lower or higher than those in the
standard table.
Under the null hypothesis,
m
\sum  ( d x − E x q x s )
x = 1
m
~ Normal(0,1)
\sum  E x q x s
x = 1
Using the data in the table, we have
m
\sum  ( d x − E x q x s )
x = 1
m
\sum  E x q x s
=
48
= 1.271.
1,426
x = 1
Since both positive and negative cumulative deviations are of interest we use a
two-tailed test.
Since |1.271|<1.96 we accept the null hypothesis.
GROUPING OF SIGNS TEST
This tests for runs of deviations of the same sign, that is for subsections of the
age range for which the mortality rates of lives in the investigation are
systematically lower or higher than the rates in the standard table.
Let G be the number of groups of positive z x s.
Let n 1 be the number of positive z x s and n 2 be the number of negative z x s.
In our case G = 2, n 1 = 7 and n 2 = 2.
Then the probability of getting 2 or fewer groups of positive signs is
%% ---  21%%%%%%%%%%%%%%%%%%%%%%%%%%%%%%%%%%%%%%%%%%%%— September 2003 — %%%%%%%%%%%%%%%%%%%%%%%%%%%%%%%%%%%%%%%%%%%%
⎛ 6 ⎞ ⎛ 3 ⎞ ⎛ 6 ⎞ ⎛ 3 ⎞
6! 3!
3!
⎜ ⎟⎜ ⎟ ⎜ ⎟⎜ ⎟
+
⎝ 1 ⎠ ⎝ 2 ⎠ + ⎝ 0 ⎠ ⎝ 1 ⎠ = 1!5!1!2! 2!1! = 21 = 0.58333
9!
36
⎛ 9 ⎞
⎛ 9 ⎞
⎜ ⎟
⎜ ⎟
7!2!
⎝ 7 ⎠
⎝ 7 ⎠
Using a one-tailed test, since only small values of
that 0.58333 > 0.05.
We accept the null hypothesis.
%% ---  22
G are of interest, we find%%%%%%%%%%%%%%%%%%%%%%%%%%%%%%%%%%%%%%%%%%%%— September 2003 — %%%%%%%%%%%%%%%%%%%%%%%%%%%%%%%%%%%%%%%%%%%%
SERIAL CORRELATIONS TEST
This tests for runs of deviations of the same sign, that is for subsections of the
age range for which the mortality rates of lives in the investigation are
systematically lower or higher than the rates in the standard table.
The correlation coefficient at lag 1 is
m − 1
r 1 =
\sum  ( z x − z * )( z x + 1 − z ** )
x − 1
m − 1 m − 1
x = 1 x = 1
+1⁄2
.
\sum  ( z x − z * ) 2 \sum  ( z x + 1 − z ** ) 2
where z * is the mean of the standard deviations from ages 1 to m − 1 and
z ** is the mean of the standard deviations from ages 2 to m .
The calculations are shown in the table below.
Age
group z x
20–24
25–29
30–34
35–39
40–44
45–49
50–54
55–59
60–64 0.17150
0.18569
−0.67612
−0.97072
0.44721
0.70165
1.09630
0.64327
0.75649
z x − z *
−0.028348
−0.014158
−0.875968
−1.170568
0.247363
0.501803
0.896453
0.443423
0.55664
z x − z **
−0.10147
−0.08728
−0.94909
−1.24369
0.17424
0.42868
0.82333
0.3703
0.48352
z * = 0.19985
z ** = 0.27297
%% ---  23%%%%%%%%%%%%%%%%%%%%%%%%%%%%%%%%%%%%%%%%%%%%— September 2003 — %%%%%%%%%%%%%%%%%%%%%%%%%%%%%%%%%%%%%%%%%%%%
Age
group ( z x − z * ) 2 ( z x − z ** ) 2
20–24
25–29
30–34
35–39
40–44
45–49
50–54
55–59
60–64 0.000804
0.000200
0.767319
1.370228
0.061188
0.251806
0.803627
0.196624
0.309848 0.010296
0.007618
0.90077
1.546765
0.030360
0.183767
0.67787
0.137122
0.23379
( z x − z * )( z x + 1 − z ** )
0.00247
0.013437
1.089433/
−0.203960
0.106040
0.413149
0.331957
0.214404
Hence
m − 1
r 1 =
\sum  ( z x − z * )( z x + 1 − z ** )
x − 1
m − 1 m − 1
x = 1 x = 1
\sum  ( z x − z * ) 2 \sum  ( z x + 1 − z ** ) 2
=
1.96693
= 0.549045.
(3.451796)(3.718062)
Now r 1 e m ~ Normal(0,1).
Since m = 9, we have r 1 e m = 3 \int 0.549045 = 1.64714.
Using a one-tailed test (since we are only interested in positive serial
correlations), the probability of getting a value as high as 1.64714 is 0.05.
Therefore we have just sufficient evidence to reject the null hypothesis.
This test uses the exact formula, as given in the Core Reading. Credit was
also given to candidates who used the approximation given in the Gold book.
(iii)
There is little evidence here to suggest that the mortality of the lives in the
investigation is significantly different from that represented by the standard
table.
The experience passes all the tests except the serial correlations test (and only
fails that marginally).
However, there is a suggestion that at ages over 40, mortality is consistently
heavier than that in the standard table.
Moreover it is at ages over 40 that most deaths occur.
%% ---  24%%%%%%%%%%%%%%%%%%%%%%%%%%%%%%%%%%%%%%%%%%%%— September 2003 — %%%%%%%%%%%%%%%%%%%%%%%%%%%%%%%%%%%%%%%%%%%%
Therefore life offices using the standard table to represent the mortality
experience of lives such as those in the investigation might find their
profitability impaired, as they would tend to charge premiums which are too
low.
%% ---  25
