\documentclass[a4paper,12pt]{article}

%%%%%%%%%%%%%%%%%%%%%%%%%%%%%%%%%%%%%%%%%%%%%%%%%%%%%%%%%%%%%%%%%%%%%%%%%%%%%%%%%%%%%%%%%%%%%%%%%%%%%%%%%%%%%%%%%%%%%%%%%%%%%%%%%%%%%%%%%%%%%%%%%%%%%%%%%%%%%%%%%%%%%%%%%%%%%%%%%%%%%%%%%%%%%%%%%%%%%%%%%%%%%%%%%%%%%%%%%%%%%%%%%%%%%%%%%%%%%%%%%%%%%%%%%%%%

\usepackage{eurosym}
\usepackage{vmargin}
\usepackage{amsmath}
\usepackage{graphics}
\usepackage{epsfig}
\usepackage{enumerate}
\usepackage{multicol}
\usepackage{subfigure}
\usepackage{fancyhdr}
\usepackage{listings}
\usepackage{framed}
\usepackage{graphicx}
\usepackage{amsmath}
\usepackage{chngpage}

%\usepackage{bigints}
\usepackage{vmargin}

% left top textwidth textheight headheight

% headsep footheight footskip

\setmargins{2.0cm}{2.5cm}{16 cm}{22cm}{0.5cm}{0cm}{1cm}{1cm}

\renewcommand{\baselinestretch}{1.3}

\setcounter{MaxMatrixCols}{10}

\begin{document}1
An investigation into the marital status of males in a particular country used a
multiple state Markov model. Men were observed over a fixed period and the
numbers of transitions made between the three states “married”, “not married” and
“dead” were noted. The total exposure time in the states “married” and “not married”
was also recorded. The diagram below shows the possible transitions.
m
not married
married
n
r
s
dead
2
If n is the total exposure time in the state “not married”;
m is the total exposure time in the state “married”;
a is the observed number of marriages;
b is the observed number of deaths of “not married” people;
c is the observed number of transitions from the state “married” to the state
“not married” and
d is the observed number of deaths of “married” people.
(i) Write down an expression for the likelihood of these data from which the
values of the constant transition intensities m, n, r and s may be estimated. [2]
(ii) Derive the maximum likelihood estimate of the transition intensity from the
state “not married” to the state “married”.
[4]
[Total 6]
A mortality table has been estimated for the ages 4 to 100 inclusive. The rates have
been graduated by fitting a mathematical formula to the crude estimates. The
deviations of the observed number of deaths from the expected number of deaths at
each age using the graduated mortality rates have been calculated. The results are:
Number
Positive deviations
Negative deviations
57
40
Test this graduation using the Signs Test by:
(a) stating the Null Hypothesis being tested
(b) stating the sampling distribution of the test statistic if the Null Hypothesis is
true
(c) completing the test and stating your conclusions
104 A2002—2
[6]3
(i)
The random variable T 18 measures the future lifetime of a life aged 18 from
the Assured Lives 1967–1970 Ultimate mortality experience for male lives.
Draw a sketch of the force of mortality m 18+t , t > 0. Add labels to your sketch
to explain its important features.
[2]
(ii)
The force of mortality m 18+t is to be modelled using the Gompertz-Makeham
family of curves where:
m 18+t = GM(r,s) = a 1 + a 2 t + a 3 t 2 ... + a r t r-1
+ exp{a r+1 + a r+2 t + a r+3 t 2 ... + a r+s t s-1 }
and a 1 , a 2 , a 3 , ..., a r+s are constants which do not depend on t.
Using your sketch in (i) explain why a GM(2, 2) curve may prove to be a
suitable model for m 18+t .
[4]
[Total 6]
4
In a mortality investigation the number of deaths at age x during the period of the
investigation is q x , where x is defined as:
x = [age last birthday at 6th April prior to date of issue of policy]
+ [number of 5th April’s passed since date of issue of policy]
(i) State the rate year implied by this definition and the age range of lives at the
start of this interval.
[3]
(ii) (a)
State the Principle of Correspondence.
(b)
Using this principle describe the central exposed to risk, E x c , that
would correspond to the above classification of deaths.
(iii)
[3]
The estimated force of mortality for those lives classified as aged x is
m ˆ x =
q x
E x c
which estimates the force of mortality, m x + f . Determine the value of f stating
any assumptions that you make.
[3]
[Total 9]
104 A2002—3
PLEASE TURN OVER5
A life insurance company issued a 10-year temporary immediate annuity of £5,000
per annum to a life aged 44 exact on 1 January 1990. Annuity payments were
deferred for 10 years, so that the first payment was made on 1 January 2001.
Premiums were payable annually in advance until the end of the deferred period or
earlier death.
(i) State the Principle of Equivalence.
[1]
(ii) Calculate the level annual premium payable.
[5]
(iii) Calculate the Expected Death Strain for the calendar year 2004.
[4]
(iv) If the annuitant died during the calendar year 2004, calculate the Actual Death
Strain for this calendar year.
[1]
Basis: Mortality: A1967–70 Ultimate Males
Interest: 5% per annum interest effective throughout
Expenses are ignored
6
[Total 11]
A life insurance company has investigated the recent mortality experience of its male
annuitants. The following is an extract from the results.
Age
x Exposed
to risk
E x Observed
deaths
q x
70
71
72
73
74
75 600
750
725
650
700
675 23
31
33
29
35
39
(i) Use the Chi-squared goodness of fit test to compare this experience with the
a(55) Ultimate Mortality Table for Male Annuitants. This was the mortality
basis used to determine the price of these annuities. State the null hypothesis
you are testing and comment on the results of your test.
[8]
(ii) Comment on the financial impact on the company if it continues to sell these
annuities with an unchanged mortality basis.
[1]
(iii) State how your test in (i) would be varied if you were testing graduated
o
mortality rates for adherence to the above data. The graduated rates, q x , were
determined by fitting the relationship
o
q x = a + bq x s
where q x s are rates from the a(55) Ultimate Mortality Table for Male
Annuitants. No further calculations are required.
%%%%%%%%%%%%%%%%%%%%%%%%%%%%%%%%%%%%%%%%%%%%%%%%%%%%%%%%%%%%%%%%%%%%%%%%%%%%%%%%%%%%%%%%%%%%%%%%%%%%%%%%%%%%%
[Total 11]
104 A2002—47
An investigation into the risk factors associated with mortality from lung cancer
among men was undertaken. The purpose of the investigation was to establish
whether a new treatment was effective in prolonging survival. Two groups of patients
were identified. One group was given the “new” treatment and the other was given
the “existing” treatment. Other factors taken into consideration were the patients’
general state of health at time of diagnosis (recorded as “able to care for self” or
“unable to care for self”), and the type of tumour (recorded as “large”, “squamous”,
“small” or “adeno”).
A Cox proportional hazards model of the hazard of death was estimated. The table
below shows an extract from the results.
Covariate
General state of health at time of diagnosis
Able to care for self
Unable to care for self
Treatment
New
Existing
Type of tumour
Large
Squamous
Small
Adeno
Parameter Standard error
- 0.60
0.00 0.05
0.25
0.00 0.25
0.00
- 0.40
0.45
0.75
0.28
0.26
0.28
(i) Defining all the terms you use, write down a general expression for the Cox
proportional hazards model in terms of a set of covariates, their associated
parameters and a baseline hazard function.
[2]
(ii) In the context of the investigation described above, state the class of men to
which the baseline hazard refers.
[2]
(iii) Compare the new treatment with the previous one. Does it improve the
chances of survival, make them worse, or is it not possible to say? Justify
your answer.
[4]
(iv) Calculate the proportion by which the risk of death for men with “adeno” type
tumours who were “able to care for themselves” at the time of diagnosis is
greater than that for men with “large” type tumours who were “unable to care
for themselves” at the time of diagnosis.
[3]
[Total 11]
104 A2002—5
PLEASE TURN OVER8
9
An investigation took place of the mortality of persons between exact ages 60 and 61
years. The table below gives an extract from the results. For each person it gives the
age at which they were first observed, the age at which they ceased to be observed,
and whether their departure from observation was because of their death or
withdrawal from the investigation.
Person Age at entry
years months Age at exit
years months Died or withdrew
1
2
3
4
5
6
7
8
9
10 60
60
60
60
60
60
60
60
60
60 60
61
60
61
60
61
60
61
60
61 withdrew
withdrew
died
withdrew
died
withdrew
died
withdrew
died
withdrew
0
1
1
2
3
4
5
7
8
9
6
0
3
0
9
0
11
0
10
0
(i) Estimate q 60 using the actuarial estimate, stating any assumptions that you
make.
(ii) Estimate q 60 using a two-state Markov model, stating any assumptions that
you make.
[4]
(iii) Comment on the differences between the two estimates. Include a statement
about which estimate you consider to be the more reliable and why.
[3]
[Total 12]
(i) K 70 is a random variable which measures the curtate future lifetime of a life
aged exactly 70. Using a mortality basis of English Life Table No. 12 Males
complete the following table for the probability function of K 70 :
t
P K 70 ( t )
(ii)
0
5
10
15
20
[3]
Z is a random variable which measures the present value of the benefits of a
term assurance policy with a 20 year term issued to a life aged exactly 70.
The sum assured of £1,000 is payable at the end of the policy year of death.
Basis: Mortality: English Life Table No. 12 Males
Interest: 5% per annum effective
104 A2002—6
(a) Using the results from (i), or otherwise, draw a graph of the probability
function of Z. Add labels to your graph to explain its important
features.
(b) Find P[Z > 500], stating any assumptions that you make in determining
this value.
[9]
[Total 12]10
(i) Show that A x : n = 1 - da && x : n
(ii) The force of mortality at age x in a special mortality table, m ¢ x , is related to the
corresponding force of mortality in a standard table, m x , by the relationship:
m ¢ x + t = m x + t + k
[3]
t > 0
where k is a constant which does not depend on t.
(a) Show that t p ¢ x = t p x exp{ - kt }
(b) Hence or otherwise show that the expected present value of an immediate temporary annuity-due of 1 per annum issued to a life aged
x with a term of n years valued using the special mortality table and a
valuation rate of interest, i per annum is given by:
t > 0
a && x : n
where this function is evaluated using the standard mortality table and
a valuation rate of interest, j per annum, where:
j = (1 + i) exp{k} - 1
(iii)
[6]
A 20 year term insurance policy is issued to a life aged exactly 60. The sum
assured is £10,000 payable at the end of the year of death. Calculate the net
level annual premium for this policy using a basis of 6% per annum effective
and a special mortality table where:
m ¢ x + t = m x + t + 0.018692 .
The standard mortality table is the a(55) Ultimate Mortality Table for male
lives.
[7]
[Total 16]
104 A2002—7
