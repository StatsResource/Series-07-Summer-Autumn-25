\documentclass[a4paper,12pt]{article}

%%%%%%%%%%%%%%%%%%%%%%%%%%%%%%%%%%%%%%%%%%%%%%%%%%%%%%%%%%%%%%%%%%%%%%%%%%%%%%%%%%%%%%%%%%%%%%%%%%%%%%%%%%%%%%%%%%%%%%%%%%%%%%%%%%%%%%%%%%%%%%%%%%%%%%%%%%%%%%%%%%%%%%%%%%%%%%%%%%%%%%%%%%%%%%%%%%%%%%%%%%%%%%%%%%%%%%%%%%%%%%%%%%%%%%%%%%%%%%%%%%%%%%%%%%%%

\usepackage{eurosym}
\usepackage{vmargin}
\usepackage{amsmath}
\usepackage{graphics}
\usepackage{epsfig}
\usepackage{enumerate}
\usepackage{multicol}
\usepackage{subfigure}
\usepackage{fancyhdr}
\usepackage{listings}
\usepackage{framed}
\usepackage{graphicx}
\usepackage{amsmath}
\usepackage{chngpage}

%\usepackage{bigints}
\usepackage{vmargin}

% left top textwidth textheight headheight

% headsep footheight footskip

\setmargins{2.0cm}{2.5cm}{16 cm}{22cm}{0.5cm}{0cm}{1cm}{1cm}

\renewcommand{\baselinestretch}{1.3}

\setcounter{MaxMatrixCols}{10}

\begin{document}
\begin{enumerate}
10
\begin{enumerate}[(i)]
\item (i) Show that A x : n = 1 - da && x : n
\item (ii) The force of mortality at age x in a special mortality table, m ¢ x , is related to the
corresponding force of mortality in a standard table, m x , by the relationship:
m ¢ x + t = m x + t + k
[3]
t > 0
where k is a constant which does not depend on t.
(a) Show that t p ¢ x = t p x exp{ - kt }
(b) Hence or otherwise show that the expected present value of an
immediate temporary annuity-due of 1 per annum issued to a life aged
x with a term of n years valued using the special mortality table and a
valuation rate of interest, i per annum is given by:
t > 0
a && x : n
where this function is evaluated using the standard mortality table and
a valuation rate of interest, j per annum, where:
j = (1 + i) exp{k} - 1
\item (iii)
[6]

A 20 year term insurance policy is issued to a life aged exactly 60. The sum assured is £10,000 payable at the end of the year of death. Calculate the net level annual premium for this policy using a basis of 6\% per annum effective and a special mortality table where:
m ¢ x + t = m x + t + 0.018692 .
The standard mortality table is the a(55) Ultimate Mortality Table for male
lives.
\end{enumerate}
%%%%%%%%%%%%%%%%%%%%%%%%%%%%%%%%%%%%%%%%%%%%%%%%%%%%%%%%%%%%%%%%%%%%%%%%%%%%%%%%%%%%%%
%%% Page 13Subject 104 (Survival Models) — April 2002 — Examiners’ Report
%% Solution to Q 10
\newpage

(i)
a && x : n = E é a && min( K + 1, n ) ù
x
ë
û
é 1 - v min( K x + 1, n ) ù
= E ê
ú
d
ë ê
û ú
So
= 1 1 é min( K x + 1, n ) ù
- E v
û
d d ë
= 1 1
- A
d d x : n
A x : n = 1 - da && x : n
ALTERNATIVELY
Let
C x
= v x+1 d x
= v x+1 ( l x - l x+ 1 )
= v . v x l x - v x + 1 l x + 1
= vD x - D x+1
Then
t =¥
å C x + t
= M x
t = 0
t =¥ t =¥
t = 0 t = 0
= v å D x + t - å D x + t + 1
= vN x - N x +1
So
A x : n =
=
Page 14
M x - M x + n + D x + n
D x
vN x - N x + 1 - vN x + n + N x + n + 1 + D x + n
D x

%%--- Subject 104 (Survival Models) — April 2002 — Examiners’ Report
since N x + n +1 + D x + n - N x +1
= N x + n - (N x - D x )
= N x + n - N x + D x
then
A x : n =
(1 - d )( N x - N x + n ) - ( N x - N x + n ) + D x
D x
= 1 - da && x : n
ALTERNATIVELY
A x : n =
k = n - 1
å
v k + 1 . k 1⁄2 q x + v n n p x
k = 0
Now k 1⁄2 q x = k p x - k +1 p x
Substitute
=
k = n - 1
å
v k + 1 k p x -
k = 0
k = n - 1
å
k = 0
v k + 1 k + 1 p x + v n n p x
Then EITHER
= va && x : n - ( a && x : n + 1 - 1) + v n n p x
= va && x : n - a && x : n + 1
= 1 - (1 - v ) a && x : n
= 1 - da && x : n
OR
= v + v 2 p x ...+ v nn - 1 p x - (vp x + v 22 p x ... + v n - 1 n - 1 p x + v nn p x ) + v n p x
= 1 - (1 - v)(1 + vp x ... + v n - 1 n - 1 p x )
= 1 - da && x : n
%%-- Page 15
%%-- Subject 104 (Survival Models) — April 2002 — Examiners’ Report
(ii)
(a)
t
ì ï u = t
ü ï
p ¢ x = exp í - ò m ¢ x + u . du ý
ï î u = 0
ï þ
ì ï u = t
ü ï
= exp í - ò m x + u + k . du ý
î ï u = 0
þ ï
ì ï u = t
ü ï
ì ï u = t
ü ï
= exp í - ò m x + u . du ý exp í - ò k . du ý
ï î u = 0
ï þ
ï î u = 0
ï þ
= t p x .exp( - kt )
(b)
a && ¢ x : n =
=
t = n - 1
å
t p ¢ x . v t
t p x . e - kt . v t
t p x ( v . e - k )
t p x v t j
t = 0
t = n - 1
å
t = 0
=
t = n - 1
å
t
t = 0
=
t = n - 1
å
t = 0
= a && x : n j
where
1
1
1
=
. k
1 + j (1 + i ) e
i.e. j = (1 + i) e k - 1
(iii)
¢ 6% = 1 - d 6% a && 60:20
¢
Use (i): A 60:20
6%
¢
Use (ii): a && 60:20
= a && 60:20 j
6%
where j = (1.06)e 0.018692 - 1 = 0.08 i.e. 8%
%%-- Page 16
%%-- Subject 104 (Survival Models) — April 2002 — Examiners’ Report
1
60:20 6%
Also A ¢
=
¢
D 80
20
= v 6%
¢
D 60
= (1.06) - 20 .
= (1.06) - 20
¢
20 p 60
20 p 60 .
e - 20  ́ 0.018692
363,991 - 0.37384
e
859,916
= 0.31180  ́ 0.42329  ́ 0.68809
= 0.09082
Alternatively
D 80
771.2
= 0.09081
@ 8% =
8, 492.4
D 60
¢
a && 60:20
= a && 60:20 8%
6%
= a && 60 -
D 80
a && 80
D 60
= (8.432 + 1) - 0.09082  ́ (3.989 + 1)
= 8.97890 (8.97895 with 0.09081)
¢ 6% = 1 -
Then A 60:20
Then Premium =
0.06
 ́ 8.97890 = 0.49176
1.06
10, 000(0.49176 - 0.09082)
8.97890
= 446.5362
£446.54 p.a.
\end{document}
