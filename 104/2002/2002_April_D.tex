\documentclass[a4paper,12pt]{article}

%%%%%%%%%%%%%%%%%%%%%%%%%%%%%%%%%%%%%%%%%%%%%%%%%%%%%%%%%%%%%%%%%%%%%%%%%%%%%%%%%%%%%%%%%%%%%%%%%%%%%%%%%%%%%%%%%%%%%%%%%%%%%%%%%%%%%%%%%%%%%%%%%%%%%%%%%%%%%%%%%%%%%%%%%%%%%%%%%%%%%%%%%%%%%%%%%%%%%%%%%%%%%%%%%%%%%%%%%%%%%%%%%%%%%%%%%%%%%%%%%%%%%%%%%%%%

\usepackage{eurosym}
\usepackage{vmargin}
\usepackage{amsmath}
\usepackage{graphics}
\usepackage{epsfig}
\usepackage{enumerate}
\usepackage{multicol}
\usepackage{subfigure}
\usepackage{fancyhdr}
\usepackage{listings}
\usepackage{framed}
\usepackage{graphicx}
\usepackage{amsmath}
\usepackage{chngpage}

%\usepackage{bigints}
\usepackage{vmargin}

% left top textwidth textheight headheight

% headsep footheight footskip

\setmargins{2.0cm}{2.5cm}{16 cm}{22cm}{0.5cm}{0cm}{1cm}{1cm}

\renewcommand{\baselinestretch}{1.3}

\setcounter{MaxMatrixCols}{10}

\begin{document}
\begin{enumerate}

104 A2002—47
An investigation into the risk factors associated with mortality from lung cancer
among men was undertaken. The purpose of the investigation was to establish
whether a new treatment was effective in prolonging survival. Two groups of patients
were identified. One group was given the “new” treatment and the other was given
the “existing” treatment. Other factors taken into consideration were the patients’
general state of health at time of diagnosis (recorded as “able to care for self” or
“unable to care for self”), and the type of tumour (recorded as “large”, “squamous”,
“small” or “adeno”).
A Cox proportional hazards model of the hazard of death was estimated. The table
below shows an extract from the results.
Covariate
General state of health at time of diagnosis
Able to care for self
Unable to care for self
Treatment
New
Existing
Type of tumour
Large
Squamous
Small
Adeno
Parameter Standard error
- 0.60
0.00 0.05
0.25
0.00 0.25
0.00
- 0.40
0.45
0.75
0.28
0.26
0.28
(i) Defining all the terms you use, write down a general expression for the Cox
proportional hazards model in terms of a set of covariates, their associated
parameters and a baseline hazard function.
[2]
(ii) In the context of the investigation described above, state the class of men to
which the baseline hazard refers.
[2]
(iii) Compare the new treatment with the previous one. Does it improve the
chances of survival, make them worse, or is it not possible to say? Justify
your answer.
[4]
(iv) Calculate the proportion by which the risk of death for men with “adeno” type
tumours who were “able to care for themselves” at the time of diagnosis is
greater than that for men with “large” type tumours who were “unable to care
for themselves” at the time of diagnosis.
[3]
[Total 11]
104 A2002—5
PLEASE TURN OVER8
9
An investigation took place of the mortality of persons between exact ages 60 and 61
years. The table below gives an extract from the results. For each person it gives the
age at which they were first observed, the age at which they ceased to be observed,
and whether their departure from observation was because of their death or
withdrawal from the investigation.
Person Age at entry
years months Age at exit
years months Died or withdrew
1
2
3
4
5
6
7
8
9
10 60
60
60
60
60
60
60
60
60
60 60
61
60
61
60
61
60
61
60
61 withdrew
withdrew
died
withdrew
died
withdrew
died
withdrew
died
withdrew
0
1
1
2
3
4
5
7
8
9
6
0
3
0
9
0
11
0
10
0
(i) Estimate q 60 using the actuarial estimate, stating any assumptions that you
make.
[5]
(ii) Estimate q 60 using a two-state Markov model, stating any assumptions that
you make.
[4]
(iii) Comment on the differences between the two estimates. Include a statement
about which estimate you consider to be the more reliable and why.
[3]
%%%%%%%%%%%%%%%%%%%%%%%%%%%%%%%%%%%%%%%%%%%%%%

Page 8Subject 104 (Survival Models) — April 2002 — Examiners’ Report
7
(i) h(x, t) = h 0 (t)exp( b 1 x 1 + b 2 x 2 + ... + b k x k )
where h(x, t) is the hazard at duration t, h 0 (t) is some unspecified baseline
hazard, x 1 ... x k are covariates and b 1 ... b k are their associated parameters.
(ii) Men who were “unable to care for themselves” at the time of diagnosis, who
were given the “existing” treatment, and whose tumours were of the “large”
type.
(iii) The value of the parameter associated with the new treatment is 0.25. This
implies that the ratio of the hazards of death for two otherwise identical
patients, one of whom is given the new treatment and the other the existing
treatment is exp(0.25) = 1.28. Thus the new treatment appears to increase the
risk of death.
However, the standard error associated with the parameter is 0.25. The
approximate 95% confidence interval is therefore 0.25 ± 1.96(0.25) = ( - 0.24,
0.74), which includes 0. Therefore, the value of the parameter is not
significantly different from zero at the 5% level, so it is not possible to say
with the available data whether the new treatment affects the risk of death.
(iv)
The hazard for men with “adeno” type tumours who were “able to care for
themselves” at the time of diagnosis is h 0 (t)exp( - 0.60 + 0.75).
The hazard for men with “large” type tumours who were “unable to care for
themselves” at the time of diagnosis is h 0 (t), since this is the baseline category.
The ratio is thus
h 0 ( t ) exp( - 0.60 + 0.75)
= exp(0.15) = 1.16
h 0 ( t )
so the risk of death is 16% greater for men with “adeno” type tumours who
were “able to care for themselves” at the time of diagnosis.
%%%%%%%%%%%%%%%%%%%%%%%%%%%%%%%%%%%%%%%%%%%%%%%%%%%%%%%%%%%%%%%%%%%%%%%%%%%%%%%%%%%%%%%%%%%%%%%%%%%%%
8
(i)
Let the time individual i enters observation be a i and the time that individual i
leaves observation be b i . Define an indicator variable d i such that d i = 0 if individual i is not observed to die and d i = 1 if individual i dies. Then the
traditional actuarial estimate of q 60 is
10
q ˆ 60 =
å d i
10
i = 1
.
å (1 - a i - [(1 - d i )(1 - b i )])
i = 1

The calculations are shown in the table below.
Person a i b i d i 1 - a i 1 - b i 1 - a i - (1 - d i ) (1 - b i )
1
2
3
4
5
6
7
8
9
10 6/12
1
3/12
1
9/12
1
11/12
1
10/12
1 0
0
1
0
1
0
1
0
1
0 1
11/12
11/12
10/12
9/12
8/12
7/12
5/12
4/12
3/12 6/12
0
9/12
0
3/12
0
1/12
0
2/12
0 6/12
11/12
11/12
10/12
9/12
8/12
7/12
5/12
4/12
3/12
0
1/12
1/12
2/12
3/12
4/12
5/12
7/12
8/12
9/12
Totals
4
Therefore q ˆ 60 =
74/12
4
= 0.6486.
74 /12
ALTERNATIVELY
The data allow the exact calculation of the initial exposed to risk. The
approximate calculation using
E x c + 1⁄2 q x =
59 1
83
+  ́ 4 =
12 2
12
makes an unnecessary assumption, giving q̂ 60 = 0.5783.
The assumption is that the dates of death are uniformly distributed over the
life year (60, 61).
10
(ii)
In the two-state model we estimate m ˆ 60 =
is shown in the table below.
Page 10
å d i
10
i = 1
å ( b i - a i )
i = 1
. The necessary working

Subject 104 (Survival Models) — April 2002 — Examiners’ Report

Person a i b i b i - a i
1
2
3
4
5
6
7
8
9
10 0
1/12
1/12
2/12
3/12
4/12
5/12
7/12
8/12
9/12 6/12
1
3/12
1
9/12
1
11/12
1
10/12
1 6/12
11/12
2/12
10/12
6/12
8/12
6/12
5/12
2/12
3/12
Totals
So m ˆ 60 =
59/12
4
= 0.81355
59 /12
and assuming that the force of mortality is constant over (60, 61)
q̂ 60 = 1 - exp( -m ) = 0.5567.
(iii)
The estimates of q 60 differ, the actuarial estimate being higher than the
estimate produced by the two-state model.
The two-state model uses all the information we are given, including that of
times of death, whereas the actuarial estimate places relatively more weight on
the number of deaths.
In this case, m is large, so most of the information is in the times of death
rather than the number of deaths, and the actuarial estimate does not produce
acceptable results. The estimate from the two-state model is to be preferred.
Explicit statements of formulae are not necessary, provided they are implicit in the
calculations presented. The alternative solution to (i) received partial credit.
