\documentclass[a4paper,12pt]{article}

%%%%%%%%%%%%%%%%%%%%%%%%%%%%%%%%%%%%%%%%%%%%%%%%%%%%%%%%%%%%%%%%%%%%%%%%%%%%%%%%%%%%%%%%%%%%%%%%%%%%%%%%%%%%%%%%%%%%%%%%%%%%%%%%%%%%%%%%%%%%%%%%%%%%%%%%%%%%%%%%%%%%%%%%%%%%%%%%%%%%%%%%%%%%%%%%%%%%%%%%%%%%%%%%%%%%%%%%%%%%%%%%%%%%%%%%%%%%%%%%%%%%%%%%%%%%

\usepackage{eurosym}
\usepackage{vmargin}
\usepackage{amsmath}
\usepackage{graphics}
\usepackage{epsfig}
\usepackage{enumerate}
\usepackage{multicol}
\usepackage{subfigure}
\usepackage{fancyhdr}
\usepackage{listings}
\usepackage{framed}
\usepackage{graphicx}
\usepackage{amsmath}
\usepackage{chngpage}

%\usepackage{bigints}
\usepackage{vmargin}

% left top textwidth textheight headheight

% headsep footheight footskip

\setmargins{2.0cm}{2.5cm}{16 cm}{22cm}{0.5cm}{0cm}{1cm}{1cm}

\renewcommand{\baselinestretch}{1.3}

\setcounter{MaxMatrixCols}{10}

\begin{document}




%%%%%%%%%%%%%%%%%%%%%%%%%%%%%%%%%%%%%%%%%%%%%%%%%%%%%%%%%%%%%%%%%%%%%%%%%%%%%%%


An investigation took place of the mortality of persons between exact ages 60 and 61
years. The table below gives an extract from the results. For each person it gives the
age at which they were first observed, the age at which they ceased to be observed,
and whether their departure from observation was because of their death or
withdrawal from the investigation.
Person Age at entry
years months Age at exit
years months Died or withdrew
1
2
3
4
5
6
7
8
9
10 60
60
60
60
60
60
60
60
60
60 60
61
60
61
60
61
60
61
60
61 withdrew
withdrew
died
withdrew
died
withdrew
died
withdrew
died
withdrew
0
1
1
2
3
4
5
7
8
9
6
0
3
0
9
0
11
0
10
0

\begin{enumerate}
\item Estimate q 60 using the actuarial estimate, stating any assumptions that you
make.
\item Estimate q 60 using a two-state Markov model, stating any assumptions that
you make.
\item Comment on the differences between the two estimates. Include a statement
about which estimate you consider to be the more reliable and why.
\end{enumerate}
%%%%%%%%%%%%%%%%%%%%%%%%%%%%%%%%%%%%%%%%%%%%%%


%%%%%%%%%%%%%%%%%%%%%%%%%%%%%%%%%%%%%%%%%%%%%%%%%%%%%%%%%%%%%%%%%%%%%%%%%%%%%%%%%%%%%%%%%%%%%%%%%%%%%
8
(i)
Let the time individual i enters observation be a i and the time that individual i
leaves observation be $b_i$ . Define an indicator variable d i such that $d_i = 0$ if individual i is not observed to die and d i = 1 if individual i dies. Then the
traditional actuarial estimate of q 60 is

\[ \hat{q}_{60} =   \frac{\sum^{10}_{i=1} d_{i}}{\sum^{10}_{i=1} \left(  1-a_i - [(1-d_{i})(1-b_{i}) ] \right)} \]



The calculations are shown in the table below.
Person a i b i d i 1 - a i 1 - b i 1 - a i - (1 - d i ) (1 - b i )
1
2
3
4
5
6
7
8
9
10 6/12
1
3/12
1
9/12
1
11/12
1
10/12
1 0
0
1
0
1
0
1
0
1
0 1
11/12
11/12
10/12
9/12
8/12
7/12
5/12
4/12
3/12 6/12
0
9/12
0
3/12
0
1/12
0
2/12
0 6/12
11/12
11/12
10/12
9/12
8/12
7/12
5/12
4/12
3/12
0
1/12
1/12
2/12
3/12
4/12
5/12
7/12
8/12
9/12
Totals
4
Therefore q ˆ 60 =
74/12
4
= 0.6486.
74 /12
ALTERNATIVELY
The data allow the exact calculation of the initial exposed to risk. The
approximate calculation using
E x c + 1⁄2 q x =
59 1
83
+  ́ 4 =
12 2
12
makes an unnecessary assumption, giving q̂ 60 = 0.5783.
The assumption is that the dates of death are uniformly distributed over the
life year (60, 61).%%%%%%%%%%%%%%%%%%%%%%%%%%%%%%%%%%%%%%%%%%%%%%%%%%%%%%%%%%%%%%%%%%%%%%%%%5
10

(ii)
In the two-state model we estimate m ˆ 60 =
is shown in the table below.
Page 10
å d i
10
i = 1
å ( b i - a i )
i = 1
. The necessary working

Subject 104 (Survival Models) — April 2002 — Examiners’ Report

Person a i b i b i - a i
1
2
3
4
5
6
7
8
9
10 0
1/12
1/12
2/12
3/12
4/12
5/12
7/12
8/12
9/12 6/12
1
3/12
1
9/12
1
11/12
1
10/12
1 6/12
11/12
2/12
10/12
6/12
8/12
6/12
5/12
2/12
3/12
Totals
So m ˆ 60 =
59/12
4
= 0.81355
59 /12
and assuming that the force of mortality is constant over (60, 61)
\[q̂ 60 = 1 - exp( -m ) = 0.5567.\]

%%%%%%%%%%%%%%%%%%%%%%%%%%%%%%%%%%%%%%%%%%%%%%%%%%%%%%%%%%%%%%%%%%%%%%%%%%%%%%%%%5
(iii)
The estimates of q 60 differ, the actuarial estimate being higher than the estimate produced by the two-state model.
The two-state model uses all the information we are given, including that of times of death, whereas the actuarial estimate places relatively more weight on the number of deaths.

In this case, m is large, so most of the information is in the times of death rather than the number of deaths, and the actuarial estimate does not produce acceptable results. The estimate from the two-state model is to be preferred.
Explicit statements of formulae are not necessary, provided they are implicit in the
calculations presented. 

%% The alternative solution to (i) received partial credit.


\end{document}
