\documentclass[a4paper,12pt]{article}

%%%%%%%%%%%%%%%%%%%%%%%%%%%%%%%%%%%%%%%%%%%%%%%%%%%%%%%%%%%%%%%%%%%%%%%%%%%%%%%%%%%%%%%%%%%%%%%%%%%%%%%%%%%%%%%%%%%%%%%%%%%%%%%%%%%%%%%%%%%%%%%%%%%%%%%%%%%%%%%%%%%%%%%%%%%%%%%%%%%%%%%%%%%%%%%%%%%%%%%%%%%%%%%%%%%%%%%%%%%%%%%%%%%%%%%%%%%%%%%%%%%%%%%%%%%%

\usepackage{eurosym}
\usepackage{vmargin}
\usepackage{amsmath}
\usepackage{graphics}
\usepackage{epsfig}
\usepackage{enumerate}
\usepackage{multicol}
\usepackage{subfigure}
\usepackage{fancyhdr}
\usepackage{listings}
\usepackage{framed}
\usepackage{graphicx}
\usepackage{amsmath}
\usepackage{chngpage}

%\usepackage{bigints}
\usepackage{vmargin}

% left top textwidth textheight headheight

% headsep footheight footskip

\setmargins{2.0cm}{2.5cm}{16 cm}{22cm}{0.5cm}{0cm}{1cm}{1cm}

\renewcommand{\baselinestretch}{1.3}

\setcounter{MaxMatrixCols}{10}

\begin{document}
\begin{enumerate}
[Total 7]
104 S2002—25
(i)
Given that p x = 0.9, calculate 0.5 p x and 0.5 p x+0.5 using the following
assumptions about mortality between ages x and x + 1:
(a)
(b)
(ii)
6
uniform distribution of deaths
Balducci assumption
[4]
Comment on how appropriate you think each of these assumptions is.
[2]
[Total 6]
[Total 7]

104 S2002—25

(i)

Given that p x = 0.9, calculate 0.5 p x and 0.5 p x+0.5 using the following

assumptions about mortality between ages x and x + 1:

(a)

(b)

(ii)

6

uniform distribution of deaths

Balducci assumption

[4]

Comment on how appropriate you think each of these assumptions is.



[Total 6]
%%%%%%%%%%%%%%%%%%%%%%%%%%%%%%%%%%%%%%%%%%%%%%%%%%
\newpage

(i) (a)
UDD: t q x = t.q x
0.5 p x
= 1. - 0.5.q x = 1. - 0.5  ́ (1 - 0.9) = 0.95
p x = 0.5 p x +0.5 . 0.5 p x
Þ 0.5 p x +0.5 = 0.9/0.95 = 0.9474
(b)
Balducci: 1 - t q x + t = (1 - t).q x
0.5 p x +0.5
= 1 - 0.5. q x = 1 - 0.5  ́ (1 - 0.9) = 0.95
p x = 0.5 p x +0.5 . 0.5 p x
Þ 0.5 p x = 0.9 / 0.95 = 0.9474
(ii)
Under UDD assumption, 0.5 p x +0.5 < 0.5 p x , so the force of mortality is
increasing between x and x + 1. Conversely, under the Balducci assumption,
the force of mortality is decreasing. So UDD assumption seems the more
Page 5Subject 104 (Survival Models) — September 2002 — Examiners’ Report
appropriate for most ages. The Balducci assumption would be appropriate for
either very young ages or the back of the accident hump.
\end{document}
