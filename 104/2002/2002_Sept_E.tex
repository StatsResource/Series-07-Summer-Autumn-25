\documentclass[a4paper,12pt]{article}

%%%%%%%%%%%%%%%%%%%%%%%%%%%%%%%%%%%%%%%%%%%%%%%%%%%%%%%%%%%%%%%%%%%%%%%%%%%%%%%%%%%%%%%%%%%%%%%%%%%%%%%%%%%%%%%%%%%%%%%%%%%%%%%%%%%%%%%%%%%%%%%%%%%%%%%%%%%%%%%%%%%%%%%%%%%%%%%%%%%%%%%%%%%%%%%%%%%%%%%%%%%%%%%%%%%%%%%%%%%%%%%%%%%%%%%%%%%%%%%%%%%%%%%%%%%%

\usepackage{eurosym}
\usepackage{vmargin}
\usepackage{amsmath}
\usepackage{graphics}
\usepackage{epsfig}
\usepackage{enumerate}
\usepackage{multicol}
\usepackage{subfigure}
\usepackage{fancyhdr}
\usepackage{listings}
\usepackage{framed}
\usepackage{graphicx}
\usepackage{amsmath}
\usepackage{chngpage}

%\usepackage{bigints}
\usepackage{vmargin}

% left top textwidth textheight headheight

% headsep footheight footskip

\setmargins{2.0cm}{2.5cm}{16 cm}{22cm}{0.5cm}{0cm}{1cm}{1cm}

\renewcommand{\baselinestretch}{1.3}

\setcounter{MaxMatrixCols}{10}

\begin{document}

In an investigation of the mortality of whole of life insurance policyholders,
information was available about the number of policyholders aged x last birthday on
1 January 1999, 1 January 2000 and 1 January 2001. Counts of deaths for the
calendar years 1999 and 2000 were available classified by age nearest birthday on the
date of death. The table below shows an extract from the data.
Age
x Number of persons aged x last birthday on
1 Jan 1999
1 Jan 2000
1 Jan 2001
P x,1999
P x,2000
P x,2001 Deaths aged x nearest birthday
1999
2000
q x (1999) q x (2000)
40
41
42 473
450
490 17
20
21
512
470
460
491
482
480
18
18
19
Assuming that the forces of mortality at each age are constant over the whole period
from 1 January 1999 to 1 January 2001:
%%%%%%%%%%%%%%%%%%%%%%%%%%%%%%%%%%%%%%%%%%%%
5
(ii) Censoring is non-informative if it gives no information about lifetimes. In the
case of random censoring, the independence for all lives of the random
variables measuring the future lifetime and the time until censoring is
sufficient to ensure that the censoring is non-informative. Censoring would be
informative if, for example, censored lives were subject to lighter mortality
than other lives. Since persons who allow their life insurance policies to lapse
tend to have lighter mortality than those who keep up their payments, it is
likely that in most investigations, censoring due to withdrawals while still
alive is informative. Censored observations are likely to have longer lives
than lives which are not censored.
%%%%%%%%%%%%%%%%%%%%%%%%%%%%%%%%%
6
(i)
Since deaths are classified on an “age nearest birthday” basis, we need to
estimate the exposed to risk on the same basis.
The number of persons aged x last birthday on 1 January in calendar year t is
P x , t . Let the number of persons aged x nearest birthday on 1 January in
calendar year t be P* x , t . Then, assuming that dates of birth are evenly
distributed across each calendar year, P* x,t = 0.5(P x - 1, t + P x , t ).
The central exposed to risk at age x over the two year period between 1
January 1999 and 1 January 2001 may be estimated using the formula
E x c = 0.5P* x ,1999 + P* x ,2000 + 0.5P* x ,2001
assuming P * x , t is linear in t over the calendar years 1999 and 2000.
Thus, substituting for P* x , t this becomes
E x c = 0.5[0.5(P x - 1,1999 + P x ,1999 )] + 0.5(P x - 1,2000 + P x ,2000 )
+ 0.5[0.5(P x - 1,2001 + P x ,2001 )].
Then
m ˆ x =
q x (1999) + q x (2000)
E x c
estimates m x .
assuming that the force of mortality is constant between age x - 1⁄2 and age
x + 1⁄2 .
(ii)
So, using the data given, we have
c
E 41
= 0.25(473 + 450) + 0.5(512 + 470) + 0.25(491 + 482) = 965
c
E 42
= 0.25(450 + 490) + 0.5(470 + 460) + 0.25(482 + 480) = 940.5
The total deaths at ages 41 and 42 nearest birthday are 38 and 40 respectively,
so
m̂ 41 =
Page 6
38
40
= 0.03938 and m̂ 42 =
= 0.04253.
965
940.5Subject 104 (Survival Models) — September 2002 — Examiners’ Report
