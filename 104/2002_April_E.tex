\documentclass[a4paper,12pt]{article}

%%%%%%%%%%%%%%%%%%%%%%%%%%%%%%%%%%%%%%%%%%%%%%%%%%%%%%%%%%%%%%%%%%%%%%%%%%%%%%%%%%%%%%%%%%%%%%%%%%%%%%%%%%%%%%%%%%%%%%%%%%%%%%%%%%%%%%%%%%%%%%%%%%%%%%%%%%%%%%%%%%%%%%%%%%%%%%%%%%%%%%%%%%%%%%%%%%%%%%%%%%%%%%%%%%%%%%%%%%%%%%%%%%%%%%%%%%%%%%%%%%%%%%%%%%%%

\usepackage{eurosym}
\usepackage{vmargin}
\usepackage{amsmath}
\usepackage{graphics}
\usepackage{epsfig}
\usepackage{enumerate}
\usepackage{multicol}
\usepackage{subfigure}
\usepackage{fancyhdr}
\usepackage{listings}
\usepackage{framed}
\usepackage{graphicx}
\usepackage{amsmath}
\usepackage{chngpage}

%\usepackage{bigints}
\usepackage{vmargin}

% left top textwidth textheight headheight

% headsep footheight footskip

\setmargins{2.0cm}{2.5cm}{16 cm}{22cm}{0.5cm}{0cm}{1cm}{1cm}

\renewcommand{\baselinestretch}{1.3}

\setcounter{MaxMatrixCols}{10}

\begin{document}
\begin{enumerate}

[Total 12]
(i) K 70 is a random variable which measures the curtate future lifetime of a life
aged exactly 70. Using a mortality basis of English Life Table No. 12 Males
complete the following table for the probability function of K 70 :
t
P K 70 ( t )
(ii)
0
5
10
15
20
[3]
Z is a random variable which measures the present value of the benefits of a
term assurance policy with a 20 year term issued to a life aged exactly 70.
The sum assured of £1,000 is payable at the end of the policy year of death.
Basis: Mortality: English Life Table No. 12 Males
Interest: 5% per annum effective
104 A2002—6
(a) Using the results from (i), or otherwise, draw a graph of the probability
function of Z. Add labels to your graph to explain its important
features.
(b) Find P[Z > 500], stating any assumptions that you make in determining
this value.
[9]

%%%%%%%%%%%%%%%%%%%%%%%%%%%%%%%%%%%%%%%%%%%%%%%%%%%%%%%%%%%%%%%%%%%%%%%%%%%%%%%%%%%%%%%%%%%%%%%%

Page 11Subject 104 (Survival Models) — April 2002 — Examiners’ Report
9
(i)
P K 70 ( t ) = t p 70 q 70+t =
d 70 + t
l 70
Using tables we obtain
t = 0, 1, 2, 3, ...
l 70 = 54,806
t 0 5 10 15 20
d 70 + t
Value 3,051
0.05567 3,282
0.05988 2,923
0.05333 1,897.4
0.03462 779.9
0.01422
Explicit statements of formulae are not needed provided they are implicit in the
calculations presented.
(ii)
(a)
ì ï 1, 000(1.05) - t - 1 t = 0, 1, 2, ..., 19
Z = í
t 3 20
î ï 0
So values for density function are
t 0 5 10 15 19 3 20
z 952 746 585 458 377 0
0.060 0.053 0.035 0.018 l 90
= 0.056
l 70
Height of probability
function
0.056
P Z (z)
0.05
 ́
 ́
 ́
 ́
 ́
 ́
458
377
500 585
746
952
The graph should be labelled to explain its shape and the “spike” at Z = 0.
Page 12
zSubject 104 (Survival Models) — April 2002 — Examiners’ Report
(b)
From the results in (ii)(a) we can see that z = 500 will occur between
t = 10 and t = 15.
Expanding the tabulation in (ii)(a) gives
t 11 12 13 14
z 557 530 505 481
ALTERNATIVELY
We can find the “cross-over” point by solving
1,000(1.05) -t-1 = 500
(1.05) t+1 =
t =
1,000
= 2
500
log e 2
- 1 = 13.21 years
log e 1.05
So t = 13 gives Z > 500 and t = 14 gives Z < 500.
So Z > 500 corresponds to K 70 £ 13 that the life dies in the interval
(70, 84). We can write
P[Z > 500] = P[K 70 £ 13]
= 1 -
= 1 -
l 84
l 70
12,306
54,806
= 1 - 0.22454
= 0.775
