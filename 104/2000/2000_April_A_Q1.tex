\documentclass[a4paper,12pt]{article}
%%%%%%%%%%%%%%%%%%%%%%%%%%%%%%%%%%%%%%%%%%%%%%%%%%%%%%%%%%%%%%%%%%%%%%%%%%%%%%%%%%%%%%%%%%%%%%%%%%%%%%%%%%%%%%%%%%%%%%%%%%%%%%%%%%%%%%%%%%%%%%%%%%%%%%%%%%%%%%%%%%%%%%%%%%%%%%%%%%%%%%%%%%%%%%%%%%%%%%%%%%%%%%%%%%%%%%%%%%%%%%%%%%%%%%%%%%%%%%%%%%%%%%%%%%%%
\usepackage{eurosym}
\usepackage{vmargin}
\usepackage{amsmath}
\usepackage{graphics}
\usepackage{epsfig}
\usepackage{enumerate}
\usepackage{multicol}
\usepackage{subfigure}
\usepackage{fancyhdr}
\usepackage{listings}
\usepackage{framed}
\usepackage{graphicx}
\usepackage{amsmath}
\usepackage{chngpage}
%\usepackage{bigints}

\usepackage{vmargin}
% left top textwidth textheight headheight
% headsep footheight footskip
\setmargins{2.0cm}{2.5cm}{16 cm}{22cm}{0.5cm}{0cm}{1cm}{1cm}
\renewcommand{\baselinestretch}{1.3}

\setcounter{MaxMatrixCols}{10}
\begin{document}

1
Describe how smoothness is ensured when mortality rates are graduated
using each of the following methods:
(a)
(b)
fitting a parametric formula to the crude estimates; and
graduation by reference to a standard mortality table.

%%%%%%%%%%%%%%%%%%%%%%%%%%%%%%%%%%%%%%%%%%%%
1
(a)
Parametric Formula
The chosen formula, e.g. GM(r,s) can be approximated by a polynominal of
low order/a curve with few parameters.
The smoothness criterion of small second or third order differences of the
graduated rates will be ensured by the approximate low order of the
parametric formula chosen as the best fitting curve.
(b)
Reference to a Standard Table
The standard table will already have fulfilled the smoothness criterion.
The formula chosen to be fitted to the crude rates will usually represent
an approximately linear transformation of the rates in the standard table,
e.g. q x = aq x s + b so that the second and third order differences of the
graduated rates will be the same as those of the standard table.
