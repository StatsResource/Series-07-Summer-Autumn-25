\documentclass[a4paper,12pt]{article}
%%%%%%%%%%%%%%%%%%%%%%%%%%%%%%%%%%%%%%%%%%%%%%%%%%%%%%%%%%%%%%%%%%%%%%%%%%%%%%%%%%%%%%%%%%%%%%%%%%%%%%%%%%%%%%%%%%%%%%%%%%%%%%%%%%%%%%%%%%%%%%%%%%%%%%%%%%%%%%%%%%%%%%%%%%%%%%%%%%%%%%%%%%%%%%%%%%%%%%%%%%%%%%%%%%%%%%%%%%%%%%%%%%%%%%%%%%%%%%%%%%%%%%%%%%%%
\usepackage{eurosym}
\usepackage{vmargin}
\usepackage{amsmath}
\usepackage{graphics}
\usepackage{epsfig}
\usepackage{enumerate}
\usepackage{multicol}
\usepackage{subfigure}
\usepackage{fancyhdr}
\usepackage{listings}
\usepackage{framed}
\usepackage{graphicx}
\usepackage{amsmath}
\usepackage{chngpage}
%\usepackage{bigints}

\usepackage{vmargin}
% left top textwidth textheight headheight
% headsep footheight footskip
\setmargins{2.0cm}{2.5cm}{16 cm}{22cm}{0.5cm}{0cm}{1cm}{1cm}
\renewcommand{\baselinestretch}{1.3}

\setcounter{MaxMatrixCols}{10}
\begin{document}

8
X is a random variable which measures the duration from the date of a kidney
transplant until death.
(i) Express the hazard rate and the integrated hazard function at duration
x, in terms of probabilities.

(ii) If the hazard rate at duration x is
h(x) = \alpha\lambda x \alpha−1
derive an expression for the integrated hazard, H(x).
(iii)

The hazard rate h(x), as defined in part (i), varies between transplant
patients in such a way that
\alpha = \alpha 0 + \alpha 1 z_{1}
\lambda = \lambda 1 z_{1} + \lambda 2 z_{2}
where \alpha 0 , \alpha 1 , \lambda 1 , \lambda 2 are constants; z_{1} is the age of the patient at the date
of the transplant and z_{2} is the patient’s sex where z_{2} = 0 = female,
z_{2} = 1 = male.
Show that these hazards are not in general proportional, but that if
\alpha 1 = 0 the hazards are proportional.

%%%%%%%%%%%%%%%%%%%%%%%%%%%

%%%%%%%%%%%%%%%%%%%%%%%%%%%%%%%

8
(i)
Hazard Rate, h(x) = Lim
h ® 0 +
P [ x < X £ x + h 1⁄2 X > x ]
h
Alternative expressions are also correct.
u = x
Integrated Hazard H(x) =
ò
h ( u ) . du = - ln(S(x))
u = 0
= - ln . P[X > x]
where S(x) = P[X > x]
u = x
(ii)
From (i)
H(x) =
ò
al u a - 1 . du
u = 0
=
a
l u a
a
u = x
u = 0
= lx =
(iii)
x > 0
h(x) = (a 0 + a 1 z_{1} )(l 1 z_{1} + l 2 z_{2} ) x a 0 +a 1 z_{1} - 1
Then
h ( x 1⁄2 z )
h ( x 1⁄2 z *)
=
( a 0 + a 1 z_{1} )( l 1 z_{1} + l 2 z_{2} ) x a 0 +a 1 z - 1
( a 0 + a 1 z_{1} * )( l 1 z_{1} * + l 2 z_{2} * ) x a 0 +a 1 z * - 1
which is not in general independent of x, so hazards are
not proportional.
%%---- Page 7Subject 104 (Survival Models) — %%%%%%%%%%%%%%%%%%%%%%%%%%%%%%%%%%%%%%%5
If a 1 = 0, then
h ( x 1⁄2 z )
h ( x 1⁄2 z *)
=
=
a 0 ( l 1 z_{1} + l 2 z_{2} ) x a 0 - 1
a 0 ( l 1 z_{1} * + l 2 z_{2} * ) x a 0 - 1
l 1 z_{1} + l 2 z_{2}
l 1 z_{1} * + l 2 z_{2} *
which is independent of x, so the hazards are proportional.
Alternatively if a 1 = 0, then
h(x1⁄2z) = a 0 (l 1 z_{1} + l 2 z_{2} ) x a 0 - 1
which is of the form h 0 (x) = a 0 x a 0 - 1  ́ c ( z ) ; where c(z) = l 1 z_{1} + l 2 z_{2}

\end{document}
