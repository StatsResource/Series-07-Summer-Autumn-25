%%%%%%%%%%%%%%%%%%%%%%%%%%%%%%%%%%%%%%%11
A pension scheme is to be modelled using the following four-state model:
(1)
ACTIVE MEMBER
(2)
RETIRED MEMBER
(3)
WITHDRAWN
(4)
DEAD
The following functions are defined for integer x:
\mu 1j
x + t = force of transition from state 1 to state j at exact age x + t,
(j = 2, 3, 4), for 0 \leq t < 1.
t
(i)
p 1 x j = probability that a life in state 1 at exact age x is in state j at
exact age x + t, (j = 1, 2, 3, 4), for t > 0.
Derive from first principles a differential equation for t p x 11 , and show
that the solution to this equation is:
 t
11
p
=
exp
 −
t x

 0
4
∫ ∑ \mu
j = 2
1 j
x + s

. ds 


stating all assumptions made.
(ii)
[6]
State, or otherwise obtain, the differential equation for t p 1 x j (j \neq 1), and
show that the solution to this equation is:
t
1 j
t p x =
∫
r
1 j
p 11
x . \mu x + r dr ;
for j \neq 1.

0
(iii)
Assuming that the forces of transition \mu 1j
x + t have a constant value
\mu 1 x j + 1⁄2 for 0 \leq t < 1, show that:
1
p 1 x j =
\mu 1 x j + 1⁄2
4
∑
k = 2
104—6
\mu 1 x k + 1⁄2

 4
 
  1 − exp  − \mu 1 x k + 1⁄2   
 k = 2
 

∑
The values of \mu 1 x j + 1⁄2 are:
(iv)
x \mu 12
x + 1⁄2 \mu 13
x + 1⁄2 \mu 14
x + 1⁄2
60
61 .25
.10 .08
.04 .012
.014
Calculate
12
2
p 12
60 .

[Total 14]


%%%%%%%%%%%%%%%%%%%%%%%%%%%%%%%%%%%%%%%%%%%%%%%%%%%%%%%%%%%%%%%%%%%%%%%%%%%%%%%%%%%%%%%%%%%%%%%%
\new%%---- Page

11
(i)
t + h
11
11
p 11
x = t p x . h p x + t
assuming probabilities of transition are dependent upon age and the
current state only (Markov assumption).
Using 1 =
h
12
13
14
p 11
x + t + h p x + t + h p x + t + h p x + t
i.e. law of total probability
t + h
11
12
13
14
p 11
x = t p x (1 - [ h p x + t + h p x + t + h p x + t ])
Now
h
p 1 x + j t = h . m 1 x j + t + o ( h )
for j 1 1
where o(h) is such that lim
h ® 0 +
\
\
t + h
t + h
o ( h )
= 0
h
11
11
12
13
14
p 11
x - t p x = - t p x . h . ( m x + t + m x + t + m x + t ) + o ( h )
11
p 11
x - t p x
= - t p 11
x .
h
4
å m
j = 2
1 j
x + t
+
o ( h )
h
æ
ö
p 11 - t p 11
¶ t p 11
x
x
\ lim ç ç t + h x
= - t p 11
÷ ÷ =
x .
h ® 0 +
h
¶ t
è
ø
Hence
\
¶ log e t p 11
x
= -
¶ t
log e t p 11
x
-
log e o p 11
x
4
4
å m
2
1 j
x + t
å m
1 j
x + t
2
t
=
4
ò å m
-
0
2
1 j
x + s ds
%%---- Page 11Subject 104 (Survival Models) — %%%%%%%%%%%%%%%%%%%%%%%%%%%%%%%%%%%%%%%5
As 0 p 11
x = 1
then
t
log e t p 11
x
4
ò å m
-
=
2
0
æ
and t p 11
x = exp ç -
ç
è
(ii)
1 j
x + s ds
t 4
0 2
ò å m
1 j
x + s
¶ 1 j
11
1 j
t p x = t p x . m x + t
¶ t
ö
ds ÷
÷
ø
j 1 1
t
ò
Hence t p 1 x j - o p 1 x j =
r
1 j
p 11
x . m x + r dr
0
As o p 1 x j = 0
then t p 1 x j =
t
ò
r
1 j
p 11
x . m x + r dr
0
(iii)
Substituting 1 for t, assuming constant forces, and putting (i) into (ii) we
get:
1 j
1 p x
1
=
-
ò
e
r 4
ò 0 å 2 m x + 1⁄2 ds
1 k
. m 1 x j + 1⁄2 dr
0
1
ò
= m 1 x j + 1⁄2 . e
- r .
4
å m 1 x k + 1⁄2
2
dr
0
1
m 1 x j + 1⁄2
4
m 1 x k + 1⁄2
2
=
å
-
=
m 1 x j + 1⁄2
4
å m
2
%%---- Page 12
1 k
x + 1⁄2
4
é - r . å
m 1 x k + 1⁄2 ù
ê e 2
ú
ê
ú
ë ê
û ú 0
4
é
- å m 1 x k + 1⁄2 ù
ê 1 - e 2
ú
ê
ú
ê ë
ú ûSubject 104 (Survival Models) — %%%%%%%%%%%%%%%%%%%%%%%%%%%%%%%%%%%%%%%5
(iv)
2
p 12
60
Now
