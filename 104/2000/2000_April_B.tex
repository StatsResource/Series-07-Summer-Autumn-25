5
Thiele’s equation for the policy value at duration t, t V , of an immediate life
annuity payable continuously at a rate of £1 per annum from age x is:
∂
t V = \mu x+t . t V − 1 + δ . t V
∂ t
(i) Derive this result algebraically showing all the steps in your
argument.
(ii) Explain this result by general reasoning.
104—2



[Total 8]6
In a mortality investigation the actual number of deaths at age x last birthday
is d x . The goodness of fit between the data and the force of mortality, \mu x+1⁄2 ,
over the age range x 1 , x 1 + 1, x 1 + 2 ..., x 1 + m − 1 can be tested using the
statistic
x = x 1 + m − 1
∑
x = x 1
( d
x
− E x c \mu x + 1⁄2
E x c \mu x + 1⁄2
)
2
where E x c is the central exposed to risk which corresponds to d x .
For each of the following cases state the null hypothesis being tested, and give
the sampling distribution of this statistic as precisely as you can from the
given information:
(a) if \mu x+1⁄2 are taken from a standard mortality table;
(b) if \mu x+1⁄2 are the graduated rates obtained from the crude estimates using
the Gompertz-Makeham formula
g(x) = a 0 + exp{b 0 + b 1 x + b 2 x 2 };
(c)
if \mu x+1⁄2 are the graduated rates obtained from the crude estimates using
the formula
g(x) = \mu sx + k
where \mu sx are taken from a standard mortality table and k is a constant;
and
(d)
if \mu x+1⁄2 are the graduated rates obtained from a graphical graduation of
the crude estimates d x / E x c .
104—3
[8]
%%%%%%%%%%%%%%%%%%%%%%%%%%%%%%%%%%%%%%%7
Let X be a random variable representing the present value of the benefits of a
whole of life assurance, and Y be a random variable representing the present
value of the benefits of a temporary assurance with a term of n years. Both
assurances have a sum assured of 1 payable at the end of the year of death and
were issued to the same life aged x.
(i) Describe the benefits provided by the contract which has a present
value represented by the random variable X − Y.
(ii) Show that

Cov(X, Y) = 2 A x 1 : n − A x A 1 x : n
and hence or otherwise that
Var(X − Y) = 2 A x − ( n  A x ) 2 − 2 A x 1: n
where the functions A are determined using an interest rate of i, and
functions 2 A are determined using an interest rate of i 2 + 2i.
%%%%%%%%%%%%%%%%%%%%%%%%%%%%%%%%%%%%%%%
[Total 8]


5
(i)
The life annuity will be secured by a single payment at age x and so the
policy value at duration t will be
t V
= a x + t
s =¥
t V
= a x + t =
ò
e -d s . s p x+t . ds
s = 0
So:
¶
t V
¶ t
=
¶
¶ t
s =¥
=
ò
s = 0
Now
s =¥
ò
e -d s s p x+t . ds
s = 0
e -d s
¶
s p x+t . ds
¶ t
¶
¶ æ l x + t + s ö
ç
÷
s p x+t =
¶ t
¶ t è l x + t ø
=
l x + t ( -m x + t + s . l x + t + s ) - l x + t + s ( -m x + t l x + t )
l x 2 + t
= s p x+t (m x+t - m x+t+s )
\
¶
t V
¶ t
s =¥
=
ò
e -d s . s p x + t (m x+t - m x+t+s ) ds
s = 0
s =¥
= m x + t a x + t -
ò
s = 0
%%---- Page 4
e -d s s p x + t m x + t + s . dsSubject 104 (Survival Models) — %%%%%%%%%%%%%%%%%%%%%%%%%%%%%%%%%%%%%%%5
= m x + t a x + t
ì
ï
- í - e -d s . s p x + t
ï
î
¥
0
ü
ï
e -d s s p x + t . ds ý
ï
s = 0
þ
s =¥
- ( -d )
ò
= m x + t a x + t - 1 + d a x + t
= m x + t . t V - 1 + d . t V
A derivation is required for (i). Alternative “steps” are possible.
(ii)
If we consider the short time interval (t, t + dt) then equation implies
t + dt V
- t V = t V . d dt - 1 . dt + t V m x + t dt + o ( dt )
where
t V @ dt Interest earned on the reserve over (t, t + dt)
- 1 . dt Annuity payments made in (t, t + dt)
+ t V m x + t . dt Reserves “released” as a result of deaths
in (t, t + dt)
and these are all the “changes” that can happen over (t, t + dt).
[Note: the candidates may write the expression as:
t V (1
+ d . dt ) =
t + dt V
+ 1 . dt - dt . m x + t . t V + o ( dt )
which might be easier to interpret as [income] = [outgo]]
6
(a)
H 0 : the observed transition rates m ˆ x + 1⁄2 come from a population in which
the standard table rates are the true rates.
c 2 with m degrees of freedom
(b)
H 0 : the observed transition rates m ˆ x + 1⁄2 come from a population in which
the graduated transition rates are the true rates
In the graduation process four parameters have been estimated so
sampling distribution is c 2 with m - 4 degrees of freedom.
%%---- Page 5Subject 104 (Survival Models) — %%%%%%%%%%%%%%%%%%%%%%%%%%%%%%%%%%%%%%%5
(c)
H 0 : same as in (b).
1 degree of freedom is lost for each parameter estimated, plus an
unknown additional number for each constraint imposed by the standard
table; say p degrees of freedom, so:
c 2 with m - 1 - p degrees of freedom.
(d)
H 0 : same as in (b)
The best fitting curve will usually be drawn as a series of curved segments
joined smoothly.
Each segment imposes a constraint of height, slope and curvature; so lose
2 or 3 degrees of freedom for each section of about 10 ages drawn; so in 2
sections for example we have
c 2 with say m - 5 degrees of freedom.
7
(i) X - Y is the present value of a deferred whole of life assurance with a sum
assured of 1 payable at the end of the year of death of a life now aged x
provided the life dies after age x + n.
(ii) X = v k+1
all k
Cov(X, Y)
= E[XY] - E[X] E[Y]
k = n - 1
Now E[XY] =
å
k = 0
k + 1
ì
ï v
Y = í
ï
î 0
k 3 n
( v k + 1 ) 2 P [ K x = k ] +
k = n - 1
=
0 £ k < n
å ( v
k = 0
2 k + 1
)
k =¥
å v
k = n
k + 1
 ́ 0  ́ P [ K x = k ]
P [ K x = k ]
= 2 A x 1: n
Where 2 A is determined using a discount function v 2 , i.e. using an interest
rate
i* = (1 + i) 2 - 1 = 2i + i 2
Then: Cov(X, Y) = 2 A x 1 : n - A x . A 1 x : n
%%---- Page 6Subject 104 (Survival Models) — %%%%%%%%%%%%%%%%%%%%%%%%%%%%%%%%%%%%%%%5
Now: Var(X - Y) = Var(X) + Var(Y) - 2 Cov(X, Y)
= ( 2 A x - ( A x ) 2 ) + ( 2 A x 1 : n - ( A x 1 : n ) 2 ) - 2( 2 A x 1 : n - A x . A x 1 : n )
= ( 2 A x + 2 A x 1 : n | - 2 2 A x 1 : n | ) - (( A x ) 2 + ( A 1 x : n | ) 2 - 2 A 1 x : n | A x )
= 2 A x - 2 A x 1 : n - ( A x - A 1 x : n ) 2
= 2 A x - 2 A x 1 : n - ( n 1⁄2 A x ) 2
Alternative approaches are possible.
