\documentclass[a4paper,12pt]{article}
%%%%%%%%%%%%%%%%%%%%%%%%%%%%%%%%%%%%%%%%%%%%%%%%%%%%%%%%%%%%%%%%%%%%%%%%%%%%%%%%%%%%%%%%%%%%%%%%%%%%%%%%%%%%%%%%%%%%%%%%%%%%%%%%%%%%%%%%%%%%%%%%%%%%%%%%%%%%%%%%%%%%%%%%%%%%%%%%%%%%%%%%%%%%%%%%%%%%%%%%%%%%%%%%%%%%%%%%%%%%%%%%%%%%%%%%%%%%%%%%%%%%%%%%%%%%
\usepackage{eurosym}
\usepackage{vmargin}
\usepackage{amsmath}
\usepackage{graphics}
\usepackage{epsfig}
\usepackage{enumerate}
\usepackage{multicol}
\usepackage{subfigure}
\usepackage{fancyhdr}
\usepackage{listings}
\usepackage{framed}
\usepackage{graphicx}
\usepackage{amsmath}
\usepackage{chngpage}
%\usepackage{bigints}

\usepackage{vmargin}
% left top textwidth textheight headheight
% headsep footheight footskip
\setmargins{2.0cm}{2.5cm}{16 cm}{22cm}{0.5cm}{0cm}{1cm}{1cm}
\renewcommand{\baselinestretch}{1.3}

\setcounter{MaxMatrixCols}{10}
\begin{document}


5
Thiele’s equation for the policy value at duration t, t V , of an immediate life
annuity payable continuously at a rate of £1 per annum from age x is:
∂
t V = \mu x+t . t V − 1 + δ . t V
∂ t
(i) Derive this result algebraically showing all the steps in your
argument.
(ii) Explain this result by general reasoning.
104—2


%%%%%%%%%%%%%%%%%%%%%%%%%%%%%%%%%%%%%%%%%%%%%%%%%%%%%%%%%%%%%%%%%%%%%%%%%%%%%


5
(i)
The life annuity will be secured by a single payment at age x and so the
policy value at duration t will be
t V
= a x + t
s =¥
t V
= a x + t =
ò
e -d s . s p x+t . ds
s = 0
So:
¶
t V
¶ t
=
¶
¶ t
s =¥
=
ò
s = 0
Now
s =¥
ò
e -d s s p x+t . ds
s = 0
e -d s
¶
s p x+t . ds
¶ t
¶
¶ æ l x + t + s ö
ç
÷
s p x+t =
¶ t
¶ t è l x + t ø
=
l x + t ( -m x + t + s . l x + t + s ) - l x + t + s ( -m x + t l x + t )
l x 2 + t
= s p x+t (m x+t - m x+t+s )
\
¶
t V
¶ t
s =¥
=
ò
e -d s . s p x + t (m x+t - m x+t+s ) ds
s = 0
s =¥
= m x + t a x + t -
ò
s = 0
%%---- Page 4
e -d s s p x + t m x + t + s . ds

%%%%%%%%%%%% Subject 104 (Survival Models) — %%%%%%%%%%%%%%%%%%%%%%%%%%%%%%%%%%%%%%%5
= m x + t a x + t
ì
ï
- í - e -d s . s p x + t
ï
î
¥
0
ü
ï
e -d s s p x + t . ds ý
ï
s = 0
þ
s =¥
- ( -d )
ò
= m x + t a x + t - 1 + d a x + t
= m x + t . t V - 1 + d . t V
A derivation is required for (i). Alternative “steps” are possible.

%%%%%%%%%%%%%%%%%%%%%%%%%%%%%%%%%%%%%

(ii)
If we consider the short time interval (t, t + dt) then equation implies
t + dt V
- t V = t V . d dt - 1 . dt + t V m x + t dt + o ( dt )
where
t V @ dt Interest earned on the reserve over (t, t + dt)
- 1 . dt Annuity payments made in (t, t + dt)
+ t V m x + t . dt Reserves “released” as a result of deaths
in (t, t + dt)
and these are all the “changes” that can happen over (t, t + dt).
[Note: the candidates may write the expression as:
t V (1
+ d . dt ) =
t + dt V
+ 1 . dt - dt . m x + t . t V + o ( dt )
which might be easier to interpret as [income] = [outgo]]


\end{document}
