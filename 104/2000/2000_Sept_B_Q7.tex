\documentclass[a4paper,12pt]{article}

%%%%%%%%%%%%%%%%%%%%%%%%%%%%%%%%%%%%%%%%%%%%%%%%%%%%%%%%%%%%%%%%%%%%%%%%%%%%%%%%%%%%%%%%%%%%%%%%%%%%%%%%%%%%%%%%%%%%%%%%%%%%%%%%%%%%%%%%%%%%%%%%%%%%%%%%%%%%%%%%%%%%%%%%%%%%%%%%%%%%%%%%%%%%%%%%%%%%%%%%%%%%%%%%%%%%%%%%%%%%%%%%%%%%%%%%%%%%%%%%%%%%%%%%%%%%

\usepackage{eurosym}
\usepackage{vmargin}
\usepackage{amsmath}
\usepackage{graphics}
\usepackage{epsfig}
\usepackage{enumerate}
\usepackage{multicol}
\usepackage{subfigure}
\usepackage{fancyhdr}
\usepackage{listings}
\usepackage{framed}
\usepackage{graphicx}
\usepackage{amsmath}
\usepackage{chngpage}

%\usepackage{bigints}
\usepackage{vmargin}

% left top textwidth textheight headheight

% headsep footheight footskip

\setmargins{2.0cm}{2.5cm}{16 cm}{22cm}{0.5cm}{0cm}{1cm}{1cm}

\renewcommand{\baselinestretch}{1.3}

\setcounter{MaxMatrixCols}{10}

\begin{document}

7
(ii) What does the model (if correct) tell you about the survival function of a
male smoker aged 30 at entry, relative to that of a female smoker aged 40
at entry?

(iii) What does the model (if correct) tell you about the survival function of a
female smoker aged 30 at entry, relative to that of a male non-smoker
aged 40 at entry?


[Total 7]
(i) Consider an annuity payable continuously during the lifetime of ( x ), but
for at most 20 years. The rate of payment at time t is £2,000 per annum
for 0 \leq t \leq 5, and £10,000 per annum for 5 < t \leq 20. The force of mortality
to which ( x ) is subject is assumed to be constant at 0.01.
Calculate the expected present value of this annuity at a force of interest
of 0.06.

(ii)
104(S)—4
Calculate the expected present value of the benefits of an endowment
assurance issued to the same life as in part (i), paying a sum assured of
£20,000 at the end of five years or immediately on earlier death. The
mortality and interest assumptions are the same as in part (i).



7
(i)
EPV = ò 2000 e
20
− 0 . 06 s
e
− 0 . 01 s
ds + ò 10000 e − 0 . 07 s ds
0
5
= 2000a 5 | (at a force of 0.07) + 10000 e − 0 . 35 a 15 | (at force of 0.07)
= 2000 × 4.21874158 + 10000 × e − 0.35 × 9.28660358 = 73879.07
(ii)
Page 6
(
EPV = 20,000 1 − δ . a x :5
)%%%%%%%%%%%%%%%%%%%%%%%%%%%%%%%%%%%%%%%%%%%%%%5for September 2000
From (i), a x :5 = 4.2187
Hence EPV = 20,000 (1 − 0.06 × 4.2187) = £14,937.56
Other expressions will lead to the same numerical results in (i) and (ii).

\end{document}
