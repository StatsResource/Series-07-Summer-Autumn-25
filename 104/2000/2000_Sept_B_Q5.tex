\documentclass[a4paper,12pt]{article}

%%%%%%%%%%%%%%%%%%%%%%%%%%%%%%%%%%%%%%%%%%%%%%%%%%%%%%%%%%%%%%%%%%%%%%%%%%%%%%%%%%%%%%%%%%%%%%%%%%%%%%%%%%%%%%%%%%%%%%%%%%%%%%%%%%%%%%%%%%%%%%%%%%%%%%%%%%%%%%%%%%%%%%%%%%%%%%%%%%%%%%%%%%%%%%%%%%%%%%%%%%%%%%%%%%%%%%%%%%%%%%%%%%%%%%%%%%%%%%%%%%%%%%%%%%%%

\usepackage{eurosym}
\usepackage{vmargin}
\usepackage{amsmath}
\usepackage{graphics}
\usepackage{epsfig}
\usepackage{enumerate}
\usepackage{multicol}
\usepackage{subfigure}
\usepackage{fancyhdr}
\usepackage{listings}
\usepackage{framed}
\usepackage{graphicx}
\usepackage{amsmath}
\usepackage{chngpage}

%\usepackage{bigints}
\usepackage{vmargin}

% left top textwidth textheight headheight

% headsep footheight footskip

\setmargins{2.0cm}{2.5cm}{16 cm}{22cm}{0.5cm}{0cm}{1cm}{1cm}

\renewcommand{\baselinestretch}{1.3}

\setcounter{MaxMatrixCols}{10}

\begin{document}

5
A life insurance company prices its long-term sickness policies using the following
3-state continuous-time Markov model, in which the forces of transition \sigma, \rho, \mu
and υ are assumed to be constant:
\sigma
State 1
(healthy)
\mu
State 2
(sick)
\rho
υ
State 3
(dead)
For a group of policyholders observed over a 1-year period, there are:
10 transitions from state 1 to state 2;
7 transitions from state 2 to state 1;
2 deaths from state 1;
3 deaths from state 2.
The total time spent in state 1 is 512 years, and in state 2 is 20 years.
(i) Write down the likelihood function for these data. 
(ii) Hence derive the maximum likelihood estimate of \sigma. 
(iii) Calculate an estimate of the standard error of \sigmâ , where \sigmâ is the
maximum likelihood estimator of \sigma .
104(S)—3

[Total 6]


Page 5%%%%%%%%%%%%%%%%%%%%%%%%%%%%%%%%%%%%%%%%%%%%%%5for September 2000
5
(i)
(ii)
L = exp ( − 512 ( \sigma + \mu ) ) exp ( − 20 ( \rho + υ ) ) \sigma 10 \rho 7 \mu 2 υ 3
l = ln L = − 512 \sigma + 10 ln \sigma + const . ( w . r . t . \sigma )
∂ l
10
10
0 =
= − 512 +
\sigma ˆ =
= 0 . 0195 p.a.
∂ \sigma
\sigma
512
Þ
A derivation was required.
(iii)
−
10
∂ 2 l
= 2 , hence we can estimate Var \sigma ˆ by
2
∂ \sigma
\sigma
æ 10 ö
ç 2 ÷
è \sigma ø
− 1
\sigma ˆ 2
, hence by
10
æ
ç =
ç
è
\sigmâ ö
÷
512 ÷ ø
Hence estimated standard deviation of \sigmâ is \sigma ˆ
\end{document}
