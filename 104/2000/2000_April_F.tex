\documentclass[a4paper,12pt]{article}

%%%%%%%%%%%%%%%%%%%%%%%%%%%%%%%%%%%%%%%%%%%%%%%%%%%%%%%%%%%%%%%%%%%%%%%%%%%%%%%%%%%%%%%%%%%%%%%%%%%%%%%%%%%%%%%%%%%%%%%%%%%%%%%%%%%%%%%%%%%%%%%%%%%%%%%%%%%%%%%%%%%%%%%%%%%%%%%%%%%%%%%%%%%%%%%%%%%%%%%%%%%%%%%%%%%%%%%%%%%%%%%%%%%%%%%%%%%%%%%%%%%%%%%%%%%%

\usepackage{eurosym}
\usepackage{vmargin}
\usepackage{amsmath}
\usepackage{graphics}
\usepackage{epsfig}
\usepackage{enumerate}
\usepackage{multicol}
\usepackage{subfigure}
\usepackage{fancyhdr}
\usepackage{listings}
\usepackage{framed}
\usepackage{graphicx}
\usepackage{amsmath}
\usepackage{chngpage}

%\usepackage{bigints}
\usepackage{vmargin}

% left top textwidth textheight headheight

% headsep footheight footskip

\setmargins{2.0cm}{2.5cm}{16 cm}{22cm}{0.5cm}{0cm}{1cm}{1cm}

\renewcommand{\baselinestretch}{1.3}

\setcounter{MaxMatrixCols}{10}

\begin{document}


A life insurance company issues only annual premium life assurance policies.
The company keeps records of its life assurance policies in two files: an in-
force file and a claims file.
For each policy paying premiums the in-force file includes the following
information:
• age last birthday at the date of policy issue
• smoking status (smoker or non-smoker)
• sex
• type of policy (temporary assurance, whole life assurance or endowment
assurance)
On 1 January each year the company tabulates summary statistics from this
file. For each age x where:
x = age last birthday at date of policy issue
+ number of annual premiums paid since policy issue
a count of the number of policies sub-divided by smoking status, sex and type
of policy is tabulated.
For each policy for which a death claim has been paid the claims file includes
the same information as the in-force file.
On 1 January each year the company tabulates summary statistics from this
file for death claims paid in the previous calendar year. For each age x where:
x = age last birthday at date of policy issue
+ number of annual premiums paid up to the date of death
a count of the number of policies divided by smoking status, sex and type of
policy is tabulated.
The company wishes to investigate the recent mortality experience of its life
assurance policies.
104—7
%%%%%%%%%%%%%%%%%%%%%%%%%%%%%%%%%%%%%%%(i)
(a) Explain why the subdivisions of data by smoking status, sex and
type of policy are important for the company’s mortality analysis.
(b) Describe the statistical problems that can arise when data are
sub-divided in this way.

(ii) Defining suitable symbols, derive a formula for the estimation of the
force of mortality using the data for an age group x according to the
definitions given. State any assumptions that are required for your
formula. State, with reasons, the age to which your estimate of the
force of mortality applies.
%%%%%%%%%%%%%%%%%%%%%%%%%%%%%%%%%%%%%%%
(iii) The company proposes to graduate the crude estimates obtained from
its investigations before using them for determining premiums.
Describe an appropriate method of graduation indicating how the
suitability of the graduated rates would be assessed. Detailed
descriptions of graduation tests are not required.
[6]
[Total 17]
104—8


%%%%%%%%%%%%%%%%%%%%%%%%%%%%%%%%%%%%%%%%%%%%%%%%%%%%%%%%%%%%%%%%%%%%%%%%%%%%%%%%%%%%%%%%%%%%%



12
(i)
11
12
= 1 p 12
60 + 1 p 60 . 1 p 61
.25
(1 - e -.342 ) = .2117
.342
1 p 12
60 =
1 -.342
p 11
= .7103
60 = e
1 p 12
61 =
2 p 12
60 = .2117 + .7103  ́ .0927 = .2775
.10
(1 - e -.154 ) = .0927
.154
(a) There is considerable evidence from other studies that mortality
rates vary with smoking status (smokers have higher rates at all
ages),
sex (males have higher mortality at all ages)
and policy class (permanent assurances i.e. whole life and
endowment have different mortality than temporary
assurances).
To ensure homogeneity policies should be divided so that those of
the same smoking status, sex and policy class fall in one group.
All of the statistical models of mortality which might be
parameterised from the data require the data to be
homogeneous.
(b) Too many subdivisions of data can make sample sizes very
small.
·
·
·
so that the estimates will have large standard errors
and hence little meaningful interpretation can be made
may lead to identification of spurious selection
Other relevant comments given credit.
(ii)
Choose a period of investigation from time 0 to time T, where T is a whole
number of calendar years (say about 4) and 0 corresponds to the start of a
calendar year.
The x = age next birthday on policy anniversary before death.
Let Sq x be total deaths labelled x in all calendar years during period of
investigation.
%%---- Page 13Subject 104 (Survival Models) — %%%%%%%%%%%%%%%%%%%%%%%%%%%%%%%%%%%%%%%5
Let P x (t) be a census at time t after start of period of investigation of those
lives having age label x at time t. Then
E x c =
t = T
ò
P x ( t ) . dt
t = 0
; 1⁄2P x (o) +
t = T - 1
å P ( t ) + 1⁄2P (T)
t = 1
x
x
assuming P x (t) varies linearly with t over each calendar year.
Then: m ˆ x =
5G x
E x c
estimates m x
Policy year rate interval, average x - 1⁄2 at start assuming
birthdays are uniformly distributed over the policy year, and that the
force of mortality is constant over each year of age.
(iii)
The investigations are for widely sold classes of business, so graduation
with reference to a standard mortality table for similar classes of business
would be appropriate,
e.g. using a published table for male permanent (whole life and
endowment) assurances.
Separate graduations would probably be necessary by type of policy, sex,
and smoking status.
The standard table should be chosen so that the characteristics of the
company’s business are similar to those of the standard table, e.g.
geographical regions in which the business is sold, policy terms and
conditions.
Plots of the standard table and the observed experience should be
compared for “shape”. It is important that the shapes are similar.
Graphical investigations should find a suitable relationship f ( q x s ), f ( m s x ),
between the standard table rates and the current experience, e.g.
o
q x = a + bq x s reflects and approximate straight line relationship when q ˆ x
is plotted against q x s . A perfect fit is not expected particularly at ages
o
where the data are scanty. [ q x is graduated rate, q ˆ x is crude estimate].
%%---- Page 14
%%-- Subject 104 (Survival Models) — %%%%%%%%%%%%%%%%%%%%%%%%%%%%%%%%%%%%%%%5
A best fitting line should be fitted. Weighted least squares provides a
satisfactory fitting method, e.g. choose a and b to minimise
å w ( q
x
x
o
x
- q ˆ x ) 2 =
where w x = [Var( q  x )] - 1
å w ( a + bq
x
s
x
- q ˆ x ) 2
q  x is estimator of q ˆ x
The resulting graduated rates are tested for adherence to the crude
estimates.
o
Smoothness of q n will be ensured by the method of graduation, since
smoothness is “borrowed” from the standard table.
Tests for “adherence to data” should include a test for overall fit (c 2 test)
o
and tests for the pattern of residuals ( q x - q ˆ x )
by size (sign test, standardised deviations test)
and by pattern with age (cumulative deviations test, serial correlation
test.
If satisfactory adherence to data cannot be obtained then the chosen
relationship should be reassessed, i.e. choose new f ( q x s ) or f ( m sx ) and
repeat fitting and testing.
If a satisfactory relationship cannot be found, then the choice of standard
table should be reassessed and the fitting procedure repeated.
Similar description of fitting parametric formula also acceptable.
Graphical methods not acceptable.
o
ˆ s can be tabulated with sufficient
The graduated rates, e.g. q x = a ˆ + bq
x
significant figures to facilitate their use in premium rate calculations.
%%---- Page 15
