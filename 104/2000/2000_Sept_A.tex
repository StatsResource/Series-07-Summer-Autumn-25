ã Faculty of Actuaries
ã Institute of Actuaries1
(i) Explain precisely what is meant by
(ii) Write down an integral expression for
3  5
q 50 .
3  5

q 50 in terms of the hazard rates
over the appropriate age range only.
(iii)
Evaluate
3  5

q 50 using the A1967-70 Ultimate mortality table.

[Total 4]
2
A life insurance company sells whole-life assurance policies with a sum assured of
£20,000, payable at the end of the year of death. The premium is £520 payable
annually in advance until the death of the policyholder.
A life now aged 50 purchased a policy exactly one year ago, and is now due to pay
the second annual premium.
(i) Find the expected present value of the future loss to the company arising
from this policy.

(ii) Show that the variance of the present value of the future loss from this
policy can be expressed as:
′ + c
b . A 50
Determine the numerical values of b and c , and the rate of interest used
′ .

to evaluate A 50
Basis: mortality A1967-70 (Ultimate), interest 4% per annum. Ignore expenses.
[Total 5]
3
Is it possible for t V x to exceed t V x : n ? Justify your answer, either by providing
proof or by showing a suitable example, as appropriate.
4

You are given that q 75 = 0.06229.
(i)
(ii)
104(S)—2
Calculate 1⁄4 p 75 and 1⁄4 p 753⁄4 , assuming a uniform distribution of deaths
between integer ages.

Repeat part (i), using the alternative assumption that a constant hazard
rate applies between integer ages.

[Total 6]


%%%%%%%%%%%%%%%%%%%%%%%%%%%%%%%%%%%%%%%%%%%%%%%%%%%%%%%%%%%%%%%%%%%%%

1
(i) Pr(a life aged 50 dies between exact ages 53 and 58)
(ii) 3|5 q 50
= ò
8
t
3
p 50 . \mu 50 + t dt
5
= 3 p 50 ò t p 53 . \mu 53 + t dt
0
3
5
t
= exp é ê − ò \mu 50 + r dr ù ú ò exp é ê − ò \mu 53 + r dr ù ú \mu 53 + t dt
ë
û
0
ë
0
o
û
Several other alternatives were acceptable, in particular
8
ò 3
t
exp é ê − ò \mu 50 + r dr ù ú \mu 50 + t dt
ë
ì
ï
0
3
û
é
ü
ï ì
ï
ï
þ ê
ë ï
î
5
ù
ü
ï
or exp í − ò \mu 50 + r dr ý ê 1 − exp í − ò \mu 53 + r dr ý ú
ï
î
0
0
ï
þ ú
û
3
8
ì
ü
ì
ü
ï
ï
ï
ï
or exp í − ò \mu 50 + r dr ý − exp. í − ò \mu x 50 + r dr ý
ï 0
ï
ï 0
ï
î
þ
î
þ
3
5
r
ì
ü
ì
ü
ï
ï
ï
ï
or exp í − ò \mu 50 + r . dr ý ò exp. í − ò \mu 53 + s . ds ý \mu 53 + r . dr
ï 0
ï 0
ï 0
ï
î
þ
î
þ
(iii)
2
l 53 − l 58 32,143.546 − 30,795.116
=
= 0.04127
l 50
32,669.855
 
Present value of the loss = λ = 20000 v K + 1 − 520 a
where K = curtate future
K + 1 |
lifetime, of the life aged 50.
(i) EPV loss
  50 = 20 , 000 × 0 . 38450 − 520 × 16 . 003 = − 631 . 56
= Ε ( λ ) = 20 , 000 A 50 − 520 a
(ii) λ = 20000 v K + 1 − 520 ç
æ 1 − v K + 1 ö
÷
è d 0.04 ø
= constant + ( 520 d 0 . 04 + 20 , 000 ) v K + 1
(
)
Var v K + 1 = 2 A 50 − A 50 2 (where 2 A is @ 8.16% and A @ 4%)
= A 50 − 0 . 38450 2 ,
2
Then Var ( λ ) = 33 , 520 2
(
2
A 50 − 0 . 38450 2
)
So b = 33520 2 = 1,123,590,400
Page 3%%%%%%%%%%%%%%%%%%%%%%%%%%%%%%%%%%%%%%%%%%%%%%5for September 2000
C = -(33,520 x 0.38450) 2 = − 166,111,886
An approach using co-variances is possible, but much more complicated.
3
In both cases the retrospective policy value is
D x æ
ö
  − A 1 ÷ .
ç P a
x : t | ø
D x + t è x : t |
P x > P x :n |
V x will exceed t V x :n | if
t
1
1
− d >
− d
 x
a
a  x : n
a  x < a  x : n
  x = a
  + a nonnegative term.
But a
x :n |
Hence it is not possible.
Alternatively,
If
1 −
t V x
> t V x : n then

a
 x + t
a
> 1 − x + t : n − t
 x

a
a
x : n
which means
1 −

a
 x + t
a
< x + t : n − t
 x

a
a
x : n
But a  x = a  x : n + v n n P x a  x + n
a  x + t = a  x + t : n − 1 + v n − t
 x + n
n − t P x + t a
So the inequality will hold if
( a

x + t : n − t
)
which means
 x + n a


 x + n
v n − t n − t P x + t a
< a
v n n P x a
x : n
x + t : n − t


a
< v t t P x a
x : n
x + t : n − t


= a
− a
x : n
x : t
Page 4
(
 x + n a


 x + n
+ v n − t n − t P x + t a
< a
a  x : n + v n n P x a
x : n
x + t : n − t
)%%%%%%%%%%%%%%%%%%%%%%%%%%%%%%%%%%%%%%%%%%%%%%5for September 2000
which is not true, hence the original assumption is untrue.
[Or: Write in terms of commutation functions
N x + t
N − N x + n
< x + t
N x
N x − N x + n
which implies
N x+t > N x
Which is not true, hence the original assumption is untrue.
Credit was also given for labelled sketches on t V x and t V x : n against t, for well
chosen numerical examples and for careful verbal arguments.
It was appreciated that the wording of the question would have been improved if it
specifically asked for a comparison for 0 \leq t \leq n.
%%%%%%%%%%%%%%%%%%%%%%%%%%%%
4
(i)
0 . 25
p 75 = 1 − 0 . 25 q 75 = 1 − 0 . 25 × q 75 = 0 . 98443 assuming uniform distribution
of deaths
= 1− 0.25 × 0.06229 = 1 − 0.01557 = 0.98443
p 75 = 0 . 75 p 75 × 0 . 25 p 75 . 75 = (1− 0.75 × q 75 ) 0.25 p 75.75 assuming uniform
distribution of deaths
Þ
=
(ii)
0.25
p 75.75 =
1 − q 75
1 − 0.75 q 75
1 − 0.06229
0.93771
=
= 0.98366
1 − 0.75 × 0.06229 0.95328
\mu satisfies e − \mu = p 75 = 0.93771 assuming a constant force of mortality
So \mu = 0.0643145
Then
And
0.25
0.25
p 75 = e − 0.25 \mu = e − 0.016078 = 0.98405
p 75.75 = e − 0.25 \mu = e − 0.016078 = 0.98405
Can also use e − 0.25 \mu = (0.93771) 0.25 = 0.98405
A complete verbal argument that the calendar values of the hazard ratios
resulted in the relationships between the survival functions is also
acceptable.
