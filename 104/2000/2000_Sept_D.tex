

104(S)—611                               10
The following data relate to 12 patients who had an operation which was
intended to correct a life-threatening condition, where time 0 is the start of the
period of the investigation:
Patient number Time of operation
(in weeks) Time observation
ended (in weeks) Reason observation
ended
1
2
3
4
5
6
7
8
9
10
11
12 0
0
0
4
5
10
20
44
50
63
70
80 120
68
40
120
35
40
120
115
90
98
120
110 Censored
Death
Death
Censored
Censored
Death
Censored
Death
Death
Death
Death
Death
You can assume that censoring was non-informative with regard to the survival
of any individual patient.
(i) Compute the Nelson-Aalen estimate of the cumulative hazard function,
Λ ( t ) , where t is the time since having the operation.

(ii) Using the results of part (i), deduce an estimate of the survival function
for patients who have had this operation.

(iii) Estimate the probability of a patient surviving for at least 70 weeks after
undergoing the operation.

[Total 9]

%%%%%%%%%%%%%%%%%%%%%%%%%%%%%%%%%%%%%%%%%%%%%%%%%%%%%%%%%%%%%5
Consider the following multiple-decrement model, in which S ( t ), the state
occupied at time t of a life initially aged x , is assumed to follow a continuous-time
Markov process.
State 1
\mu 01
\mu 02
State 0
\mu 03
State 2
State 3
The usual notation t p x ab = P ( S ( t ) = b  S (0) = a ) is used, and the forces of
transition (\mu 0 k , k = 1, 2, 3) are assumed to be constant.
(i) Derive differential equations for t p x 00 and t p x 01 , and state the relevant
boundary condition for each equation.

(ii) Hence find simple explicit expressions for t p x 00 and t p x 01 .
(Your final answers should be expressed in terms of \mu 0 k , k = 1, 2, 3 only.)

[Total 10]
%%%%%%%%%%%%%%%%%%%%%%%%%%%%%%%%%%%%%%%%%%%%%%%%%%%%%%%%%%%
10
T
n j
d j
å d n
j
, where:
j
(i)
d j n j
ˆ ( t) =
Λ
0 \leq t < 30
30 \leq t < 35
35 \leq t < 40
40 \leq t < 50
50 \leq t < 68
68 \leq t < 71
71 \leq t < 120
(iii)
Page 8
12
12
9
8
6
5
4
S ˆ ( 70 ) = 0 . 4088
0
2
1
2
1
1
1
0
2/12
1/9
2/8
1/6
1/5
1/4
å
(ii)
d j
0
0.1667
0.2778
0.5278
0.6944
0.8944
1.1444
n j
ˆ ( t ))
S ˆ ( t ) = exp( − Λ
1
0.8465
0.7575
0.5899
0.4994
0.4088
0.3184%%%%%%%%%%%%%%%%%%%%%%%%%%%%%%%%%%%%%%%%%%%%%%5for September 2000
11
(i) (a)
t + h
p x 00 =
å
3
æ
t
k = 0
p x 0 k h p x k + 0 t = t p x 00 h p x 00 + t = t p x 00 ç 1 −
è
å
3
k = 1
ö
h
p x 0 + kt ÷
ø
making Markov assumption, all other transitions have zero probability,
by law of total probability
å h \mu
3
æ
= t p x 00 ç 1 −
è
0 k
ö
+ o ( h ) ÷ using definitions of \mu ok k = 1,2,3
ø
k = 1
then
Þ
t + h
å
3
p x 00 − t p x 00
=− t p x 00 \mu 0 k + o ( h ) h
h
k = 1
taking limits
hence
(b)
t + h
∂
∂ t
p x 01 =
(
)
t
å
3
k = 0
æ
p x 00 = lim ç ç
t
h → 0 +
t + h
è
å
p x 00 − t p x 00 ö
æ 3
ö
÷ = − ç
\mu 0 k ÷ t p x 00 , and 0 p x 00 = 1 .
÷
h
è k = 1
ø
ø
00
01
p x 0 k h p x k + 1 t = t p x 00 h p x 01 + t + t p x 01 h p 11
+ o ( h )) + t p x 01 × 1
x + t = t p x ( h \mu
Then
Þ
p x 01 − t p x 01
= \mu 01 t p x 00 + o ( h ) h ,
h
∂
hence
( t p x 01 ) = \mu 01 t p x 00 , and 0 p x 01 = 0 .
∂ t
(ii)
t + h
(a)
å
3
∂
ln t p x 00 = − \mu 0 k
∂ t
k = 1
(
)
then
00
Þ t p x
0
æ
= const × exp ç − t
è
p x 00 = 1
å \mu
0 k
k
ö
÷
ø
Þ const = 1 .
(b)
∂
∂ t
(
t
p
01
x
) = \mu
01
t
p
00
x
Þ p
t
t
01
x
− 0 p
01
x
= \mu
01
ò
s
p x 00 ds ,
0
Page 9%%%%%%%%%%%%%%%%%%%%%%%%%%%%%%%%%%%%%%%%%%%%%%5for September 2000
Now
t
p
01
x
0
p x 01 = 0 so
= \mu
01 æ
å
3
\mu
ç
è k = 1
0 k
ö
÷
ø
− 1
[ − e
− s å \mu 0 k
]
t
s = 0
=
\mu 01
å \mu
3
0 k
3
æ
− t å \mu 0 k
ç
1 − e k = 1
ç
ç
è
ö
÷
÷
÷
ø
k = 1
