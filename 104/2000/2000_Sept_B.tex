5
A life insurance company prices its long-term sickness policies using the following
3-state continuous-time Markov model, in which the forces of transition \sigma, \rho, \mu
and υ are assumed to be constant:
\sigma
State 1
(healthy)
\mu
State 2
(sick)
\rho
υ
State 3
(dead)
For a group of policyholders observed over a 1-year period, there are:
10 transitions from state 1 to state 2;
7 transitions from state 2 to state 1;
2 deaths from state 1;
3 deaths from state 2.
The total time spent in state 1 is 512 years, and in state 2 is 20 years.
(i) Write down the likelihood function for these data. 
(ii) Hence derive the maximum likelihood estimate of \sigma. 
(iii) Calculate an estimate of the standard error of \sigmâ , where \sigmâ is the
maximum likelihood estimator of \sigma .
104(S)—3

[Total 6]

%%%%%%%%%%%%%%%%%%%%%%%%%%%%%%%555
6
Suppose you have fitted the following proportional hazards regression model to
the mortality data for a sample of life-assurance policyholders:
h i ( t ) = h 0 ( t ) exp { 0 . 01 ( x i − 30 ) + 0 . 2 y i − 0 . 05 z i } ,
where:
h i (t ) denotes the hazard function for life i at duration t ;
h 0 ( t ) denotes the baseline hazard function at duration t ;
x i denotes the age at entry of life i ;
y i = 1 if life i is a smoker, otherwise zero; and
z i = 1 if life i is female, zero if male.
(i)
Describe the class of lives to which the baseline hazard function applies.

7
(ii) What does the model (if correct) tell you about the survival function of a
male smoker aged 30 at entry, relative to that of a female smoker aged 40
at entry?

(iii) What does the model (if correct) tell you about the survival function of a
female smoker aged 30 at entry, relative to that of a male non-smoker
aged 40 at entry?

[Total 7]
(i) Consider an annuity payable continuously during the lifetime of ( x ), but
for at most 20 years. The rate of payment at time t is £2,000 per annum
for 0 \leq t \leq 5, and £10,000 per annum for 5 < t \leq 20. The force of mortality
to which ( x ) is subject is assumed to be constant at 0.01.
Calculate the expected present value of this annuity at a force of interest
of 0.06.

(ii)
104(S)—4
Calculate the expected present value of the benefits of an endowment
assurance issued to the same life as in part (i), paying a sum assured of
£20,000 at the end of five years or immediately on earlier death. The
mortality and interest assumptions are the same as in part (i).



%%%%%%%%%%%%%%%%%%%%%%%%%%%%%%%%%%%%%%%%%%%%%%%%%%%%%%%%%%%%%%%%%%%%%%%

\newpage

Page 5%%%%%%%%%%%%%%%%%%%%%%%%%%%%%%%%%%%%%%%%%%%%%%5for September 2000
5
(i)
(ii)
L = exp ( − 512 ( \sigma + \mu ) ) exp ( − 20 ( \rho + υ ) ) \sigma 10 \rho 7 \mu 2 υ 3
l = ln L = − 512 \sigma + 10 ln \sigma + const . ( w . r . t . \sigma )
∂ l
10
10
0 =
= − 512 +
\sigma ˆ =
= 0 . 0195 p.a.
∂ \sigma
\sigma
512
Þ
A derivation was required.
(iii)
−
10
∂ 2 l
= 2 , hence we can estimate Var \sigma ˆ by
2
∂ \sigma
\sigma
æ 10 ö
ç 2 ÷
è \sigma ø
− 1
\sigma ˆ 2
, hence by
10
æ
ç =
ç
è
\sigmâ ö
÷
512 ÷ ø
Hence estimated standard deviation of \sigmâ is \sigma ˆ
6
(i)
(ii)
10 = 0 . 00618 .
male non-smokers aged 30 at entry.
h j ( t )
h i ( t )
=
exp( 0 + 0 . 2 + 0 )
= e − 0 . 05 = 0 . 9512
exp( 0 . 1 + 0 . 2 − 0 . 05 )
where life j is a male smoker aged 30 at entry, and
life i is a female smoker aged 40 at entry.
t
But S ( t ) = exp æ ç − ò h ( s ) ds ö ÷ , hence
S j ( t ) = ( S i ( t ) )
è
0 . 9512
ø
0
, which implies that
S j ( t ) > S i ( t ) for all t > 0.
(iii)
h j ( t )
h i ( t )
=
exp( 0 + 0 . 2 − 0 . 05 )
= e 0 . 05 = 1 . 0513 ,
exp( 0 . 1 + 0 + 0 )
where life j is a female smoker aged 30 at entry, and
life i is a male non-smoker aged 40 at entry.
Hence S j ( t ) = ( S i ( t ) )
1 . 0513
, which implies that
S j ( t ) < S i ( t ) for all t > 0.
5
7
(i)
EPV = ò 2000 e
20
− 0 . 06 s
e
− 0 . 01 s
ds + ò 10000 e − 0 . 07 s ds
0
5
= 2000a 5 | (at a force of 0.07) + 10000 e − 0 . 35 a 15 | (at force of 0.07)
= 2000 × 4.21874158 + 10000 × e − 0.35 × 9.28660358 = 73879.07
(ii)
Page 6
(
EPV = 20,000 1 − δ . a x :5
)%%%%%%%%%%%%%%%%%%%%%%%%%%%%%%%%%%%%%%%%%%%%%%5for September 2000
From (i), a x :5 = 4.2187
Hence EPV = 20,000 (1 − 0.06 × 4.2187) = £14,937.56
Other expressions will lead to the same numerical results in (i) and (ii).
