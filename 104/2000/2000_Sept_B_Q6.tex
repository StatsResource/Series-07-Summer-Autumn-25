\documentclass[a4paper,12pt]{article}

%%%%%%%%%%%%%%%%%%%%%%%%%%%%%%%%%%%%%%%%%%%%%%%%%%%%%%%%%%%%%%%%%%%%%%%%%%%%%%%%%%%%%%%%%%%%%%%%%%%%%%%%%%%%%%%%%%%%%%%%%%%%%%%%%%%%%%%%%%%%%%%%%%%%%%%%%%%%%%%%%%%%%%%%%%%%%%%%%%%%%%%%%%%%%%%%%%%%%%%%%%%%%%%%%%%%%%%%%%%%%%%%%%%%%%%%%%%%%%%%%%%%%%%%%%%%

\usepackage{eurosym}
\usepackage{vmargin}
\usepackage{amsmath}
\usepackage{graphics}
\usepackage{epsfig}
\usepackage{enumerate}
\usepackage{multicol}
\usepackage{subfigure}
\usepackage{fancyhdr}
\usepackage{listings}
\usepackage{framed}
\usepackage{graphicx}
\usepackage{amsmath}
\usepackage{chngpage}

%\usepackage{bigints}
\usepackage{vmargin}

% left top textwidth textheight headheight

% headsep footheight footskip

\setmargins{2.0cm}{2.5cm}{16 cm}{22cm}{0.5cm}{0cm}{1cm}{1cm}

\renewcommand{\baselinestretch}{1.3}

\setcounter{MaxMatrixCols}{10}

\begin{document}


%%%%%%%%%%%%%%%%%%%%%%%%%%%%%%%555
6
Suppose you have fitted the following proportional hazards regression model to
the mortality data for a sample of life-assurance policyholders:
h i ( t ) = h 0 ( t ) exp { 0 . 01 ( x i − 30 ) + 0 . 2 y i − 0 . 05 z i } ,
where:
h i (t ) denotes the hazard function for life i at duration t ;
h 0 ( t ) denotes the baseline hazard function at duration t ;
x i denotes the age at entry of life i ;
y i = 1 if life i is a smoker, otherwise zero; and
z i = 1 if life i is female, zero if male.
(i)
Describe the class of lives to which the baseline hazard function applies.




%%%%%%%%%%%%%%%%%%%%%%%%%%%%%%%%%%%%%%%%%%%%%%%%%%%%%%%%%%%%%%%%%%%%%%%

\newpage


6
(i)
(ii)
10 = 0 . 00618 .
male non-smokers aged 30 at entry.
h j ( t )
h i ( t )
=
exp( 0 + 0 . 2 + 0 )
= e − 0 . 05 = 0 . 9512
exp( 0 . 1 + 0 . 2 − 0 . 05 )
where life j is a male smoker aged 30 at entry, and
life i is a female smoker aged 40 at entry.
t
But S ( t ) = exp æ ç − ò h ( s ) ds ö ÷ , hence
S j ( t ) = ( S i ( t ) )
è
0 . 9512
ø
0
, which implies that
S j ( t ) > S i ( t ) for all t > 0.
(iii)
h j ( t )
h i ( t )
=
exp( 0 + 0 . 2 − 0 . 05 )
= e 0 . 05 = 1 . 0513 ,
exp( 0 . 1 + 0 + 0 )
where life j is a female smoker aged 30 at entry, and
life i is a male non-smoker aged 40 at entry.
Hence S j ( t ) = ( S i ( t ) )
1 . 0513
, which implies that
S j ( t ) < S i ( t ) for all t > 0.
5



\end{document}
