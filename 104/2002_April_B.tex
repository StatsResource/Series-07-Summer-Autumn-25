\documentclass[a4paper,12pt]{article}

%%%%%%%%%%%%%%%%%%%%%%%%%%%%%%%%%%%%%%%%%%%%%%%%%%%%%%%%%%%%%%%%%%%%%%%%%%%%%%%%%%%%%%%%%%%%%%%%%%%%%%%%%%%%%%%%%%%%%%%%%%%%%%%%%%%%%%%%%%%%%%%%%%%%%%%%%%%%%%%%%%%%%%%%%%%%%%%%%%%%%%%%%%%%%%%%%%%%%%%%%%%%%%%%%%%%%%%%%%%%%%%%%%%%%%%%%%%%%%%%%%%%%%%%%%%%

\usepackage{eurosym}
\usepackage{vmargin}
\usepackage{amsmath}
\usepackage{graphics}
\usepackage{epsfig}
\usepackage{enumerate}
\usepackage{multicol}
\usepackage{subfigure}
\usepackage{fancyhdr}
\usepackage{listings}
\usepackage{framed}
\usepackage{graphicx}
\usepackage{amsmath}
\usepackage{chngpage}

%\usepackage{bigints}
\usepackage{vmargin}

% left top textwidth textheight headheight

% headsep footheight footskip

\setmargins{2.0cm}{2.5cm}{16 cm}{22cm}{0.5cm}{0cm}{1cm}{1cm}

\renewcommand{\baselinestretch}{1.3}

\setcounter{MaxMatrixCols}{10}

\begin{document}
\begin{enumerate}

3

\begin{enumerate}[(i)]
\item (i)
The random variable T 18 measures the future lifetime of a life aged 18 from
the Assured Lives 1967–1970 Ultimate mortality experience for male lives.
Draw a sketch of the force of mortality m 18+t , t > 0. Add labels to your sketch
to explain its important features.
[2]
\item (ii)
The force of mortality m 18+t is to be modelled using the Gompertz-Makeham family of curves where:
m 18+t = GM(r,s) = a 1 + a 2 t + a 3 t 2 ... + a r t r-1
+ exp{a r+1 + a r+2 t + a r+3 t 2 ... + a r+s t s-1 }
and a 1 , a 2 , a 3 , ..., a r+s are constants which do not depend on t.
Using your sketch in (i) explain why a GM(2, 2) curve may prove to be a suitable model for m 18+t .
\end{enumerate}
%%%%%%%%%%%%%%%%%%%%%%%%%%%%%%%%%%%%%%%%%%%%%%%%%%%%%%%%%%%%%%%%%%%%%%%%%%%%%%%%%%%%%%%%%%%%%%
4
In a mortality investigation the number of deaths at age x during the period of the
investigation is q x , where x is defined as:
x = [age last birthday at 6th April prior to date of issue of policy]
+ [number of 5th April’s passed since date of issue of policy]
(i) State the rate year implied by this definition and the age range of lives at the
start of this interval.
[3]
(ii) (a)
State the Principle of Correspondence.
(b)
Using this principle describe the central exposed to risk, E x c , that
would correspond to the above classification of deaths.
(iii)
[3]
The estimated force of mortality for those lives classified as aged x is
m ˆ x =
q x
E x c
which estimates the force of mortality, m x + f . Determine the value of f stating
any assumptions that you make.
%%%%%%%%%%%%%%%%%%%%%%%%%%%%%%%%%%%%%%%%%%%%%%%%%%%%%%%%%%%%%%%%%%%%%%%%%%%%%%%%%%%%%%%%%%%%%%%%%%%%%%

3
(i)
m 18+t
mortality steadily
increases with age
from mid
20s(t ¦ 10)
accident hump at
age 18(t=0)
t
7
(ii)
GM(2, 2) = a 1 + a 2 t + exp( a 3 + a 4 t)
= A + Ht + Bc t
relabelling constants
This curve has 3 components:
A
independent of age
Ht varying linearly with age
Bc t varying geometrically or exponentially with age
Page 4Subject 104 (Survival Models) — April 2002 — Examiners’ Report
m 18+t
Bc t
Ht
A
t
which add together to give required shape.
A
gives “general level” to curve
Ht a linearly decreasing function gives “accident hump”
Bc t reflects the rapid increase in mortality with increasing age
4
(i)
Age x =
=
=
=
[age last at 6th April prior start of policy] + [5th April’s passed]
age x last on 6th April prior to death
age (x + 1) next on 6th April prior to death
age on birthday in calendar year (6th April to 5th April) of death
Age changes on 6 April each year, so calendar year rate interval starting on
6 April.
Age range at start of calendar year = x to x + 1
Derivation is not required as the question says “state”.
(ii)
(a) Principle of correspondence: a life alive at time t should be included in
E x c if and only if, were that life to die immediately , the life would be
included in the death data q x .
(b) E x c = total number of years for which lives were exposed to the risk of
dying whilst aged x last birthday on the immediately preceding
6th April.
Or total number of years for which lives were exposed to the risk of
dying whilst aged x + 1 next birthday on the immediately preceding
6th April during the Period of the Investigation.
%%%%%%%%%%%%%%%%%%%%%%%%%%%%%%%%%%%%%%%%%%%%%%%%%%%%%%%%%%%%%%%%%%%%%%%%%%%%%%%%%%%%%%%%%%%
Or
T
ò 0 P x ( t ). dt where P x (t) is a census at time t after the start of the
Period of Investigation of those lives aged x last birthday on 6th April
immediately prior to t; the period of investigation is (0, T).
(iii)
x + f is the mean age of lives half way through the rate interval assuming force
of mortality is constant over the calendar year rate interval.
Assuming birthdays are uniformly distributed over the calendar year, the
average age at the start of the rate interval is x + 1⁄2 and halfway through the
interval is x + 1.
So, x + f = x + 1.
In (ii)(b) a correctly specified census formula received partial credit.
\end{document}
