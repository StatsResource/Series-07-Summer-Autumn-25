
\documentclass[a4paper,12pt]{article}

%%%%%%%%%%%%%%%%%%%%%%%%%%%%%%%%%%%%%%%%%%%%%%%%%%%%%%%%%%%%%%%%%%%%%%%%%%%%%%%%%%%%%%%%%%%%%%%%%%%%%%%%%%%%%%%%%%%%%%%%%%%%%%%%%%%%%%%%%%%%%%%%%%%%%%%%%%%%%%%%%%%%%%%%%%%%%%%%%%%%%%%%%%%%%%%%%%%%%%%%%%%%%%%%%%%%%%%%%%%%%%%%%%%%%%%%%%%%%%%%%%%%%%%%%%%%

\usepackage{eurosym}
\usepackage{vmargin}
\usepackage{amsmath}
\usepackage{graphics}
\usepackage{epsfig}
\usepackage{enumerate}
\usepackage{multicol}
\usepackage{subfigure}
\usepackage{fancyhdr}
\usepackage{listings}
\usepackage{framed}
\usepackage{graphicx}
\usepackage{amsmath}
\usepackage{chngpage}

%\usepackage{bigints}
\usepackage{vmargin}

% left top textwidth textheight headheight

% headsep footheight footskip

\setmargins{2.0cm}{2.5cm}{16 cm}{22cm}{0.5cm}{0cm}{1cm}{1cm}

\renewcommand{\baselinestretch}{1.3}

\setcounter{MaxMatrixCols}{10}

\begin{document}
[Total 5]
\item  A random variable X measures the duration until some event occurs. Write
down a definition of the hazard, h(t) of the event occurring at duration t, in
terms of probabilities relating to the random variable X, and state what the
definition means in words.

\item  You wish to investigate the effect of two factors, Z 1 and Z 2 , on the duration
until the event occurs. Someone suggests that you use a proportional hazards
model. Explain what is meant by the term proportional hazards model.

\item  The Weibull distribution has a survival function given by the formula
S ( t )
where and
exp[ ( t ) ]
are parameters.
Show that, by choosing so that it depends on Z 1 and Z 2 only, the Weibull
distribution can be used as a proportional hazards model to investigate the
effects of Z 1 and Z 2 on the duration.

[Total 8]
104 A2004

%%%%%%%%%%%%%%%%%%%%%%%5
(i)
h(t) = lim (1/dt)
dt
Pr[X
t + dt | X > t]
0
For a small interval of time dt, after duration t, the probability that the event
will occur is dt multiplied by the hazard.
(ii)
In a proportional hazards model, the hazard factorises into two parts,
algebraically,
\[h ( t ) = h 0 ( t ) f ( Z 1 , Z 2 ) .\]
One part (the baseline hazard) is the same for all individuals and depends only
on duration t, and
the other depends on the values of the covariates (in this case Z 1 and Z 2 ),
which vary among individuals.
A feature of this model is that the ratio between the hazards for two
individuals with different values of Z 1 and Z 2 does not depend on duration t.
Page 6 %%%%%%%%%%%%%%%%%%%%%%%%%%%%%%%%%%%%%
(iii)
April 2004
%%%%%%%%%%%%%%%%%%%%%%%%%%%%%%%%%%%%%
We have
S(t) = exp [ ( t) ].
Therefore
log S(t) = ( t) .
Since the hazard h(t) = d[ log S(t)]/dt,
h(t) = d[( t) ]/dt =
t
1 .
Letting be a function of Z 1 and Z 2 , so that
g ( Z 1 , Z 2 ) ,
and substituting into the expression above for h(t)
h(t) = [g(Z 1 , Z 2 )]
t
1 ,
which is of the form
\[h(t) = h 0 (t) . f(Z 1 , Z 2 ),\]
where
h 0 (t) = t
1
and f(Z 1 , Z 2 ) = [g(Z 1 , Z 2 )] .
Therefore the hazard factorises into two parts, one depending only on t and
one depending only on Z 1 and Z 2 , so the model is a proportional hazards
model.
Page 7 %%%%%%%%%%%%%%%%%%%%%%%%%%%%%%%%%%%%%
\end{document}
