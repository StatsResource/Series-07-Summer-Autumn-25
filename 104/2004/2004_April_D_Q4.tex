
\documentclass[a4paper,12pt]{article}

%%%%%%%%%%%%%%%%%%%%%%%%%%%%%%%%%%%%%%%%%%%%%%%%%%%%%%%%%%%%%%%%%%%%%%%%%%%%%%%%%%%%%%%%%%%%%%%%%%%%%%%%%%%%%%%%%%%%%%%%%%%%%%%%%%%%%%%%%%%%%%%%%%%%%%%%%%%%%%%%%%%%%%%%%%%%%%%%%%%%%%%%%%%%%%%%%%%%%%%%%%%%%%%%%%%%%%%%%%%%%%%%%%%%%%%%%%%%%%%%%%%%%%%%%%%%

\usepackage{eurosym}
\usepackage{vmargin}
\usepackage{amsmath}
\usepackage{graphics}
\usepackage{epsfig}
\usepackage{enumerate}
\usepackage{multicol}
\usepackage{subfigure}
\usepackage{fancyhdr}
\usepackage{listings}
\usepackage{framed}
\usepackage{graphicx}
\usepackage{amsmath}
\usepackage{chngpage}

%\usepackage{bigints}
\usepackage{vmargin}

% left top textwidth textheight headheight

% headsep footheight footskip

\setmargins{2.0cm}{2.5cm}{16 cm}{22cm}{0.5cm}{0cm}{1cm}{1cm}

\renewcommand{\baselinestretch}{1.3}

\setcounter{MaxMatrixCols}{10}

\begin{document}
[Total 5]
104 A2004
24
The following data come from the national life table of a developed country.
\item 
Age x Survivors to age x, l x
80
81 22,933
20,010
Estimate
(a)
(b)
0.5 p 80
assuming:
that deaths between exact ages 80 and 81 are uniformly distributed
that the force of mortality is constant between exact ages 80 and 81

5
\item  Explain why the two estimates in \item  are different. State with reasons which
estimate you would prefer to use.


%%%%%%%%%%%%%%%%%%%%%%%%%%%%%%%%%%%%%%%%%%%%%5
4
(i)
(a)
The assumption of a uniform distribution of deaths implies that
t q x
tq x
Since q 80 = 1
0.5 q 80
(for 0 t 1 ).
(20,010/22,933) = 0.12746,
= 0.5 0.12746 = 0.06373
and 0.5 p 80 = 1
0.5 q 80
= 1 0.06373 = 0.93627.
Page 5 %%%%%%%%%%%%%%%%%%%%%%%%%%%%%%%%%%%%%
(b)
%%%%%%%%%%%%%%%%%%%%%%%%%%%%%%%%%%%%%
Let the constant force of mortality be . Then
= log p 80 = log
and
(ii)
April 2004
0.5 p 80
20, 010
= 0.13634
22,933
= exp( 0.5 ) = exp( 0.5 0.13634) = 0.93410 .
The two estimates in (i) are different because they make different assumptions
about the distribution of the force of mortality within the year of age from 80
to 81 years.
The uniform distribution of deaths (UDD) assumption implies an increasing
force of mortality within this age range.
Since it is likely that the true force of mortality is increasing with age in this
age range, the UDD estimate is to be preferred.
