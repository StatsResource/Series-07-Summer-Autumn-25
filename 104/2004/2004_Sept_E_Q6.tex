
\documentclass[a4paper,12pt]{article}

%%%%%%%%%%%%%%%%%%%%%%%%%%%%%%%%%%%%%%%%%%%%%%%%%%%%%%%%%%%%%%%%%%%%%%%%%%%%%%%%%%%%%%%%%%%%%%%%%%%%%%%%%%%%%%%%%%%%%%%%%%%%%%%%%%%%%%%%%%%%%%%%%%%%%%%%%%%%%%%%%%%%%%%%%%%%%%%%%%%%%%%%%%%%%%%%%%%%%%%%%%%%%%%%%%%%%%%%%%%%%%%%%%%%%%%%%%%%%%%%%%%%%%%%%%%%

\usepackage{eurosym}
\usepackage{vmargin}
\usepackage{amsmath}
\usepackage{graphics}
\usepackage{epsfig}
\usepackage{enumerate}
\usepackage{multicol}
\usepackage{subfigure}
\usepackage{fancyhdr}
\usepackage{listings}
\usepackage{framed}
\usepackage{graphicx}
\usepackage{amsmath}
\usepackage{chngpage}

%\usepackage{bigints}
\usepackage{vmargin}

% left top textwidth textheight headheight

% headsep footheight footskip

\setmargins{2.0cm}{2.5cm}{16 cm}{22cm}{0.5cm}{0cm}{1cm}{1cm}

\renewcommand{\baselinestretch}{1.3}

\setcounter{MaxMatrixCols}{10}

\begin{document}

A life insurance company uses the following 3-state Markov model in the pricing of
its long term sickness policies. The forces of transition are assumed to be constant.
1: able
2: sick
3: dead
For a group of policyholders, over a one year period the following data were
recorded:
Transition from Number
1 to 2
1 to 3
2 to 1
2 to 3 15
6
5
1
The total waiting times recorded were:
State 1
State 2
\item 
Write down the likelihood function for this model and show that this is
maximized when = 0.024.
\item 
104 S2004
625 years
35 years
4
(a) State the asymptotic distribution of , the maximum likelihood
estimator of .
(b) Calculate an estimate of the standard deviation of .




%%%%%%%%%%%%%%%%%%%%%%%%%%%%%%%%%%%%%%%%%%%%%%%%%%%%%
6
\item 
L
625
e
September 2004
%%%%%%%%%%%%%%%%%%%%%%%%%%%%%%%%%%%%%
35 6 15 5 1
e
Taking logs gives:
log L
625
625
35 6 ln
15ln
15ln
5ln
ln
K
Differentiate with respect to :
ln L
625
15
Set to zero to find maximum:
0
15
625
15
625
0.024
Check that this is a maximum:
2
ln L
15
2
\item 
(a)
~ N
2
,
E V
0
, where V is the waiting time in state 1
Note that the question asks for the asymptotic distribution of the maximum
likelihood estimator. Inserting numbers or gives the formula we use, in
practice, to estimate this distribution, rather than the distribution itself.
(b)
ALTERNATIVE 1
Var
Estimate
Var
E V
and E[V] by the observed values of
15
625 2
So, estimate of sd of
ALTERNATIVE 2
Page 10
15
625
0.00620
and v respectively. %%%%%%%%%%%%%%%%%%%%%%%%%%%%%%%%%%%%%
September 2004
2
Var
Estimate
1
2
2
15
by the observed value of .
So, estimate of sd of
7
L
%%%%%%%%%%%%%%%%%%%%%%%%%%%%%%%%%%%%%
15
15
625
0.00620
\end{document}
