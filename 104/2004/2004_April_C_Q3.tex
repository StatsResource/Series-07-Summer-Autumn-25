
\documentclass[a4paper,12pt]{article}

%%%%%%%%%%%%%%%%%%%%%%%%%%%%%%%%%%%%%%%%%%%%%%%%%%%%%%%%%%%%%%%%%%%%%%%%%%%%%%%%%%%%%%%%%%%%%%%%%%%%%%%%%%%%%%%%%%%%%%%%%%%%%%%%%%%%%%%%%%%%%%%%%%%%%%%%%%%%%%%%%%%%%%%%%%%%%%%%%%%%%%%%%%%%%%%%%%%%%%%%%%%%%%%%%%%%%%%%%%%%%%%%%%%%%%%%%%%%%%%%%%%%%%%%%%%%

\usepackage{eurosym}
\usepackage{vmargin}
\usepackage{amsmath}
\usepackage{graphics}
\usepackage{epsfig}
\usepackage{enumerate}
\usepackage{multicol}
\usepackage{subfigure}
\usepackage{fancyhdr}
\usepackage{listings}
\usepackage{framed}
\usepackage{graphicx}
\usepackage{amsmath}
\usepackage{chngpage}

%\usepackage{bigints}
\usepackage{vmargin}

% left top textwidth textheight headheight

% headsep footheight footskip

\setmargins{2.0cm}{2.5cm}{16 cm}{22cm}{0.5cm}{0cm}{1cm}{1cm}

\renewcommand{\baselinestretch}{1.3}

\setcounter{MaxMatrixCols}{10}

\begin{document}
3
A particularly morbid actuarial student is considering the mortality of the population
of the village in which she lives. The village graveyard contains tombstones which
bear inscriptions of the form A N Other beloved mother of . , Born 1903,
Died 1976 .
\begin{itemize}
\item  If the student were to use the inscriptions as a source of death data, state the
exact age to which the calculated mortality rate would apply assuming a
Binomial model was used.

\item  Comment on the difficulties the student is likely to face in attempting to
calculate mortality rates in this way.
\end{itemize}
%%%%%%%%%%%%%%%%%%%%%%%%%%%%%%%%%%%%%%%%
3

(i)
April 2004
%%%%%%%%%%%%%%%%%%%%%%%%%%%%%%%%%%%%%
\begin{itemize}
The death data will be classified as age = calendar year of death calendar
year of birth. This is the same as age at the birthday in the calendar year of
death.
This is a calendar year rate interval, and the range of exact ages for a life
labelled x at the start of the rate interval is ( x 1, x).
So the average age at the start of the rate interval is x 0.5 and
because we are using a binomial model, this is the age to which the calculated
rate would apply.
\item (ii)
Some of the difficulties will be:
low volume of death data
deaths will cover maybe 200+ years, during which time mortality
patterns will have changed greatly
there could be selection issues
what about people too poor to have a
tombstone?
it will be almost impossible to estimate a corresponding exposed to
risk
due to lack of data, migration etc.
some tombstones could be more weathered than others (and so
illegible)
people may have been buried in the graveyard who did not live in the
village. It may not be possible to identify this and eliminate them from
the analysis
For the older tombstones in particular, the data might be incorrect (e.g.
the correct date of birth might not have been known)
Credit was given for other valid comments concerning the particular
circumstances
\end{itemize}
\end{document}
