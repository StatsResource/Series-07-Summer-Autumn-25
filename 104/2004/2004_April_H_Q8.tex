
\documentclass[a4paper,12pt]{article}

%%%%%%%%%%%%%%%%%%%%%%%%%%%%%%%%%%%%%%%%%%%%%%%%%%%%%%%%%%%%%%%%%%%%%%%%%%%%%%%%%%%%%%%%%%%%%%%%%%%%%%%%%%%%%%%%%%%%%%%%%%%%%%%%%%%%%%%%%%%%%%%%%%%%%%%%%%%%%%%%%%%%%%%%%%%%%%%%%%%%%%%%%%%%%%%%%%%%%%%%%%%%%%%%%%%%%%%%%%%%%%%%%%%%%%%%%%%%%%%%%%%%%%%%%%%%

\usepackage{eurosym}
\usepackage{vmargin}
\usepackage{amsmath}
\usepackage{graphics}
\usepackage{epsfig}
\usepackage{enumerate}
\usepackage{multicol}
\usepackage{subfigure}
\usepackage{fancyhdr}
\usepackage{listings}
\usepackage{framed}
\usepackage{graphicx}
\usepackage{amsmath}
\usepackage{chngpage}

%\usepackage{bigints}
\usepackage{vmargin}

% left top textwidth textheight headheight

% headsep footheight footskip

\setmargins{2.0cm}{2.5cm}{16 cm}{22cm}{0.5cm}{0cm}{1cm}{1cm}

\renewcommand{\baselinestretch}{1.3}

\setcounter{MaxMatrixCols}{10}

\begin{document}
%%---  Question 8
The main activity of a certain charity is training and providing guide dogs for blind
people. The charity wishes to estimate the lifetime of the dogs it provides, and has
therefore attempted to keep track of 100 dogs that were all born in 1990. Of these 100
dogs, 81 remained in service on reaching age 10 years. Of the remaining dogs, some
died and some left active guide service for other reasons.
The ages of the deaths and other exits were recorded as follows:
Age
(years)
0.25
0.5
1
2.25
3.5
4
4.5
6
7.5
9
\item 
Number
of deaths
Number
of other exits
2
1
3
3
2
2
1
1
2
2
Calculate the Nelson-Aalen estimate of the integrated hazard for these guide
dogs.

Guide dogs undergo an initial period of one year s training, which costs \$3,000
payable continuously over the year. They then work for up to 9 years. The charity
undertakes to provide food, medical expenses, equipment and so on throughout the
ten years of the dog s working life. This is estimated to cost \$2,000 per year, payable
continuously over the year. If the dog reaches age 10, the charity will find it a new
home for its retirement, the expense of this resettlement being \$300.
\item 
Using the integrated hazard calculated in \item , calculate the capital cost to the
charity of supporting a single guide dog.
[10]
The charity has two principal sources of finance. The first is a new scheme whereby
members of the public commit to make regular payments for ten years (but ceasing on
death) to the charity. Payments are to be made annually in advance. The total annual
income from this scheme at the outset is \$125,000. The average age of the
contributors is now exactly 50.
The second source of income is a commitment to leave the charity a legacy when the
benefactor dies. The charity has commitments to legacies worth a total of \$900,000
from benefactors, the average age of whom is now exactly 70. Due to delays in
settling estates, the charity generally receives these legacies one year after the death of
the benefactor.
104 A2004
6\item 
Calculate the expected present value of the current commitments to the
charity. Hence estimate the number of guide dogs that the charity can
undertake to provide.

Basis: Mortality of humans: PFA92C20
Interest:
4% per annum throughout
[Total 23]
104 A2004
7

%%%%%%%%%%%%%%%%%%%%%%%%%%%%%%%%%%%%%%%%%%%%%%%%%%%%%%%%%%%%%%%%%%%%%%%8
(i)
The relevant calculations are:
Time
t j Number of
deaths
d j Number of
other exits
c j Lives
exposed
n j
0
0.5
4
6 0
1
2
1 2
8
1
4 100
98
89
86
j
0
1/98
2/89
1/86
j
0
0.0102
0.03268
0.0443
Thus the Nelson-Aalen estimate of the integrated hazard is:
t
0 t < 0.5
0.5 t < 4
4 t < 6
6 t
(ii)
Page 12
t
0
0.0102
0.03268
0.0443
The values given for the integrated hazard mean that the survival function is
given by t p x = exp{ t }
t p 0 1 0 t
t p 0 0.9899 0.5 t
t p 0 0.9678 4 t
t p 0 0.9567 6 t
0.5
4
6 %%%%%%%%%%%%%%%%%%%%%%%%%%%%%%%%%%%%%
April 2004
%%%%%%%%%%%%%%%%%%%%%%%%%%%%%%%%%%%%%
And so the cost of training is given by
V
\$3, 000
\$3, 000
1
0 t
a 0.5
p 0 exp(
t ) dt
0.9899 v 0.5 a 0.5
\$3, 000 (0.49513 0.9899 0.98058 0.49513)
\$2,927
Similarly, the cost of the food etc is calculated as:
F \$2, 000 ( a 0.5 v 0.5 0.9899 a 3.5 0.9678 v 4 a 2
0.9567 v 6 a 4 )
\$2, 000 (0.49513 0.9899 0.98058 3.27040 0.9678 0.85480
1.92357 0.9567 0.79031 3.70202)
\$16,120
and finally, the value of the retirement benefit is
\$300 10 p 0 v 10 \$300 0.9567 0.67556 \$194
so the capital cost of providing a guide dog is
\$194 + \$16,120 + \$2,927 = \$19,241.
(iii)
The value can be calculated in 2 parts. Firstly, annual commitments are
worth:
\$125, 000 a 50:10
Now a 50:10
a 50
10
p 50 v 10 a 60 19.539
9848.431
1.04
9952.697
so the annual commitments are worth \$125,000
10
16.652 8.407
8.407 = \$1,050,875.
The values of the legacies are
\$900, 000 v A 70 \$900, 000 v 1.04 1⁄2 A 70
and we can calculate the assurances via premium conversion:
A 70 1 da 70 1 0.04 /1.04 (12.934) 0.5025
So the value of the legacies is \$900,000
1.04
1⁄2
0.5025 = \$443,468.
The total value of the benefactions is
\$1,050,875 + \$443,468 = \$1,494,343
and the number of guide dogs that can be provided is
1,494,343 / 19,241 = 77.

\end{document}
