
\documentclass[a4paper,12pt]{article}

%%%%%%%%%%%%%%%%%%%%%%%%%%%%%%%%%%%%%%%%%%%%%%%%%%%%%%%%%%%%%%%%%%%%%%%%%%%%%%%%%%%%%%%%%%%%%%%%%%%%%%%%%%%%%%%%%%%%%%%%%%%%%%%%%%%%%%%%%%%%%%%%%%%%%%%%%%%%%%%%%%%%%%%%%%%%%%%%%%%%%%%%%%%%%%%%%%%%%%%%%%%%%%%%%%%%%%%%%%%%%%%%%%%%%%%%%%%%%%%%%%%%%%%%%%%%

\usepackage{eurosym}
\usepackage{vmargin}
\usepackage{amsmath}
\usepackage{graphics}
\usepackage{epsfig}
\usepackage{enumerate}
\usepackage{multicol}
\usepackage{subfigure}
\usepackage{fancyhdr}
\usepackage{listings}
\usepackage{framed}
\usepackage{graphicx}
\usepackage{amsmath}
\usepackage{chngpage}

%\usepackage{bigints}
\usepackage{vmargin}

% left top textwidth textheight headheight

% headsep footheight footskip

\setmargins{2.0cm}{2.5cm}{16 cm}{22cm}{0.5cm}{0cm}{1cm}{1cm}

\renewcommand{\baselinestretch}{1.3}

\setcounter{MaxMatrixCols}{10}

\begin{document}
%%- Question 7
A Company is modelling its workforce using the multiple state model shown below.
Working
Left Company
voluntarily
Left Company
involuntarily
Dead
The Company is interested in how many of its employees leave employment
voluntarily each year and so wishes to estimate the transition intensity from the
working state to the left company voluntarily state. Consider employees aged
between ages 30 and 31 exact and suppose that the ith individual spends time W i
employed by the Company between those ages.
\item  Derive the maximum likelihood estimate
of .
of
and write down the variance

\item  Suppose the Company has available data on the number of employees who left
service voluntarily in 2002, classified by age last birthday on the previous
anniversary of the date they started employment.
The Company also has census data P(x, t) for t = 0 and t = 1 where P(x, t)
represents the number of employees aged x nearest birthday on 1 January in
year 2002 + t.
Assuming that the dates on which employment began are distributed
uniformly across both calendar years and life years:
\item 
104 A2004
(a) Derive a formula to approximate the exposed to risk in the year 2002
for lives classified as age 30.
(b) If you use the exposed to risk in (a) to estimate , state the exact age to
which your formula applies.

You subsequently discover that, actually, all workers begin employment with
the Company on their 16 th birthdays. Write down the revised estimate of the
exposed to risk based on the data available.

[Total 14]

%%%%%%%%%%%%%%%%%%%%%%%%%%%%%%%%%%%%%%%%%%%%%%%%%%%%%%%%%%%%%%%%%%%%%%%%%%%%%%%%%%%%%%%%%%%7
(i)
April 2004
%%%%%%%%%%%%%%%%%%%%%%%%%%%%%%%%%%%%%
\begin{itemize}
\item Suppose there are N voluntary leavers, M involuntary leavers and D deaths.
Then if we let W
W i be the total waiting time in the working state, the
i
likelihood function can be written as:
L = K exp( W (
))
N
D
M
where K is some constant. The log-likelihood function is therefore
l = log L = K
N log
W
where K is another constant, independent of . Differentiating with respect
to gives
dl N
=
W
d
and equating this expression to 0 gives
N
W = 0 i.e.
=
N
W
\item To check we have a maximum, note that
dl 2
d
2
=
N
2
0 .
\item The formula for the variance of the estimator is
Var ( ) =
E ( W )
.
Page 9 %%%%%%%%%%%%%%%%%%%%%%%%%%%%%%%%%%%%%
(ii)
(a)
April 2004
%%%%%%%%%%%%%%%%%%%%%%%%%%%%%%%%%%%%%
\item The age classification of the exit data is age last birthday on
employment anniversary prior to exit . By the principle of
correspondence, we must estimate the exposure on the same basis. If
we knew P (30, t) which is defined as the population aged 30 last
birthday at employment anniversary prior to t then the exposure would
be:
1
c
E 30
= P (30, t ) dt
0
and assuming the population varies approximately linearly over the
year, we could approximate this by
c
E 30
= 0.5 ( P (30, 0) P (30,1)) .
The range of exact ages that could apply to a life aged 30 last birthday
on employment anniversary prior to leaving is (30, 32).
The uniformity assumptions mean that the time elapsing between the
previous employment anniversary and the date of death is uniformly
distributed, as is the time between the previous employment
anniversary and the birthday before that. Therefore the time between
the birthday before the previous employment anniversary and the date
of death is distributed as the sum of two independent identical uniform
distributions and
P (30, t ) =
1
( P (30, t ) 6 P (31, t ) P (32, t ))
8
A diagram may be used to demonstrate this:
Age 32
nearest
1/8 th here
Age 31
nearest
Age 30
nearest
0.5
1.0
employment anniversary to census date
Page 10 %%%%%%%%%%%%%%%%%%%%%%%%%%%%%%%%%%%%%
April 2004
%%%%%%%%%%%%%%%%%%%%%%%%%%%%%%%%%%%%%
and substituting into the approximation above, we have
c
E 30
1
P (30, 0) 6 P (31, 0) P (32, 0)
1 8
2 1
P (30,1) 6 P (31,1) P (32,1)
8
ie,
c
E 30
1
P (30, 0) 6 P (31, 0) P (32, 0) P (30,1) 6 P (31,1) P (32,1)
16
\item Credit was also given for the following approximation (although it
should be noted that the majority of the marks for this part of the
question were available for the derivation rather than the final
answer):
Assuming employment anniversaries are uniformly distributed across
calendar years, then the average anniversary prior to 1 January 2002
will be 1 July 2001. A life aged 30 last birthday on 1 July 2001 is aged
between 30 and 31 on that date and so will be aged between 30.5 and
31.5 on 1 January 2002. So, we can approximate
$P (30, t )$
on average, the number of employees on 1 January 2002
aged between 30.5 and 31.5
the number of employees on 1 January 2002 aged 31
nearest
P 31, 0)
and substituting into the approximation above, we have
c
E 30
(b)
1
2
P (31, 0) P (31,1)
If we denote the number of voluntary exits in 2002 by employees aged
30 last birthday on last employment anniversary by N 30 then the
estimator is
30
c
= N 30 / E 30
.
This is a central rate, applying to ages at the middle of the rate interval.
Since the age range at the start of the rate interval (crossing an
employment anniversary) is (30, 31) the average age at the start of the
rate interval is 30.5 and the average age in the middle is 31.
Page 11 %%%%%%%%%%%%%%%%%%%%%%%%%%%%%%%%%%%%%
(iii)
April 2004
%%%%%%%%%%%%%%%%%%%%%%%%%%%%%%%%%%%%%
In this case, the age definition is the same as age last birthday so that the
census formula would have to be adapted (indeed simplified) accordingly.
If we define P ' ' ( 30 , t ) as the population aged 30 last birthday at time t then the
exposure would be:
P ' ' ( 30 , t ) 0 . 5 P 30 , t P 31 , t
And, using (as before)
E 30 c 0 . 5 P ' ' 30 , 0 P ' ' 30 , 1
The relevant formula is then
c
E 30
0.25( P (30, 0) P (31, 0) P (30,1) P (31,1))
with the calculated rate applying at age 30.5.
\end{itemize}
\end{document}
