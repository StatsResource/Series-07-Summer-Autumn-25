\documentclass[a4paper,1pt]{article}

%%%%%%%%%%%%%%%%%%%%%%%%%%%%%%%%%%%%%%%%%%%%%%%%%%%%%%%%%%%%%%%%%%%%%%%%%%%%%%%%%%%%%%%%%%%%%%%%%%%%%%%%%%%%%%%%%%%%%%%%%%%%%%%%%%%%%%%%%%%%%%%%%%%%%%%%%%%%%%%%%%%%%%%%%%%%%%%%%%%%%%%%%%%%%%%%%%%%%%%%%%%%%%%%%%%%%%%%%%%%%%%%%%%%%%%%%%%%%%%%%%%%%%%%%%%%

\usepackage{eurosym}
\usepackage{vmargin}
\usepackage{amsmath}
\usepackage{graphics}
\usepackage{epsfig}
\usepackage{enumerate}
\usepackage{multicol}
\usepackage{subfigure}
\usepackage{fancyhdr}
\usepackage{listings}
\usepackage{framed}
\usepackage{graphicx}
\usepackage{amsmath}
\usepackage{chngpage}

%\usepackage{bigints}
\usepackage{vmargin}

% left top textwidth textheight headheight

% headsep footheight footskip

\setmargins{.0cm}{.5cm}{16 cm}{cm}{0.5cm}{0cm}{1cm}{1cm}

\renewcommand{\baselinestretch}{1.}

\setcounter{MaxMatrixCols}{10}

\begin{document}

2 A total of N independent lives are observed during a finite period of observation.
Between the ages of x and x + 1, x + a i is the age at which observation of the ith
life starts, and x + t i is the age at which observation of the ith life ceases. For
each life, an indicator variable D i indicates whether life i is observed to die.
ì 1 if life i dies at x + t i
D i = í
î 0 otherwise
\item 
3
i = 1, 2, ... N
If the force of mortality at age x + t, \mu x + t = \mu (0 \leq  t < 1), derive the
maximum likelihood estimate of \mu.

\item  State the asymptotic sampling distribution of the maximum likelihood
estimator, \hat{\mu} .

[Total 8]

%%%%%%%%%%%%%%%%%%%%%%%%%%%%%%%%%%%%%%%%%%%%%%%%%%%%%%%%%%%%%%%%%%%%%%%%%55
2 (i)
If the ith life dies the contribution to the likelihood is
t i - a i
p x + a i m x + t i
= e -m ( t i - a i ) m with assumption of constant force, m
If the ith life survives to x + t i the contribution to the likelihood is
t i - a i
p x + a i = e -m ( t i - a i ) with assumption of constant force, m.
So the overall contribution for the ith life can be written
e -m ( t i - a i ) m d i
where d i = 1 for a death
d i = 0 for a survivor
So for N independent lives we have:
L
=
i = N
Õ e
-m ( t i - a i )
m d i
i = 1
i = N
= e
-m å ( t i - a i )
i = 1
Then \log_{e}L = - m
i = N
å ( t
i
i = N
å d i
m i = 1
- a i ) +
i = 1
i = N
å d
i
\log_{e}m
i = 1
i = N
i = N
¶ \log_{e}L
So
= -
( t i - a i ) +
¶m
i = 1
å
å d
i
i = 1
m
And the maximum likelihood estimator is
i = N
å d
i
m̂ =
i = N
i = 1
å ( t
i
i = 1
Page 2
- a i ) %%%%%%%%%%%%%%%%%%%%%%%%%%%%%%%%%%%%% — 
i = N
Now
(ii)
å
d i
¶ \log_{e}L
i = 1
= -
< 0 so solution is a maximum
m 2
¶m 2
2
Asymptotic Sampling Variance is given by
-
1
é ¶ 2 \log_{e}L ù
E ê
ú
2
ë ¶m
û
=
m 2
E [ D ]
which can be estimated by
m 2
= i = N
m ( t i - a i )
å
i = 1
=
m
i = N
å ( t
i
- a i )
i = 1
æ
So m̂ ~ N ç m ,
ç
è
