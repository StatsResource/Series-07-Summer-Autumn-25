\documentclass[a4paper,1pt]{article}

%%%%%%%%%%%%%%%%%%%%%%%%%%%%%%%%%%%%%%%%%%%%%%%%%%%%%%%%%%%%%%%%%%%%%%%%%%%%%%%%%%%%%%%%%%%%%%%%%%%%%%%%%%%%%%%%%%%%%%%%%%%%%%%%%%%%%%%%%%%%%%%%%%%%%%%%%%%%%%%%%%%%%%%%%%%%%%%%%%%%%%%%%%%%%%%%%%%%%%%%%%%%%%%%%%%%%%%%%%%%%%%%%%%%%%%%%%%%%%%%%%%%%%%%%%%%

\usepackage{eurosym}
\usepackage{vmargin}
\usepackage{amsmath}
\usepackage{graphics}
\usepackage{epsfig}
\usepackage{enumerate}
\usepackage{multicol}
\usepackage{subfigure}
\usepackage{fancyhdr}
\usepackage{listings}
\usepackage{framed}
\usepackage{graphicx}
\usepackage{amsmath}
\usepackage{chngpage}

%\usepackage{bigints}
\usepackage{vmargin}

% left top textwidth textheight headheight

% headsep footheight footskip

\setmargins{.0cm}{.5cm}{16 cm}{cm}{0.5cm}{0cm}{1cm}{1cm}

\renewcommand{\baselinestretch}{1.}

\setcounter{MaxMatrixCols}{10}

\begin{document}

%%---  Question 7
On 1 January 1995 an office issued a number of level annual premium policies to
independent lives, each of whom was aged exactly 40. The sum assured for each
policy was \$10,000. There were two types:
• whole-life assurances, with the sum assured being payable at the end of
the year of death
• pure endowments with a term of 20 years, under which the only benefit
was the payment of the sum assured on survival to the maturity date
On 1 January 2000 there were 600 of the whole-life assurances and 400 of the
pure endowments in force. During the calendar year 2000, 2 whole-life
policyholders and 1 pure endowment policyholder died. There were no exits
other than by death.
The premium and policy value bases used were A1967–70 ultimate mortality, 4%
interest p.a. and expenses can be ignored.
\item  Show that the annual premiums payable were approximately \$144.65 for
a whole-life policy and \$296.96 for a pure endowment policy.

\item  Find the total policy value in respect of each class of policy, at the
beginning and at the end of the calendar year 2000.

\item  Find the mortality profit or loss for the calendar year 2000.

\item  If on 1 January 2000 the office held funds equal to the total of the policy
values of all in force policies, what rate of interest would the office have
had to earn on its funds during the calendar year 2000 for there to be
neither a profit nor a loss on this business in that year?

[Total 13]
%%%%%%%%%%%%%%%%%%%%%%%%%%%%%%%%%%%
7
(i)
1 p̂ 1 = exp{-0.010353} = 0.989700
1 p̂ 2 = exp{-0.004805} = 0.995207
p̂ 0 = 0.951654  ́ 0.989700  ́ 0.995207
= 0.937338
So 3
and 3 q̂ 0
= 0.062662
Whole-life policies: 10,000(1 / a && 40 - d ) = 144.653166
Pure endowments:
10,000D 60
10,000  ́ 2,855.5942
=
= 296.960725
132,001.93 - 35,841.261
N 40 - N 60
(ii)
Policy values, per policy:
Whole-life, time 5: 10,000 (1 - a && 45 / a && 40 ) = 719.805229
6: 10,000 (1 - a && 46 / a && 40 ) = 874.880915
Pure endowment, time 5: 10,000(D 60 / D 45 ) - 296.96a && 45:15
= 10,000  ́ 0.501934 - 296.96  ́ 11.235 = 1,682.998848
time 6: 10,000(D 60 / D 46 ) - 296.96a 46:14
= 10,000  ́ (0.523392) -296.96  ́ 10.672 = 2,064.76288
Page 9 %%%%%%%%%%%%%%%%%%%%%%%%%%%%%%%%%%%%% — 
Hence totals:
Whole-life
time 5
(beginning of
CY 2000)
time 6
(end of CY 2000)
(iii)
PE All
431,883 673,200 1,105,083
523,179 823,840 1,347,019
Alternatively 6 V can be calculated from 5 V.
Policy values, time 5
+ premiums
+4% interest
- death claims
- policy values reqd
at time 6
431,883
86,790
518,673
20,747
20,000 673,200
118,784
791,984
31,679
0 1,105,083
205,574
1,310,657
52,426
20,000
523,179
-3,759 823,840
-177 1,347,019
-3,936
Mortality loss of approximately \$3,936
Alternatively Expected and Actual Death Strain can be calculated to
determine mortality profit. Numercial answers will not usually agree
exactly because of rounding errors.
(iv)
(1,105,083 + 205,574) (i + i) = 1,347,019 + 2,000
1 + i = 1.043002
