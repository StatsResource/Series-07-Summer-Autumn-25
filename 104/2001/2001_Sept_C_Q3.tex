\documentclass[a4paper,1pt]{article}

%%%%%%%%%%%%%%%%%%%%%%%%%%%%%%%%%%%%%%%%%%%%%%%%%%%%%%%%%%%%%%%%%%%%%%%%%%%%%%%%%%%%%%%%%%%%%%%%%%%%%%%%%%%%%%%%%%%%%%%%%%%%%%%%%%%%%%%%%%%%%%%%%%%%%%%%%%%%%%%%%%%%%%%%%%%%%%%%%%%%%%%%%%%%%%%%%%%%%%%%%%%%%%%%%%%%%%%%%%%%%%%%%%%%%%%%%%%%%%%%%%%%%%%%%%%%

\usepackage{eurosym}
\usepackage{vmargin}
\usepackage{amsmath}
\usepackage{graphics}
\usepackage{epsfig}
\usepackage{enumerate}
\usepackage{multicol}
\usepackage{subfigure}
\usepackage{fancyhdr}
\usepackage{listings}
\usepackage{framed}
\usepackage{graphicx}
\usepackage{amsmath}
\usepackage{chngpage}

%\usepackage{bigints}
\usepackage{vmargin}

% left top textwidth textheight headheight

% headsep footheight footskip

\setmargins{.0cm}{.5cm}{16 cm}{cm}{0.5cm}{0cm}{1cm}{1cm}

\renewcommand{\baselinestretch}{1.}

\setcounter{MaxMatrixCols}{10}

\begin{document}
\item  If T x and K_x are random variables measuring the complete and curtate
length of life for a life aged x, define a x : n and a  x : n as expected values of
functions of T x and K_x respectively.
\item 
Using the definition in \item  derive the following result
a x : n =
ò
n
0
v t t p x dt ,
and state the corresponding result for a  x : n .
\item 

Hence or otherwise show that a x : n can be approximated using the formula
(
a x : n ≈ a  x : n − 1⁄2 1 − v n
\item 

n
)
p x .
Hence or otherwise find a simple approximation for a x .
104 S2001—2



%%%%%%%%%%%%%%%%%%%%%%%%%%%%%3
(i)
å
ö
m
÷
( t i - a i ) ÷ ø
a x : n = E ( a S ) where S = min(T x , n)
a && x : n = E ( a && U ) where U = min(K_x + 1, n)
n
(ii)
ò
E ( a S ) =
¥
ò
a t f x ( t ) dt + a n f x ( t ) dt
n
0
where f x (t) = t p x m x + t
The second term = a n
n p x
n t
òò
n
v s ds f x ( t ) dt =
0 0
n
=
ò
0
, and the first term is:
n
ò ò
v s f x ( t ) dtds
0
s
n
ò
v s ( s p x - n p x ) ds = v s s p x ds - a n n p x
0
Page 3 %%%%%%%%%%%%%%%%%%%%%%%%%%%%%%%%%%%%% — 
Hence result.
n - 1
å v
Also a && x : n =
(iii)
a x : n =
n - 1
= 1⁄2 +
k
p x
k = 0
å ò
k = 0
k
k + 1
k
n - 1
å 1⁄2( v
v s s p x ds »
k
k
k = 0
n - 1
å v
k
k
p x + v k + 1 k + 1 p x )
p x + 1⁄2 v n n p x
k = 1
= - 1⁄2 +
n - 1
å v
k
k
p x + 1⁄2 v n n p x
k = 0
= a && x : n - 1⁄2(1 - v n n p x )
4
(iv) a x = lim a x : n ¦ lim a && x : n - 1⁄2(1 - 0) = a && x - 1⁄2


%%%%%%%%%%%%%%%%%%%%%%%%%%%%%%%%%%%%%%%%%%%%%%%%%%%%%%%%%%%%%%%%%%5
\end{document}
