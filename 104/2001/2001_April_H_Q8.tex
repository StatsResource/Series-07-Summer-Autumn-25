\documentclass[a4paper,1pt]{article}

%%%%%%%%%%%%%%%%%%%%%%%%%%%%%%%%%%%%%%%%%%%%%%%%%%%%%%%%%%%%%%%%%%%%%%%%%%%%%%%%%%%%%%%%%%%%%%%%%%%%%%%%%%%%%%%%%%%%%%%%%%%%%%%%%%%%%%%%%%%%%%%%%%%%%%%%%%%%%%%%%%%%%%%%%%%%%%%%%%%%%%%%%%%%%%%%%%%%%%%%%%%%%%%%%%%%%%%%%%%%%%%%%%%%%%%%%%%%%%%%%%%%%%%%%%%%

\usepackage{eurosym}
\usepackage{vmargin}
\usepackage{amsmath}
\usepackage{graphics}
\usepackage{epsfig}
\usepackage{enumerate}
\usepackage{multicol}
\usepackage{subfigure}
\usepackage{fancyhdr}
\usepackage{listings}
\usepackage{framed}
\usepackage{graphicx}
\usepackage{amsmath}
\usepackage{chngpage}

%\usepackage{bigints}
\usepackage{vmargin}

% left top textwidth textheight headheight

% headsep footheight footskip

\setmargins{.0cm}{.5cm}{16 cm}{cm}{0.5cm}{0cm}{1cm}{1cm}

\renewcommand{\baselinestretch}{1.}

\setcounter{MaxMatrixCols}{10}

\begin{document}
[10]
[Total 11]8
\item  If K_x is a random variable representing the curtate future lifetime of a life
now aged x, and W is a random variable which represents the present
value of the benefits payable from a temporary immediate annuity with a
term of 20 years sold to a life aged 40, write down an expression for W in

terms of K_x .
\item  Using as a basis the A67–70 Ultimate Mortality Table and an effective
rate of interest of 5% p.a. calculate
\item 
(a) P[W < E[W]]
(b) Using suitable diagrams or otherwise, explain why the solution to
(a) is not 0.5.

Show that the variance of W is
(
(
′
441 A 40:21
− A 40:21
)
2
)
State the rates of interest that should be used to evaluate A ′ and A
respectively.

[Total 15]
%% ---- 104 A2001—5
%%%%%%%%%%%%%%%%%%%%%%%%%%%%%%%%%%%%%%%%%%%%%%%%%%%%%%%%%%%%%%%%%%%%%%%%%%%%%%
Page 9 %%%%%%%%%%%%%%%%%%%%%%%%%%%%%%%%%%%%% — April 2001 — Examiners’ Report
8
(i)
W = a min( K
or
W = a K
40 ,20)
40
= a 20
(ii)
(a)
K 40 ≥ 0
K 40 = 0, 1, 2, ...20
K 40 > 20
E[W] = a 40:20 A67/70 ultimate 5%
= a && 40:20 +
l 60 v 20
− 1
l 40
= 12.760 +
30,039.787 (1.05) − 20
− 1
33,542.311
= 12.0975
5% tables give: a 19 = 12.0853 a 20 = 12.4622
So W < 12.0975 if life dies in (40, 60)
= 1 −
l 60
30,039.787
= 1 −
33,542.311
l 40
= 1 − 0.89558 = 0.10442
(b)
The probability function of the discrete random variable W is very
skew so P[W < E[W]] = the very small “left hand tail area”.
f W (w)
E[W]
a 1 a 2
Page 10
a 18 a 19 a 20
W %%%%%%%%%%%%%%%%%%%%%%%%%%%%%%%%%%%%% — April 2001 — Examiners’ Report
(iii)
ï a  K + 1 − 1 K 40 = 0,1, 2, ... 20
Now W = í 40
K 40 > 20
ï
î a  21 − 1
ì 1 − v K 40 + 1
− 1 K 40 = 0,1, 2, ...19, 20
ï
ï
d
= í
21
ï 1 − v − 1
K 40 > 20
ï
î d
ì 1
− 1 −
ï
ï d
= í
ï 1 − 1 −
ï
î d
1 K 40 + 1
v
d
1 21
v
d
K 40 > 20
1
Var(Y) where
d 2
So Var(W) =
ï v K 40 + 1
Y = í 21
ï
î v
Now E[Y] =
K 40 = 0,1, 2, ...19, 20
20
å
K 40 = 0,1, 2, ...,19, 20
K 40 > 20
k =∞
v k + 1 k  q 40 +
k = 0
å v
21
k  q 40
k = 21
= A 40:21
Where A is determined at 5% p.a.
E[Y 2 ] =
20
å ( v
k + 1 2
) k  q 40 +
k = 0
k =∞
å ( v
21 2
) k  q 40
k = 21
′
= A 40:21
Where A ′ is determined at (1.05) 2 − 1 = 10.25% p.a.
æ 1.05 ö
So Var(W) = ç
÷
è 0.05 ø
2
{ A ′
40:21
(
2
− A 40:21
2
′
= 441 A 40:21
− A 40:21
0.05
} where d = 1.05
)
The result can be derived in a similar way using
a min( K
40 ,20)
= a  min( K
40 + 1,
21)
− 1
Page 11 %%%%%%%%%%%%%%%%%%%%%%%%%%%%%%%%%%%%% — April 2001 — Examiners’ Report
