\documentclass[a4paper,1pt]{article}

%%%%%%%%%%%%%%%%%%%%%%%%%%%%%%%%%%%%%%%%%%%%%%%%%%%%%%%%%%%%%%%%%%%%%%%%%%%%%%%%%%%%%%%%%%%%%%%%%%%%%%%%%%%%%%%%%%%%%%%%%%%%%%%%%%%%%%%%%%%%%%%%%%%%%%%%%%%%%%%%%%%%%%%%%%%%%%%%%%%%%%%%%%%%%%%%%%%%%%%%%%%%%%%%%%%%%%%%%%%%%%%%%%%%%%%%%%%%%%%%%%%%%%%%%%%%

\usepackage{eurosym}
\usepackage{vmargin}
\usepackage{amsmath}
\usepackage{graphics}
\usepackage{epsfig}
\usepackage{enumerate}
\usepackage{multicol}
\usepackage{subfigure}
\usepackage{fancyhdr}
\usepackage{listings}
\usepackage{framed}
\usepackage{graphicx}
\usepackage{amsmath}
\usepackage{chngpage}

%\usepackage{bigints}
\usepackage{vmargin}

% left top textwidth textheight headheight

% headsep footheight footskip

\setmargins{.0cm}{.5cm}{16 cm}{cm}{0.5cm}{0cm}{1cm}{1cm}

\renewcommand{\baselinestretch}{1.}

\setcounter{MaxMatrixCols}{10}

\begin{document}

%%---  Question 

[Total 12]
104 S2001—46
You have been asked by a government department to help them calculate
childhood mortality rates. They have provided you with the following data
collected from a recent investigation.
Age category
(t) Age range Exposure
(in years)
E t c Deaths
(d t )
1
2
3
4
5
6 age < 1 month
1 month \leq  age < 3 months
3 months \leq  age < 6 months
6 months \leq  age < 12 months
1 year \leq  age < 2 years
2 years \leq  age < 3 years
Total 1,536.2
3,041.9
4,498.1
8,792.8
16,999.2
16,440.2
51,308.4 396
139
144
219
176
79
1,153
Exposure is defined as “the number of lives who survived through the interval
multiplied by the length of the interval in years, plus the fraction of the year
lived in the interval by those lives that died in the interval”. Exposure is
measured in years.
Estimate each of the following probabilities:
\item  the probability of dying in the first month of life
\item  (a)
q 0
(b)
3 q 0

[10]
In each case state any assumptions you have made in calculating your estimates.
[Total 13]
104 S2001—5
%%%%%%%%%%%%%%%%%%%%%%%%%%%%
6
(i)
The probability of dying the first month
=
D
E
where D is the deaths in the first month of life = d 1 = 396
and E is the initial exposed to risk of dying in the first month
= E 1 c  ́ 12 + d 1  ́ 0.5 (since the unit we are dealing with is a month).
= 1,536.2  ́ 12 + 396  ́ 0.5 = 18,632.4.
Thus the probability of dying in the first month = 0.021253.
(ii)
q 0 =
D
E
where D is the deaths in the first year of life = d 1 + d 2 + d 3 + d 4 = 898
and E is the initial exposed to risk of dying in the first year
Page 6 %%%%%%%%%%%%%%%%%%%%%%%%%%%%%%%%%%%%% — 
3
= E 1 c + E 2 c + E 3 c + E 4 c + d 1 ( 111⁄2
+ d 2 ( 10
+ d 3 ( 71⁄2
+ d 4 ( 12
)
12 )
12 )
12 )
= 17,869 + 640.083 = 18,509.083
Thus the probability of dying in the first year = 0.048517
(iii)
3 q 0
=
D
E
where D is the deaths in the first three years of life
= d 1 + d 2 + d 3 + d 4 + d 5 + d 6 = 1,153
and E is the initial exposed to risk of dying in the first three years
= 1
3 æ 6 c
ö
111⁄2
10
71⁄2
3
ç E t + d 1 ( 2 12 ) + d 2 ( 2 12 ) + d 3 ( 2 12 ) + d 4 ( 2 12 ) + 1.5 d 5 + 0.5 d 6 ÷
è t = 1
ø
= 1
3 (51,308.4 + 2,739.6) = 18,016.0
å
Thus the probability of dying in the first three years = 0.063999
Alternative is to estimate each of q 1 q 2 separately and then use
3 q 0 = 1 - (1 - q 0 ) (1 - q 1 ) (1 - q 2 )
q 1 = 176
176
=
= 0.010300
16,999.2 + 0.5  ́ 176
17,087.2
q 2 = 79
16,440.2 + 0.5  ́ 79
3 q 0 =
=
= 1 - (1 - 0.048517) (1 - 0.010300) (1 - 0.004794)
1 - 0.937168
0.062832
=
79
= 0.004794
16,479.7
These calculations assume that deaths in each age interval occur on
1⁄2 2 41⁄2 9
, ,
, ,11⁄2
average at the mid point of the interval i.e. at times
12 12 12 12
and 21⁄2 years.
Page 7 %%%%%%%%%%%%%%%%%%%%%%%%%%%%%%%%%%%%% — 
ALTERNATIVELY
If we assume that the force of mortality is constant over each age interval, then
(i)
The assumed constant force over the first month of life can be estimated
by
396
= 0.257779
1,536.2
and the probability of surviving the first month of life is
1 ü
ì
exp í - 0.257779  ́ ý = 0.978748
12 þ
î
So the probability of dying in the first month of life is 0.021252.
(ii)
The assumed constant forces over the remaining age intervals in the first
year of life can be estimated by
1 to 3 months 139
=
3,041.9 0.045695
3 to 6 months 144
=
4,498.1 0.032014
6 to 12 months 219
=
8,792.8 0.024907
The survival probabilities for these intervals are estimated by
Page 8
1 to 3 months 2 ü
ì
exp í - 0.045695  ́ ý = 0.992413
12 þ
î
3 to 6 months 3 ü
ì
exp í - 0.032014  ́ ý = 0.992028
12 þ
î
6 to 12 months 6 ü
ì
exp í - 0.024907  ́ ý = 0.987624
12
î
þ
Then 1
and 1 q̂ 0
p̂ 0 = 0.978748  ́ 0.992413  ́ 0.992028  ́ 0.987624
= 0.951654
= 0.048346 %%%%%%%%%%%%%%%%%%%%%%%%%%%%%%%%%%%%% — 
(iii)
The assumed constant force over (1, 2) (2, 3) can be estimated by
(1, 2) 176
16,999.2 = 0.010353
(2, 3) 79
16,440.2 = 0.004805
and the survival probabilities can be estimated by
