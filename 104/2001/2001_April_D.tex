\documentclass[a4paper,1pt]{article}

%%%%%%%%%%%%%%%%%%%%%%%%%%%%%%%%%%%%%%%%%%%%%%%%%%%%%%%%%%%%%%%%%%%%%%%%%%%%%%%%%%%%%%%%%%%%%%%%%%%%%%%%%%%%%%%%%%%%%%%%%%%%%%%%%%%%%%%%%%%%%%%%%%%%%%%%%%%%%%%%%%%%%%%%%%%%%%%%%%%%%%%%%%%%%%%%%%%%%%%%%%%%%%%%%%%%%%%%%%%%%%%%%%%%%%%%%%%%%%%%%%%%%%%%%%%%

\usepackage{eurosym}
\usepackage{vmargin}
\usepackage{amsmath}
\usepackage{graphics}
\usepackage{epsfig}
\usepackage{enumerate}
\usepackage{multicol}
\usepackage{subfigure}
\usepackage{fancyhdr}
\usepackage{listings}
\usepackage{framed}
\usepackage{graphicx}
\usepackage{amsmath}
\usepackage{chngpage}

%\usepackage{bigints}
\usepackage{vmargin}

% left top textwidth textheight headheight

% headsep footheight footskip

\setmargins{.0cm}{.5cm}{16 cm}{cm}{0.5cm}{0cm}{1cm}{1cm}

\renewcommand{\baselinestretch}{1.}

\setcounter{MaxMatrixCols}{10}

\begin{document}
%% ---- 104 A2001—24
A life insurance company has investigated the recent mortality experience of its
male term assurance policy holders by estimating the initial rate of mortality q x .
The crude estimates q ˆ x , of these rates will be graduated by reference to a
standard mortality table for male permanent assurance policy holders with rates
q s so that the graduated rates q o are given by
x
x
q o x = a + bq x s
(A)
where a and b are constants. The estimates of a and b will be determined by
minimising
å w ( q ˆ
x
x
x
− a − bq x s
)
2
(B)
where w x is a suitably chosen weighting function.
5
\item  Describe how the suitability of the formula (A) could be investigated.
%%---  Question 
\item  Explain why it is important to use the weighting function, w x , in
formula (B). Determine an expression for a suitable choice of function for
w x .

\item  
Explain how the smoothness of the graduated rates, q o x , is ensured.

%%%%%%%%%%%%%%%%%%%%%%%%%%%%%%%%%%%%%%%%%%%%%%%%%%%%%%%%%%%%%%%%%%%%%%
4
(i) Plot q ˆ x against q x s and look for an approximate straight line fit.
(ii) At each age there will be a different sample size/exposed to risk, E x . This
will usually be largest at ages where many term assurances are sold e.g.
25 to 50 and smaller at other ages.
The estimation procedure should pay more attention to ages where there
are lots of data. These ages should have a greater influence on the choice
of a and b than other ages.
Other relevant comments also received marks.
So weights w x α E x
Suitable choice is w x = é ë Var ( q ˆ x ) ù û
− 1
= E x
q x (1 − q x )
; E x
as q x ; 10 −2
q x
These weights can be estimated by
E 2
E x
= x
q ˆ x
\theta x
Page 5 %%%%%%%%%%%%%%%%%%%%%%%%%%%%%%%%%%%%% — April 2001 — Examiners’ Report
(iii)
The graduated rates q o x are a linear function of the rates in the standard
table q x s . The standard table rates will already be smooth.
Smoothness is based on the size of the third differences of the graduated
rates ∆ 3 q o , which because the relationship is linear will be equal to
x
b∆ 3 q x s . ∆ 3 q x s will already be acceptably small because the standard table
rates will already be smooth.
