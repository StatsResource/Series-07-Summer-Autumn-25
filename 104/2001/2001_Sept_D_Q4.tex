\documentclass[a4paper,1pt]{article}

%%%%%%%%%%%%%%%%%%%%%%%%%%%%%%%%%%%%%%%%%%%%%%%%%%%%%%%%%%%%%%%%%%%%%%%%%%%%%%%%%%%%%%%%%%%%%%%%%%%%%%%%%%%%%%%%%%%%%%%%%%%%%%%%%%%%%%%%%%%%%%%%%%%%%%%%%%%%%%%%%%%%%%%%%%%%%%%%%%%%%%%%%%%%%%%%%%%%%%%%%%%%%%%%%%%%%%%%%%%%%%%%%%%%%%%%%%%%%%%%%%%%%%%%%%%%

\usepackage{eurosym}
\usepackage{vmargin}
\usepackage{amsmath}
\usepackage{graphics}
\usepackage{epsfig}
\usepackage{enumerate}
\usepackage{multicol}
\usepackage{subfigure}
\usepackage{fancyhdr}
\usepackage{listings}
\usepackage{framed}
\usepackage{graphicx}
\usepackage{amsmath}
\usepackage{chngpage}

%\usepackage{bigints}
\usepackage{vmargin}

% left top textwidth textheight headheight

% headsep footheight footskip

\setmargins{.0cm}{.5cm}{16 cm}{cm}{0.5cm}{0cm}{1cm}{1cm}

\renewcommand{\baselinestretch}{1.}

\setcounter{MaxMatrixCols}{10}

\begin{document}
%%---  Question 
[Total 10]4
A cash-flow is payable continuously at a rate of ρ(t) per annum at time t provided
a life who is aged x at time 0 is still alive. T x is a random variable which
measures the complete future lifetime in years of a life aged x.
\item  Write down an expression, in terms of T x , for the present value at time 0
of this cash-flow, at a constant force of interest δ p.a.
%%---  Question 
\item  Hence or otherwise show that the expected present value at time 0 of this
cash-flow is equal to:
∞
ò e
−δ s
ρ(s) s p x ds.

0
\item 
An annuity is payable continuously during the lifetime of a life now
aged 30, but for at most 10 years. The rate of payment at all times t
during the first 5 years is \$5,000 p.a., and thereafter it is \$10,000 p.a.
The force of mortality to which this life is subject is assumed to be
0.01 p.a. at all ages between 30 and 35, and 0.02 p.a. between 35 and 40.
Find the expected present value of this annuity at a force of interest of
0.05 p.a.

\item 
If the mortality and interest assumptions are as in \item , find the expected
present value of the benefits of a term assurance, issued to the life in \item ,
which pays \$40,000 immediately on death within 10 years.

[Total 12]
104 S2001—3
(i) The present value =
n ®¥
n ®¥
T x
ò r ( s ) e
-d s
ds
0
(ii)
The expected present value
é T x
ù ¥
= E ê r ( s ) e -d s ds ú = æ ç
ê ë 0
ú û 0 è
ò
ò ò
t
0
r ( s ) e -d s ds ö ÷ f ( t ) dt
ø
where f denotes the probability density function of T x .
¥
Hence EPV =
5
(iii)
EPV =
ò
0
r ( s ) e -d s æ ç
è
0
ò
ò
¥
s
f ( t ) dt ö ÷ ds =
ø
¥
ò r ( s ) e
-d s
s
p x ds .
0
5
ò
e - 0.05 s 5,000 e - 0.01 s ds + e - 5  ́ 0.05 . e - 5  ́ 0.01 e - 0.05 s 10,000 e - 0.02 s ds
0
= 5,000a 5 @ force of 0.06 + 10,000e - 5(0.06) a 5 @ 0.07 = 52,851.69
Page 4 %%%%%%%%%%%%%%%%%%%%%%%%%%%%%%%%%%%%% — 
(iv)
10
æ 5
ö
40,000 ç e - 0.05 t 0.01 e - 0.01 t dt + e - 0.05 t 0.02 e - 0.01  ́ 5 e - 0.02( t - 5) dt ÷
ç
÷
5
è 0
ø
ò
ò
(
)
= 0.01 a 5 d= 0.06 + 0.02 e - 5(0.06) a 5 d= 0.07  ́ 40,000
= 4,228.14
Other approaches are possible, in particular the use of premium conversion
relationships in (iv).
