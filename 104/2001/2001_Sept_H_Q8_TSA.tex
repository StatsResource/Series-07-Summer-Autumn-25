\documentclass[a4paper,1pt]{article}

%%%%%%%%%%%%%%%%%%%%%%%%%%%%%%%%%%%%%%%%%%%%%%%%%%%%%%%%%%%%%%%%%%%%%%%%%%%%%%%%%%%%%%%%%%%%%%%%%%%%%%%%%%%%%%%%%%%%%%%%%%%%%%%%%%%%%%%%%%%%%%%%%%%%%%%%%%%%%%%%%%%%%%%%%%%%%%%%%%%%%%%%%%%%%%%%%%%%%%%%%%%%%%%%%%%%%%%%%%%%%%%%%%%%%%%%%%%%%%%%%%%%%%%%%%%%

\usepackage{eurosym}
\usepackage{vmargin}
\usepackage{amsmath}
\usepackage{graphics}
\usepackage{epsfig}
\usepackage{enumerate}
\usepackage{multicol}
\usepackage{subfigure}
\usepackage{fancyhdr}
\usepackage{listings}
\usepackage{framed}
\usepackage{graphicx}
\usepackage{amsmath}
\usepackage{chngpage}

%\usepackage{bigints}
\usepackage{vmargin}

% left top textwidth textheight headheight

% headsep footheight footskip

\setmargins{.0cm}{.5cm}{16 cm}{cm}{0.5cm}{0cm}{1cm}{1cm}

\renewcommand{\baselinestretch}{1.}

\setcounter{MaxMatrixCols}{10}

\begin{document}

104 S2001—68
\item  Explain why graduated rates, rather than crude estimates of mortality
rates are used in the construction of standard mortality tables.

\item  A graduation of the mortality experience of the male population of a
region of the United Kingdom has been carried out using a graphical
method. The following is an extract from the results.
Age x Actual number
of deaths, \theta x Graduated
mortality rate, q o x Initial exposed
to risk, E x
14
15
16
17
18
19
20
21
22 3
8
5
14
17
9
15
10
10 0.00038
0.00043
0.00048
0.00053
0.00059
0.00066
0.00074
0.00083
0.00093 12,800
15,300
12,500
15,000
16,500
10,100
12,800
13,700
11,900
Total 91
E x q o x
4.86
6.58
6.00
7.95
9.74
6.67
9.47
11.37
11.07
73.71
Use the Chi-squared test to test the adherence of the graduated rates to
the data. State clearly the null hypothesis you are testing and comment
on the result.

\item 
Perform two other tests which detect different aspects of the adherence of
the graduation to the data. For each test state clearly the features of the
graduation which the test is able to detect, and comment on your results.

[Total 15]
104 S2001—7

%%%%%%%%%%%%%%%%%%%%%%%%%%%%%%%%%%%%%%%%%%%%%%%%%%%%%%%%%%%%%%%%%%%%
8
(i)
4.3% p.a. effective.
Mortality rates are estimated separately for each year of age.
So there is no reason why the estimated rates should progress smoothly
from age to age as would be expected, a priori.
This fluctuation is the result of sampling error attaching to each estimate.
The sampling error can be smoothed out by using information from
adjacent ages to adjust the rate at any age. The adjusted rate is called the
graduated rate.
Graduated rates will be smooth (from age to age) and will not deviate
significantly from the observed rates.
Page 10 %%%%%%%%%%%%%%%%%%%%%%%%%%%%%%%%%%%%% — 
So they will reflect the observed mortality experience and conform to a
priori assumptions about rates. In calculations e.g. premium rates the
resulting calculated values will change smoothly from age to age.
Examiners’ Comment: Not all points were necessary for full marks.
(ii)
x
14
15
16
17
18
19
20
21
22
Total
q x q o x  ́ 10 5 E x E x q o x q x - E x q o x
3
8
5
14
17
9
15
10
10
91 38
43
48
53
59
66
74
83
93 12,800
15,300
12,500
15,000
16,500
10,100
12,800
13,700
11,900 4.86
6.58
6.00
7.95
9.74
6.67
9.47
11.37
11.07
73.71 -1.86
+1.42
-1.00
+6.05
+7.26
+2.33
+5.53
-1.37
-1.07
17.29
( q x - E x q o x ) / E x q o x
-0.84
0.55
-0.41
2.15
2.33
0.90
1.80
-0.41
-0.32
All the q o x are small so E x q o x can be used to approximate the variance.
Chi-Squared Test
H 0 : The observed rates look as if they come from a population in which the
graduated rates are the true rates.
Observed Value of test Statistic:
Sum of squared standardised deviations 15.55
which will be c 29-deduction if null hypothesis is true.
Say c 26 or c 7 2
deduct one degree of freedom from each of height, slope and curvature.
c 26 (0.95) = 12.59 c 7 2 (0.95) = 14.07
So result leads to rejection of null hypothesis. The graduated rates do not
appear to fit the observed rates.
There seems to be “under-estimation” at ages 17, 18, 19 and 20.
Page 11 %%%%%%%%%%%%%%%%%%%%%%%%%%%%%%%%%%%%% — 
(iii)
Any two of the following (suitably chosen so as not to overlap in what
they test).
Individual Standardised Deviations:
Table of Values
Values -¥, -2 -2, -1 -1, 0 0, +1 1, 2 2, ¥
Expected 0.2 1.22 3.07 3.07 1.22 0.2
1111 11 1 11
4 2 1 2
Tally
Observed
0
0
The standarised deviations seem to be skewed towards +ve values
compared to the distribution expected under the null hypothesis.
There are 4 negative and 5 positive deviations compared to the expected
number of each of 4.5. So no reason to worry here.
Five deviations are >1⁄20.671⁄2 and four deviations are less than 1⁄20.671⁄2,
compared to the expected number of each of 4.5. No reason to reject null
hypothesis.
There are 2 outliers > 1⁄221⁄2 in a sample of size 9. This is much greater
than the expected number of 0.4, so reason to reject the null hypothesis.
Here we are examining the general level of the graduated rates, are they
too high or too low?
We must conclude that they appear to be too low compared to the
observed rates.
Grouping of Signs Test:
There are 9 deviations with 3 groups of negative signs and 2 groups of
positive signs.
The probability of observing 2 or fewer groups of positive signs when there
are 5 positive and 4 negative deviations is
æ 4 ö æ 5 ö
ç ÷ç ÷
è 0 ø è 1 ø +
æ 9 ö
ç ÷
è 5 ø
Page 12
æ 4 ö æ 5 ö
ç ÷ç ÷
è 1 ø è 2 ø
æ 9 ö
ç ÷
è 5 ø %%%%%%%%%%%%%%%%%%%%%%%%%%%%%%%%%%%%% — 
= 5
4  ́ 10
+
126
126
= 45
= 0.357
126
So there is no reason to reject the null hypothesis.
This test looks for “clumping” of deviations of the same sign, i.e. that the
graduated curve “cuts through” the curve of observed rates.
Change of Sign Test:
There are 9 deviations and 4 observed changes of signs.
Under the null hypothesis the observed number of changes of sign should
follow a Binomial (8, 0.5) sampling distribution.
If the null hypothesis is true the probability of observing 4 or fewer sign
changes is:
8
8
8
8
8.7 æ 1 ö
8.7.6 æ 1 ö
8.7.6.5 æ 1 ö
æ 1 ö
æ 1 ö
ç 2 ÷ + 9 ç 2 ÷ + 2 ç 2 ÷ + 6 ç 2 ÷ + 24 ç 2 ÷
è ø
è ø
è ø
è ø
è ø
8
8
æ 1 ö
= 163 ç ÷ = 0.637
è 2 ø
So there is no reason to reject the null hypothesis.
This test detects if the graduated rates are consistently higher or lower
than the observed rates.
Cumulative Deviations Test:
This text tests for overall bias or long runs of deviations of the same sign.
The observed value of the cumulative deviations statistic is 17.29.
If the null hypothesis is true we expect this statistic to follow a
N(0, 73.71) distribution.
Observed Standard Normal Value:
17.29
73.71
= 2.01
which is significant at the 5% level. There is reason to reject the null
hypothesis.
Page 13 %%%%%%%%%%%%%%%%%%%%%%%%%%%%%%%%%%%%% — 
This test looks for significant under or over estimation by the graduated
rates compared to the observed rates.
Serial Correlations Test:
Mean standardised deviation =
i
5.75
= 0.64
9
1 2 3 Z i - Z -1.48 -0.09 -1.05 1.51 1.69 0.26
Z i +1 - Z -0.09 -1.05 1.51 1.69 0.26
i = 8
å ( Z
i - Z ) ( Z i + 1 - Z ) = 1.7251
i - Z ) 2 = 11.8745
i = 1
i = 9
å ( Z
4
5
6
7 8 9
1.16 -1.05 -0.96
1.16 -1.05 -0.96
i = 1
Then: r 1 =
1.7251
8
 ́ 11.8745
9
= 0.1634
Standardised r 1 = 0.1634 9 = 0.4903
Examiners’ Comment: using “separate” residual means gives r 1 = 0.4598.
which is standardised normal if H 0 is true.
So no reason to reject null hypothesis.
This test looks for “runs” of deviations of the same sign, i.e. looks to see if
graduated rates ignore “key” features of the observed rates.
Page 14 %%%%%%%%%%%%%%%%%%%%%%%%%%%%%%%%%%%%% — 
\end{document}
