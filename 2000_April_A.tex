\documentclass{article}
\usepackage{framed}
\usepackage[utf8]{inputenc}



\begin{document}
\begin{enumerate}
%%%%%%%%%%%%%%%%%%%%%%%%%%%%%%%%%%%%%%%%%%%%%%%%%%%%%%%
\item  Fourteen economists were asked to provide forecasts for the percentage rate
of inflation for the third quarter of 2002. They produced the forecasts given
below.
\begin{verbatim}
1.2 1.4 1.5 1.5 1.7 1.8 1.8
1.9 1.9 2.1 2.7 3.2 3.9 5.0    
\end{verbatim}

Calculate the median and the upper and lower quartiles of these forecasts. 
%%%%%%%%%%%%%%%%%%%%%%%%%%%%%%%%%%%%%%%%%%%%%%%%%%%%%%%
\item Insurance policies providing car insurance are such that the sizes of claims are
normally distributed with mean £1,870 and standard deviation £610. In one
month 50 claims are made. Assuming that claims are independent, calculate
the probability that the total of the claim sizes is more than £100,000. 
%%%%%%%%%%%%%%%%%%%%%%%%%%%%%%%%%%%%%%%%%%%%%%%%%%%%%%%
\item  In an investigation into the proportion (q) of lapses in the first year of a
certain type of policy, the uncertainty about q is modelled by taking q to have a
beta distribution with parameters a = 1 and b = 9, that is, with density
f(q) = 9(1 - q)8 : 0 < q < 1.
Using this distribution, calculate the probability that q exceeds 0.2. 
%%%%%%%%%%%%%%%%%%%%%%%%%%%%%%%%%%%%%%%%%%%%%%%%%%%%%%%
4 Consider the following three probability statements concerning an F variable
with 6 and 12 degrees of freedom.
\item P(F6,12 > 0.250) = 0.95
\item P(F6,12 < 4.821) = 0.99
\item P(F6,12 < 0.130) = 0.01
State, with reasons, whether each of these statements is true. 
%%%%%%%%%%%%%%%%%%%%%%%%%%%%%%%%%%%%%%%%%%%%%%%%%%%%%%%
\item  An insurance company’s records suggest that experienced drivers (those aged
over 21) submit claims at a rate of 0.1 per year, and inexperienced drivers
(those 21 years old or younger) submit claims at a rate of 0.15 per year. A
driver can submit more than one claim a year. The company has 40
experienced and 20 inexperienced drivers insured with it.
The number of claims for each driver can be modelled by a Poisson
distribution, and claims are independent of each other. Calculate the
probability the company will receive three or fewer claims in a year. 

%%%%%%%%%%%%%%%%%%%%%%%%%%%%%%%%%%%%%%%%%%%%%%%%%%%%%%%

\item The number of claims which arise under a policy of a particular type in a year
is to be modelled as a Poisson(l) random variable. A random sample of 600
such policies gave rise to a total of 72 claims in 1999.
Determine the approximate probability value of this result in a test of
H0 : l = 0.14 v H1 : l < 0.14 . 
\end{enumerate}
%%%%%%%%%%%%%%%%%%%%%%%%%%%%%%%%%%%%%%%%%%%%%%%%%%%%%%%

%%  101 April 2000
%%  Subject 101 — Statistical Modelling

%%%%%%%%%%%%%%%%%%%%%%%%%%%%%%%%%%%%%%%%%%%%%%%%%%%%%%%%%%%%%%%%%%%%%
Page 2
1 As n = 14 the median is half way between the 7th and 8th value
i.e. m = (1.8 + 1.9)/2=1.85.
The quartiles are the 4th and 11th values, so Q1 = 1.5 and Q3 = 2.7 .
OR: Using the definition of the quartiles as the 15/4th and 45/4th value gives
Q1 = 1.5 and Q3 = 2.8.
%%%%%%%%%%%%%%%%%%%%%%%%%%%%%%%%%%%%%%%%%%%%%%%%%%%%%%%%%%%%%%%%%%%%%
2 The total claim, T, will be normally distributed with mean 50  1870 = 93500
and variance 50  6102 = 18,605,000 = 43132.
(Alternatively, we can work with the mean claim.)
Thus, the probability that the total claim is greater than £100,000 is
100,000 93,500
1
4313
  
  
 
= 1  (1.507) = 0.066.
3 P( > 0.2) =
1
8
0.2
 9 (1  ) d
=
1
9
0.2
(1  ) 
= 0 + (1  0.2)9 = 0.89 = 0.13
4 For (a) to be true, 0.250 must be lower 5% pt of F6,12 i.e. reciprocal of upper 5% pt
of F12,6 which is
1
4.000
= 0.250  true.
For (b) to be true, 4.821 must be upper 1% pt of F6,12 which is 4.821 true.
For (c) to be true, 0.130 must be lower 1% pt of F6,12 i.e. reciprocal of upper 1% pt
of F12,6 which is
1
7.718
= 0.130 true.


%%%%%%%%%%%%%%%%%%%%%%%%%%%%%%%%%%%%%%%%%%%%%%%%%%%%%%%%%%%%%%%%%%%%%
\end{document}
