\documentclass[a4paper,12pt]{article}

%%%%%%%%%%%%%%%%%%%%%%%%%%%%%%%%%%%%%%%%%%%%%%%%%%%%%%%%%%%%%%%%%%%%%%%%%%%%%%%%%%%%%%%%%%%%%%%%%%%%%%%%%%%%%%%%%%%%%%%%%%%%%%%%%%%%%%%%%%%%%%%%%%%%%%%%%%%%%%%%%%%%%%%%%%%%%%%%%%%%%%%%%%%%%%%%%%%%%%%%%%%%%%%%%%%%%%%%%%%%%%%%%%%%%%%%%%%%%%%%%%%%%%%%%%%%

\usepackage{eurosym}
\usepackage{vmargin}
\usepackage{amsmath}
\usepackage{graphics}
\usepackage{epsfig}
\usepackage{enumerate}
\usepackage{multicol}
\usepackage{subfigure}
\usepackage{fancyhdr}
\usepackage{listings}
\usepackage{framed}
\usepackage{graphicx}
\usepackage{amsmath}
\usepackage{chngpage}

%\usepackage{bigints}
\usepackage{vmargin}

% left top textwidth textheight headheight

% headsep footheight footskip

\setmargins{2.0cm}{2.5cm}{16 cm}{22cm}{0.5cm}{0cm}{1cm}{1cm}

\renewcommand{\baselinestretch}{1.3}

\setcounter{MaxMatrixCols}{10}

\begin{document}

\begin{enumerate}
%%%%%%%%%%%%%%%%%%%%%%%%%%%%%%%%%%%%%%%%%%%%%%%%%%%%%%%%%%%%%%%%%%%%%%%%%%%%%%%%%%%%%%%%%%%%%%%%%%%%%%%%%%%%%%%%%%%%%%%%%%%%%%%%%%%%%%%%%%
\item 5 In a quality control test, a random sample of 100 items is selected, of which 90 are
found to be satisfactory. The value of p, the probability of an item being satisfactory,
is unknown.
Write down the probability of observing 90 satisfactory items out of 100 as a function
of p, and thus derive the maximum likelihood estimate of p. [3]
%%%%%%%%%%%%%%%%%%%%%%%%%%%%%%%%%%%%%%%%%%%%%%%%%%%%%%%%%%%%%%%%%%%%%%%%%%%%%%%%%%%%%%%%%%%%%%%%%%%%%%%%%%%%%%%%%%%%%%%%%%%%%%%%%%%%%%%%%%
\item 6 A national survey research company has past data which indicate that the interview
time for a consumer opinion study has a standard deviation of 6 minutes.
Calculate the size of the sample that should be taken if the company requires a 99%
confidence interval for the mean interview time to be within 2 minutes. 
\item 7 The following information on white blood cell count (WBCC) was collected from
subjects one week after the start of chemotherapy treatment. One group of subjects
(A) received steroids in addition to the chemotherapy treatment and the other group
(B) received a placebo in addition to the chemotherapy. The subjects were assigned
to the groups at random.
Group A — Steroid
WBCC (millions of cells per ml.)
12.4 15.2 12.7 15.9 12.2 14.2 12.9 14.2 12.4 14.6
12.7 13.6 12.5 13.3 12.1 13.9 17.1 13.6 17.2 13.1
Group B — Placebo
WBCC (millions of cells per ml.)
17.0 13.5 15.4 14.1 15.4 14.8 12.9 14.4 13.2 13.1
12.9 13.9 13.0 13.6 13.0 13.4 12.9 13.1 14.4 13.8
\begin{enumerate}[(a)]
\item Construct stem and leaf diagrams for Group A and Group B separately. 
\item Comment on the results in the context of investigating an association between
WBCC and the treatment with or without steroids. 
[Total 4]
\end{enumerate}

%%%%%%%%%%%%%%%%%%%%%%%%%%%%%%%%%%%%%%%%%%%%%%%%%%%%%%%%%%%%%%%%%%%%%%%%%%%%%%%%%%
8 Consider a negative binomial variable X with probability function given by
1
( = ) = k x : = 0,1, 2, ... k x
P X x p q x
x
   
 
 
where 0 < p < 1 and q = 1  p.
\begin{itemize}
\item Show that the moment generating function is given by
( ) =
1
k
t
M t p
qe
 
    
for qet < 1. 
\item Determine E(X) and E(X2) by expanding M(t) as a power series as far as the
term in t2, and hence verify that the mean and variance of X are given by
kq
p
and 2
kq
p
, respectively. [3]
%%%%%%%%%%%%%%%%%%%%%%%%%%%%%%%%%%%%%%%%%%%%%%%%%%%%%%%%%%%%%%%%%%%%%%%%%%%%%%%%%%%%%%%%%%%%%%%%%%%%%%%%%%%%%%%%%%%%%%%%%%%%%%%%%%%%%%%%%%


5 Using binomial model, P(90 satisfactory items out of 100) = 100 90 10
(1 )
90
p p
 
  
 
Therefore the log likelihood for p is given by:
log L( p) = 90log p 10log(1 p) + constant
log = 90 10 ˆ = 90 = 0.9
1 100
d L p
dp p p
 

(MLE for p)
(MLE is sample proportion.)
%%%%%%%%%%%%%%%%%%%%%%%%%%%%%%%%%%%%%%%%%%%%%%%%%%%%%%%%%%%%%%%%%%%%%%%%%%%%%%%%%%%%%%%%%%%%%%%%%%%%%%%%%%%%%%%%%%%%%%%%%%%%%%%%%%%%%%%%%%
6 Let n denote the sample size which is determined by the limits of the 99% confidence
interval, i.e.
2.58 6 2
n
 
n  (3 2.58)2 = 59.9. Therefore n should be at least 60.
%%%%%%%%%%%%%%%%%%%%%%%%%%%%%%%%%%%%%%%%%%%%%%%%%%%%%%%%%%%%%%%%%%%%%%%%%%%%%%%%%%%%%%%%%%%%%%%%%%%%%%%%%%%%%%%%%%%%%%%%%%%%%%%%%%%%%%%%%%
7 \item Using leaves with units of 0.1, the stem and leaf diagrams are:
Group A: 12 1 2 4 4 5 7 7 9
(Steroid) 13 1 3 6 6 9
14 2 2 6
15 2 9
16
17 1 2
Group B: 12 9 9 9
(Placebo) 13 0 0 1 1 2 4 5 6 8 9
14 1 4 4 8
15 4 4
16
17 0
\item The distribution of WBCC is similar for both groups in that the medians are
similar (Group A median = (13.3 + 13.6)/2 = 13.45; Group B median = (13.5
+ 13.6)/2 = 13.55), and both distributions are slightly positively skewed.
There is a slightly greater variability in the steroid group compared to the
placebo group. It appears that the steroid treatment has no difference in effect
from the placebo.
Subject 101 (Statistical Modelling) — April 2002 — Examiners’ Report
Page 4
8 \item
0
1
( ) = ( tx ) = tx k x
x
k x
M t Ee e pq
x


   
 
 

0
1
= k ( t )x
x
k x
p qe
x


   
 
 

0
1
= (1 ) ( )
1
k
t k t x
t
x
p k x qe qe
qe x


     
         

=
1
k
t
p
qe
 
    
as the sum equals 1, being the sum of a probability
distribution.
Alternative:
2
0
1 ( 1) ( ) = 1 . .( ) ...
2
k t x k t t
x
k x k k p qe p kqe qe
x


      
      
   
 = 1 
k t k p qe 

\item
2
2 ( ) = = 1 ( ...)
2
1 (1 ...)
2
k
k
M t p q t t
t p q t
%%%%%%%%%%%%%%%%%%%%%%%%%%%%%%%%%%%%%%%%%%%%%%%%%%%%%%%%%%%%%%%%%%%%%%%%%%%%%%%%%%%
2
=1 ( ...) ( 1) ( )2 ( ...)2 ...
2 2
k q t t k k q t
p p

     
coefficient of t: E(X ) = k q
p
coefficient of
2
2!
t : E(X 2 ) = k q k(k 1)( q )2
p p
 
mean = E(X ) = kq
p
variance = E(X 2 ) E2 (X ) = k q k(k 1)( q )2 (k q )2
p p p
   
2
2 = k q k( q ) = kq (1 q ) = kq
p p p p p

    \item 
\end{itemize}
\end{document}
