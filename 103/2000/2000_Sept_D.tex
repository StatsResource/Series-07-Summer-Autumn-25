
\documentclass[a4paper,12pt]{article}

%%%%%%%%%%%%%%%%%%%%%%%%%%%%%%%%%%%%%%%%%%%%%%%%%%%%%%%%%%%%%%%%%%%%%%%%%%%%%%%%%%%%%%%%%%%%%%%%%%%%%%%%%%%%%%%%%%%%%%%%%%%%%%%%%%%%%%%%%%%%%%%%%%%%%%%%%%%%%%%%%%%%%%%%%%%%%%%%%%%%%%%%%%%%%%%%%%%%%%%%%%%%%%%%%%%%%%%%%%%%%%%%%%%%%%%%%%%%%%%%%%%%%%%%%%%%

\usepackage{eurosym}
\usepackage{vmargin}
\usepackage{amsmath}
\usepackage{graphics}
\usepackage{epsfig}
\usepackage{enumerate}
\usepackage{multicol}
\usepackage{subfigure}
\usepackage{fancyhdr}
\usepackage{listings}
\usepackage{framed}
\usepackage{graphicx}
\usepackage{amsmath}
\usepackage{chngpage}

%\usepackage{bigints}
\usepackage{vmargin}

% left top textwidth textheight headheight

% headsep footheight footskip

\setmargins{2.0cm}{2.5cm}{16 cm}{22cm}{0.5cm}{0cm}{1cm}{1cm}

\renewcommand{\baselinestretch}{1.3}

\setcounter{MaxMatrixCols}{10}

\begin{document}
\begin{enumerate}
103—4
[1]7
A client wishes to model the behaviour of a stochastic process {X t : t ≥ 0} which
represents the average annual return for a particular class of asset. After a
number of observations the client has determined that Corr(X t , X t − 1 ) = 0.7 and
Corr(X t , X t − 2 ) = 0.5. He thinks that one of the two models
I: X t = μ + 0.7(X t − 1 − μ) + 0.5(X t − 2 − μ) + e t
II: X t = μ + e t + 0.7e t −1 + 0.5e t − 2
will be best, but cannot decide which. He has simulated both processes from time
t = 1 to time t = 200, but has not obtained the results he expected, so is seeking
your advice.
(i)
(a) Outline a suitable method of simulating a second-order
autoregression, assuming you have access to a reliable stream
{u k : k ≥ 0} of pseudo-random numbers uniformly distributed over
the range [0, 1].
(b) Explain why might it be desirable to ensure that the stream {u k }
can be re-used if necessary.
[4]
(ii) State why neither of the suggested models is suitable.
(iii) (a)

Derive the lag-1 and lag-2 autocorrelations, ρ 1 and ρ 2 , of a second-
order autoregressive process
X t = μ + α 1 (X t − 1 − μ) + α 2 (X t − 2 − μ) + e t .
(b)
103—5
Find values of the parameters α 1 and α 2 which would provide a
suitable AR(2) model for {X t : t = 0, 1, 2, ...}.
[5]
[Total 10]
%%PLEASE TURN OVER
%%--Question 8
\item 
Consider a time-homogeneous Markov jump process {X(t) : t ≥ 0} with two states
denoted by 0, 1, and transition rates σ 0,1 = λ, σ 1,0 = μ.
(i) State Kolmogorov’s forward equation for the probability P 0,0 (t) that X is in
state 0 at time t, given that it starts in state 0.
[1]
(ii) Show that P 0,0 (t) =
(iii) Let O t denote the total amount of time spent in state 0 up until time t,
μ
λ
e − ( λ + μ ) t .
+
λ + μ λ + μ
[3]
ì 1 if X s = 0
I s ds , where I s = í
. Derive,
î 0 if X s ≠ 0
using the result in part (ii), an expression for E[O t X(0) = 0], the expected occupation time in state 0 by time t for the two-state continuous-time Markov chain starting in state 0.
[2]
which may be expressed as O t =
9
t
z 0
(iv) Write down the expected occupation time in state 1 by time t for the two-state continuous-time Markov chain starting in state 0.
[1]
(v) A health insurance scheme labels members as “healthy” (state 0) or “unhealthy” (state 1) at any time. When in state 0, members pay
contributions at rate α; when in state 1 they receive benefit at rate β.
Expenses amount to a constant γ per member per unit time.
(a) Explain how the above model can be used to calculate α in terms of β and γ.
(b) State the assumptions which you make in applying the model.
(c) Discuss whether they are likely to be satisfied in practice.
%%-- [4]
%%--- [Total 11]
%%%%%%%%%%%%%%%%%%%%%%%%%%%%%%%%%%%%%%%%%%
7
(i)
(a) A Lévy process is the sum of three independent components: a deterministic part of the form $\mu + \alpha t$, a Brownian part of the form
σB t and a pure jump part which may be thought of as a compound Poisson process.
(b) One problem would be in estimating the distribution of the jump sizes, particularly with only 250 observations. Even if a family were assumed for the distribution (e.g. double exponential), there would be the additional difficulty that small jumps would not be detectable against the background of the Gaussian noise.
(a) First the u k need to be transformed so that their distribution is something suitable for the white noise sequence of a time series,
since at the very least the mean of the sequence needs to be zero.
N (0, σ 2 e ) is the standard choice: one method of achieving this is to
define, for each integer t,
e 2 t
= σ e − 2 log u 2 t sin(2 π u 2 t + 1 )
e 2 t +1 = σ e − 2log u 2 t cos(2 π u 2 t + 1 ),
but there are others, such as the polar method, inverse transform method or acceptance-rejection sampling.
The values of the e t can now be fed into the formula to give the values of the X t , whichever model is in use.
(b)
(ii)
%%Page 4
\begin{itemize}
\item The ability to re-use a pseudo-random number sequence is important when comparing the ability of different mechanisms to
control a process which is affected by randomness: in order to ensure fair comparison of the mechanisms, the must be subjected
to the same degree of “random” input.
\item 
The models do not possess the correct correlation structure.
\end{itemize}
%%--- Subject 103 (Stochastic Modelling) — September 2000 — Examiners’ Report
(iii)
(a)
ρ 1 = Corr(X t ,X t − 1 ) = α 1 Corr(X t − 1 ,X t − 1 ) + α 2 Corr (X t − 2 ,X t − 1 ) = α 1 + α 2 ρ 1 .
Hence ρ 1 = α 1 / (1 − α 2 )
ρ 2 = Corr(X t ,X t − 2 ) = α 1 Corr(X t −1 ,X t − 2 ) + α 2 Corr(X t − 2 ,X t − 2 ) = α 1 ρ 1 + α 2 .
(b)
We have 0.7 = ρ 1 = α 1 / (1 − α 2 )
and 0.5 = ρ 2 = α 2 + α 1 2 / (1 − α 2 ) = α 2 + 0.7α 1 . Two equations in two
35
1
, α 2 =
.
unknowns. Solution: α 1 =
51
51
(2 marks for the observation that α 1 = 0.7 and α 2 = 0 is very close
to giving the right answer, as it gives ρ 2 = 0.49.)
8
(i)
′ ( t ) = μP 0,1 (t) − λP 0,0 (t), or a more general form such as P 0,0
′ ( t ) =
P 0,0
ΣP 0,k (t)σ k,0
(ii)
Since P 0,1 (t) = 1 − P 0,0 (t),
′ ( t ) = μ(1 − P 0,0 (t)) − λP 0,0 (t). Any solution method will do,
we have P 0,0
d ( λ + μ ) t
μ
[e
P 0,0 (t)] = μe ( λ + μ ) t , solved by P 0,0 (t) =
+ Ce − ( λ + μ ) t , with C
dt
λ+μ
being determined by the fact that P 0,0 (0) = 1.
e.g.
(iii)
E 0 O t = E 0 ò t 0 I s ds =
=
t
ò 0
E 0 I s ds =
t
ò 0
P 0,0 (s)ds
μ
λ
t +
(1 − e − ( λ + μ ) t )
2
λ+μ
( λ + μ )
(iv) Since the process must be in state 0 or state 1 at all times, the solution is
λ
λ
t −
just t − E 0 0 t =
(1 − e − ( λ + μ ) t ).
λ+μ
( λ + μ ) 2
(v) (a)
Assuming a member who is initially healthy, expected outgoings
(including expenses) by time t and expected income by time t, are
respectively
γt + β
and
æ
ç
è λ
ö
λ
λ
t −
(1 − e − ( λ +μ ) t ) ÷
2
+μ
( λ + μ )
ø
ö
μ
λ
t +
(1 − e − ( λ+μ ) t ) ÷ .
2
( λ + μ )
è λ +μ
ø
æ
α ç
%%Page 5Subject 103 (Stochastic Modelling) — September 2000 — Examiners’ Report
In the long run, then, as t → ∞, we require αμ = βλ + γ(λ + μ) to
break even.
%%%%%%%%%%%%%%%%%%%%%%%%%%%%%%%%%%%%%%%%%%%%%%%%%%%%%%%%%%%%%%%%%%%%%%%%%%%%%%%%
\end{document}
