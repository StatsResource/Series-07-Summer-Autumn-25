
\documentclass[a4paper,12pt]{article}

%%%%%%%%%%%%%%%%%%%%%%%%%%%%%%%%%%%%%%%%%%%%%%%%%%%%%%%%%%%%%%%%%%%%%%%%%%%%%%%%%%%%%%%%%%%%%%%%%%%%%%%%%%%%%%%%%%%%%%%%%%%%%%%%%%%%%%%%%%%%%%%%%%%%%%%%%%%%%%%%%%%%%%%%%%%%%%%%%%%%%%%%%%%%%%%%%%%%%%%%%%%%%%%%%%%%%%%%%%%%%%%%%%%%%%%%%%%%%%%%%%%%%%%%%%%%

\usepackage{eurosym}
\usepackage{vmargin}
\usepackage{amsmath}
\usepackage{graphics}
\usepackage{epsfig}
\usepackage{enumerate}
\usepackage{multicol}
\usepackage{subfigure}
\usepackage{fancyhdr}
\usepackage{listings}
\usepackage{framed}
\usepackage{graphicx}
\usepackage{amsmath}
\usepackage{chngpage}

%\usepackage{bigints}
\usepackage{vmargin}

% left top textwidth textheight headheight

% headsep footheight footskip

\setmargins{2.0cm}{2.5cm}{16 cm}{22cm}{0.5cm}{0cm}{1cm}{1cm}

\renewcommand{\baselinestretch}{1.3}

\setcounter{MaxMatrixCols}{10}

\begin{document}
\begin{enumerate}
eSuppose that the evolution of the price of an asset follows the lognormal model
log(S t ) = Y t = y + μt + αB t where B t denotes the standard Brownian motion and μ
is a negative drift. The asset will be liquidated at the stopping time
T α = inf{t : Y t = a} when its value reduces to e a , where a is some number less than
y. Consider now the exponential V t = exp(uY t − c(u)t).
(i) Derive the condition on c(u) under which {V t : t ≥ 0} is a martingale.
[3]
(ii) State the optional stopping theorem and explain how it is used.
[3]
(iii) Derive the moment generating function f(y, v) = E [ e − vT u Y 0 = y] of the
bankruptcy time for positive v by applying the optional stopping theorem
[5]
to the martingale V t .
[Total 11]
103—610
Consider a survival model with two states “alive” (A) and “dead” (D), with time-
dependent transition rate from A to D equal to μ(t) = μt. The time parameter, t,
represents the age of the individual under consideration.

\begin{enumerate}[(i)]
\item Calculate the transition probability P AA (s, t), defined by
P AA (s, t) = P(X(t) = AX(s) = A).
(ii)
[2]
\item Show, by making use of the formula
E(X) =
∞
z 0
P[X ≥ x] dx
for a positive random variable $X$, that the expected future lifespan of an
individual aged s is
E[R s ] =
1 1 − G ( s μ )
μ
g ( s μ )
,
where G is the standard Gaussian probability distribution function and g
is its density.
[4]
(iii)
\item It is desired to calibrate the above model so that the expected future
lifespan of an individual aged 70 is 6 years. Derive an approximation to
the corresponding value of μ, using the double inequality
1
1 1 − G ( x ) 1
−
≤
≤ .
x x 3
g ( x )
x
\item 
% (iv)
% [5]
A company wishes to test the validity of the above model. They assume that the true force of mortality from age 70 onwards is of the form $\mu(t) = a + bt$ and intend to test whether $a = 0$. The testing method will be to simulate one sample of size 1000 when a = 0 and another when $a \neq 0$,
then to see which most resembles the data which the company has
collected.
Explain how to simulate a value from the proposed distribution, for
arbitrary values of $a$ and $b$.
\end{enumerate}

%%%%%%%%%%%%%%%%%%%%%%
\newpage
9
(i)
\begin{itemize}
\item (b) The assumptions required are that the rate of becoming ill and
rate of recovery from illness are constant.
\item (c) This will certainly not be true of any individual member but, if
membership is large and the age and health profiles of the
members are constant by virtue of a constant influx of new
members, it may be a reasonable approximation.
\item If V t is a martingale, then its expectation must be constant and equal to
its initial value e uy .
Therefore E e u ( y +μ t +α B t ) − c ( u ) t = e uy + ( u μ+ u α
2 2
/ 2 − c ( u )) t
= e uy .
Thus we must have c ( u ) = u μ + u 2 α 2 / 2.
\item (ii)
The optional stopping theorem states that if M t is a martingale, and T is a
random stopping time, then under some additional technical conditions
(such as M t∧T being uniformly bounded) we have:
EM T = M 0 .
It is frequently used to evaluate the expectation of a function of T , such as
the moment generating function (as in this instance).
(iii)
\item Applying the optional stopping theorem to the martingale V t we find that
E V T a = e uy = E e ua − c ( u ) T a .
\item The equation c(u) = v has two roots u + , u − , one being negative and the
other positive (since v is positive).
Now V t ∧ T a = e u ( v ) Y t − vt and Y t ≥ a for 0 ≤ t ≤ T a . If u(v) < 0, then
0 < V t ∧ T a ≤ e u ( v ) a for all t, so that the technical condition is satisfied; the
same cannot be said if u(v) > 0.
\item Therefore the positive root is unacceptable and f(y, v) = E y e − vT a = e u − ( v )( y − a ) .
Comment: For the record, there were two very slight errors in this question, both
appearing as subscripts. In line 4, first formula: T { α } should have read T { a } , and in part
(iii) line 1: T { u } should have read T { a } . This was taken into account by the markers, and
the examiners ensured that no marks were lost by students because of either small error.

\end{enumerate}

%%--Page 6Subject 103 (Stochastic Modelling) — September 2000 — Examiners’ Report
10
t
(i) P AA (s, t) = e −
(ii) P[R s ≥ w] = P AA (s, s + w) = e −μ (( s + w )
ò s
μ xdx
= e −μ ( t
2
− s 2 ) / 2
2
∞
E[R s ] =
ò 0
e −μ sw −μ w
2
/2
− s 2 ) / 2
= e −μ sw −μ w
2
/2
. Therefore
dw .
Complete the square at the exponent to get
E[R s ] = e μ s
(iii)
2
/2
∞
ò 0
2
e −μ ( s + w )
/2
dw = e μ s
2
/2
ò
∞
s μ
e − x
2
/2
dx
μ
=
1 1 − G ( s μ )
μ
g ( s μ )
.
From (ii) and the given bound
E [ R s ] ≤
E [ R s ] ≥
1
1
μ s μ
=
1 æ 1
μ
ç
ç s
è
μ
−
1
sμ
1
1
1 ö
− 3 2 .
÷ =
3 3/2 ÷
s μ s μ
s μ ø
The first inequality yields
μ≤
1
1
=
= 0.00238 . (year) −2 .
s E [ R s ]
420
The second inequality can be written as
μ 2 E[R s ] −
μ 1
+
≥ 0,
s s 3
so μ must lie outside the interval:
1
±
s
1 4 E [ R s ]
4 E [ R s ]
24
−
1 ± 1 −
1 ± 1 −
2
3
s
70
s
s
=
=
= [0.00023, 0.00215]
2 E [ R s ]
2 s E [ R s ]
2 × 6 × 70
In fact, since clearly μ − 
1
we see that μ must lie in the interval
s E [ R s ]
[0.00215, 0.00238].
(iv)
Use the inverse transform method, X = F − 1 (U).
x
[ a + bt ] dt ) = exp{−a(x − 70) − 1⁄2b(x 2 − 70 2 )}.
In this case 1 − F(x) = exp( − ò 70
%%Page 7Subject 103 (Stochastic Modelling) — September 2000 — Examiners’ Report
Therefore 1⁄2bx 2 + ax = −log(1 − F(x)) + 1⁄270 2 b + 70a. Replace F(x) by u to
get
x = F − 1 (u) =
− a + a 2 + 2 b [ − log(1 − u ) + 1⁄270 2 b + 70 a ]
b
= −r + 70 2 + 140 r + r 2 − 2 b − 1 log(1 − u ),
where r = ab −1 . If u is an observation of a uniform pseudo-random variate,
then x is an observation from the required distribution.

\end{document}
