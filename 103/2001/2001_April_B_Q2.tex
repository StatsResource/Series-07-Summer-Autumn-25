\documentclass[a4paper,12pt]{article}

%%%%%%%%%%%%%%%%%%%%%%%%%%%%%%%%%%%%%%%%%%%%%%%%%%%%%%%%%%%%%%%%%%%%%%%%%%%%%%%%%%%%%%%%%%%%%%%%%%%%%%%%%%%%%%%%%%%%%%%%%%%%%%%%%%%%%%%%%%%%%%%%%%%%%%%%%%%%%%%%%%%%%%%%%%%%%%%%%%%%%%%%%%%%%%%%%%%%%%%%%%%%%%%%%%%%%%%%%%%%%%%%%%%%%%%%%%%%%%%%%%%%%%%%%%%%

\usepackage{eurosym}
\usepackage{vmargin}
\usepackage{amsmath}
\usepackage{graphics}
\usepackage{epsfig}
\usepackage{enumerate}
\usepackage{multicol}
\usepackage{subfigure}
\usepackage{fancyhdr}
\usepackage{listings}
\usepackage{framed}
\usepackage{graphicx}
\usepackage{amsmath}
\usepackage{chngpage}

%\usepackage{bigints}
\usepackage{vmargin}

% left top textwidth textheight headheight
% headsep footheight footskip
\setmargins{2.0cm}{2.5cm}{16 cm}{22cm}{0.5cm}{0cm}{1cm}{1cm}

\renewcommand{\baselinestretch}{1.3}

\setcounter{MaxMatrixCols}{10}

\begin{document}


%%%%%%%%%%%%%%%%%%%%%%%%%%%%%%%%%%%%%%%%%%%%%%%%%%%%%%%%%%%%%%%%%%%%%%%%
 An insurance company wishes to test the assumption that claims of a particular
type arrive according to a Poisson process model. The times of arrival of the next
20 incoming claims of this type are to be recorded, giving a sequence $T_1 , ..., T_{20}$ .
\begin{enumerate}[(a)]
\item (i) Give reasons why tests for the goodness of fit should be based on the inter-
arrival times X i = T i − T i−1 rather than on the arrival times T i .
\item (ii) Write down the distribution of the inter-arrival times if the Poisson process model is correct and state one statistical test which could be
applied to determine whether this distribution is realised in practice.
\item 
(iii) State the relationship between successive values of the inter-arrival times if the Poisson process model is correct and state one method which could
be applied to determine whether this relationship holds in practice.
\end{enumerate}

%%%%%%%%%%%%%%%%%%%%%%%%%%%%%%%%%%%%%%%%%%%%%%%%%%%%%%%%%%%%%%%%%%%%%%%%
\newpage


%%%%%%%%%%%%%%%%%%%%%%%%%%%%%%%%%%%%%%%%%%%%%%%%%%%%%%%%%%%%%%%%%%%%%%%%
2
\begin{itemize}
\item (i) The inter-arrival times are much more suitable because they are independent.
\item (ii) They should be exponentially distributed with the same mean.
Kolmogorov-Smirnov, Anderson-Darling or $\chi^2$ goodness-of-fit test can all be used.
\item (iii)
Successive values should be independent.
Regress X i on X i−1 using ordinary least squares, or fit an AR(1) and test the \alpha 1 parameter for significance (equivalent to Durbin-Watson test).
\end{itemize}

\end{document}
