\documentclass[a4paper,12pt]{article}

%%%%%%%%%%%%%%%%%%%%%%%%%%%%%%%%%%%%%%%%%%%%%%%%%%%%%%%%%%%%%%%%%%%%%%%%%%%%%%%%%%%%%%%%%%%%%%%%%%%%%%%%%%%%%%%%%%%%%%%%%%%%%%%%%%%%%%%%%%%%%%%%%%%%%%%%%%%%%%%%%%%%%%%%%%%%%%%%%%%%%%%%%%%%%%%%%%%%%%%%%%%%%%%%%%%%%%%%%%%%%%%%%%%%%%%%%%%%%%%%%%%%%%%%%%%%

\usepackage{eurosym}
\usepackage{vmargin}
\usepackage{amsmath}
\usepackage{graphics}
\usepackage{epsfig}
\usepackage{enumerate}
\usepackage{multicol}
\usepackage{subfigure}
\usepackage{fancyhdr}
\usepackage{listings}
\usepackage{framed}
\usepackage{graphicx}
\usepackage{amsmath}
\usepackage{chngpage}

%\usepackage{bigints}
\usepackage{vmargin}

% left top textwidth textheight headheight

% headsep footheight footskip

\setmargins{2.0cm}{2.5cm}{16 cm}{22cm}{0.5cm}{0cm}{1cm}{1cm}

\renewcommand{\baselinestretch}{1.3}

\setcounter{MaxMatrixCols}{10}

\begin{document}
\begin{enumerate}[(a)]
\item (i) State the expectation and variance of $dY t = Y t+h − Y t$ , the increment of the process Y t over a small interval of size h, conditional on Y t = y.

\item (ii) Calculate the expectation and variance of the increment X n+1 − X n of the autoregression, conditional on X n = y.
[2]
\item (iii) Find, by equating the first and second moments of the increments in (i) and (ii) above, an expression for the drift \mu(y) of the approximating diffusion in a form which does not involve the time increment h.
\item 
(iv) State a condition under which the approximating process in (iii) is a Brownian motion with drift.
\item 
(v) State a condition under which the approximating process in (iii) is an Ornstein-Uhlenbeck process.
\end{enumerate}
\begin{itemize}
\item (i) E ( dY t | Y t = y ) = \mu ( y ) h + o ( h ) , Var( dY t | Y t = y ) = h + o ( h )
\item (ii) \[E ( X n + 1 − X n | X n = y ) = \theta − \alpha y ,\] \[Var( X n + 1 − X n | X n = y ) = \tau^2 .\]
\item (iii) \mu(y)h = (\theta − \alpha\;y) and h = \tau^2 , so \mu(y) = (\theta − \alpha\;y) / \tau^2 .
\item (iv) The increments of a brownian motion do not depend on its current value,
i.e. \alpha = 0.
\item (v) An OU process drifts towards zero, so that \theta = 0.]
\end{itemize}

\end{document}
