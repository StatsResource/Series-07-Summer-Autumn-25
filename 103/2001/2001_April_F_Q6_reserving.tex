\documentclass[a4paper,12pt]{article}

%%%%%%%%%%%%%%%%%%%%%%%%%%%%%%%%%%%%%%%%%%%%%%%%%%%%%%%%%%%%%%%%%%%%%%%%%%%%%%%%%%%%%%%%%%%%%%%%%%%%%%%%%%%%%%%%%%%%%%%%%%%%%%%%%%%%%%%%%%%%%%%%%%%%%%%%%%%%%%%%%%%%%%%%%%%%%%%%%%%%%%%%%%%%%%%%%%%%%%%%%%%%%%%%%%%%%%%%%%%%%%%%%%%%%%%%%%%%%%%%%%%%%%%%%%%%

\usepackage{eurosym}
\usepackage{vmargin}
\usepackage{amsmath}
\usepackage{graphics}
\usepackage{epsfig}
\usepackage{enumerate}
\usepackage{multicol}
\usepackage{subfigure}
\usepackage{fancyhdr}
\usepackage{listings}
\usepackage{framed}
\usepackage{graphicx}
\usepackage{amsmath}
\usepackage{chngpage}

%\usepackage{bigints}
\usepackage{vmargin}

% left top textwidth textheight headheight

% headsep footheight footskip

\setmargins{2.0cm}{2.5cm}{16 cm}{22cm}{0.5cm}{0cm}{1cm}{1cm}

\renewcommand{\baselinestretch}{1.3}

\setcounter{MaxMatrixCols}{10}

\begin{document}
6
A motor insurance company has 80,000 policy holders, paying an average annual
premium of £400. The company receives claims at a rate of 2000 per month, the sizes of the claims having mean £1,200, standard deviation σ = £200.
Let $S(t)$ denote the company’s total surplus at time t, with $S(0)$ equal to the initial reserve, set at £20,000,000.

\begin{enumerate}
    \item 
(i) Calculate the expectation of the total amount paid out in claims in a given month and the safety loading employed by the company.
\item 
(ii) State, with reasons, whether it would be appropriate to use a diffusion approximation to calculate the probability of ruin, that is the probability
that the process ${S(t) : t \geq 0}$ ever hits 0.
\item 
(iii) Describe a simulation-based method for estimating the probability of ruin.
Indicate why it would be important to use a method of generating pseudo-random variables which gives rise to reproducible sequences.
\end{enumerate}
\newpage
%%%%%%%%%%%%%%%%%%%%%%%%

6
\begin{itemize}
    \item 

(i)
Expectation is £2.4m. Safety loading is
ρ =
C − \alpha\mu
(400 /12) × 80,000 − 2,000 × 1,200
=
= 11.11%
2,000 × 1,200
\alpha\mu
where \alpha denotes the mean arrival rate of claims and $\mu$ the mean claim
size.
\item (ii)
Conditions for validity of diffusion approximation are
u
\mu
large, ρ small,
ρ u \mu moderate, where u is the reserve.
In this case
ρ u \mu =
u
is large, ρ is not particularly small and
\mu
2 × 10 7 × 11.1 × 10 − 2
1,200
= 1,852, clearly too large. We conclude that the
diffusion approximation is not appropriate.
\item (iii)
Decide on a quantum of time, which may be a month or may be smaller.
For each time period generate a Poisson variate to indicate the number of
claims received and, conditional on this, a Normal variate with
appropriate mean and variance to represent the total sum claimed.
Subtract this from the total premium income over the period
(deterministic), using the resulting quantity as the increment of the
surplus process. Run the simulation for an extended period of time,
stopping if/when it goes below zero. A large number of simulations should
be performed, with the probability of ruin being estimated using standard
techniques based on the Binomial distribution.
The importance of reproducibility is for sensitivity analysis. The
estimated probability may depend heavily on the values assumed for
mean and standard deviation of the claim size, or on other numerical
parameters. It is necessary to vary the initial assumptions and run the
simulation again, just to ensure that conclusions are not substantially
changed if the parameter values used do not adequately reflect the actual
conditions experienced.
\end{itemize}
\end{document}
