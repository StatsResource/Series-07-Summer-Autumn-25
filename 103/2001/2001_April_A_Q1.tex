\documentclass[a4paper,12pt]{article}

%%%%%%%%%%%%%%%%%%%%%%%%%%%%%%%%%%%%%%%%%%%%%%%%%%%%%%%%%%%%%%%%%%%%%%%%%%%%%%%%%%%%%%%%%%%%%%%%%%%%%%%%%%%%%%%%%%%%%%%%%%%%%%%%%%%%%%%%%%%%%%%%%%%%%%%%%%%%%%%%%%%%%%%%%%%%%%%%%%%%%%%%%%%%%%%%%%%%%%%%%%%%%%%%%%%%%%%%%%%%%%%%%%%%%%%%%%%%%%%%%%%%%%%%%%%%

\usepackage{eurosym}
\usepackage{vmargin}
\usepackage{amsmath}
\usepackage{graphics}
\usepackage{epsfig}
\usepackage{enumerate}
\usepackage{multicol}
\usepackage{subfigure}
\usepackage{fancyhdr}
\usepackage{listings}
\usepackage{framed}
\usepackage{graphicx}
\usepackage{amsmath}
\usepackage{chngpage}

%\usepackage{bigints}
\usepackage{vmargin}

% left top textwidth textheight headheight
% headsep footheight footskip
\setmargins{2.0cm}{2.5cm}{16 cm}{22cm}{0.5cm}{0cm}{1cm}{1cm}

\renewcommand{\baselinestretch}{1.3}

\setcounter{MaxMatrixCols}{10}

\begin{document}
\begin{enumerate}
%%ã Institute of Actuaries1
\item ${N(t) : t \geq 0}$ is a Poisson process with rate $\lambda$ and {. t : t \geq 0} is the filtration
associated with N.
\begin{enumerate}[(i)]
\item(i)
Write down the conditional distribution of N(t + s) − N(t) given . t , where
s > 0 and use your answer to find E(θ N(t+s) . t ).
\item (ii)
2
3
Find a process of the form M(t) = η(t)θ N(t) which is a martingale.
\end{enumerate}

%%%%%%%%%%%%%%%%%%%%%%%%%%%%%%%%%%%%%%%%%%%%%%%%%%%%%%%%%%%%%%%%%%%%%%%%
\newpage

1
(i)
Given . t we know that N(t + s) − N(t) ~ Poisson(\lambdas).
Hence E(θ N(t+s) . t ) = θ N(t) e (θ−1)\lambdas .
(ii)
Now E(η(t + s) θ N(t+s) . t ) = η(t + s) θ N(t) e (θ−1)\lambdas , which needs to be equal to
M(t) = η(t) θ N(t) . It follows that η(t) = e −(θ−1)\lambdat .

\end{document}
