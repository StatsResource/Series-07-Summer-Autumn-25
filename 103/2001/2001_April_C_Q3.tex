\documentclass[a4paper,12pt]{article}

%%%%%%%%%%%%%%%%%%%%%%%%%%%%%%%%%%%%%%%%%%%%%%%%%%%%%%%%%%%%%%%%%%%%%%%%%%%%%%%%%%%%%%%%%%%%%%%%%%%%%%%%%%%%%%%%%%%%%%%%%%%%%%%%%%%%%%%%%%%%%%%%%%%%%%%%%%%%%%%%%%%%%%%%%%%%%%%%%%%%%%%%%%%%%%%%%%%%%%%%%%%%%%%%%%%%%%%%%%%%%%%%%%%%%%%%%%%%%%%%%%%%%%%%%%%%

\usepackage{eurosym}
\usepackage{vmargin}
\usepackage{amsmath}
\usepackage{graphics}
\usepackage{epsfig}
\usepackage{enumerate}
\usepackage{multicol}
\usepackage{subfigure}
\usepackage{fancyhdr}
\usepackage{listings}
\usepackage{framed}
\usepackage{graphicx}
\usepackage{amsmath}
\usepackage{chngpage}

%\usepackage{bigints}
\usepackage{vmargin}

% left top textwidth textheight headheight
% headsep footheight footskip

\setmargins{2.0cm}{2.5cm}{16 cm}{22cm}{0.5cm}{0cm}{1cm}{1cm}
\renewcommand{\baselinestretch}{1.3}

\setcounter{MaxMatrixCols}{10}

\begin{document}
\begin{enumerate}
%%-- [2]
\item [Total 5]
\begin{enumerate}[(i)]
\item (i) Give a definition of the spectral density of a stationary time series, expressed in terms of the autocovariance function {γ k : k ∈ Z} of the
process. Use this definition to derive the spectral density of a first-order moving average process and of a first-order autoregression.

\item (ii) Suppose the “inverse” of a time series model with spectral density f(ω) is 1
. Using part (i), state
defined to be the model with spectral density
f ( ω )
the form of the inverse of a first-order moving average and state the way in which the inverse of an invertible MA(1) differs from the inverse of a non-invertible MA(1).
[2]
\end{enumerate}
%-- [Total 7]
%%%%%%%%%%%%%%%%%%%%%%%%%%%%%%%%
%-- 103 A2001—24
\item Let $X$ n denote an autoregressive high frequency time series modelled by:
\[X n+1 = (1 − \alpha) X n + θ + τe n ,\]
where $e_n = \pm 1$ with equal probabilities and \alpha, θ, τ are constant parameters. An analyst wishes to investigate whether this series may be approximated by some continuous time diffusion, i.e. X n ≈ Y nh , where $Y_t$ satisfies a stochastic differential
equation
\[dY t = \mu(Y t ) dt + dB t\]
and B t denotes standard Brownian motion.

%%%%%%%%%%%%%%%%%%%%%%%%%%%%%%%%%%%%%%%%%%%%%%%5

\newpage

3
\begin{enumerate}
\item (i)
Spectral density f(ω) =
1
2 π
Σ ∞−∞ γ k e ik ω , or equivalent.
For MA(1), therefore, we have
f(ω) =
σ 2 e
(1 + β 2 + 2β cos ω).
2 π
And for AR(1),
f(ω) =
(ii)
σ 2 e
1
.
2
2 π 1 + α − 2 α cos ω
\item Clearly from (i) the inverse of the MA(1) is an AR(1), with α = −β and with
a different value of σ 2 e .
The word “invertible” attached to a MA(1) indicates that the inverse is a
stationary AR(1), whereas a non-“invertible” MA(1) has as inverse an
AR(1) model which cannot be stationary, such as X t = 2X t − 1 + e t .
\end{itemize}
\end{document}
