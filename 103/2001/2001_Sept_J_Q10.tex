\documentclass[a4paper,12pt]{article}

%%%%%%%%%%%%%%%%%%%%%%%%%%%%%%%%%%%%%%%%%%%%%%%%%%%%%%%%%%%%%%%%%%%%%%%%%%%%%%%%%%%%%%%%%%%%%%%%%%%%%%%%%%%%%%%%%%%%%%%%%%%%%%%%%%%%%%%%%%%%%%%%%%%%%%%%%%%%%%%%%%%%%%%%%%%%%%%%%%%%%%%%%%%%%%%%%%%%%%%%%%%%%%%%%%%%%%%%%%%%%%%%%%%%%%%%%%%%%%%%%%%%%%%%%%%%

\usepackage{eurosym}
\usepackage{vmargin}
\usepackage{amsmath}
\usepackage{graphics}
\usepackage{epsfig}
\usepackage{enumerate}
\usepackage{multicol}
\usepackage{subfigure}
\usepackage{fancyhdr}
\usepackage{listings}
\usepackage{framed}
\usepackage{graphicx}
\usepackage{amsmath}
\usepackage{chngpage}

%\usepackage{bigints}
\usepackage{vmargin}

% left top textwidth textheight headheight

% headsep footheight footskip

\setmargins{2.0cm}{2.5cm}{16 cm}{22cm}{0.5cm}{0cm}{1cm}{1cm}

\renewcommand{\baselinestretch}{1.3}

\setcounter{MaxMatrixCols}{10}

\begin{document}
\begin{enumerate} For a given driver, any period j is either accident free (Y j = 0) or gives rise to one
accident (Y j = 1). The probability of having no accident during the next period is
estimated using the driver’s past record as follows (all values y j are either 0 or 1):
P[Y n+1 = 0Y 1 = y 1 , Y 2 = y 2 , ..., Y n = y n ] = pe −\lambda ( y 1 + y 2 + ... + y n ) ,
where 0 < p < 1, \lambda ≥ 0. The cumulative number of accidents suffered by the
driver over the time from period 1 up to period n is
X n =
n
å Y .
j
j = 1

\begin{enumerate}[(a)]

\item (i) Verify that the Markov property holds for the sequence $X 1 , X 2 , ..., X n , ...$
and explain why the sequence Y 1 , Y 2 , ..., Y n , ... does not form a Markov
chain.
\item 
(ii) Draw the transition graph of the Markov chain X and write down its
transition matrix.

\item (iii) Determine, being careful to explain your reasons in each case:
(a)
(b)
(c)
(iv)
(v)
[4]
whether the Markov chain X is time-homogeneous
whether it is irreducible
whether it admits a stationary probability distribution 

\item Starting from the state X t = j, calculate the probability of suffering no
further accident for the next n successive periods. 
Suppose you are provided with full claims records for a number of a
company’s policy holders.
(a) Describe a method for estimating the parameters \lambda and p.
(b) Explain how to test the assumption that the probability of an accident depends only on the cumulative number of accidents, X n ,
and does not have a direct dependence on n.

\end{enumerate}
%%%%%%%%%%%%%%%%%%%%%%%%%%%%%%%
\newpage
%%%%%%%%%%%%%%%%%%%%%%%%%%%%%%%%%%%%%%%%%%%%%%%%%%%%%%%%%%%%%%%%%%%%%%%%%%%%%
10
(i)
\begin{itemize}
\item There is an explicit dependence on the past behaviour of Y j , j \leq n in the
probability distribution of Y n+1 ; hence the Markov property does not hold.
On the other hand
P[X n+1 = j1⁄2X 1 = i 1 , X 2 = i 2 , ..., X n - 1 = i n - 1 , X n = i]
= P[Y n+1 = j - i1⁄2Y 1 = i 1 , Y 2 = i 2 - i 1 , ..., Y n - 1 = i n - 1 - i n - 2 , Y n = i - i n - 1 ]
ì ï pe -l i
= í
-l i
ï
î 1 - pe
if j - i = 0,
if j - i = 1.
This is independent of i 1 , i 2 , ..., i n - 1 .
\item (ii)
Transition graph
1 - p
1 - pe -l
1 - pe -2l
0 1 2
p pe -l pe -2l
3
...
Transition matrix:
æ p 1 - p
ö
ç
÷
-l
-l
pe
1 - pe
0
ç
÷ .
ç
pe - 2 l
1 - pe - 2 l ÷
ç ç
÷ ÷
O
O
è 0
ø
Page 7Subject 103 (Stochastic Modelling) — 
%%%%%%%%%%%%%%%%%%%%%%%%%%%%%%%%

\item (iii)
(a) Chain is time-homogeneous since transition probabilities calculated in (i) do not depend on time n.
(b) It is not irreducible since the number of accidents can never go down.
(c) There are no recurrent states, hence there can be no stationary distribution. Alternatively, a stationary distribution, p, if it exists,
must obey
p 0 p = p 0
p 0 (1 - p) + p 1 pe -l = p 1
p 1 (1 - pe -l ) + p 2 pe - 2 l = p 2 .
M
Since p < 1 we have p 0 = 0, and then p 1 = 0 etc. Hence no
stationary probability distribution exists.
(iv)
No new accident:
(pe - j l ) n = p n e - nj l
(v)
(a) Maximum likelihood would be very easy in this case: choose l and
p to maximise P {( pe -l x k ) 1 - y k (1 - pe -l x k ) y k }.
(b) Change the model to:
P[Y n +1 = 01⁄2Y 1 = y 1 , Y 2 = y 2 , ..., Y n = y n ] = pe -l ( x n , n ) ,
then test the hypothesis that l(x, n) o l(x) for all n.
\end{itemize}
\end{document}
