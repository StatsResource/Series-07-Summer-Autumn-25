
\documentclass[a4paper,12pt]{article}

%%%%%%%%%%%%%%%%%%%%%%%%%%%%%%%%%%%%%%%%%%%%%%%%%%%%%%%%%%%%%%%%%%%%%%%%%%%%%%%%%%%%%%%%%%%%%%%%%%%%%%%%%%%%%%%%%%%%%%%%%%%%%%%%%%%%%%%%%%%%%%%%%%%%%%%%%%%%%%%%%%%%%%%%%%%%%%%%%%%%%%%%%%%%%%%%%%%%%%%%%%%%%%%%%%%%%%%%%%%%%%%%%%%%%%%%%%%%%%%%%%%%%%%%%%%%

\usepackage{eurosym}
\usepackage{vmargin}
\usepackage{amsmath}
\usepackage{graphics}
\usepackage{epsfig}
\usepackage{enumerate}
\usepackage{multicol}
\usepackage{subfigure}
\usepackage{fancyhdr}
\usepackage{listings}
\usepackage{framed}
\usepackage{graphicx}
\usepackage{amsmath}
\usepackage{chngpage}

%\usepackage{bigints}
\usepackage{vmargin}

% left top textwidth textheight headheight

% headsep footheight footskip

\setmargins{2.0cm}{2.5cm}{16 cm}{22cm}{0.5cm}{0cm}{1cm}{1cm}

\renewcommand{\baselinestretch}{1.3}

\setcounter{MaxMatrixCols}{10}

\begin{document}

%%%%%%%%%%%%%%%%%%%%%%%%%%%%%%%%%%%%%%
%%103 A2001—48
%%--Question 8 
\begin{enumerate}

\item A company assesses the credit-worthiness of various firms every quarter; the ratings are, in order of decreasing merit, A, B, C and D (default). Historical data support the view that the credit rating of a typical firm evolves as a Markov
chain with transition matrix
æ 1 − \alpha − \alpha 2
\alpha
\alpha 2
ç
\alpha
1 − 2 \alpha − \alpha 2
\alpha
P = ç
2
ç
\alpha
\alpha
1 − 2 \alpha − \alpha 2
ç ç
0
0
0
è
0 ö
÷
\alpha 2 ÷
\alpha ÷
÷
1 ÷ ø
for some parameter \alpha
\end{itemize}

\begin{enumerate}[(i)]
\item (i) Draw the transition graph of the chain. [2]
\item (ii) Determine the range of values of \alpha for which the matrix P is a valid
transition matrix. [2]
\item (iii) State, with reasons, whether the chain is irreducible and aperiodic.
\item (iv) Derive a stationary probability distribution for the chain and establish
whether it is unique.
[4]
\item (v) For the value \alpha = 0.1, calculate the probability that the company’s rating
in the third quarter, X 3 , is in the default state D:
[2]
(a) in the case where the company’s rating in the first quarter, X 1 , is
equal to A
(b) in the case X 1 = B
(c) in the case X 1 = C
(d) in the case X 1 = D
\end{enumerate}
\newpage


8
(i)
Transition Graph
1 − \alpha − \alpha 2
1 − 2\alpha − \alpha 2
\alpha
A
B
\alpha
\alpha 2
\alpha 2
\alpha 2
\alpha
\alpha
C
D
\alpha
1
1 − 2\alpha − \alpha 2
(ii) All transition probabilities must lie in [0,1].
Now 1 − 2\alpha − \alpha 2 ≤ 1 − \alpha − \alpha 2 ≤ 1 for \alpha \geq 0, so it suffices to ensure that 1 − 2\alpha − \alpha 2 \geq 0 i.e. \alpha ≤ 2 − 1. So the range of possible values of
\alpha is [0, 2 − 1].
(iii) The chain is not irreducible since D is a trap state.
The chain is aperiodic by inspection.
(iv) A stationary probability distribution, if it exists, must obey
(1 − \alpha − \alpha 2 ) \pi A + \alpha\pi B + \alpha 2 \pi C
\alpha\pi A + (1 − 2\alpha − \alpha 2 ) \pi B + \alpha\pi C
\alpha 2 \pi A + \alpha\pi B + (1 − 2\alpha − \alpha 2 ) \pi C
\alpha 2 \pi B + \alpha\pi C + \pi D
=
=
=
=
\pi A
\pi B
\pi C
\pi D
The last equation implies \pi B = \pi C = 0, and this in turn shows that \pi A = 0.
Hence the stationary probability distribution is \pi = (0, 0, 0, 1) T .
%%Page 5
%%Subject 103 (Stochastic Modelling) — April 2001 — Examiners’ Report
It is unique: there is just one recurrent class and it is aperiodic. (Or point out that there is no other solutions to the equations.)
(v)
With \alpha = 0.1, the transition matrix is
0 ö
æ 0.89 0.1 0.01
ç
÷
ç 0.1 0.79 0.1 0.01 ÷
ç 0.01 0.1 0.79 0.1 ÷
ç ç
÷
0
0
1 ÷ ø
è 0
Its square is
æ 0.8022 0.169 0.0268 0.002 ö
ç
÷
ç 0.169 0.6441 0.159 0.0279 ÷
ç 0.0268 0.159 0.6342 0.18 ÷
ç ç
÷
0
0
1 ÷ ø
è 0
the relevant entries being the last column.

\end{document}
