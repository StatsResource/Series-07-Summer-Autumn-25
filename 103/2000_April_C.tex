
\documentclass[a4paper,12pt]{article}

%%%%%%%%%%%%%%%%%%%%%%%%%%%%%%%%%%%%%%%%%%%%%%%%%%%%%%%%%%%%%%%%%%%%%%%%%%%%%%%%%%%%%%%%%%%%%%%%%%%%%%%%%%%%%%%%%%%%%%%%%%%%%%%%%%%%%%%%%%%%%%%%%%%%%%%%%%%%%%%%%%%%%%%%%%%%%%%%%%%%%%%%%%%%%%%%%%%%%%%%%%%%%%%%%%%%%%%%%%%%%%%%%%%%%%%%%%%%%%%%%%%%%%%%%%%%

\usepackage{eurosym}
\usepackage{vmargin}
\usepackage{amsmath}
\usepackage{graphics}
\usepackage{epsfig}
\usepackage{enumerate}
\usepackage{multicol}
\usepackage{subfigure}
\usepackage{fancyhdr}
\usepackage{listings}
\usepackage{framed}
\usepackage{graphicx}
\usepackage{amsmath}
\usepackage{chngpage}

%\usepackage{bigints}
\usepackage{vmargin}

% left top textwidth textheight headheight

% headsep footheight footskip

\setmargins{2.0cm}{2.5cm}{16 cm}{22cm}{0.5cm}{0cm}{1cm}{1cm}

\renewcommand{\baselinestretch}{1.3}

\setcounter{MaxMatrixCols}{10}

\begin{document}
\begin{enumerate}
%%%%%%%%%%%%%%%%%%%%%
5
Let B t be a standard Brownian motion, and let F t = σ(B s , 0 ≤ s ≤ t) be its natural
filtration.
(i)
Derive the conditional expectations E[ B t 2 F s ] and E[ B t 4 F s ] , where s ≤ t.
You may assume that the fourth moment of a random variable with
distribution N(0, σ 2 ) is 3σ 4 .
%%%%%%%%%%%%%%%%%%%%%%%%%%%%%%%%%%%%%%%%%%%%%%%
6
(ii) Hence construct a martingale out of B t 4 .
(i) Apply the inverse transform method to generate an observation from
the density
f 1 (x) =
1
( 1 + x ) 2
[4]
[3]
[Total 7]
(x > 0)
using a pseudo-random number u in the range 0 < u < 1. Explain how
this can be extended to generate an observation from the symmetrised
form of the same density
f 2 (x) =
(ii)
1
2 ( 1 + x  ) 2
(x ∈ R)
[3]
The Cauchy distribution has density function
f(xθ) =
θ
π ( x + θ 2 )
2
(x ∈ R)
where θ is a positive parameter.
Show that
f(xθ) ≤ Cf 2 (x)
for all x ∈ R
2
(θ + θ −1 ). Hence devise a method based on Acceptance-
π
Rejection sampling for generating observations from the Cauchy
distribution.
[4]
[Total 7]
as long as C ≥
103—3
%%%%%%%%%%%%%%%%%%%%%%%%%%%%%%%%%%%%%%%%%%%%%%%%%%%%%%%%%%%%%%%%%%%%%%
Page 3Subject 103 (Stochastic Modelling) — April 2000 — Examiners’ Report
5
(i)
E[ B t 2 1⁄2F s ] = E[(B t - B s + B s ) 2 1⁄2F s ]
= E[(B t - B s ) 2 + 2(B t -B s ) B s + B s 2 1⁄2F s ]
= E[(B t - B s ) 2 1⁄2F s ] + 2B s E[B t - B s 1⁄2F s ] + B s 2 ,
by the property of conditional expectations which allows one to “take out
what is known”. Moreover, by independence of the increments, the above is
E[(B t - B s ) 2 ] + B s 2 = t - s + B s 2 .
Similarly,
E[ B t 4 1⁄2F s ] = E[(B t - B s + B s ) 4 1⁄2F s ]
= E[(B t - B s ) 4 + 4(B t - B s ) 3 + 6(B t - B s ) 2 B s 2 + 4(B t - B s ) B s B s 3 + B s 4 1⁄2F s ]
= E[(B t - B s ) 4 ] + 6 B s 2 E[(B t - B s ) 2 ] + B s 4 ,
where we used the independence of increments property as well as the fact
that moments of odd order of N(0, s 2 ) vanish. Finally
E[ B t 4 1⁄2F s ] = B s 4 + 6(t - s) B s 2 + 3(t - s) 2 .
(ii)
From above
E[ B t 4 - 6 tB t 2 1⁄2F s ] = B s 4 + 6(t - s) B s 2 + 3(t - s) 2 - 6t(t - s + B s 2 )
= B s 4 + 6 sB s 2 + 3(t - s) 2 - 6t(t - s) = B s 4 - 6 sB s 2 + 3(s 2 - t 2 )
\ B t 4 - 6 tB t 2 + 3 t 2 is a martingale.
6
(i)
u = F 1 (x) =
x
u
is solved by x = F 1 - 1 ( u ) =
.
1 + x
1 - u
For the symmetrised version, the simplest thing is to multiply x by a
variable y which takes ±1 depending on whether another pseudo-random
uniform number v is in the range (0, 0.5) or (0.5, 1).
(ii)
Page 4
2 q ( 1 + x ) 2
By symmetry we only need consider x > 0, so we find max x>0
.
p ( q 2 + x 2 )
Differentiating the logarithm of this fraction and setting equal to 0, we get
2 x
2
, with solution x = q 2 . Substituting this value in, we obtain
= 2
1 + x
q + x 2
the required value of C.Subject 103 (Stochastic Modelling) — April 2000 — Examiners’ Report
f ( x 1⁄2q )
, which we observe is less than or equal to 1 everywhere.
Cf 2 ( x )
The method of Acceptance-Rejection sampling goes as follows: use (i) to
generate a variable y from density f 2 . We accept y as a valid observation
from f(x1⁄2q) with probability g(y), otherwise reject it. (Generate a uniform
variable u, and reject if u > g(y).) If we reject it, go back and generate
another y from f 2 , and continue to do the same until eventual acceptance.
Let g(x) =
\end{document}
