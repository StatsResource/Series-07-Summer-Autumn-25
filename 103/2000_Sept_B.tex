
\documentclass[a4paper,12pt]{article}

%%%%%%%%%%%%%%%%%%%%%%%%%%%%%%%%%%%%%%%%%%%%%%%%%%%%%%%%%%%%%%%%%%%%%%%%%%%%%%%%%%%%%%%%%%%%%%%%%%%%%%%%%%%%%%%%%%%%%%%%%%%%%%%%%%%%%%%%%%%%%%%%%%%%%%%%%%%%%%%%%%%%%%%%%%%%%%%%%%%%%%%%%%%%%%%%%%%%%%%%%%%%%%%%%%%%%%%%%%%%%%%%%%%%%%%%%%%%%%%%%%%%%%%%%%%%

\usepackage{eurosym}
\usepackage{vmargin}
\usepackage{amsmath}
\usepackage{graphics}
\usepackage{epsfig}
\usepackage{enumerate}
\usepackage{multicol}
\usepackage{subfigure}
\usepackage{fancyhdr}
\usepackage{listings}
\usepackage{framed}
\usepackage{graphicx}
\usepackage{amsmath}
\usepackage{chngpage}

%\usepackage{bigints}
\usepackage{vmargin}

% left top textwidth textheight headheight

% headsep footheight footskip

\setmargins{2.0cm}{2.5cm}{16 cm}{22cm}{0.5cm}{0cm}{1cm}{1cm}

\renewcommand{\baselinestretch}{1.3}

\setcounter{MaxMatrixCols}{10}

\begin{document}

During a long motorway journey a child amuses himself by noting down, at the
end of each minute, the lane in which the car is travelling. The motorway has
three lanes and the journey lasts N minutes.
(i) Describe how to fit a three-state time-homogeneous Markov chain model
to the data, writing down formulae for the estimates of the transition
probabilities.
[2]
(ii) Describe one test which could be applied to determine whether the process
possesses the Markov property. [2]
[Total 4]
A standard Ornstein-Uhlenbeck process may be defined as a stationary zero-
mean Gaussian process {U t : t ∈ R} with autocovariance function given by
Cov(U t , U s ) =
τ 2 − θ  t − s 
e
2 θ
(s, t ∈ R).
(i) Show that the process {U n : n = 1, 2, ...}, obtained by observing U only at
integer times, is a first-order autoregression.
[2]
(ii) Derive expressions for the parameters α and σ 2 of the autoregression in
terms of θ and τ 2 .
[2]
[Total 4]
103—244

3
(i) Let N ij denote the number of minutes when the car was in lane i at the
start of the minute and in lane j at the end. The estimate of the
transition probability p ij is N ij / N i+ , where N i+ = Σ j N ij .
(ii) The problem here is the alternative hypothesis. It would be possible to
test whether the distribution of X n+1 conditional on X n = i was really
independent of X n−1 . Or one might test, using a standard goodness-of-fit
test, whether the distribution of the number of consecutive minutes spent
in lane i really was geometrically distributed with parameter determined
by i.
(i) Since {U n } is Gaussian and stationary, it is determined uniquely by its
mean and autocovariance functions. For k > 0 we have γ k = Cov(U n , U n−k )
=
(ii)
τ 2 −θk
τ 2
e , so that the ACF is ρ k = e −θk and the variance γ 0 =
.
2 θ
2 θ
Compare this with the corresponding values for an AR(1): γ 0 =
σ 2
and
1 − α 2
ρ k = α k for k > 0.
The two are seen to match as long as α = e −θ and σ 2 = (1 − e − 2 θ )
Page 2
τ 2
.
2 θSubject 103 (Stochastic Modelling) — September 2000 — Examiners’ Report
4
(i)
If X satisfies dX t = Y t dB t + Z t dt, then f(X t ) satisfies
d[f(X t )] = f ′(X t ) Y t dB t + { f ′(X t ) Z t + 1⁄2 f ′′(X t ) Y t 2 } dt.
(ii)
(x 4 )′ = 4x 3 , (x 4 )′′ = 12x 2 .
d ( B t 4 ) = 4 B t 3 dB t +
(iii)
t
ò 0
. 12 B t 2 dt = 4 B t 3 dB t + 6 B t 2 dt .
d ( B s 4 ) = 4 ò t 0 B s 3 dB s + 6 ò t 0 B s 2 ds .
Therefore\end{document}
