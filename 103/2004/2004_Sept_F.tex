%%- Subject 103 (Stochastic Modelling) September 2004 Examiners Report\documentclass[a4paper,12pt]{article}

%%%%%%%%%%%%%%%%%%%%%%%%%%%%%%%%%%%%%%%%%%%%%%%%%%%%%%%%%%%%%%%%%%%%%%%%%%%%%%%%%%%%%%%%%%%%%%%%%%%%%%%%%%%%%%%%%%%%%%%%%%%%%%%%%%%%%%%%%%%%%%%%%%%%%%%%%%%%%%%%%%%%%%%%%%%%%%%%%%%%%%%%%%%%%%%%%%%%%%%%%%%%%%%%%%%%%%%%%%%%%%%%%%%%%%%%%%%%%%%%%%%%%%%%%%%%

\usepackage{eurosym}
\usepackage{vmargin}
\usepackage{amsmath}
\usepackage{graphics}
\usepackage{epsfig}
\usepackage{enumerate}
\usepackage{multicol}
\usepackage{subfigure}
\usepackage{fancyhdr}
\usepackage{listings}
\usepackage{framed}
\usepackage{graphicx}
\usepackage{amsmath}
\usepackage{chngpage}

%\usepackage{bigints}
\usepackage{vmargin}

% left top textwidth textheight headheight

% headsep footheight footskip

\setmargins{2.0cm}{2.5cm}{16 cm}{22cm}{0.5cm}{0cm}{1cm}{1cm}

\renewcommand{\baselinestretch}{1.3}

\setcounter{MaxMatrixCols}{10}

\begin{document}
\begin{enumerate}
\item 
%%--10 
Assume that the spot rate of interest at time t , S(t) , can be modelled by
S(t) = e 2 W(t ) , where W(t) is a Wiener process with drift coefficient and
diffusion coefficient 1 such that W(0) = 0.
(i) Write down an expression for W(t) in terms of a standard Brownian motion
B(t) . [1]
(ii) Show that {S(t) t 0} is a continuous-time martingale. [4]
(iii) Let Ta = inf{t S(t) = a} for some 0 a 1.
(a) Prove that P[S(t) a] 1 as t .
(b) Deduce that P[Ta ] = 1 and that E[S(Ta )] = a .
(c) Explain why the fact that E[S(Ta )] S(0) does not contradict the
optional stopping theorem.
[5]
(iv) Now suppose instead that a 1 and define Ta as before.
(a) Explain why E[S(Ta )] = aP[Ta ] .
(b) Apply the optional stopping theorem to find P Ta .
[4]
[Total 14]
%%%%%%%%%%%%%%%%%%%%%%%%%%%%%%%%%%%%%%%
%% - solution
%% - Subject 103 (Stochastic Modelling) September 2004 Examiners Report
10 (i) The Wiener process can be defined as Wt t Bt . In this case 1.
(ii) We need to show that

Page 12
E St Ss Ss 0 s t
as well as proving that E[ St ] .
But W(t) N( t t) , so that
2 2 ( ) x tx
Mt x e
We have
2 W(t )
E St Ss E e Ss
2 W(t) W(s) W(s)
E e Ss
e 2 W(s)E e 2 (W(t s)) (1)
using stationarity and independence of the increments.
From the definition of a Wiener process with drift above, we have
W(t s) N( (t s) (t s)) (2)
so
2 (W(t s)) ( 2 )
E e Mt s
where Mt s is the moment generating function of the normal distribution in
(2). But for this distribution we know that
( ) ( ) 2 2 ( ) t s x t s x
Mt s x e .
Therefore
2 2 ( ) ( )(2 )2 2 ( 2 ) t s t s 1
Mt s e
now (1) shows that E St Ss Ss as required.
Finally, check the expectation: St 0 , so
2 ( ) ( 2 ) ( 2 )2 2 [ ] [ ] [ W t ] ( 2 ) t t
E St E St E e Mt e
(iii) (a) P[S(t) a] P[e 2 (B(t) t) a] P[ 2 B(t) b 2 2t] , where
b = log a. If 0 , this becomes
Subject 103 (Stochastic Modelling) September 2004 Examiners Report
Page 13
( )
2 2
b b
P B t t t
t
whereas in the case 0 we have
( )
2 2
b b
P B t t t
t
and in both cases the RHS tends to 1 as t .
(b) P[Ta t] P[S(t) a] 0 as t .
By definition, S(Ta ) a . Therefore E[S(Ta )] a .
(c) This is not a contradiction because the conditions of the optional
stopping theorem are not satisfied. Neither S(t) nor Ta is bounded
above, even though S(t) is a convergent martingale.
(iv) (a) In this case Ta is only finite if S(t) hits a , which is not certain.
However, as above it is certain that S(t) 0 almost surely.
Therefore
if
( )
0 if
a
a
a
a T
S T
T
It follows that E[S(Ta )] aP[Ta ] .
(b) Now the optional stopping theorem applies, since S(t Ta ) is bounded
below by 0 and above by a .
We may deduce that
1
1 S(0) aP[Ta ] i e that P[Ta ]
a
In part (ii) a large number of candidates did not even attempt to prove that (| |) t E S : this condition is a requirement for S to be a martingale and should not be omitted. However, most candidates had a good idea of how to prove that S satisfied the conditional expectation condition.

Parts (iii) and (iv) attracted at best sketchy answers. The examiners were unsure whether this was due to pressure of time or to lack of familiarity with applications of the optional stopping theorem.
\end{document}
