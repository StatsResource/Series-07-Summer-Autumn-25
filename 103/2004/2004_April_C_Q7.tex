\documentclass[a4paper,12pt]{article}

%%%%%%%%%%%%%%%%%%%%%%%%%%%%%%%%%%%%%%%%%%%%%%%%%%%%%%%%%%%%%%%%%%%%%%%%%%%%%%%%%%%%%%%%%%%%%%%%%%%%%%%%%%%%%%%%%%%%%%%%%%%%%%%%%%%%%%%%%%%%%%%%%%%%%%%%%%%%%%%%%%%%%%%%%%%%%%%%%%%%%%%%%%%%%%%%%%%%%%%%%%%%%%%%%%%%%%%%%%%%%%%%%%%%%%%%%%%%%%%%%%%%%%%%%%%%

\usepackage{eurosym}
\usepackage{vmargin}
\usepackage{amsmath}
\usepackage{graphics}
\usepackage{epsfig}
\usepackage{enumerate}
\usepackage{multicol}
\usepackage{subfigure}
\usepackage{fancyhdr}
\usepackage{listings}
\usepackage{framed}
\usepackage{graphicx}
\usepackage{amsmath}
\usepackage{chngpage}

%\usepackage{bigints}
\usepackage{vmargin}

% left top textwidth textheight headheight

% headsep footheight footskip

\setmargins{2.0cm}{2.5cm}{16 cm}{22cm}{0.5cm}{0cm}{1cm}{1cm}

\renewcommand{\baselinestretch}{1.3}

\setcounter{MaxMatrixCols}{10}

\begin{document}
\begin{enumerate}

%%%%%%%%%%%%%%%%%%%%%%%%%%%%%%%%%%%%%%%%%%%%%%%%%%%%%%%%%%%%%%%%%%%%%%%%%%%%%%%%%%%%%%%%
7 (i) Suppose that {Xt : t 0} is a time-homogeneous Markov jump process with
generator matrix A and transition matrix P(t). Let be a (column) vector of
probabilities such that TA = 0T, where 0 is a (column) vector whose
components are all equal to zero.
(a) Prove that
T ( ) = T. d
P t
dt
0
(b) Explain why is known as the stationary distribution for X.
%------------------------%
(ii) A telephone call centre receives calls from customers at an average rate of 0.5
per minute. Each call has a random duration which is exponentially distributed with mean 3 minutes, independently of the number or duration of
any other calls. Two operators are assigned to handle calls. If a call arrives when both operators are busy, the call is put on hold unless there are already
two calls on hold, in which case the new call is lost. When a call ends, one of the calls on hold is immediately put through to the newly free operator.
(a) Identify the five states which are required if this system is to be
modelled as a Markov jump process.
(b) Draw the transition diagram for this system.
(c) Write down the generator matrix for the process.
(d) Evaluate the stationary distribution of the system.

\newpage
%%%%%%%%%%%%%%%%%%%%%%%%%%%%%%%%%%%%%%%%%%%%%%%%5
7 (i) (a) We will use the matrix form of the Kolmogorov DE, which states that
P'(t) AP(t) .
It follows that T ( ) T T ( ) 0T ( ) 0T. d dP
P t AP t P t
dt dt
[The other Kolmogorov DE, P'(t) P(t)A, cannot be used to complete the proof]
(b) If X0 is random with distribution , then the distribution of Xt is given by TP(t).
The fact that the differential of this is equal to zero implies that TP(t)
= TP(0) = T for all t, in other words Xt has the same distribution for all t.
For an irreducible Markov chain on a finite state space, the stationary distribution is also the limiting (equilibrium, long-term, steady state)
distribution.
(ii) (a) The states can be labelled as 0, 1, 2:0, 2:1, 2:2, where 0 and 1 represent
the number of operators occupied and 2:j means that both operators are occupied and j calls are on hold.
Candidates with a different collection of states can earn 0.5 marks, as long as their states are states. For example, Call arrives is not a
state, but an event triggering a transition from one state to another.

(b)
The labels on the arcs represent the transition rates.
(c) From the transition diagram, the generator matrix is as below.
0 1 2:0 2:1 2:2
1/2 1/2 1/2 1/2
1/3 2/3 2/3 2/3
Subject 103 (Stochastic Modelling) April 2004 Examiners Report
Page 9
1/ 2 1/ 2 0 0 0
1/ 3 5 / 6 1/ 2 0 0
0 2 / 3 7 / 6 1/ 2 0
0 0 2 / 3 7 / 6 1/ 2
0 0 0 2 / 3 2 / 3
A
(d) We have
0 1
1 1
2 3 1 0
3
2
1 0 2:0
5 1 2
6 2 3 2:0 0
9
8
2:0 1 2:1
7 1 2
6 2 3 2:1 0
27
32
2:1 2:0 2:2
7 1 2
6 2 3 2:2 0
81
128
To solve this we add the condition that the i must sum to 1.
Therefore the stationary distribution is
128 192 144 108 81
653 653 653 653 653
(in decimal form this is
0.1960 0.2940 0.2205 0.1654 0.1240 )
%%%%%%%%%%%%%%%%%%%%%%%%%%%%%%%%%%%%%%%%%%%%%%%%%%%%%%%5
\newpage
Many candidates made a good attempt at part (i), though not many scored full marks. In part
(ii) a surprising number of candidates were unable to distinguish between states in which
the process stays for a certain length of time and events, which occur instantaneously and
trigger transitions from one state to another.
\end{document}
