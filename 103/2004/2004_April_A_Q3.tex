\documentclass[a4paper,12pt]{article}

%%%%%%%%%%%%%%%%%%%%%%%%%%%%%%%%%%%%%%%%%%%%%%%%%%%%%%%%%%%%%%%%%%%%%%%%%%%%%%%%%%%%%%%%%%%%%%%%%%%%%%%%%%%%%%%%%%%%%%%%%%%%%%%%%%%%%%%%%%%%%%%%%%%%%%%%%%%%%%%%%%%%%%%%%%%%%%%%%%%%%%%%%%%%%%%%%%%%%%%%%%%%%%%%%%%%%%%%%%%%%%%%%%%%%%%%%%%%%%%%%%%%%%%%%%%%

\usepackage{eurosym}
\usepackage{vmargin}
\usepackage{amsmath}
\usepackage{graphics}
\usepackage{epsfig}
\usepackage{enumerate}
\usepackage{multicol}
\usepackage{subfigure}
\usepackage{fancyhdr}
\usepackage{listings}
\usepackage{framed}
\usepackage{graphicx}
\usepackage{amsmath}
\usepackage{chngpage}

%\usepackage{bigints}
\usepackage{vmargin}

% left top textwidth textheight headheight

% headsep footheight footskip

\setmargins{2.0cm}{2.5cm}{16 cm}{22cm}{0.5cm}{0cm}{1cm}{1cm}

\renewcommand{\baselinestretch}{1.3}

\setcounter{MaxMatrixCols}{10}

\begin{document}

%%%%%%%%%%%%%%%%%%%%%%%%%%%%%%%%%%%%%%%%%%%%%%%%%%%%%%%%%%%%%%%%%%%%%%%%%%%%%%%%%%
3 The number, N(t), of members of a pension scheme who are receiving benefits at time
t, is subject to change of two kinds:
it increases by 1 when an active member reaches retirement age
it decreases by 1 when a retired member dies
Assume that retirements occur according to a Poisson process with rate and that
each retired member, independently, has a probability dt of dying within the time
interval (t, t + dt).
(i) Explain why, under these assumptions, N(t) is a Markov jump process. [1]
(ii) Write down the transition rates of N(t). [1]
(iii) Using the notation pn(t) = P(N(t) = n), obtain a differential equation satisfied
by pn(t). [2]
(iv) Verify that one solution of the equation in (iii) is given by
1 /
( ) = , = 0,1,...
!
n
pn t e n
n
[2]
(v) State what conclusions can be drawn from (iv). [1]

%%%%%%%%%%%%%%%%%%%%%%%%%%%%%%%%%%%%%%%%%%%%%%%%%%%%%%%%%%%%%%%%%%%%%%%%%%%%%%%%%%
3 (i) It is a jump process because it remains in one state for a period of time and
then jumps to another state, or alternatively because it is a continuous-time
process with a discrete state space.
Given the entire past history of the process, the probability of a member
retiring and beginning to receive benefits in the next dt interval is dt, i.e.
independent of the past. The same applies to the probability of death between
times t and t + dt: it is dependent only on the state (the number of retired
members) at time t, and not on anything which happened before that. Thus the
Markov property holds.
(ii) n,n 1 , n,n 1 n
Subject 103 (Stochastic Modelling) April 2004 Examiners Report
(iii) From the Kolomogorov Forward equations we have that
, 1 , 1 1, 1 1, 1
1 1
( )
( ) ( 1)
n
n n n n n n n n n n n
n n n
dp
p p p
dt
n p n p p
,.
(iv) The suggested form of pn does not depend on t, so its derivative is zero. The
RHS is
1 1
( ) ( 1) 0
! ( 1)! ( 1)!
n n n
n e n e e
n n n
,
where = / .
(v) The implication is that Poisson( ) is a stationary distribution for the Markov
jump process. (In this case it is also a limiting distribution, but that is not
deducible from the Core Reading.)
%%%%%%%%%%%%%%%%%%%%%%%%%%%%%%%%%%%%%%%%%%%%%%%%%%%%%%%%%%%%%%%%%%%%%%%%%%%%%%%%%%%%
\newpage

Very few candidates suggested that the rate at which deaths take place should be
proportional to the number of retired scheme members.
In (iv) examiners awarded marks for correct attempts to carry out the verification procedure even when the differential equation in (iii) was wrong.
In (v), terms such as limiting distribution or equilibrium distribution were given full credit.

\end{document}
