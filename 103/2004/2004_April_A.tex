\documentclass[a4paper,12pt]{article}

%%%%%%%%%%%%%%%%%%%%%%%%%%%%%%%%%%%%%%%%%%%%%%%%%%%%%%%%%%%%%%%%%%%%%%%%%%%%%%%%%%%%%%%%%%%%%%%%%%%%%%%%%%%%%%%%%%%%%%%%%%%%%%%%%%%%%%%%%%%%%%%%%%%%%%%%%%%%%%%%%%%%%%%%%%%%%%%%%%%%%%%%%%%%%%%%%%%%%%%%%%%%%%%%%%%%%%%%%%%%%%%%%%%%%%%%%%%%%%%%%%%%%%%%%%%%

\usepackage{eurosym}
\usepackage{vmargin}
\usepackage{amsmath}
\usepackage{graphics}
\usepackage{epsfig}
\usepackage{enumerate}
\usepackage{multicol}
\usepackage{subfigure}
\usepackage{fancyhdr}
\usepackage{listings}
\usepackage{framed}
\usepackage{graphicx}
\usepackage{amsmath}
\usepackage{chngpage}

%\usepackage{bigints}
\usepackage{vmargin}

% left top textwidth textheight headheight

% headsep footheight footskip

\setmargins{2.0cm}{2.5cm}{16 cm}{22cm}{0.5cm}{0cm}{1cm}{1cm}

\renewcommand{\baselinestretch}{1.3}

\setcounter{MaxMatrixCols}{10}

\begin{document}
\begin{enumerate}
1 The stochastic differential equation for geometric Brownian motion is given by
dXt = Xt dt + Xt dBt
where Bt is a standard Brownian motion. Show that the solution to the differential
equation which also satisfies
t0 X x is
0
1 2
2 0 Xt x exp Bt Bt t t
for t t0. [4]

%%%%%%%%%%%%%%%%%%%%%%%%%%%%%%%%%%%%%%%%%%%%%%%%%%%%%%%%%%%%%%%%%%%%%%%%%%%%%%%%%%
2 A Box-Jenkins model-fitting procedure suggests that the best fitting model for a set of
normalised share price data x1, , xn is ARMA(1,2), with equation
Xt 0.63Xt 1 et 0.45et 1 0.34et 2 ,
where {e1, e2, } is a sequence of uncorrelated, zero-mean random variables with
variance 2.
(i) Determine whether the model is stationary and invertible. [2]
(ii) Calculate 0, 1, 2, the autocovariance function of the fitted model at lags 0, 1,
and 2, in terms of 2. [4]
[Total 6]

%%%%%%%%%%%%%%%%%%%%%%%%%%%%%%%%%%%%%%%%%%%%%%%%%%%%%%%%%%%%%%%%%%%%%%%%%%%%%%%%%%
3 The number, N(t), of members of a pension scheme who are receiving benefits at time
t, is subject to change of two kinds:
it increases by 1 when an active member reaches retirement age
it decreases by 1 when a retired member dies
Assume that retirements occur according to a Poisson process with rate and that
each retired member, independently, has a probability dt of dying within the time
interval (t, t + dt).
(i) Explain why, under these assumptions, N(t) is a Markov jump process. [1]
(ii) Write down the transition rates of N(t). [1]
(iii) Using the notation pn(t) = P(N(t) = n), obtain a differential equation satisfied
by pn(t). [2]
(iv) Verify that one solution of the equation in (iii) is given by
1 /
( ) = , = 0,1,...
!
n
pn t e n
n
[2]
(v) State what conclusions can be drawn from (iv). [1]

%%%%%%%%%%%%%%%%%%%%%%%%%%%%%%%%%%%%%%%%%%%%%%%%%%%%%%%%%%%%%%%%%%%%%%%%%%%%%%%%%%
1 First method (direct method):
Let f(Xt) = ln Xt . Using Itô s Lemma we have
df = 1 2 2
2 f Xtdt XtdBt f ( Xt )dt
= 1 2 2
2 2
1 1 1
t . t t t
t t t
X X dt X dB
X X X
= 1 2
2 dt dBt
Integrating from t0 to t we have
0 0
1 2
2 0 ln Xt ln Xt t t Bt Bt
and using the initial condition we get
0
1 2
2 0 Xt x exp Bt Bt t t as required.
Alternative method (working backwards from the solution)
Let
0
1 2
2 0 f (t, Bt ) x exp Bt Bt t t . Then
2
2 2
2
1
, ,
2
f f f
f f f
t B B
, so that
2 2 1 1
2 2 t t t t t t t t df f dt f dB f dt f dt f dB
This implies that Xt = f(t,Bt) satisfies the stochastic differential equation.
We need to verify that the initial condition holds for this solution. (It clearly does, but
the check needs to be performed.)
Considering that this stochastic differential equation is the most frequently used of all,
this question was on average very poorly answered.

%%%%%%%%%%%%%%%%%%%%%%%%%%%%%%%%%%%%%%%%%%%%%%%%%%%%%%%%%%%%%%%%%%%%%%%%%%%%%%%%%%
2 (i) The solution to 1 0.63z 0 is z = 1/0.63, which is greater than 1. Therefore
the model is stationary.
For invertibility, we should check that the roots of 1 0.45z 0.34z2 0 are
outside the unit circle. They are ( 0.45 1.25)/( 2*0.34) = 2.5 or 1.18, both
OK.
Subject 103 (Stochastic Modelling) April 2004 Examiners Report
Page 3
(ii) Let 0 = Cov(Xt, et), 1 = Cov(Xt, et 1), 2 = Cov(Xt, et 2). Then
0 = 0.63 1 + 0 + 0.45 1 0.34 2
1 = 0.63 0 + 0.45 0 0.34 1
2 = 0.63 1 0.34 0
0 = 2 (this may be regarded as obvious and not stated explicitly)
1 = 0.63 0 + 0.45 2 = 1.08 2
2 = 0.63 1 0.34 2 = 0.3404 2
An alternative expression for 0 may be obtained using
0 1 1 2
2 2 2 2
0 0 1
Var(0.63 0.45 0.34 )
0.63 (1 0.45 0.34 ) 2 0.63[0.45 0.34 ]
t t t t X e e e
which removes the need to calculate 2.
Having derived the equations, we need to solve them.
1 = 0.63 0 + 0.0828 2
0 = 0.63(0.63 0 + 0.0828 2) + 1.370 2 = 0.3969 0 + 1.422 2,
implying that
0 = 2.358 2, 1 = 1.5684 2, 2 = 0.6481 2.
Candidates were often unsure of the procedure required to derive the equations for the k but
did rather better at solving them. In particular, many candidates did not take correct
account of the relationship between Xt and past values of et. Marks were awarded for correct
methodology when deriving the solutions, even if the equations being solved were not the
right ones.

%%%%%%%%%%%%%%%%%%%%%%%%%%%%%%%%%%%%%%%%%%%%%%%%%%%%%%%%%%%%%%%%%%%%%%%%%%%%%%%%%%
3 (i) It is a jump process because it remains in one state for a period of time and
then jumps to another state, or alternatively because it is a continuous-time
process with a discrete state space.
Given the entire past history of the process, the probability of a member
retiring and beginning to receive benefits in the next dt interval is dt, i.e.
independent of the past. The same applies to the probability of death between
times t and t + dt: it is dependent only on the state (the number of retired
members) at time t, and not on anything which happened before that. Thus the
Markov property holds.
(ii) n,n 1 , n,n 1 n
Subject 103 (Stochastic Modelling) April 2004 Examiners Report
(iii) From the Kolomogorov Forward equations we have that
, 1 , 1 1, 1 1, 1
1 1
( )
( ) ( 1)
n
n n n n n n n n n n n
n n n
dp
p p p
dt
n p n p p
,.
(iv) The suggested form of pn does not depend on t, so its derivative is zero. The
RHS is
1 1
( ) ( 1) 0
! ( 1)! ( 1)!
n n n
n e n e e
n n n
,
where = / .
(v) The implication is that Poisson( ) is a stationary distribution for the Markov
jump process. (In this case it is also a limiting distribution, but that is not
deducible from the Core Reading.)
Very few candidates suggested that the rate at which deaths take place should be
proportional to the number of retired scheme members.
In (iv) examiners awarded marks for correct attempts to carry out the verification procedure
even when the differential equation in (iii) was wrong.
In (v), terms such as limiting distribution or equilibrium distribution were given full
credit.
