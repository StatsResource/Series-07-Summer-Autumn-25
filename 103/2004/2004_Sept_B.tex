\documentclass[a4paper,12pt]{article}

%%%%%%%%%%%%%%%%%%%%%%%%%%%%%%%%%%%%%%%%%%%%%%%%%%%%%%%%%%%%%%%%%%%%%%%%%%%%%%%%%%%%%%%%%%%%%%%%%%%%%%%%%%%%%%%%%%%%%%%%%%%%%%%%%%%%%%%%%%%%%%%%%%%%%%%%%%%%%%%%%%%%%%%%%%%%%%%%%%%%%%%%%%%%%%%%%%%%%%%%%%%%%%%%%%%%%%%%%%%%%%%%%%%%%%%%%%%%%%%%%%%%%%%%%%%%

\usepackage{eurosym}
\usepackage{vmargin}
\usepackage{amsmath}
\usepackage{graphics}
\usepackage{epsfig}
\usepackage{enumerate}
\usepackage{multicol}
\usepackage{subfigure}
\usepackage{fancyhdr}
\usepackage{listings}
\usepackage{framed}
\usepackage{graphicx}
\usepackage{amsmath}
\usepackage{chngpage}

%\usepackage{bigints}
\usepackage{vmargin}

% left top textwidth textheight headheight

% headsep footheight footskip

\setmargins{2.0cm}{2.5cm}{16 cm}{22cm}{0.5cm}{0cm}{1cm}{1cm}

\renewcommand{\baselinestretch}{1.3}

\setcounter{MaxMatrixCols}{10}

\begin{document}
\begin{enumerate}

4 (i) Define a standard Brownian motion {Bt t 0}. [2]
(ii) Assume that St , the price of a share at time t , satisfies the stochastic
differential equation
dSt = St dt St dBt , t 0.
Solve the equation for St in terms of S0 and Bt . [3]
(iii) For the values = 20% and =15% on an annual basis, calculate the
probability that the price of the share will exceed 120 in three months time if
the current price of the share is 96. [4]
[Total 9]
103 S2004 4
5 A bag contains N balls, all green or red. At each stage a ball is taken out of the bag at
random and is replaced by a ball of the other colour. Let Xn denote the number of
green balls in the bag after n stages.
(i) Explain why Xn, n 0 is a discrete time Markov chain with state space
S = {0, 1, 2, .., N} and with transition probabilities
i,i 1 =
N i
p
N
and i,i 1 =
i
p
N
[2]
(ii) Show that the Markov chain Xn, n 0 is irreducible and periodic and state its
period. [3]
(iii) Show that for this process the stationary distribution = ( 0, 1, 2, .., N)
is given by
1
=
2 i N
N
i
for i = 0, 1, 2, ., N. [4]
(iv) State, with a reason, whether it is the case that
lim ( n = | 0 = 0) = j .
n
P X j X [1]
[Total 10]

%%%%%%%%%%%%%%%%%%%%%%%%%%%

END OF PAPER
4 (i) A standard Brownian motion {Bt} is defined by the following properties:
B(0) 0 and Bt has independent increments; Bt Bs is independent of
Bu for 0 u s and s t .
Bt has stationary and Gaussian increments; Bt Bs N(0 t s) .
Bt has continuous sample paths, i.e. t Bt is continuous.
(ii) Using Itô s Lemma gives
2
2
2
1 1 1
(log ) ( )
2
2
t t
t t
t
d S dS dSt
S S
dt dB dt
This implies that
2
log log 0
2 St S t Bt
or, finally,
2
2
0
t Bt
St S e
(iii) We have
Subject 103 (Stochastic Modelling) September 2004 Examiners Report
Page 5
2
2
0
2
2
log
2
1
log
2
log
1
t t
t
x
y
x
P S x S y P B t
y
x
P B t
y
t
t
Substituting the values x 120 y 96 0 2 0 15 and t 0 25 years
above we find that the answer is
1 (2.3461) 1 0 9905 0 0095.
The calculations in parts (ii) and (iii) were well done. The definition in part
(i) caused some problems: it is necessary to mention the stationary,
independent increments property; then either of the two remaining properties
(continuous sample paths, normally distributed increments) implies the other.
5 (i) The Markov property is clear: the chain jumps either up or down by 1, with
probabilities depending only on the current state, not the past history.
P(Xn 1 i 1| Xn i) is the probability that the (n+1)th ball selected is red,
which is just 1/N of the number of red balls at time n, which is N i.
(ii) From any state i it is possible to reach any other state j in just |j i| steps,
either all upwards or all downwards. This means that the chain is irreducible.
Every transition takes the chain from an even state to an odd one or vice versa,
which implies that the period must be an even number.
On the other hand, starting from state 0 it is possible to return to 0 in two
steps. Therefore state 0 has period 2 and, by irreducibility, all states have
period 2.
(iii) To find the stationary distribution we can use the relationship suggested by the
Detailed Balance Equations:
1
1
i i
N i i
N N
for i = 0, 1, 2, ., N 1.
Thus we get the recursive relationship
1 1 i i
N i
i
for i = 0, 1, 2, ., N 1
Subject 103 (Stochastic Modelling) September 2004 Examiners Report
Page 6
Starting with i = 0 and working forwards we get
1 0 1
N
2 1
1
2
N
= 0
( 1)
(2)(1)
N N
3 2
2
3
N
= 0
( 1)( 2)
(3)(2)(1)
N N N
and in general
i = 0
( 1)( 2) ( 1)
( 1)( 2) (2)(1)
N N N N i
i i i
= 0
!
( )! !
N
N i i
= 0
N
i
Alternatively, write down the transition matrix P and use the equation TP =
T to obtain
1 0 1 0
1
N
N
0 2 1 2 0
2 ( 1)
2
N N
N
1 3 2 3 0
1 3 ( 1)( 2)
,
3!
N N N N
N N
etc.
To find 0, we use the fact that 0 + 1 + 2 + ..+ N = 1
i.e.
1 1
1
0
1 0
! ! 1
1 2
( )! ! ! 2
N N
N
N
i i
N N
N i i i
Therefore the stationary distribution = ( 0, 1, 2, .., N) is given by
1
2 i N
N
i
for i = 0, 1, 2, ., N
(iv) P(Xn j | X0 0) does not converge, being alternately zero and non-zero,
since X is periodic.
The derivation of the stationary distribution in part (iii) caused difficulties
with many candidates, but otherwise candidates showed a good understanding
of discrete-time Markov chains.
Subject 103 (Stochastic Modelling) September 2004 Examiners Report
Page 7
