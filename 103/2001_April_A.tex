\documentclass[a4paper,12pt]{article}

%%%%%%%%%%%%%%%%%%%%%%%%%%%%%%%%%%%%%%%%%%%%%%%%%%%%%%%%%%%%%%%%%%%%%%%%%%%%%%%%%%%%%%%%%%%%%%%%%%%%%%%%%%%%%%%%%%%%%%%%%%%%%%%%%%%%%%%%%%%%%%%%%%%%%%%%%%%%%%%%%%%%%%%%%%%%%%%%%%%%%%%%%%%%%%%%%%%%%%%%%%%%%%%%%%%%%%%%%%%%%%%%%%%%%%%%%%%%%%%%%%%%%%%%%%%%

\usepackage{eurosym}
\usepackage{vmargin}
\usepackage{amsmath}
\usepackage{graphics}
\usepackage{epsfig}
\usepackage{enumerate}
\usepackage{multicol}
\usepackage{subfigure}
\usepackage{fancyhdr}
\usepackage{listings}
\usepackage{framed}
\usepackage{graphicx}
\usepackage{amsmath}
\usepackage{chngpage}

%\usepackage{bigints}
\usepackage{vmargin}

% left top textwidth textheight headheight

% headsep footheight footskip

\setmargins{2.0cm}{2.5cm}{16 cm}{22cm}{0.5cm}{0cm}{1cm}{1cm}

\renewcommand{\baselinestretch}{1.3}

\setcounter{MaxMatrixCols}{10}

\begin{document}
\begin{enumerate}
ã Institute of Actuaries1
{N(t) : t \geq 0} is a Poisson process with rate \lambda and {. t : t \geq 0} is the filtration
associated with N.
(i)
Write down the conditional distribution of N(t + s) − N(t) given . t , where
s > 0 and use your answer to find E(θ N(t+s) . t ).
(ii)
2
3
Find a process of the form M(t) = η(t)θ N(t) which is a martingale.
[3]
[2]
[Total 5]
%%%%%%%%%%%%%%%%%%%%%%%%%%%%%%%%%%%%%%%%%%%%%%%%%%%%%%%%%%%%%%%%%%%%%%%%
An insurance company wishes to test the assumption that claims of a particular
type arrive according to a Poisson process model. The times of arrival of the next
20 incoming claims of this type are to be recorded, giving a sequence T 1 , ..., T 20 .
(i) Give reasons why tests for the goodness of fit should be based on the inter-
[1]
arrival times X i = T i − T i−1 rather than on the arrival times T i .
(ii) Write down the distribution of the inter-arrival times if the Poisson process model is correct and state one statistical test which could be
applied to determine whether this distribution is realised in practice.

(iii) State the relationship between successive values of the inter-arrival times if the Poisson process model is correct and state one method which could
be applied to determine whether this relationship holds in practice.


%%%%%%%%%%%%%%%%%%%%%%%%%%%%%%%%%%%%%%%%%%%%%%%%%%%%%%%%%%%%%%%%%%%%%%%%
1
(i)
Given . t we know that N(t + s) − N(t) ~ Poisson(\lambdas).
Hence E(θ N(t+s) . t ) = θ N(t) e (θ−1)\lambdas .
(ii)
Now E(η(t + s) θ N(t+s) . t ) = η(t + s) θ N(t) e (θ−1)\lambdas , which needs to be equal to
M(t) = η(t) θ N(t) . It follows that η(t) = e −(θ−1)\lambdat .
%%%%%%%%%%%%%%%%%%%%%%%%%%%%%%%%%%%%%%%%%%%%%%%%%%%%%%%%%%%%%%%%%%%%%%%%
2
(i) The inter-arrival times are much more suitable because they are
independent.
(ii) They should be exponentially distributed with the same mean.
Kolmogorov-Smirnov, Anderson-Darling or χ 2 goodness-of-fit test can all
be used.
(iii)
Successive values should be independent.
Regress X i on X i−1 using ordinary least squares, or fit an AR(1) and test
the \alpha 1 parameter for significance (equivalent to Durbin-Watson test).
%%%%%%%%%%%%%%%%%%%%%%%%%%%%%%%%%%%%%%%%%%%%%%%%%%%%%%%%%%%%%%%%%%%%%%%%
3
(i)
Spectral density f(ω) =
1
2 π
Σ ∞−∞ γ k e ik ω , or equivalent.
For MA(1), therefore, we have
f(ω) =
σ 2 e
(1 + \beta 2 + 2\beta cos ω).
2 π
And for AR(1),
f(ω) =
(ii)
σ 2 e
1
.
2
2 π 1 + \alpha − 2 \alpha cos ω
Clearly from (i) the inverse of the MA(1) is an AR(1), with \alpha = −\beta and with
a different value of σ 2 e .
The word “invertible” attached to a MA(1) indicates that the inverse is a
stationary AR(1), whereas a non-“invertible” MA(1) has as inverse an
AR(1) model which cannot be stationary, such as X t = 2X t − 1 + e t .
Page 2Subject 103 (Stochastic Modelling) — April 2001 — Examiners’ Report
4
