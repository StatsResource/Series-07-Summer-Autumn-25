\documentclass[a4paper,12pt]{article}

%%%%%%%%%%%%%%%%%%%%%%%%%%%%%%%%%%%%%%%%%%%%%%%%%%%%%%%%%%%%%%%%%%%%%%%%%%%%%%%%%%%%%%%%%%%%%%%%%%%%%%%%%%%%%%%%%%%%%%%%%%%%%%%%%%%%%%%%%%%%%%%%%%%%%%%%%%%%%%%%%%%%%%%%%%%%%%%%%%%%%%%%%%%%%%%%%%%%%%%%%%%%%%%%%%%%%%%%%%%%%%%%%%%%%%%%%%%%%%%%%%%%%%%%%%%%

\usepackage{eurosym}
\usepackage{vmargin}
\usepackage{amsmath}
\usepackage{graphics}
\usepackage{epsfig}
\usepackage{enumerate}
\usepackage{multicol}
\usepackage{subfigure}
\usepackage{fancyhdr}
\usepackage{listings}
\usepackage{framed}
\usepackage{graphicx}
\usepackage{amsmath}
\usepackage{chngpage}

%\usepackage{bigints}
\usepackage{vmargin}

% left top textwidth textheight headheight

% headsep footheight footskip

\setmargins{2.0cm}{2.5cm}{16 cm}{22cm}{0.5cm}{0cm}{1cm}{1cm}

\renewcommand{\baselinestretch}{1.3}

\setcounter{MaxMatrixCols}{10}

\begin{document}
\begin{enumerate}
\item A family agrees an expenditure target, $Y_n$ , for year n, in such a way that the annual increase in the expenditure target is proportional to the increase in the
family income over the previous year. The actual expenditure during the year, X n , is assumed to be related to the expenditure target, but incorporating an element of randomness and a factor accounting for the family’s propensity to
overspend. The family income, $I_n$ , is assumed to grow at a constant annual rate,
before randomness is taken into account.

The head of the household believes that the following three equations form an
appropriate representation of the above information:
Y n = Y n − 1 + \beta(I n − 1 − I n − 2 )
X n = (1 + π) Y n + e n (1)
I n = (1 + \alpha) I n − 1 + e n (2)
where {( e n (1) , e n (2) ) : n = 1, 2, ...} is a sequence of zero-mean bivariate Normal random variables and \alpha, \beta and π are positive parameters (with \beta < 1).
103 A2001—6
\begin{enumerate}
\item (i) Express the first of the equations in terms of the backshift operator, B, and deduce that a linear relationship exists between Y n and I n − 1 .
\item (ii) Show that the process Z n = (X n , I n ) is a first-order multivariate autoregressive process.
\item (iii) State, with reasons, whether {I n : n \geq 1} is a stationary time series, and hence determine whether {Z n : n \geq 1} is I(0), I(1) or neither.
\item (iv) Find an estimator for the parameter \alpha by minimising the quantity
Σ t n = 2 ( e t (2) ) 2 .
\item (v)
The head of the household wishes to perform a simulation to investigate whether the propensity to overspend will result in negative net savings.
It is assumed that Var( e n (1) ) = σ 1 2 , Var( e n (2) ) = σ 22 and Cov( e n (1) , e n (2) ) = ρσ 1 σ 2 , where $−1 < ρ < 1$.
\item (vi)
(a) Describe a method of simulating an observation of the pair
( e n (1) , e n (2) ) starting from two uniformly distributed pseudo-random variables U 1 , U 2 .
(b) Describe the role of sensitivity analysis in drawing conclusions from the simulation.
\end{enumerate}
An alternative model is proposed, involving the logarithms of the
quantities I n , X n and Y n :
ln Y n = ln Y n−1 + ln I n−1 − ln I n−2
ln X n = θ + ln Y n + e n (1)
ln I n = φ + ln I n−1 + e n (2)
Discuss whether this model is more suitable than the original model. [2]
[Total 19]
103 A2001—7
\end{document}
