1
A stochastic interest rate model postulates that the base lending rate in month t,
i t , follows the model:
i t = 5% + 0.9 (i t−1 − 5%) + e t ,
where e t , e 2 , ...is a sequence of independent Normal random variables with mean
0, variance σ 2 .
(i)
(ii)
2
(a) Determine an expression for i t in terms of e t , e t−1 , e t−2 , ..., e 1 and i 0 .
(b) Calculate the conditional mean and variance of the base lending
rate in month t given that i 0 = 8%.
[4]
Derive an estimator for σ 2 based on observations i 1 , i 2 , ..., i n of the base
lending rate in n successive months.

[Total 6]
Let {B(t) : t ≥ 0} be a standard Brownian motion and define
æ 1 ö
B 1 (t) = tB ç ÷ for t > 0, with B 1 (0) = 0.
è t ø
(i) Calculate EB 1 (t), Var(B 1 (t)) and Cov{B 1 (s), B 1 (t)) for s < t. Deduce that B 1
is a standard Brownian motion.
[3]
(ii) (a)
Show that the two probabilities
P[B(t) < ct for all t ≥ 1]
and
P[B(t) < c for all 0 ≤ t ≤ 1]
are equal to one another where c > 0 is a constant.
(b)
103 S2001—2
Find an expression for the value of these probabilities by stating
the probability density function of M 1 = max 0≤t≤1 B(t).
[4]
[Total 7]

%%%%%%%%%%%%%%%%%%%%%%%%%%%%%%%%%%%%%%%%%%%%%%%%%%%%%%%%%%%%%%%%%%%%%%%%%%%%%%%%%%%%%%%%%%%%%%%%%%%%%%%%%%%%%%
1
(i)
(a) Applying the equation iteratively for t, t - 1, 1⁄4 , 1, we obtain
i t - 5% = e t + 0.9e t-1 + 0.9 2 e t-2 + ... + 0.9 t-1 e 1 + 0.9 t (i 0 - 5%).
(b) The RHS has expectation 0.9 t  ́ 3%, since each e u has expectation
zero, and has variance s 2 (1 + 0.9 2 + 0.9 4 + ... + 0.9 2t-2 ) = s 2
(ii)
(1 - 0.9 2 t )
.
0.19
We have the expression
e t = i t - 5% - 0.9(i t- 1 - 5%),
so that a sensible estimator for s 2 would be
1
n
n
å ( i
t
t = 2
- 5% - 0.9( i t - 1 - 5%)) 2 .
(This is both least squares estimator and maximum likelihood estimator.)
2
(i)
EB 1 (t) = tEB(t - 1 ) = 0;
Var B 1 (t) = t 2 Var B(t - 1 ) = t;
Cov (B 1 (s), B 1 (t)) = stCov (B(s - 1 ), B(t - 1 )) = stt - 1 = s for s < t.
These are identical to the corresponding quantities for B.
(ii)
(a)
é æ 1 ö c
ù
1
P[B(t) < ct for all t 3 1] = P ê B ç ÷ < for all 3 1 ú
u
ë è u ø u
û
= P[B 1 (u) < c for all u \leq 1]
(b)
We know that the density of M 1 is 2f(y) for y > 0, where f is the
standard Normal density.
It follows that the required probability is
P[M 1 < c] = 2 ò c 0 f(y)dy = 2F(c) - 1.
Page 2Subject 103 (Stochastic Modelling) — 
%%%%%%%%%%%%%%%%%%%%%%%%%%%%%%%%

3
(i)
g 1 = 0.8g 0 -0.4g 1 + 0 and g 2 = 0.8g 1 - 0.4g 0 + 0
Hence g 1 =
4
7
g 0 and g 2 =
f 1 = r 1 =
4
4
7
, f 2 =
2
35
g 0 , implying that r 1 =
r 2 - r 1 2
1 - r 1 2
4
7
and r 2 =
The ACF will reduce to zero as k increases;
the PACF, however, will be equal to zero for all k > 2.
(i) The generator is
0 ö
æ - 4 a 4 a
ç
÷
A = ç a
- 4 a 3 a ÷ = a
ç 0
0
0 ÷ ø
è
d
dt
.
= -0.4.
(ii)
The forward equation is
2
35
æ - 4 4 0 ö
ç
÷
ç 1 - 4 3 ÷
ç 0
0 0 ÷ ø
è
P t = P t A.
Calculate the right hand side
æ e - 2 a t ( - 2 + 1) + e - 6 a t ( - 2 - 1) e - 2 a t (2 - 4) + e - 6 a t (2 + 4)
ç
P t A = a ç e - 2 a t ( - 1 + 1 2 ) + e - 6 a t (1 + 1 2 ) e - 2 a t (1 - 2) + e - 6 a t ( - 1 - 2)
ç
0
0
è
æ -a e - 2 a t - 3 a e - 6 a t
ç
= ç -a
e - 2 a t + 3 2 a e - 6 a t
2
ç
0
è
=
(ii)
d
dt
-a e - 2 a t - 3 a e - 6 a t
0
3 a e - 2 a t - 3 a e - 6 a t ö
÷
3 a - 2 a t
e
+ 2 3 a e - 6 a t ÷
2
÷
0
ø
P t
P AA (t) \leq P AD (t)
1 -
- 2 a e - 2 a t + 6 a e - 6 a t
3 e - 2 a t - 3 e - 6 a t ö
÷
3 - 2 a t
e
+ 3 2 e - 6 a t ÷
2
÷
0
ø
t 3 t implies
3 - 2 at
1
e
= e - 2 at .
2
2
Hence,
t =
log 2
.
2a
Page 3Subject 103 (Stochastic Modelling) — 
%%%%%%%%%%%%%%%%%%%%%%%%%%%%%%%%

