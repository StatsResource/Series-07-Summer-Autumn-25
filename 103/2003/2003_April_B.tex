
4 (i) Given a pseudo-random number U uniformly distributed over [0,1], obtain an
expression in terms of U and  for a non-negative pseudo-random variable X
which has density function
f (x) = e x 
 [2]
(ii) A sequence of simulated observations is required from the density function
( )= ( ) , 0
1
e x g x k x
x

 

where  is a non-negative parameter and k() is a constant of integration not
involving x.
(a) Describe a procedure that applies the Acceptance-Rejection method to
obtain the required observations.
(b) Derive an expression involving  and k() for the expected number of
pseudo-random variables required to generate a single observation
from the density g using this method.
[6]
[Total 8]
103 A2003—4
5 (i) Let W(t) be defined by W(t) = 	4B(kt), where B(t) is a standard Brownian
motion.
(a) Calculate the value of k which gives W(t) the same expectation and
covariance function as B(t).
(b) Prove that, for this value of k, W is a standard Brownian motion.
[4]
(ii) Prove that exp[2B(t)	2t] is a martingale. [5]
[Total 9]

%%%%%%%%%%%%%%%%%%%%%%%%

4 (i) Let X =  (log U)/. Then, for x > 0,
P(X > x) = P(log U < x) = exp(x).
Differentiating, f (x) e x 
  , as required.
Subject 103 (Stochastic Modelling) — April 2003 — Examiners’ Report
Page 4
Alternatively: use the inverse distribution function method.
0
F(x) x e ydy 1 e x    
    . We need to invert this: set U = F(X) and express
X as a function of U. This gives X =   1 log (1U).
(ii) (a) Use f(x) =  exp(x) as the base density. We need to find a constant
C such that ( )
1
x
k e C e x
x


  

for all x > 0.
C = k()/ is the best that we can do.
The procedure is:
1. Generate a value y from the density f(x) =  exp(x).
2. Take another Uniform pseudo-random variable U2; if this is less
than g(y)/(C f(y)) [which is equal to 1/(1 + y)] then we accept the
value y, otherwise reject it and return to 1.
(b) On average it takes C repetitions of steps 1 and 2 to generate a value.
Each such repetition requires two uniform pseudo-random variables.
So the answer is 2 k()/.
Most candidates did well here, although a few people inverted the density function
instead of the distribution function.
Some experienced real difficulty in providing a clear statement of the algorithm for
Acceptance-Rejection method. A lot of people answered this part in abstract without
realising that they should use the density from part (i) as the base density, and hence
didn’t get the marks for calculating C.
5 (i) (a) EW(t) = 0, Cov(W(s), W(t)) = 16Cov(B(ks), B(kt)) = 16k min(s, t).
Thus k = 1/16 is the required value.
To prove that a process is a BM, it is necessary to check the
covariance, not just the variance.
(b) In addition to the expectation and covariance function, we need to
show
  that W is Normally distributed. Any linear transformation of a
Normal random variable is itself Normal.
  that W has continuous sample paths. For small h, we have
W(t + h)  W(t) =  4(B(kt + kh)  B(kt)), which clearly tends to 0
as h approaches 0.
Subject 103 (Stochastic Modelling) — April 2003 — Examiners’ Report
Page 5
  An alternative to proving continuity is to state that the increments
of W are independent of past values and are also Normally
distributed.
(ii) Consider exp[2B(t)2t] with respect to the filtration Ft.
B(t) – B(s) is independent of Ft and B(s) is Fs-measurable. Then
 2B(t) |  =  2[B(t) B(s)] 2B(s) | 
E e Fs E e e Fs 
= 2B(s)  2[B(t) B(s)] | 
e E e Fs 
= e2B(s)Ee2[B(t) B(s)]  
The increment B(t) – B(s) has the normal distribution with mean 0 and
variance t 	s, so the expectation of e2[B(t) – B(s)] is equal to
M(2) = exp(2(t  s)), where M(
) is the moment generating function of the
N(0, t  s) distribution.
It follows that
 2B(t) 2t |  = 2B(s) 2s
E e Fs e  
and therefore e2[B(t) t] is a martingale.
For part (i), in (a) many candidates just checked variance rather than the covariance,
but got 1/16 correct. Not so many people got part (b)  generally people assumed
that (a) implied (b), rather than considering other properties of Brownian motion
Most candidates fared well on part (ii), identifying where necessary the mgf of a
normal random variable to successfully demonstrate the given process is a
martingale.
Subject 103 (Stochastic Modelling) — April 2003 — Examiners’ Report
Page 6
