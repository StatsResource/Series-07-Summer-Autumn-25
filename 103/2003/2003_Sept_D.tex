\documentclass[a4paper,12pt]{article}

%%%%%%%%%%%%%%%%%%%%%%%%%%%%%%%%%%%%%%%%%%%%%%%%%%%%%%%%%%%%%%%%%%%%%%%%%%%%%%%%%%%%%%%%%%%%%%%%%%%%%%%%%%%%%%%%%%%%%%%%%%%%%%%%%%%%%%%%%%%%%%%%%%%%%%%%%%%%%%%%%%%%%%%%%%%%%%%%%%%%%%%%%%%%%%%%%%%%%%%%%%%%%%%%%%%%%%%%%%%%%%%%%%%%%%%%%%%%%%%%%%%%%%%%%%%%

\usepackage{eurosym}
\usepackage{vmargin}
\usepackage{amsmath}
\usepackage{graphics}
\usepackage{epsfig}
\usepackage{enumerate}
\usepackage{multicol}
\usepackage{subfigure}
\usepackage{fancyhdr}
\usepackage{listings}
\usepackage{framed}
\usepackage{graphicx}
\usepackage{amsmath}
\usepackage{chngpage}

%\usepackage{bigints}
\usepackage{vmargin}

% left top textwidth textheight headheight

% headsep footheight footskip

\setmargins{2.0cm}{2.5cm}{16 cm}{22cm}{0.5cm}{0cm}{1cm}{1cm}

\renewcommand{\baselinestretch}{1.3}

\setcounter{MaxMatrixCols}{10}

\begin{document}
\begin{enumerate}

8 A no-claims discount scheme has six classes of discount numbered from 0 (no
discount) to 5 (maximum discount). A claim-free year results in a move to the next
higher discount status (or in the retention of the maximum discount status); similarly a
year with one or more claims results in a move to the next lower discount class (or in
the retention of the no-discount status).
(i) Assuming that the probability of a claim-free period is , where 0 <  < ½,
write down the transition graph and the transition matrix of the discrete-time
Markov chain which models this scheme. [3]
(ii) (a) Write down the equations obeyed by the stationary probability
distribution 
 = (
0, 
1, …, 
5) of the Markov chain.
(b) Solve successively for 
1, 
2, 
3, …, 
5 in terms of 
0.
(c) Derive an expression, in terms of  only, for the value of 
0.
[5]
(iii) Explain how this analysis can be used to help the insurance company to set its
premium levels. [2]
[Total 10]
9 A model of mortality after retirement suggests that the age at death, T, of an
individual aged 65 at retirement is a random variable satisfying
65
( )=exp () ,
P T t  t x dx     
 	 
where (x), the force of mortality, is given by (x) = a + bx, and a and b are positive
parameters whose values are known.
(i) Determine a formula involving a uniform pseudo-random variable U for
simulating the age at death, T, of a single individual currently aged 65. [5]
(ii) As part of a simulation exercise, a trainee is required to consider a sample of
10 individuals now aged 65 and to generate a single observation of Y, the
number of individuals in that sample who will still be alive at age 75.
(a) Describe a method for doing this which does not involve 10 repetitions
of the method in (i).
(b) Explain the main advantages of the method in (a) when a large number
of simulated copies of the random variable Y are needed. [4]
(iii) The trainee is asked to investigate the effect of a reduction in the mortality rate
by repeating the whole simulation for different values of a and b. State with
reasons whether the trainee should use the same sequence of pseudo-random
numbers as before or whether a different sequence would be preferable. [1]
[Total 10]
103 S2003—7 PLEASE TURN OVER

%%%%%%%%%%%%%%%%%%5
8 (i) Transition graph:
(Transition probabilities are to the right and 1 to the left.)
The transition matrix is
P =
1 0 0 0 0
1 0 0 0 0
0 1 0 0 0
0 0 1 0 0
0 0 0 1 0
0 0 0 0 1
(ii) (a) The equations P = read
0(1 ) + 1(1 ) = 0
0 + 2(1 ) = 1
1 + 3(1 ) = 2
2 + 4(1 ) = 3
3 + 5(1 ) = 4
4 + 5 = 5
(b) Discard last equation and solve first one in terms of 0:
1(1 ) = 0(1 (1 ))
1 = 1 0
2(1 ) =
2
0 = 0
1 1
2 =
2
1 0
In general j = 0.
1
j
(c) Find 0 by normalisation:
0 1 2 3 4 5
Subject 103 (Stochastic Modelling) September 2003 Examiners Report
Page 10
1 =
6
5
0 0 0 0
1
1
= =
1 1
1
j
n
j j j
0 = 6
1
1 .
1
1
(iii) For a consistent profit the company requires that min0< 0.5 P( ) > 0, where
P( ) is the expected profit when annual claim rate is .
Expected long-term annual income from one customer is 5
0
j j , j
P where Pj
is the premium payable in discount level j, and expected annual claims
= C(1 ), so expected profit is
P( ) = 5
0 6
1
(1 ),
1
j
j j
P C where = .
1
Candidates showed a good understanding of Markov chains here, with many candidates achieving a
high score. For the final part of the question, a large number of candidates only gave a cursory
explanation and were unable to score the full marks.
9 (i) Use the inverse distribution function method.
F(t) = 1 P(T > t), so that
2 2
65
( ) = 1 exp ( ) = 1 exp 0.5 ( 65 ) ( 65) .
t
F t a bx dx b t a t
Rearranging,
2 652 65 log(1 ( )) = 0,
2 2
b b
t at a F t
or, in other words,
2 2
1 (65 ) 2 65 log(1 )
( ) = .
a a b b a u
F u
b
Subject 103 (Stochastic Modelling) September 2003 Examiners Report
Page 11
Since the simulated variable must be positive, the positive root is required.
The method, then, is to generate a pseudo-uniform random variable U in the
range [0,1] and to set
2 (65 )2 2 65 log(1 )
= .
a a b b a U
T
b
(ii) (a) Y is a Binomial random variable, with parameters n = 10 and
75
65
p exp (a bx)dx exp( 700b 10a).
Let G(y) be the distribution function of Y and let U be a single uniform
pseudo-random variable on [0,1]. Then set Y min y :G( y) U .
(b) Although this method requires a certain amount of computing time to
evaluate the distribution function G, this only has to be done once;
thereafter, only one value of U is needed to generate each value of Y,
as opposed to the ten values of U which are required in the other
method.
(iii) It is important to use the same sequence of pseudo-random numbers in each
case, otherwise we are not comparing like with like.
Many candidates correctly identified that the inverse transformation method was required, although
in a surprisingly high proportion of cases marks were lost because of errors in the algebra /
integration. The later parts of this question, dealing with the comparison of two methods of
generating discrete random variables, were in general less well done.
