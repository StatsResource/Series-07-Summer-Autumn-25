\documentclass[a4paper,12pt]{article}

%%%%%%%%%%%%%%%%%%%%%%%%%%%%%%%%%%%%%%%%%%%%%%%%%%%%%%%%%%%%%%%%%%%%%%%%%%%%%%%%%%%%%%%%%%%%%%%%%%%%%%%%%%%%%%%%%%%%%%%%%%%%%%%%%%%%%%%%%%%%%%%%%%%%%%%%%%%%%%%%%%%%%%%%%%%%%%%%%%%%%%%%%%%%%%%%%%%%%%%%%%%%%%%%%%%%%%%%%%%%%%%%%%%%%%%%%%%%%%%%%%%%%%%%%%%%

\usepackage{eurosym}
\usepackage{vmargin}
\usepackage{amsmath}
\usepackage{graphics}
\usepackage{epsfig}
\usepackage{enumerate}
\usepackage{multicol}
\usepackage{subfigure}
\usepackage{fancyhdr}
\usepackage{listings}
\usepackage{framed}
\usepackage{graphicx}
\usepackage{amsmath}
\usepackage{chngpage}

%\usepackage{bigints}
\usepackage{vmargin}

% left top textwidth textheight headheight

% headsep footheight footskip

\setmargins{2.0cm}{2.5cm}{16 cm}{22cm}{0.5cm}{0cm}{1cm}{1cm}

\renewcommand{\baselinestretch}{1.3}

\setcounter{MaxMatrixCols}{10}

\begin{document}


6 A No-Claims Discount system operated by a motor insurer has four levels:
\begin{itemize}
    \item Level 1: 0% discount
    \item Level 2: 25% discount
    \item Level 3: 40% discount
    \item Level 4: 50% discount
\end{itemize}
The rules for moving between these levels are as follows:
Following a claim-free year, move to the next higher level, or remain at level 4.
Following a year with one or more claims:
move back one level, or remain at level 1, if, in the year before the
most recent year, there were no claims;
move back two levels, or move to level 1 or remain at level 1 if, in the year before the most recent year, there was one or more claims.
For a given policyholder the probability of no claims in a year is 0.8.
\begin{enumerate}
\item (i) Let X(t) denote the state, either 1, 2, 3 or 4, of the policyholder in year t.
Explain why 1 { ( )}t X t 
 is not a Markov chain. 
\item (ii) (a) By increasing the number of states, define a new stochastic process
1 { ( )}t Y t 
 which is Markov and is such that Y(t) indicates the discount
level for the policyholder in year t.
(b) Write down the transition matrix for the Markov chain {Y(t)}t 1. 

(c) Calculate the long-run probability that the motorist is in discount
level 3.
\end{enumerate}
\newpage
%%%%%%%%%%%%%%%
6 (i) {X(t)} is not Markov because, for example, P[Xt+1 = 2Xt = 3, Xt1 = 2, …]
cannot be reduced to P[Xt+1 = 2Xt = 3].
(ii) (a) Define new states
3a = Level 3 this year following Level 2 last year
3b = Level 3 this year following Level 4 last year
(b) The transition matrix is then
1 2 3a 4 3b
1 0.2 0.8 0 0 0
2 0.2 0 0.8 0 0
3a 0 0.2 0 0.8 0
4 0 0 0 0.8 0.2
3b 0.2 0 0 0.8 0
(c) We have
\begin{eqnarray*}
\pi_1 &=& 0.2\pi_1 + 0.2\pi_2 + 0.2\pi_3b = \pi_1 = 0.25\pi_2 + 0.25\pi_3b\\
\pi_2 &=& 0.8\pi_1 + 0.2\pi_3a  \pi_2 = 0.25(\pi_3a + \pi_3b)\\
\pi_3a &=& 0.2\pi_3b + 0.2\pi_3a  \pi_3a = 0.25\pi_3b\\
\pi_3b &=& 0.2\pi_4\\
\end{eqnarray*}
Applying the condition i = 1, we obtain the solution as
(\pi_1, \pi_2, \pi_3a, \pi_4, \pi_3b) = 1
441
(21, 20, 16, 320, 64).
It follows that the long-run proportion of time spent in Level 3 is
(16 + 64)/441 = 80/441.
% A large number of candidates fared well on this question. Some committed the mostly harmless error of including too many (redundant) states. Most people had a reasonable attempt at (ii)(c) too, although there were some errors of arithmetic or
% matrix multiplication.

\end{document}
