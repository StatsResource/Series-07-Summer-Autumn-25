\documentclass[a4paper,12pt]{article}

%%%%%%%%%%%%%%%%%%%%%%%%%%%%%%%%%%%%%%%%%%%%%%%%%%%%%%%%%%%%%%%%%%%%%%%%%%%%%%%%%%%%%%%%%%%%%%%%%%%%%%%%%%%%%%%%%%%%%%%%%%%%%%%%%%%%%%%%%%%%%%%%%%%%%%%%%%%%%%%%%%%%%%%%%%%%%%%%%%%%%%%%%%%%%%%%%%%%%%%%%%%%%%%%%%%%%%%%%%%%%%%%%%%%%%%%%%%%%%%%%%%%%%%%%%%%

\usepackage{eurosym}
\usepackage{vmargin}
\usepackage{amsmath}
\usepackage{graphics}
\usepackage{epsfig}
\usepackage{enumerate}
\usepackage{multicol}
\usepackage{subfigure}
\usepackage{fancyhdr}
\usepackage{listings}
\usepackage{framed}
\usepackage{graphicx}
\usepackage{amsmath}
\usepackage{chngpage}

%\usepackage{bigints}
\usepackage{vmargin}

% left top textwidth textheight headheight

% headsep footheight footskip

\setmargins{2.0cm}{2.5cm}{16 cm}{22cm}{0.5cm}{0cm}{1cm}{1cm}

\renewcommand{\baselinestretch}{1.3}

\setcounter{MaxMatrixCols}{10}

\begin{document}
\begin{enumerate}

%%--- Question 4 
\item (i) Given a pseudo-random number U uniformly distributed over [0,1], obtain an expression in terms of $U$ and $\theta$  for a non-negative pseudo-random variable $X$ which has density function
f (x) = e x 
 
\item (ii) A sequence of simulated observations is required from the density function
( )= ( ) , 0
1
e x g x k x
x

 

where $\theta$ is a non-negative parameter and $k(\theta )$ is a constant of integration not
involving $x$.
(a) Describe a procedure that applies the Acceptance-Rejection method to obtain the required observations.
(b) Derive an expression involving $\theta$  and $k(\theta )$ for the expected number of pseudo-random variables required to generate a single observation
from the density g using this method.
\end{enumerate}


%%%%%%%%%%%%%%%%%%%%%%%%
\newpage
4 (i) Let X =  (log U)/. Then, for x > 0,
P(X > x) = P(log U < x) = exp(x).
Differentiating, f (x) e x 
  , as required.
%%---Subject 103 (Stochastic Modelling) — April 2003 — Examiners’ Report
Page 4
Alternatively: use the inverse distribution function method.
0
F(x) x e ydy 1 e x    
    . We need to invert this: set $U = F(X)$ and express $X$ as a function of U. This gives X =   1 log (1U).

(ii) (a) Use f(x) =  exp(x) as the base density. We need to find a constant $C$ such that ( )
1
x
k e C e x
x


  

for all x > 0.
C = k()/ is the best that we can do.
%%%%%%%%%%%%%%%%%%%%%%%%%%%%%%%%%%%%%%%%%%%%%%%%%%%%
The procedure is:
\begin{enumerate}
\item Generate a value y from the density f(x) =  exp(x).
\item Take another Uniform pseudo-random variable U2; if this is less than g(y)/(C f(y)) [which is equal to 1/(1 + y)] then we accept the
value y, otherwise reject it and return to 1.
\end{enumerate}
(b) On average it takes C repetitions of steps 1 and 2 to generate a value.
Each such repetition requires two uniform pseudo-random variables.
So the answer is 2 k()/.

%% Most candidates did well here, although a few people inverted the density function instead of the distribution function.
%% Some experienced real difficulty in providing a clear statement of the algorithm for
%% Acceptance-Rejection method. A lot of people answered this part in abstract without realising that they should use the density from part (i) as the base density, and hence didn’t get the marks for calculating C.

\end{document}
