
9 In a simple discrete-time model for the price of a share, the change in price at time t,
Xt, is assumed to be independent of anything that has happened before time t and to
have distribution:
Xt =
1 with probability
1 with probability ,
0 with probability = 1
p
q
r p q

 
  
where p, q, r > 0.
Let S0 = m (where m is a positive integer) be the original price of the share,
Sn = S0 + 1
n
t t X
  be the price after n time units and define
Yn = (q / p)Sn
(i)  Show that {Yn : n  1} is a martingale and that for any positive integer n,
E(Yn) = (q/p)m [4]
(ii)  Let T be the time until the share price reaches either 0 or N for the first time,
where N is an integer greater than m.
(a) Show that 
Yn
  C for n  T, for some constant C.
(b) Write down an expression for E(YT).
[2]
(iii) Assuming that p  q, calculate the probability that, starting from m, the share
price reaches 0 before it reaches N, i.e. P(ST = 0 
S0 = m). [3]
(iv) In the case p = q you may assume that P(ST = 0 
S0 = m) = (Nm)/N. Now
define Zn = Sn2  2np.
(a) Prove that {Zn : n  1} is a martingale.
(b) Show that the expected value of the time until absorption, T, is given
by
E(T
S0 = m) = ( ).
2
m N m
p

[5]
[Total 14]
103 A2003—8

9 (i) We need to prove that E(Yn+1X1, X2, …Xn) = Yn.
We have
E(Yn+1X1, X2, …Xn) = 1 (( / ) n n 1, 2,... ) S X
E q p  X X Xn
 
= 1 ( / ) n (( / ) n 1, 2,... ) S X
q p E q p  X X Xn
Subject 103 (Stochastic Modelling) — April 2003 — Examiners’ Report
Page 10
= ( / )Sn ( ( / ) ( / ) 1 (1 )) ,
q p p q p q q p p q Yn      
which shows that Yn is a martingale.
By taking expectations, we obtain that E(Yn+1) = E(Yn), and this in turn implies
that E(Yn) = E(Y0) = (q/p)m.
(ii) (a) It is clear that T is a stopping time with respect to this martingale, since
the event that the share has reached 0 or N by time n depends only on
X1, X2, …, Xn and m.]
For n  T, we have 0  Sn  N,
Yn = (q / p)Sn  (q / p)N if q > p, or Yn  1 if q  p.
(b) The conditions of the optional stopping theorem are satisfied. It
follows that E(YT) = E(Y0) = (q/p)m.
(iii)
0
= ( )= ( = ) ( =0) 1 1 ( = ).
m N N
T T T T
q E Y q P S N q P S q P S N
p p p p
         
               	  	  	   	   	
Thus ( = ) = 1 ( / )
1 ( / )
m
T N
P S N q p
q p


so that ( = 0) = ( / ) ( / ) .
1 ( / )
m N
T N
P S q p q p
q p


(iv) (a) We have
E(Zn+1X1, X2, …, Xn) = 2 2
E(Sn 2SnXn 1 Xn 1 2(n 1) p X1, X2,..., Xn )       
= 2 2
Sn 2(n 1) p 2SnE(Xn 1) E(Xn 1).  
   
But E(Xn+1) = 0, 2
E(Xn 1)  = p + q = 2p, and the last equation gives
E(Zn+1X1, X2, …, Xn) = Sn2  2np = Zn , so that {Zn} is a martingale.
(b) The conditions of the optional stopping theorem are not satisfied, since
2 2 Sn np  is not bounded below for 0  n  T. But we can work with a
truncated stopping time TK = min(T, K), for which the conditions of the
OST are satisfied, then let K  .
Applying the optional stopping theorem we get
E(ZT) = E(Z0) = m2.
Subject 103 (Stochastic Modelling) — April 2003 — Examiners’ Report
Page 11
But E(ZT) = E(ST2 ) 	2pE(T) = N2P(ST = N)  2p E(T) and we are
given that P(ST = N) = m/N. Thus m2 = E(ZT) = Nm  2pE(T), which
implies that E(T) = m(N  m)/(2p), as required.
Parts (i) and (iv)(a) were generally answered successfully.
Part (ii): in (a), the bounds on (q/p)Sn must be shown to apply for any p and q, not
just some; the main problem with (b) was that many did not mention the Optional
Stopping Theorem as justification for claiming that E(YT) = E(Y0).
Of the candidates who attempted (iv)(b), very few noticed that the stopping time did
not meet the requirements of the OST, so needs to be truncated.
