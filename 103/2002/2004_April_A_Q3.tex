\documentclass[a4paper,12pt]{article}

%%%%%%%%%%%%%%%%%%%%%%%%%%%%%%%%%%%%%%%%%%%%%%%%%%%%%%%%%%%%%%%%%%%%%%%%%%%%%%%%%%%%%%%%%%%%%%%%%%%%%%%%%%%%%%%%%%%%%%%%%%%%%%%%%%%%%%%%%%%%%%%%%%%%%%%%%%%%%%%%%%%%%%%%%%%%%%%%%%%%%%%%%%%%%%%%%%%%%%%%%%%%%%%%%%%%%%%%%%%%%%%%%%%%%%%%%%%%%%%%%%%%%%%%%%%%

\usepackage{eurosym}
\usepackage{vmargin}
\usepackage{amsmath}
\usepackage{graphics}
\usepackage{epsfig}
\usepackage{enumerate}
\usepackage{multicol}
\usepackage{subfigure}
\usepackage{fancyhdr}
\usepackage{listings}
\usepackage{framed}
\usepackage{graphicx}
\usepackage{amsmath}
\usepackage{chngpage}

%\usepackage{bigints}
\usepackage{vmargin}

% left top textwidth textheight headheight

% headsep footheight footskip

\setmargins{2.0cm}{2.5cm}{16 cm}{22cm}{0.5cm}{0cm}{1cm}{1cm}

\renewcommand{\baselinestretch}{1.3}

\setcounter{MaxMatrixCols}{10}

\begin{document}

%%%%%%%%%%%%%%%%%%%%%%%%%%%%%%%%%%%%%%%%%%%%%%%%%%%%%%%%%%%%%%%%%%%%%%%%
3 Consider the accident proneness model in which the cumulative number of accidents
Xt suffered by a driver is a Markovian birth process with linear transition rates given
by
i,j =
( 1) if = 1
0 otherwise
j i i    
(i) Denoting by ai(t) the probability that a driver who has had no accidents at
time 0 has had at least i accidents by time t, explain why
ai(t + dt) = ai(t) + (ai1(t)  ai(t))i
dt + o(dt)
as dt  0 and hence derive a differential equation satisfied by ai(t). [3]
(ii) Show that
ai(t) = (1  et)i
and deduce the value of P[X(t) = i|X(0) = 0]. [4]
103 S2002—3 PLEASE TURN OVER
(iii) A colleague suggests that a better model would involve transition rates i,i+1(t)
which are dependent on t as well as on i. Comment on this suggestion. [1]
(iv) The same colleague proposes the model i,i+1(t) = (i 1)
t
 . Comment on the
proposed model. [1]
[Total 9]


%%%%%%%%%%%%%%%%%%%%%%%%%%%%%%

3 (i) If the driver is to have had at least i accidents by time t + dt, either there must
have been i accidents by time t or there must have been exactly i 	 by time t
and another between t and t + dt.
P(exactly i 1) = P(at least i 1) P(at least i).   
Therefore
dai
dt
= i
(ai1  ai).
(ii) Verification: d
dt
{(1  et)i}= i
et(1  et)i1 .
ai1  ai = (1  et)i1(1 [1  et]).
We should also verify that ai(0) is correct: the value should be 0 for i > 0,
which it is.
Then, defining Ti as the time the process first hits i, we have
P0,i(t) = P{Ti t} P{Ti 1 
    t} = (1  et)i  (1  et)i+1 = et(1 et)i
(iii)  Probably a good suggestion. A driver who has had 2 accidents in 10 years is
less likely to have another than a driver who has had 2 accidents in a month.
(iv) The proposed model does address the issue raised in (iii) but leads to a very
high accident rate when t is close to 0, so is unsuitable.
Many candidates attempted part (i), but were unable to give a sufficiently clear description to
convincingly display their understanding and so did not earn full marks. In some cases,
Subject 103 (Stochastic Modelling) — September 2002 — Examiners’ Report
Page 5
candidates did not correctly interpret the definition of ai(t). In part (ii) many candidates tried
to solve the differential equation, where they could instead have simply shown the solution
given to be valid. Candidates generally came up with sensible suggestions for part (iii), but
only a few candidates were able to make suitable comments on part (iv).
