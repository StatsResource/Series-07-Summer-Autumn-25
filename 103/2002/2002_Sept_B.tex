\documentclass[a4paper,12pt]{article}

%%%%%%%%%%%%%%%%%%%%%%%%%%%%%%%%%%%%%%%%%%%%%%%%%%%%%%%%%%%%%%%%%%%%%%%%%%%%%%%%%%%%%%%%%%%%%%%%%%%%%%%%%%%%%%%%%%%%%%%%%%%%%%%%%%%%%%%%%%%%%%%%%%%%%%%%%%%%%%%%%%%%%%%%%%%%%%%%%%%%%%%%%%%%%%%%%%%%%%%%%%%%%%%%%%%%%%%%%%%%%%%%%%%%%%%%%%%%%%%%%%%%%%%%%%%%

\usepackage{eurosym}
\usepackage{vmargin}
\usepackage{amsmath}
\usepackage{graphics}
\usepackage{epsfig}
\usepackage{enumerate}
\usepackage{multicol}
\usepackage{subfigure}
\usepackage{fancyhdr}
\usepackage{listings}
\usepackage{framed}
\usepackage{graphicx}
\usepackage{amsmath}
\usepackage{chngpage}

%\usepackage{bigints}
\usepackage{vmargin}

% left top textwidth textheight headheight

% headsep footheight footskip

\setmargins{2.0cm}{2.5cm}{16 cm}{22cm}{0.5cm}{0cm}{1cm}{1cm}

\renewcommand{\baselinestretch}{1.3}

\setcounter{MaxMatrixCols}{10}

\begin{document}
\begin{enumerate}

4 An investor believes that the price of gold increases when the volatility of the equities
market is high and decreases when the volatility is low. The investor therefore wishes
to model the price of gold Xt, as an Itô process defined by dXt = VtdBt + 2
Vt dt , where
Bt is standard Brownian motion and Vt is a measure of market volatility calculated
from the equity price information available at time t.
(i) Comment briefly on the suitability of this model, mentioning in particular its
behaviour when Vt is large and when Vt is small. [2]
(ii) (a) State Itô’s Lemma.
(b) Use Itô’s Lemma to find an expression for dMt, where Mt = e 2Xt  .
(c) Deduce that Mt is a martingale. [5]
(iii) Jensen’s Inequality implies that
e 2E(Xt ) E(e 2Xt ).   
Show that E(Xt )  X0 whatever the value of X0 and comment briefly again
on the suitability of the model. [2]
[Total 9]
103 S2002—4
5 (i) The classification of stochastic models according to discrete or continuous
time variable, discrete or continuous state space gives rise to a four-way
classification. Give four examples, one of each type, of stochastic models
which may be used to model observed processes. [2]
(ii) For each of the following observed processes, identify a type of model which
could be used to model the process, stating which features of the process lead
you to make this choice:
(a) a monthly index of food prices
(b) an index of prices of shares on the London Stock Exchange, constantly
updated
(c) the status (active, retired, dead) of a member of a pension scheme [3]
(iii) (a) Discuss briefly the importance of model verification.
(b) A simple random walk model, whose single parameter  is the
probability of an upwards jump, is to be fitted to a set of observations
{x1, x2, …, xn} which have the property that |xi+1  xi| = 1 for each 1 
 i
< n. Write down the estimate for the parameter of the model and
describe one test which you would use in the process of model
verification. [5]
[Total 10]


%%%%%%%%%%%%%%%%%%

4 (i) When volatility is high, dXt is strongly upwards; when volatility is low, dXt is
close to zero.
The first of these fits the assumptions, whereas for the second something more
negative would be preferable. (It looks as though the process X has more
opportunity to increase than to decrease.)
(ii) (a) Itô (time-independent case): if dXt = Ytdt + ZtdBt then
df(Xt) =
2
2
( t t t ) ½ 2 t .
df Y dt Z dB d f Z dt
dx dx
  or simply
’( ) ½ ’’( )( )2 , t t t t f X dX  f X dX with an explanation of what is meant by (dXt)2.
Equally acceptable is the time-dependent version:
df(Xt,t) =
2
2
2 ( t t t ) ½ t .
f dt f Y dt Z dB f Z dt
t x x
  
  
  
(b) Here
df(Xt) = f (Xt )dXt ½ f (Xt )(dXt )2
= f (Xt )[YtdBt Yt2dt]½ f (Xt )Yt2dt
= f (Xt )YtdBt Yt2[ f (Xt ) ½ f (Xt )]dt.
In this instance, df(Xt) = 2 2Xt .
e VtdBt 

(c) This is a martingale by the disappearance of the dt term, as E(dBt | Ft) = 0.
Alternatively,
e 2XT  = 2 0 2
0
2 t ,
X T X
e e VtdBt  
 
which is a martingale.
(iii) 2 0 2 2 ( )
= ( 0 ) = ( ) = ( t ) t . X X EX
e E M E Mt E e e     It follows that E(Xt )  X0 ,
whatever the initial state.
Subject 103 (Stochastic Modelling) — September 2002 — Examiners’ Report
Page 6
This confirms the initial suggestion that downward movements are too
unlikely in comparison to upward ones.
The question generally showed good attempts, with a variety of different but correct versions
of Itô’s Lemma being given. The only real problem evident here was in part (iv), where many
candidates failed to use the given inequality to prove the point.
5 (i) Discrete state space, discrete time: Markov chain, simple random walk,
anything like that;
Discrete state space, continuous time: Poisson process, Markov jump process;
Continuous state space, discrete time: time series, general random walk;
Continuous state space, continuous time: Itô process, Brownian motion, etc.
(ii) (a) by definition, is monthly. Therefore a continuous time variable is not
appropriate. Something like a time series would do, containing an element of
Autoregression.
(b) functions in continuous time; Brownian motion, Geometric BM or any
kind of diffusion or Itô process would be a suitable candidate.
(c) has a discrete state space. It would be possible to review the member’s
status only once a year, say in which case a Markov chain would fit, but in the
absence of a remark about frequency of membership status review a
continuous-time model would seem more appropriate; a Markov jump process
is what we might look for.
(iii) (a) Once a model has been decided upon, parameters may be estimated by
standard methods. But it is necessary to check that the model, with
parameters given by their estimated values, has sample paths which
resemble the data actually observed, otherwise incorrect inferences can
be drawn.
(b) Let yi = xi  xi1 = xi and let nu be the number of time yi = 1. The
parameter,  (or p) is the probability of an up-jump, estimated as
ˆ = nu / (n 1).
We need to check that the yi form a sequence of i.i.d. variables. A test
based on chi-squared is not appropriate except as indicated below.
The test to apply is one which determines whether there is significant
clustering of the +1’s and the 1’s, as opposed to their being randomly
scattered through the sample: the runs test on the yi, or equivalently the
turning points test on the xi, would be fine; a test based on the sample
autocorrelation function or a contingency table (yi = 1 or +1 as the
Subject 103 (Stochastic Modelling) — September 2002 — Examiners’ Report
Page 7
row labels, yi+1 = 1 or +1 as the column labels) would also be
acceptable.
In part (i) many candidates gave specific examples of observable processes, rather than
describing the stochastic models themselves as requested in the question; some credit was
however given for these cases. Part (ii) and (iii) (a) generally had good answers. In part
(iii) (b) many candidates incorrectly suggested that a chi-squared test should be used,
