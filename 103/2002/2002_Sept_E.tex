\documentclass[a4paper,12pt]{article}

%%%%%%%%%%%%%%%%%%%%%%%%%%%%%%%%%%%%%%%%%%%%%%%%%%%%%%%%%%%%%%%%%%%%%%%%%%%%%%%%%%%%%%%%%%%%%%%%%%%%%%%%%%%%%%%%%%%%%%%%%%%%%%%%%%%%%%%%%%%%%%%%%%%%%%%%%%%%%%%%%%%%%%%%%%%%%%%%%%%%%%%%%%%%%%%%%%%%%%%%%%%%%%%%%%%%%%%%%%%%%%%%%%%%%%%%%%%%%%%%%%%%%%%%%%%%

\usepackage{eurosym}
\usepackage{vmargin}
\usepackage{amsmath}
\usepackage{graphics}
\usepackage{epsfig}
\usepackage{enumerate}
\usepackage{multicol}
\usepackage{subfigure}
\usepackage{fancyhdr}
\usepackage{listings}
\usepackage{framed}
\usepackage{graphicx}
\usepackage{amsmath}
\usepackage{chngpage}

%\usepackage{bigints}
\usepackage{vmargin}

% left top textwidth textheight headheight

% headsep footheight footskip

\setmargins{2.0cm}{2.5cm}{16 cm}{22cm}{0.5cm}{0cm}{1cm}{1cm}

\renewcommand{\baselinestretch}{1.3}

\setcounter{MaxMatrixCols}{10}

\begin{document}
\begin{enumerate}
9 Let {xt: 1 
 t 
 n} be a time series to which an ARMA(1, 1) model is to be fitted.
(i) Write down the defining equation of an ARMA(1, 1) process, identifying the
parameters. [2]
(ii) (a) Outline the Method of Moments parameter estimation technique.
(b) State the assumptions underlying the Maximum Likelihood parameter
estimation technique.
(c) Describe briefly the technique known as backforecasting and explain
why it, or something similar, is necessary in this case. [5]
(iii) The model fitted to the data is
xt = 5.67 + 0.61xt1 + et 0.23et1
The most recently observed value in the series is x20 = 8.2, with estimated
residual eˆ20 = 1.38.
(a) Evaluate estimates xˆ20 (1) and xˆ20 (2) for x21 and x22.
(b) The simplest form of the method of exponential smoothing used at
time 19 gave a forecast for x20 of 8.37. Assuming the smoothing
parameter is equal to 0.2, find the forecast of x21.
(c) Give an example of a circumstance in which a form of exponential
smoothing might be expected to outperform Box-Jenkins forecasting in
the prediction of future values of the time series. [5]
[Total 12]
103 S2002—8





9 (i) Xt =  + 
(Xt1  ) + et + et1 is the equation.
The parameters are 
 (the autoregressive parameter),  (the moving average
parameter),  (the mean level) and 	 (the innovation standard deviation).
(ii) (a) Calculate theoretical ACF 
1, 
2 of ARMA(1, 1) in terms of 
 and .
Find sample ACF r1, r2 from the data. The required estimates are the
values of 
 and  which ensure that 
1 = r1 and 
2 = r2.
The value of 	
2 is estimated using for example the calculated value of
0 and the sample variance.
(b) The assumptions are that the en are independent and Normally
distributed.
(c) MLE in this case tries to minimise 2
et . Now et can be expressed as
a function of xt, xt1 and et1. et1 is unknown, but can be expressed as
a function of xt1, xt2 and et2, etc. We need to estimate a suitable
value of e0.
This is done iteratively: assume that e0 = 0 and estimate parameters on
that basis; then use forecasting techniques on the time-reversed process
{xn, xn1, …, x1} to gain a more accurate estimate of e0; repeat this
process until everything converges.
(iii) (a) xˆ20 (1) = 5.67 + 0.61(8.2) + 0  0.23(1.38) = 10.99.
xˆ20 (2) = 5.67 + 0.61(10.99) + 0  0 = 12.37.
(b) For exponential smoothing the equation is
xˆ20 (1) = xˆ19 (1)  (x20  xˆ19 (1)) = 8.37 + 0.2(0.17) = 8.34.
(c) A variety of exponential smoothing might be better if the mean
changes by some means other than a linear trend, or if there is
multiplicative seasonal variation.
In part (i) a number of candidates omitted σ from the list of parameters. Part (ii) (a) showed
some good answers, however in most cases insufficient details was provided to earn full
marks. Part (ii) (c) was very poorly answered, with very few candidates showing that they
understood the concept of backforecasting. In part (iii) most candidates understood what
was generally required, though in some cases there were errors in applying the formulae.
