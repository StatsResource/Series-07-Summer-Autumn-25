
\documentclass[a4paper,12pt]{article}

%%%%%%%%%%%%%%%%%%%%%%%%%%%%%%%%%%%%%%%%%%%%%%%%%%%%%%%%%%%%%%%%%%%%%%%%%%%%%%%%%%%%%%%%%%%%%%%%%%%%%%%%%%%%%%%%%%%%%%%%%%%%%%%%%%%%%%%%%%%%%%%%%%%%%%%%%%%%%%%%%%%%%%%%%%%%%%%%%%%%%%%%%%%%%%%%%%%%%%%%%%%%%%%%%%%%%%%%%%%%%%%%%%%%%%%%%%%%%%%%%%%%%%%%%%%%

\usepackage{eurosym}
\usepackage{vmargin}
\usepackage{amsmath}
\usepackage{graphics}
\usepackage{epsfig}
\usepackage{enumerate}
\usepackage{multicol}
\usepackage{subfigure}
\usepackage{fancyhdr}
\usepackage{listings}
\usepackage{framed}
\usepackage{graphicx}
\usepackage{amsmath}
\usepackage{chngpage}

%\usepackage{bigints}
\usepackage{vmargin}

% left top textwidth textheight headheight

% headsep footheight footskip

\setmargins{2.0cm}{2.5cm}{16 cm}{22cm}{0.5cm}{0cm}{1cm}{1cm}

\renewcommand{\baselinestretch}{1.3}

\setcounter{MaxMatrixCols}{10}

\begin{document}
\begin{enumerate}
PLEASE TURN OVER7
(i)
(a)
Calculate the autocovariance function {γ k : k ≥ 0} and autocorrelation function {ρ k : k ≥ 0} of a first-order Moving
Average process
\[X t = μ + e t + β 1 e t−1 ,\]
where {e t : t ≥ 0} is a sequence of uncorrelated, zero-mean random
variables with common variance $\sigma^2_e$.
(b)
(ii)
State the conditions on the values of the parameters such that the process is invertible.

A sequence of observations x 1 , x 2 , ..., x n has sample variance γ $ 0 = 14.5, sample lag-1 autocovariance γ $ 1 = 5.0. Show that there is more than one first-order moving average process which can be fitted to these data,
but verify that only one of the fitted processes is invertible.

%%%%%%%%%%%%%%%%%%%%%%%%%%%%%%%%%%%%%%%%%%%%%
% 8
The members of a health insurance scheme are classified as contributors or beneficiaries; a member who is a contributor in one period becomes a beneficiary in the next period if he or she becomes seriously ill, and this happens with probability 0.1. The probability of a serious illness continuing into the next period is 0.2. The rules of the scheme specify that any member
who is a beneficiary for three successive periods must become a contributor for the next period; if the illness still persists the member may thereafter revert to being a beneficiary.
%%%%%%%%%%%%%%%%%%%%%%%%%%%%%%%%%%%%%%
\begin{enumerate}[(i)]
(i)
(a) Construct a discrete time Markov chain to model this health scheme, introducing if necessary various classes of beneficiaries
and contributors (a five state model is suggested).

(b) Draw the transition graph.
(c) Write down the transition matrix of the chain.
(ii) Explain whether the above Markov chain is irreducible, periodic or both.

\item (iii) (a)
Calculate the stationary probability distribution of the chain.
(b)
Determine the proportion of beneficiaries among the membership in the stationary régime.
\item (iv)
Let b be the average gross payout per beneficiary and c the average gross payout per contributor per period; this means that the nett
payments are $b − f$ and $c − f$ respectively, where $f$ is the membership fee per period (assumed to be uniform over members and over time).
(a) Explain how b, c and f should be related if the scheme is to be viable.
(b) Calculate the average profit per period per member in the stationary régime if b = 600, c = 150 and f = 300.
\end{enumerate}
%%%%%%%%%%%%%%%%%%%%%%%%%%%%%%%%%%%%%%%%%%%%%%%%%%%
7
(i)
(a)
\[g 0 = Var(e t + b 1 e t-1 ) = ( 1 + b 12 ) s e 2 \]and 
\[g 1 = Cov(e t + b 1 e t-1 , e t-1 + b 1 e t-2 )
= b 1 s 2 e , with g k = 0 for k > 1.\]

This gives r 0 = 1, r 1 = b 1 / (1 + > 12 ), r k = 0 otherwise.
(b)
%------------------%
Invertibility requires that 1⁄2b 1 1⁄2 < 1, so that the sum X t - b 1 X t-1 +
b 12 X t- 2 + ... converges. m and s e are irrelevant.
We need to solve ( 1 + b 12 ) s e 2 = 14.5, b 1 I 2 e = 5.0. Eliminating s 2 e , we
(ii)
have 1 + b 12 = 2.9b 1 , or b 1 = 1⁄2(2.9 ± 2 . 9 2 - 4 ) = 2.5 or 0.4.
b 1 = 2.5 corresponds to I 2 e = 2, whereas b 1 = 0.4 corresponds to I 2 e =
12.5.
For invertibility, solve 1 + b 1 z = 0. In the first case, z = -0.4 (no
good); in the second, z = -2.5 (OK).

%%%%%%%%%%%%%%%%%%%%%%%%%%%%%%%%%%%%%%%%%%%%%%%%%%%%%%%%%%%%%%%%%%%%%%%%%%%%%%%%%%%%%%%%
8
(i)
(a)
States:
C: healthy contributor
C ¢: contributor but ill
B 1 , B 2 , B 3 : beneficiary, with index giving duration of illness
(b)
Transition
graph:
0.9
C
0.1
0.8
0.8
B 1
0.8
B 2
0.2
0.8
0.2
B 3
0.2
C ¢
0.2
%%%%%%%%%%%%%%%%%%%%%%%%%%%%%%%%%%%%%%%%%%%%%%%%%%%%%%%%%%%%%%%%%%
(c)
Transition matrix (states ordered C, C ¢, B 1 , B 2 , B 3 ):
F 0 . 9
G 0 . 8
G 0 . 8
G H 0 0 . . 8 8
P =
0 01
. 0
0 0 . 2 0
0
0 0
0 0 . 2
0
0 . 2 0 0
I
0 J
J
0 J
0 . 2 J
J
0 K
0
(ii) The chain is irreducible by inspection: every state is accessible from every other state. State C is clearly aperiodic because of the one-step loop from C to C; because of irreducibility, every other state must be aperiodic too.
(iii) (a)
p = pP reads
p c = 0.9p c + 0.8(p c¢ + p 1 + p 2 + p 3 )
p c¢ = 0.2p 3
p 1 = 0.1p c + 0.2p c¢
p 2 = 0.2p 1
p 3 = 0.2p 2 .
Discard first equation and choose p c¢ as working variable:
p 3 = 1
p c¢ = 5p c¢
0 . 2
p 2 = 1
p 3 = 5p 3 = 25p c¢
0 . 2
\ p = p c¢ (1248, 1, 125, 25, 5)
p 1 = 1
p 2 = 5p 2 = 125p c¢
0 . 2
p c = 1
0 . 2
p 1 -
p c¢ = 10p 1 - 2p c¢ = 1248p c¢ .
01
.
01
.
Find p c¢ by normalisation: p c¢ (1248 + 1 + 125 + 25 + 5) = 1,
\ p c¢ =
(b)
Page 6
1
.
1404
Proportion of beneficiaries:
125 + 25 + 5
= 11.04%.
1404

(iv)
(a)
Average profit per period per member in stationary régime is
Z = (f - c)
= f - c
F G 1248 + 1 I J + (f - b) F G 125 + 25 + 5 I J
H 1404 K
H 1404 K
1249
155
- b
.
1404
1404
For Z > 0 you need f > c
(b)
With the given data
Z = 300 -
\end{document}
7
