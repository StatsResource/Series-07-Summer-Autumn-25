\documentclass[a4paper,1pt]{article}

%%%%%%%%%%%%%%%%%%%%%%%%%%%%%%%%%%%%%%%%%%%%%%%%%%%%%%%%%%%%%%%%%%%%%%%%%%%%%%%%%%%%%%%%%%%%%%%%%%%%%%%%%%%%%%%%%%%%%%%%%%%%%%%%%%%%%%%%%%%%%%%%%%%%%%%%%%%%%%%%%%%%%%%%%%%%%%%%%%%%%%%%%%%%%%%%%%%%%%%%%%%%%%%%%%%%%%%%%%%%%%%%%%%%%%%%%%%%%%%%%%%%%%%%%%%%

\usepackage{eurosym}
\usepackage{vmargin}
\usepackage{amsmath}
\usepackage{graphics}
\usepackage{epsfig}
\usepackage{enumerate}
\usepackage{multicol}
\usepackage{subfigure}
\usepackage{fancyhdr}
\usepackage{listings}
\usepackage{framed}
\usepackage{graphicx}
\usepackage{amsmath}
\usepackage{chngpage}

%\usepackage{bigints}
\usepackage{vmargin}

% left top textwidth textheight headheight

% headsep footheight footskip

\setmargins{.0cm}{.5cm}{16 cm}{cm}{0.5cm}{0cm}{1cm}{1cm}

\renewcommand{\baselinestretch}{1.}

\setcounter{MaxMatrixCols}{10}

\begin{document}

\begin{enumerate}
\item


%%%%%%%%%%%%%%%%%%%%%%%%%%%%%%%%%%%%%
10
(i) State the Lévy decomposition theorem which describes the constituent parts of
a Lévy process.
[]
(ii) Let M t = exp( - ab + bB t - b  t), where B t is standard Brownian motion and
where a and b are positive constants. Define T as the first time that M t = 1,
with the definition T = ¥ if M never hits the point 1. You may assume that
M t ® 0 as t ® ¥ , so that
ì if M hits 1
M T = í 1
î 0 if M never hits 1
(a) Show that M is a martingale and that 0 £ M t £ 1 for all 0 £ t £ T.
(b) State the optional stopping theorem and use it to prove that the
probability that the Brownian motion B t ever hits the line a + bt is
e - ab .
(iii)
[6]/
An insurance company earns premium income at a constant rate c per unit
time. Claims arrive according to a Poisson process with rate l ; each claim
may be assumed to be of fixed size k. Let X t denote the total value of all
claims received up until time t and denote by S t the company’s surplus at
time t,
\[S t = s 0 + ct - X t ,\]
where s 0 is a positive constant. Show that {S t : t  0} is a Lévy process and
identify the components of the Lévy decomposition of S.
[]
(iv) Calculate the expectation and variance of S t .
(v) The company is interested in the probability of ruin, defined as the probability
that S t ever goes below 0. An investigator proposes using a Brownian model
[]
S t * = s 0 + m t + s B t ,
where B t is a standard Brownian motion, as an approximation to the surplus
process S t .
(a) Calculate the appropriate values of m and s .
(b) State the significance of the condition c > k l in this situation.
(c) Write down the probability that the approximating process S* ever hits
0 assuming that c > k l .
(d) Outline the principal difference between S and S* and state whether
you consider that the probability obtained in (c) would be an
acceptable approximation to the probability of ruin.
[7]
[Total 0]
10 A00—10

%%%%%%%%%%%%%%%%%%%%%%%%%%%%%%%
10
(i) A Lévy process Y t can be decomposed as Y t = y 0 + m t + s B t + N t , where m t is a
deterministic component, s B t a continuous random component (Brownian
motion) and N t a purely discontinuous component which may be regarded as a
compound Poisson process, independent of the Brownian component.
(ii) (a)
E ( M t + s 1⁄F t ) = exp( -  ab -  b  ( t + s ) +  bB t ) E ( e  b ( B t + s - B t ) 1⁄F t ). The
independent increment property implies that this conditional
expectation is exp( b  s ).
Therefore E ( M t + s 1⁄F t ) = exp( -  ab -  b  ( t + s ) +  bB t +  b  s ) = M t .
Ought to check that E (| M t |) < ¥ . Since M t  0, E (| M t |) = E ( M t ) = e -ab .
M t  0 by definition. Since B t is continuous, it follows that M t is
continuous. Further, M 0 < 1. Therefore M t cannot exceed 1 without
first passing through 1, which does not happen until time T .
(b)
The optional stopping theorem states that, for any martingale Y and stopping time T adapted to the same filtration, EY T = Y 0 if T is bounded
or Y is bounded or Y t Ù T is bounded.
The last of these conditions holds in this case.
We conclude that P ( B hits a + bt ) = P ( M hits 1) = E ( M T ) = M 0 = e -  ab .
(iii) ct is deterministic, - X t is a compound Poisson process with constant jump
height - k and the multiplier of the Brownian component is s = 0.
(iv) Since X t / k ~ P ( l ), we have E ( - X t ) = -l kt , Var( - X t ) = k  l t . Therefore
E ( S t ) = s 0 + ( c - k l ) t , Var( S t ) = k  l t .
Page 10Subject 10 (Stochastic Modelling) — 
%%%%%%%%%%%%%%%%%%%%%%%%%%%%%%%%%%%%%

(v)
(a) In order for the mean and variance to match we require m = c - k l ,
s = k l .
(b) c > k l is the condition for premium income to outstrip outgoings on
average.
(c) s 0 + m t + s B t = 0 if and only if B t = -
s 0 m
- t .
s s
From above, the probability that B ever hits the line a + bt is e -  ab .
Therefore the required approximation to the probability of ruin is
æ m s ö
æ ( c - k l ) s 0 ö
exp ç -   0 ÷ = exp ç - 
÷ if c > k l .
s ø
k  l ø
è
è
(d)
The difference is that S has a discontinuous component, whereas S *
approximates this with a continuous one.
The difference will be significant if s 0 is small and c - k l large, as then
S * will be much less likely to hit 0 than S ; in the opposite case the
approximation may be quite reasonable.
Most candidates were not able to attempt every part of this question.
Very few made the connection between part (ii)(b) and part (v)(c).
There seems to be a general lack of confidence when dealing with
martingales or Lévy processes.
Page 11
