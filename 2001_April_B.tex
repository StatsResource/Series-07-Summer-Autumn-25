\documentclass[a4paper,12pt]{article}

%%%%%%%%%%%%%%%%%%%%%%%%%%%%%%%%%%%%%%%%%%%%%%%%%%%%%%%%%%%%%%%%%%%%%%%%%%%%%%%%%%%%%%%%%%%%%%%%%%%%%%%%%%%%%%%%%%%%%%%%%%%%%%%%%%%%%%%%%%%%%%%%%%%%%%%%%%%%%%%%%%%%%%%%%%%%%%%%%%%%%%%%%%%%%%%%%%%%%%%%%%%%%%%%%%%%%%%%%%%%%%%%%%%%%%%%%%%%%%%%%%%%%%%%%%%%

\usepackage{eurosym}
\usepackage{vmargin}
\usepackage{amsmath}
\usepackage{graphics}
\usepackage{epsfig}
\usepackage{enumerate}
\usepackage{multicol}
\usepackage{subfigure}
\usepackage{fancyhdr}
\usepackage{listings}
\usepackage{framed}
\usepackage{graphicx}
\usepackage{amsmath}
\usepackage{chngpage}

%\usepackage{bigints}
\usepackage{vmargin}

% left top textwidth textheight headheight

% headsep footheight footskip

\setmargins{2.0cm}{2.5cm}{16 cm}{22cm}{0.5cm}{0cm}{1cm}{1cm}

\renewcommand{\baselinestretch}{1.3}

\setcounter{MaxMatrixCols}{10}

\begin{document}
\begin{enumerate}
\item 5 Show that the probability generating function for a binomial (n, p) distribution is
GX (t) = (1 − p + pt)n .
Deduce the moment generating function. 
%%%%%%%%%%%%%%%%%%%%%%%%%%%%%%%%%%%%%%%%%%%%%%%%%%%%%%%%%%%%%%%%%%%%%%%%%%%%%%%%%%%%%%%%%%%%%%%%%%%%%%%%%%%%%%%%%%%
\item 6 Let X have a normal distribution with mean μ and standard deviation σ, and let
the ith cumulant of the distribution of X be denoted κi .
Assuming the moment generating function of X, determine the values of κ2 , κ3 ,
and κ4 . 
%%%%%%%%%%%%%%%%%%%%%%%%%%%%%%%%%%%%%%%%%%%%%%%%%%%%%%%%%%%%%%%%%%%%%%%%%%%%%%%%%%%%%%%%%%%%%%%%%%%%%%%%%%%%%%%%%%%
\item 7 The number of policies (N) in a portfolio at any one time is modelled as a Poisson
random variable with mean 10.
The number of claims (Xi) arising on a policy is also modelled as a Poisson
random variable with mean 2, independently for each policy and independent
of N.
Determine the moment generating function for the total number of claims,
1
N
i
i
X
= 
,
arising for the portfolio of policies. 
%%%%%%%%%%%%%%%%%%%%%%%%%%%%%%%%%%%%%%%%%%%%%%%%%
\item Consider two independent lives A and B. The probabilities that A and B die
within a specified period are 0.1 and 0.2 respectively. If A dies you lose £50,000,
whether or not B dies. If B dies you lose £30,000, whether or not A dies.
\begin{enumerate}[(a)]
    \item (i) Calculate the mean and standard deviation of your total losses in the
period.
\item Calculate your expected loss within the period, given that one, and only
one, of A and B dies. 
\end{enumerate}

\end{enumerate}
\newpage

5 PGF for binomial (n,p)
0
( ) ( ) ( )
(1 )
( ) (1 )
n
X x
X
x
x x n x
x nx
G t Et t P X x
n
t p p
x
n
pt p
x
=
−
−
= = =
 
=   −
 
 
=   −
 



= (1 − p + pt)n
MGF obtained by replacing t by et, i.e. ( ) (1 t )n.
X M t = − p + pe
%%%%%%%%%%%%%%%%%%%%%%%%%%%%%%%%%%%%%%%%%%%%%%%%%%%%%%%%%%%%%%%%%%%%%%%%%%%%%%%%%%%%%%%%%%%%%%%%%%%%%%%%%%%%%%%%%%%
6 C(t) = logM(t) = log[exp(μt + σ2t2/2)] from Green Book
= μt + σ2t2/2
κr is the coefficient of tr/r! , r = 2,3,…
∴ κ2 = σ2, κ3 = κ4 = 0
7 Using formulae from the Green book
mgf of N is exp(10(et − 1)) and mgf of Xi  is exp(2(et − 1))
mgf of
1
N
i
i
X
= 
is exp{10[exp(2(et − 1)) − 1]}
%%%%%%%%%%%%%%%%%%%%%%%%%%%%%%%%%%%%%%%%%%%%%%%%%%%%%%%%%%

%%%%%%%%%%%%%%%%%%%%%%%%%%%%%%%%%%%%%%%%%%%%%%%%%%%%%%%%%%%%%%%%%%%%%%%%%%%%%%%%%%%%%%%%%%%%%%%%%%%%%%%%%%%%%%%%%%%
8 (i) Let XA = 1 if A dies and 0 if not. Similarly XB for life B.
E(XA) = 0.1, V(XA) = 0.1 × 0.9 = 0.09;
E(XB) = 0.2, V(XB) = 0.2 × 0.8 = 0.16
Total losses T = 5XA + 3XB (units of £10,000)
so E(T) = (5 × 0.1) + (3 × 0.2) = 1.1 i.e. £11,000
and V(T) = (25 × 0.09) + (9 × 0.16) = 3.69 so SD(T) = 1.921
i.e. £19,210
[OR: T takes values 0, 3, 5, and 8 with probabilities 0.9 × 0.8 = 0.72, 0.2 ×
0.9 = 0.18, 0.1 × 0.8 = 0.08, and 0.1 × 0.2 = 0.02 respectively; hence find
E(T), E(T2), and V(T).]

(ii) P(exactly one of A, B dies) = 0.18 + 0.08 = 0.26
P(A dies|exactly one dies) = P(A dies and B does not die) / P(exactly one dies)
= 0.08/0.26 = 4/13 ; so P(B dies|exactly one dies) = 9/13
So E(loss|exactly one dies) = 5(4/13) + 3(9/13) = 47/13 = 3.615
i.e. £36,150
%%%%%%%%%%%%%%%%%%%%%%%%%%%%%%%%%%%%%%%%%%%%%%%%%%%%%%%%%%%%%%%%%%%%%%%%%%%%%%%%%%%%%%%%%%%%%%%%%%%%%%%%%%%%%%%%%%%
9 (i) The change after a large number of time periods equals the sum of i.i.d.
r.v.’s.
The Central Limit theorem leads to an approximate normal distribution.
(ii)
x : +1 0 −1
p : 0.35 0.35 0.30
E(X) = 0.35 − 0.30 = 0.05
E(X2) = 0.35 + 0.30 = 0.65
∴Var(X) = 0.65 − (0.05)2 = 0.6475 ∴s.d.(X) = 0.8047
Let Y = price change after 500 periods
∴Y has mean 500(0.05) = 25 and s.d. 500(0.8047) =17.99
P(Y > 20)
20 25
( = 0.28) 0.61
17.99
≈ P Z > − − =
[Continuity correction can be used if desired]
\end{document}
