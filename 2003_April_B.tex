
5 Consider a random sample of size n from a normal distribution N(,2) and let S2
denote the sample variance.
(i) State the sampling distribution for
2
2
(n 1)S

, and specify an approximate
sampling distribution for this expression when n is large. [2]
(ii) For n = 101 calculate an approximate value for the probability that S2 exceeds
2 by more than a factor of 10%, i.e. P(S2 > 1.1 2). [1]
[Total 3]
6 Calculate the maximum possible width of a symmetrical two-sided 95% confidence
interval for the proportion of a population who possess a particular characteristic,
based on the corresponding information in a random sample of size 1600 from the
population. [3]
%%%%%%%%%%%%%%%%%%%%%%%%%%%%%%%%%%%%%%%%%%%%%%%%%%%%%%%%%%%%%%%%%%%%%%%%%%%%%%%%%%%%%%%%%%%%%%%%
7 A group of 500 insurance policies gave rise to a total of 83 claims during the last year.
Assuming a Poisson model for the occurrence of claims, calculate an approximate
95% confidence interval for , the claim rate per policy per year. [3]
8 The ratio of the standard deviation to the mean of a random variable is called the
coefficient of variation.
For each of the following distributions, decide whether increasing the mean of the
random variable increases, decreases, or has no effect on the value of the coefficient
of variation:
(a) Poisson with mean 
(b) exponential with mean 
(c) chi-square with n degrees of freedom [6]


5 (i)
2
2
2 1
( 1) ~ n
n S




 N n 1, 2n 1 for large n.
(ii)
2
2
(100 110) ( 110 100 0.707) 1 0.76 0.24
200
P S P Z 
      

OR: can interpolate in the Yellow Tables (p.169)
6 Approximate CI is “observed proportion  {1.96  standard error}”
Maximum value of s.e. is 0.5/1600 = 0.0125
so maximum width of CI = 2  1.96  0.0125 = 0.049
Subject 101 (Statistical Modelling) — April 2003 — Examiners’ Report
Page 4
7 ˆ 83 0.166
500
  
95% confidence interval is:
ˆ ˆ 1.96
500


giving 0.166 1.96 0.166 0.166 0.036 (0.130,0.202)
500
   
%%%%%%%%%%%%%%%%%%%%%%%%%%%%%%%%%%%%%%%%%%%%%%%%%%%%%%%%%%%%%%%%%%%%%%%%%%%%%%%%%%%%%%%%%%%%%%%%
8 (a) Mean = variance =  so c.o.v. = / = 1/
C.o.v. decreases as mean increases
(b) Mean = standard deviation =  so c.o.v. = 1
C.o.v. is unaffected by increasing the mean
(c) Mean = n, variance = 2n so c.o.v. = (2n)1/2/n = (2/n)1/2
C.o.v. decreases as mean increases
