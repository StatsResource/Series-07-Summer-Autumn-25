
12 The following data give the invoiced amounts for work carried out on 12 jobs
performed by a plumber in private customers’ houses. The durations of the jobs are
also given.
duration x (hrs) 1 1 2 3 4 4 5 6 7 8 9 10
amount y (£) 45 65 80 95 100 125 145 180 180 210 330 240
60, 2 402, 1795, 2 343,725, 11,570 xi  xi  yi  yi  xi yi 
The plumber claims to calculate his total charge for each job on the basis of a single
call-out charge plus an hourly rate for the time spent working on the job.
(i) (a) Draw a scatterplot of the data on graph paper and comment briefly on
your plot.
(b) The equation of the fitted regression line of y on x is y = 22.4 + 25.4x
and the coefficient of determination is R2 = 87.8% (you are not asked
to verify these results).
Draw the fitted line on your scatterplot. [5]
(ii) (a) Calculate the fitted regression line of invoiced amount on duration of
job using only the 11 pairs of values remaining after excluding the
invoice for which x = 9 and y = 330.
(b) Calculate the coefficient of determination of the fit in (ii)(a) above.
(c) Add the second fitted line to your scatterplot, distinguishing it clearly
from the first line you added (in part (i)(b) above).
(d) Comment on the effect of omitting the invoice for which x = 9 and
y = 330.
(e) Carry out a test to establish whether or not the slope in the model fitted
in (ii)(a) above is consistent with a rate of £25 per hour for work
carried out. [13]
%%%%%%%%%%%%%%%%%%%%%%%%%%%%%%%%%%%%%%%%%%%%%%%%%%%%%%%%%%%%%%%%%%%%%%%%%%%%%%%%%%%%%%%%%%%%%%%%
13 The random variable, X, has a gamma distribution with probability density function
given by:
 
 
 
 
1 exp /
0 ,
m
m
x x
f x x
m
  
 
 
where m and  are positive constants. This distribution has mean m and variance
m2. Let x1,  , xn denote a random sample of n observations on X.
(i) Suppose that m is known.
(a) Show that the maximum likelihood estimate of  is given by
1
ˆ / . n
i i x mn

 
(b) Show that ˆ
is an unbiased estimator of .
(c) Obtain the Cramer-Rao lower bound for estimators of .
(d) Show that the maximum likelihood estimator of  has variance equal
to the lower bound given in (i)(c). [10]
Suppose now that m is unknown, and is also to be estimated by maximum likelihood.
It is assumed that m is large enough so that (m) is well approximated by
g(m) exp( m)m(m 0.5) (2 )0.5 
   .
(ii) Determine the approximate maximum likelihood estimates of  and m,
substituting g(m) for (m) in the likelihood function. [7]
(iii) Suppose that the sample values are:
32 48 51 43 82 155
Obtain the approximate maximum likelihood estimates for  and m given
in (ii). [2]
%%%%%%%%%%%%%%%%%%%%%%%%%%%%%%%%%%%%%%%%%%%%%%%%%%%%%%%%%%%%%%%%%%%%%%%%%%%%%%%%%%%%%%%%%%%%%%%%
14 A social researcher is interested in the gender distribution among children in families,
and has collected data for her investigation as follows.
Three hundred families were selected at random. The table below shows frequency
distributions of the numbers of girls in families of size 1, 2, 3, and 4, that is with 1, 2,
3, and 4 children.
Number of girls in family
Size of
family 0 1 2 3 4
Number of
families
1 23 27 - - - 50
2 30 46 24 - - 100
3 9 36 43 12 - 100
4 4 17 15 11 3 50
(i) The researcher wants to investigate whether the proportion of girls within
families is independent of family size.
(a) Construct a suitable 2  4 contingency table, and calculate the overall
proportion of girls.
(b) State appropriate hypotheses to use in the researcher’s investigation.
(c) Calculate the value of an appropriate test statistic and state whether or
not its probability value exceeds 0.05.
(d) State your conclusion. [12]
(ii) (a) Suggest a model (with all parameter values stated or estimated) for the
number of girls in a family, for each family size (1, 2, 3, and 4), using
your conclusion from part (i).
(b) Suppose the researcher was to test the goodness-of-fit of the models
you have suggested in part (ii)(a) for family sizes 2, 3, and 4 and that
the models were rejected as being unsuitable. Discuss briefly how you
would interpret this lack of fit. [4]
[Total 16]


%%%%%%%%%%%%%%%%%%%%%%%%%%%%%%%%%%%%%%%%%%%%%%%%%%%%%%%%%%%%%%%%%%%%%%%%%%%%%%%%%%%%%%%%%%%%%%%%
12 (i) (a)
The point (9,330) is an “outlier” from the general pattern, which is
strongly linear.
(b)
(ii) (a) Now Σx = 51, Σx2 = 321, Σy = 1,465, Σy2 = 234,825, Σxy = 8,600
Sxx = 84.545, Syy = 39,713.636, Sxy = 1807.727
 ˆ 1807.727 /84.545  21.382, ˆ 1465 /11ˆ(51/11)  34.048
Fitted line is y = 34.0 + 21.4x
(b) Coefficient of determination R2 = 1807.7272/(84.545  39713.636)
= 0.973 i.e. 97.3%
(or find these by first calculating the three sums of squares SSTOT
= 39,713.636 as above, SSREG = 1807.7272/84.545 = 38652.515, and
so SSRES = 1061.121)
0 1 2 3 4 5 6 7 8 9 10
400
300
200
100
0
duration
amount
0 1 2 3 4 5 6 7 8 9 10
0
100
200
300
400
duration
amount
Subject 101 (Statistical Modelling) — April 2003 — Examiners’ Report
Page 8
(c)
(d) Removing the influence of the single point (9,330) results in a fitted
line with a lower slope and a much better fit for the remaining data (R2
has increased from 87.8% to 97.3%).
(e) The hourly rate corresponds to the slope in the model
H0: slope = 25 v H1: slope ≠ 25
Estimate of error variance = 1061.121/9
 standard error of slope estimate = [(1061.121/9)/84.545]1/2 = 1.181
t = (21.382 – 25)/1.181 = 	3.06 on 9 df
Upper tail probability is between 0.005 and 0.01 so P-value is between
0.01 and 0.02, so we have quite strong evidence against H0. We
conclude that the data are not consistent with an hourly rate of £25.
0 1 2 3 4 5 6 7 8 9 10
0
100
200
300
400
duration
amount
line (outlier omitted)
original line
%%%%%%%%%%%%%%%%%%%%%%%%%%%%%%%%%%%%%%%%%%%%%%%%%%%%%%%%%%%%%%%%%%%%%%%%%%%%%%%%%%%%%%%%%%%%%%%%
13
1 ( ) exp( / ) ( 0)
( )
m
m
f x x x x
m
  
 
 
1 1
1
1
exp
( )
( )
n n m i i
i i n
i i mn n
x x
L fx
m
 


  
 
  	 
 

 


(i) m is known case.
(a) 1
1
log ( 1) log log log ( )
n
n i
i
i
i
x
l L m x mn n m 

      



1 1
2
Then 0 ˆ
n n
i i
i i
x x
l mn l
mn
   

  
 


   
 
(MLE)
(b) E(ˆ) nE(Xi ) nm ˆ
mn mn

    is unbiased
(c)
2
1
2 3 2 2
n
i
i
x
l mn  
  
  

2
2 3 2 2
E l 2 nE(X ) mn mn
  
       
 	  	 	 	
 CRlb =
2 1
2
E l
    
 		 

    

=
2
mn

(d) Variance of ˆ :
2 2
2 2 2 2 2 2
1
( ˆ) 1 ( ) ( )
n
i
i
V V X n V X nm
m n m n m n mn 
 
     
Cramer-Rao lb is attained.
%%%%%%%%%%%%%%%%%%%%%%%%%%%%%%%%%%%%%%%%%%%%%%%%%%%%%%%%%%%%%%%%%%%%%%%%%%%%%%%%%%%%%%%%%%%%%%%%
(ii) m is unknown case.
If m has also to be estimated, ML equations are (approx):
l l 0
m
 
 
 
  1
1 log ( 1) log log
n
n i i
i i
x
l L m x mn 

     



1 1
2 2 n m (m ) logm log 2
1
2 1
ˆ ˆ ˆ ˆ ˆ 0 /
n
i i n
i i
l x mn x nm 


    
  

 (MLE of )
 
1
2
1
ˆ ˆ log log log ˆ 0
ˆ
n
i i
l m x n n n n m
m  m
   
     		    
 

Substituting for ˆ
gives
 
1
1 log log log ˆ 0
ˆ 2 ˆ
n n
i i
n x n x n n m
 m m
 
     
 	

2mˆ 1/ log(x / x) where  
1
1 1 , / n n n
i i i i x x x x n
 
   
 mˆ 1/ log  x / x 2   
 (MLE of m)
There are alternative versions, for example,
 

i x
n
x
m
log 1 log
ˆ 0.5
Most candidates found Question 13(ii) difficult to complete successfully (the Examiners
recognise that this part was on the “hard side”).
(iii) x  68.5, x  59.147 mˆ  3.406, ˆ  20.114
%%%%%%%%%%%%%%%%%%%%%%%%%%%%%%%%%%%%%%%%%%%%%%%%%%%%%%%%%%%%%%%%%%%%%%%%%%%%%%%%%%%%%%%%%%%%%%%%
14 (i) (a) Suitable table is gender  family size.
Family size 3: no. of girls = 36 + 43(2) + 12(3) = 158
so no. of boys = 300 – 158 = 142, etc.
family size
1 2 3 4 total
no. of girls 27 94 158 92 371
no. of boys 23 106 142 108 379
total 50 200 300 200 750
Overall proportion of girls = 371/750 = 0.495
(b) H0: gender and family size are independent
H1: gender and family size are not independent
(c) Expected frequency (under H0) in brackets
27 94 158 92
(24.73) (98.93) (148.40) (98.93)
23 106 142 108
(25.27) (101.07) (151.60) (101.07)
2 = (27-24.73)2/24.73 + …..
= 0.208 + 0.246 + 0.621 + 0.486 +
0.203 + 0.241 + 0.608 + 0.476 = 3.088
df = 3, upper 5% point is 7.815 so P-value exceeds 0.05.
(d) We have no evidence against H0, which can therefore stand, and so we
conclude that the proportion of girls is independent of family size.
(ii) (a) Models: no. of girls ~ binomial(n, 0.495) for n = 1, 2, 3, 4
(b) The model assumes that the “trials are independent” i.e. that the gender
of each child is independent of that of all other children in the family.
We would interpret the lack of fit as evidence that the genders of
children in a family are not independently determined.
In addition, the gender of the first child (or the genders of the first and
second children) may have an influence on – or even decide - the
family size.
Another possible reason is variation in P(girl) across families.
%%%%%%%%%%%%%%%%%%%%%%%%%%%%%%%%%%%%%%%%%%%%%%%%%%%%%%%%%%%%%%%%%%%%%%%%%%%%%%%%%%%%%%%%%%%%%%%%
The Examiners did not anticipate that so many candidates would be unable to construct the
basic 2  4 contingency table appropriate for investigating the matter in question.
