\documentclass[a4paper,12pt]{article}

%%%%%%%%%%%%%%%%%%%%%%%%%%%%%%%%%%%%%%%%%%%%%%%%%%%%%%%%%%%%%%%%%%%%%%%%%%%%%%%%%%%%%%%%%%%%%%%%%%%%%%%%%%%%%%%%%%%%%%%%%%%%%%%%%%%%%%%%%%%%%%%%%%%%%%%%%%%%%%%%%%%%%%%%%%%%%%%%%%%%%%%%%%%%%%%%%%%%%%%%%%%%%%%%%%%%%%%%%%%%%%%%%%%%%%%%%%%%%%%%%%%%%%%%%%%%

\usepackage{eurosym}
\usepackage{vmargin}
\usepackage{amsmath}
\usepackage{graphics}
\usepackage{epsfig}
\usepackage{enumerate}
\usepackage{multicol}
\usepackage{subfigure}
\usepackage{fancyhdr}
\usepackage{listings}
\usepackage{framed}
\usepackage{graphicx}
\usepackage{amsmath}
\usepackage{chngpage}

%\usepackage{bigints}
\usepackage{vmargin}

% left top textwidth textheight headheight

% headsep footheight footskip

\setmargins{2.0cm}{2.5cm}{16 cm}{22cm}{0.5cm}{0cm}{1cm}{1cm}

\renewcommand{\baselinestretch}{1.3}

\setcounter{MaxMatrixCols}{10}

\begin{document}
%%%%%%%%%%%%%%%%%%%%%%%%%%%%%%%%%%%%%%%%%%%%%%%%%%%%%%%
\enumerate}
\item The discrete random variable X has the following probability function:
P(X = i) = 0.2 + ai : i = -2, -1, 0, 1, 2.
(i) State the possible values that a can take. [1]
(ii) Given a random sample x1 , x2 , ..., xn from this distribution, determine
the method of moments estimate of a and show that this can result in
inadmissible estimates (i.e. estimates outside the range of possible
values of a). 

 %%%%%%%%%%%%%%%%%%%%%%%%%%%%%%%%%%%%%%%%%%%%%%%%%%%%%%%%%%%%%%%%%%%%%
9 (i) This will be a probability function provided the specified probabilities are
non-negative; i.e. if and only if 0.1  a  0.1.
(ii) The method of moments estimate of a is obtained by equating the sample
mean to the population mean. To do this note that
 = 2
2
 i(0.2 + ai) = a 2
2 
 i2 = 10a.
Thus, the method of moments estimate is 10
X .

As X can take any value between 2 and +2, the method of moments
estimate can take any value between –0.2 and +0.2. Thus it can be
outside the range (0.1, 0.1).
\newpage
%%%%%%%%%%%%%%%%%%%%%%%%%%%%%%%%%%%%%%%%%%%%%%%%%%%%%%%
\item Under a particular model for the evolution of the size of a population over
time, the probability generating function of Xt , the size at time t, ( )
Xt G s , is
given by
(1 )
( ) = where ( ) ( ) and 0
1 (1 )
t
t t
X
X X
s t s
G s G s E s
t s
ì + l - ü
í ý = l >
î + l - þ
.
If the population dies out, it remains in this extinct state for ever.
\begin{enumerate}
    \item Show that the expected size of the population at any time t is 1. 
\item  Show that the probability that the population has become extinct by
time t is given by lt / (1 + lt). 
\item Comment briefly on the future prospects for the population. [1]
\end{enumerate}

%%%%%%%%%%%%%%%%%%%%%%%%%%%%%%%%%%%%%%%%%%%%%%%%%%%
10 (i)
t
X G (s) = (1  t){1 + t(1  s)}1 {s + t(1  s)}{t}{1 + t(1  s)}2
  =
t
X G (1) = 1  t + t = 1
(ii) P(extinct by time t) = P(population size at time t is zero)
= (0)
t
X G = t / (1 + t)
(iii) P(extinct by time t) 
 1 as t 
 , so eventual extinction is certain.
%%%%%%%%%%%%%%%%%%%%%%%%%%%%%%%%%%%%%%%%%%%%%%%%%%%%%%%%%%%%%%

\newpage
\item A charity sent eight hundred of its supporters information packs about its activities and asked for donations to help it continue its work. Two hundred were sent a pack about its work in education, two hundred a pack about its
work in health care, and four hundred a pack about its anti-poverty programmes. Ninety-seven of the eight hundred people responded with donations; the breakdown is shown in the table below.

Education Health Poverty Total
Donate 26 31 40 97
Don’t donate 174 169 360 703
Total 200 200 400 800

Perform a c2 test on this table to investigate whether the proportions who
send donations are affected by the type of pack received. 

%%%%%%%%%%%%%%%%%%%%%%%%%%%%%%%%%%%%%%%%%%%%%%%%%%%%%%%
\medskip 
Complete the table of Expected values:
Expected Education Health Poverty Total
Donate 24.25 24.25 48.5 97
Don’t donate 175.75 175.75 351.5 703
200 200 400 800
Calculate 2
 = 3.98.
The 5\% point of a 2
 random variable on 2 degrees of freedom is 5.991,
so the 2
 test is not significant at the 5% level.
On the basis of the data collected, it is plausible that the three packs are equally
effective.
\newpage
%%%%%%%%%%%%%%%%%%%%%%%%%%%%%%%%%%%%%%%%%%%%%%%%%%%%%%%
\item A random sample of 200 pairs of observations (x, y) from a discrete bivariate
distribution (X, Y) is as follows:
\begin{itemize}
    \item the observation (-2, 2) occurs 50 times
\item the observation (0, 0) occurs 90 times
\item the observation (2, -1) occurs 60 times.
\end{itemize}

Calculate the sample correlation coefficient for these data.
%%%%%%%%%%%%%%%%%%%%%%%%%%%%%%%%%%%%%%%%%%%%%%%%%%%%%%%
12 x = 50(2) + 0 + 60(2) = 20 x2 = 50(4) + 0 + 60(4) = 440
y = 50(2) + 0 + 60(1) = 40 y2 = 50(4) + 0 + 60(1) = 260
xy = 50(4) + 0 + 60(2) = 320
so r = [320  (20  40)/200]/[(440  202/200)(260  402/200)]1/2 = 0.975
%%%%%%%%%%%%%%%%%%%%%%%%%%%%%%%%%%%%%%%%%%%%%%%%%%%%%%%
\end{enumerate}
\end{document}
