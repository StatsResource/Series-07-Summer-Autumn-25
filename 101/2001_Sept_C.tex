\documentclass[a4paper,12pt]{article}

%%%%%%%%%%%%%%%%%%%%%%%%%%%%%%%%%%%%%%%%%%%%%%%%%%%%%%%%%%%%%%%%%%%%%%%%%%%%%%%%%%%%%%%%%%%%%%%%%%%%%%%%%%%%%%%%%%%%%%%%%%%%%%%%%%%%%%%%%%%%%%%%%%%%%%%%%%%%%%%%%%%%%%%%%%%%%%%%%%%%%%%%%%%%%%%%%%%%%%%%%%%%%%%%%%%%%%%%%%%%%%%%%%%%%%%%%%%%%%%%%%%%%%%%%%%%

\usepackage{eurosym}
\usepackage{vmargin}
\usepackage{amsmath}
\usepackage{graphics}
\usepackage{epsfig}
\usepackage{enumerate}
\usepackage{multicol}
\usepackage{subfigure}
\usepackage{fancyhdr}
\usepackage{listings}
\usepackage{framed}
\usepackage{graphicx}
\usepackage{amsmath}
\usepackage{chngpage}

%\usepackage{bigints}
\usepackage{vmargin}

% left top textwidth textheight headheight

% headsep footheight footskip

\setmargins{2.0cm}{2.5cm}{16 cm}{22cm}{0.5cm}{0cm}{1cm}{1cm}

\renewcommand{\baselinestretch}{1.3}

\setcounter{MaxMatrixCols}{10}

\begin{document}
\begin{enumerate}
%%%%%%%%%%%%%%%%%%%%%%%%%%%%%%%%%%%%%%%%%%%%%%%%%%%%%%%%%%%%%%%%%%%%%%%%%%%%%%%%%%%%%%%%%%%%%%%%%%%%%%%%%
\item Two independent random samples of sizes n1 and n2 are selected from a normal
population with variance $\sigma^2$. The sample variances are denoted by $S^2_1$ and $S^2_2$
respectively. Let $S^2_W$ denote a weighted average of the sample variances given by
\[ S^2_W = \alpha S^2_1 + (1 - \alpha)S^2_2\]
where $\alpha$ is a constant such that $0 \leq \alpha \leq 1$.
\begin{enumerate}[(a)]
\item Show that $S^2_W$
 is an unbiased estimator of $\sigma^2$, and obtain an expression for
the mean square error of $S^2_W$
 .
(You may use
${ \displaystyle \operatorname{Var}(S^2_i)  = \frac{2 \sigma^4}{n_i - 1} }$ 

, i = 1, 2.) 
\item Show that $S^2_W$
 has minimum mean square error if ${ \displaystyle  \alpha = \frac{n_1 - 1}{ n_1 + n_2 - 2} }$
\end{enumerate}

%%%%%%%%%%%%%%%%%%%%%%%%%%%%%%%%%%%%%%%%%%%%%%%%%%%%%%%%%%%%%%%%%%%%%%%%%%%%%%%%%%%%%%%%%%%%%%%%%%%%%%%%%
\item 10 The number of incomplete insurance proposals Y, in a batch of x proposals, is to
be modelled as a Poisson random variable with mean $\lambda x$, where $\lambda$ is unknown.
Data are available from n independent batches of proposals as follows: batch
number i contains xi proposals of which yi are incomplete, $i = 1,2, \ldots, n$.
The least squares estimator of $\lambda$ is that value of $\lambda$for which
\[( )2
1
( )
n
i i
i
Y EY
=
 −\]
is minimised.
\begin{enumerate}[(a)]
\item Show that the least squares estimator of $\lambda$ is given by:
\[2 = i i
i
x Y
x
\lambda
 .\] 
\item Determine $\hat{lambda}$ , the maximum likelihood estimator of $\lambda$. 
\item Determine whether neither, one, or both of $\lambda$ and $\hat{lambda}$ provide unbiased
estimators of $\lambda$. 
\end{enumerate}


%%%%%%%%%%%%%%%%%%%%%%%%%%%%%%%%%%%%%%%%%%%%%%%%%%%%%%%%%%%%%%%%%%%%%%%%%%%%%%%%%%%%%%%%%%%%%%%%%%%%%%%%%
\item 11 A random sample of 11 policies on the contents of private houses was examined
for each of three insurance companies and the sum insured under each policy
noted. The observations were rounded to the nearest $$\$100$ and expressed in units
of $\$1,000$.
The sums and sums of squares of the observations are as follows:
\begin{center}
\begin{tabular}{ccc}
& Sum & Sum of squares\\
Company 1 & 129.1 & 1,534.37\\
Company 2 & 109.8 & 1,109.88\\
Company 3 & 123.5 & 1,401.73\\
\end{tabular}
\end{center}
The data are to be analysed under the one-way analysis of variance model to
examine whether company effects are present.
The ANOVA table is given below, with three entries deleted.
\begin{verbatim}
Source of variation d.f. SS MSS
Between companies 2 *** ***
Residual 30 48.24 1.61
32 ***
\end{verbatim}

\begin{enumerate}[(i)]
\item Copy the table into your answer book, filling in the three values which
have been deleted. 
\item Test the null hypothesis of no company effects and state your conclusion.

\end{enumerate}


%%%%%%%%%%%%%%%%%%%%%%%%%%%%%%%%%%%%%%%%%%%%%%%%%%%%%%%%%%%%%%%%%%%%%%%%%%%%%%%%%%%%%%%%%%%%%%%%%%%%%%%%%
\item 12 Claims are classified on inception into one of three categories, “simple”,
“standard” and “complex”.
\begin{itemize}
    \item Last year the percentage of all claims classified in
each of these categories was 18.4\%, 70.3\% and 11.3\% respectively.
\item A random sample of 120 of this year’s claims to date shows that the numbers
classified in each category are 15, 87 and 18 respectively.
\item Perform a goodness-of-fit test to investigate whether this year’s pattern to date
differs from that of last year, and state your conclusion. 
\end{itemize}
\end{enumerate}
%%%%%%%%%%%%%%%%%%%%%%%%%%%%%%%%%%%%%%%%%%%%%%%%%%%%%%
\newpage
% 9 2 2 2
% 1 2 ( )= ( ) (1 ) ( ) W E S E S    E S
% = \sigma^2 + (1  )\sigma^2 = \sigma^2 
(using unbiasedness of sample variance)
Therefore 2
W S is unbiased for $\sigma^2$. MSE = ( 2 ) W Var S since unbiased.
MSE = ( 2 ) W Var S 2 2 2 2
% 1 2 =  Var(S )  (1  ) Var(S )
%  2 2 4
4 2
1 2
1 2
= 2 using ( ) = ; 1,2
1 1 1 i
i
Var S i
n n n

% \item 4
1 2
2 2(1 )
= 2
1 1
dMSE
d n n


Setting equal to zero gives
1 2
1
n 1 n 1

% = 0  (n2  1)   (n1  1)(1  ) = 0

${ \displaystyle \operatorname{Var}(S^2_i)  = \frac{2 \sigma^4}{n_i - 1} }$ 
%%%%%%%%%%%%%%%%%%%%%%%%%%%%%%%%%%%%%%%%%%%%%%%%%%%%%%%%%%%%%%%%%%%%%%%%%%%%%%%%%%%%%%%%%%%%%%%%%%%%%%%%%
Thus giving 1
1 2
1
=
2
n
n n
%
which clearly minimises MSE.
10 
\begin{itemize}
\item 2 2
1 1
= ( ) =
n n
i i i i
i i
S Y EY Y x
%%%%%%%%%%%%%%%%%%%%%%%%%%%5
= 2 ( ) i i i
dS
x Y x
d
%%%%%%%%%%%%%%%%%%%%%%%%%%%%%%%
 Setting to 0  2 = i i
i
xY
x
%%%%%%%%%%%%%%%%%%%%%%%%%%%5
\item %( ) = xi  Yi const. so log =   log const.
i i i L e x L x Y %
%%%%%%%%%%%%%%%%%%%%%%%%%%%%%%%%%%%%%%
log 1
i i
d L
x Y
d
%%%%%%%%%%%%%%%%%%%%%%%%%%%%%
% Setting to 0  ˆ i
i
Y
x%%%%%%%%%%%%%%%%%%%%%%%%%%%
\item %  2 2
1 1
( )= = = i i i i
i i
E E x Y x x
x x
%     
%  
%  hence unbiased
%  
% ˆ 1 1 ( )= = = i i
i i
E E Y x
x x
    \item 
\end{itemize}
hence unbiased
Some candidates did not appear to understand that the two methods of deriving
estimators could produce different estimators.
%%%%%%%%%%%%%%%%%%%%%%%%%%%%%%%%%%%%%%%%%%%%
11 

129.1 + 109.8 + 123.5 = 362.4 , 1,534.37 + 1,109.88 + 1,401.73 = 4,045.98
% SST = 4,045.98 – 362.42/33 = 66.17
% So SSB = 66.17 – 48.24 = 17.93 **
Table is:
\begin{verbatim}
    Source of variation d.f. SS MSS
Between companies 2 17.93 8.97
Residual 30 48.24 1.61
32 66.17
\end{verbatim}



F = 8.97/1.61 = 5.57 on 2,30 d.f.
P-value is less than 0.01, so reject null hypothesis.
There is strong evidence that there are differences among the (population)
means of the sums insured for the three companies.

%%%%%%%%%%%%%%%%%%%%%%%%%%%%%%%%%%%%%%%%%%%%%%%%%%%%%%%%%%%%%%%%%%%%%%%%%%%%%%%%%%%%%%%%%%%%%%%%%%%%%%%%%
** OR: Calculate SSB directly as
SSB = (129.12 + 109.82 + 123.52) / 11 – 362.42/33 = 17.93
12 H0 : this year’s pattern is the same as last year’s v. H1 : not the same
\begin{itemize}
    \item Under H0 , the expected frequencies are:
% 120  0.184 ; 0.703 ; 0.113 = 22.08 ; 84.36 ; 13.56
% oi ei (o 
e)2/e
15 22.08 2.270
87 84.36 0.083
18 13.56 1.454
3.807 on 2 df
5% point from 22
 is 5.991. So cannot reject H0 at 5% level.
\item These data provide no evidence to suggest that this year’s pattern differs from
that of last year.
\item A few candidates worked with percentages of claims instead of numbers of claims
(when using the chi-squared goodness-of-fit statistic, one must work with observed
and expected frequencies). Such work received few if any marks.
\end{itemize}


\end{document}
