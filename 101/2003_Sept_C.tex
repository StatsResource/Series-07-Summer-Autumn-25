\documentclass[a4paper,12pt]{article}

%%%%%%%%%%%%%%%%%%%%%%%%%%%%%%%%%%%%%%%%%%%%%%%%%%%%%%%%%%%%%%%%%%%%%%%%%%%%%%%%%%%%%%%%%%%%%%%%%%%%%%%%%%%%%%%%%%%%%%%%%%%%%%%%%%%%%%%%%%%%%%%%%%%%%%%%%%%%%%%%%%%%%%%%%%%%%%%%%%%%%%%%%%%%%%%%%%%%%%%%%%%%%%%%%%%%%%%%%%%%%%%%%%%%%%%%%%%%%%%%%%%%%%%%%%%%

\usepackage{eurosym}
\usepackage{vmargin}
\usepackage{amsmath}
\usepackage{graphics}
\usepackage{epsfig}
\usepackage{enumerate}
\usepackage{multicol}
\usepackage{subfigure}
\usepackage{fancyhdr}
\usepackage{listings}
\usepackage{framed}
\usepackage{graphicx}
\usepackage{amsmath}
\usepackage{chngpage}

%\usepackage{bigints}
\usepackage{vmargin}

% left top textwidth textheight headheight

% headsep footheight footskip

\setmargins{2.0cm}{2.5cm}{16 cm}{22cm}{0.5cm}{0cm}{1cm}{1cm}

\renewcommand{\baselinestretch}{1.3}

\setcounter{MaxMatrixCols}{10}

\begin{document}

\begin{enumerate}
\item
9 Suppose that the joint probability distribution of two random variables X and Y is
given by the following table:
Y
2 4 6
1 0.2 0.0 0.2
X 2 0.0 0.2 0.0
3 0.2 0.0 0.2
(i) Show that X and Y are uncorrelated, but are not independent. 
(ii) Leaving the probabilities in the first and third rows of the table the same,
change the entries in the second row so that X and Y are independent. 
%%%%%%%%%%%%%%%%%%%%%%%%%%%%%%%%%%%%%%%%%%%%%%%%%%%%%%%%%%%%%%%%%%%%%%%%%%%%%%%%%%%%%%%%%%%%%%%%%%
\item  The number of claims on a portfolio of policies was observed as follows:
Number of claims per day 0 1 2 3 4 5 Total
Frequency 48 32 17 2 0 1 100
Use a 2 goodness-of-fit test to test the hypothesis that the number of claims each day
follows a Poisson distribution. [7]
\item  The following table contains 10 claim amounts for repair costs arising from a
particular type of storm damage to private houses, for each of four different postcode
regions:
Regions
A B C D
Claim amount (£) 961 1,507 1,303 1,022
1,263 1,349 959 997
1,304 1,521 1,297 1,335
1,532 1,134 1,051 1,216
1,294 1,293 1,163 1,277
1,605 993 993 1,135
1,308 1,126 978 1,273
1,393 1,140 891 1,244
1,255 1,305 1,177 1,105
1,131 1,224 1,153 1,524
Sums 13,046 12,592 10,965 12,128
Sums of squares 17,322,090 16,116,822 12,208,021 14,929,994
(i) Show that a one-way analysis of variance to compare the mean claim amounts
for the regions produces a significant result at the 5% level, but not at the 1%
level. 
(ii) Compare the mean claim amounts for the regions A, B, C, and D by using a
least significant difference approach with a significance level of 5%. 
\item  Let X denote the number of accidents a manual worker in a particular factory has in a
year. For a given worker the distribution of X is modelled as a Poisson distribution
with unknown parameter u that varies across the workforce. U is regarded as a
random variable which has a gamma distribution with parameters  and , i.e.
X | (U = u) 	 Poisson(u),
U 	 gamma(, ).
(i) Show that the marginal distribution of X has mean 

and variance 2
 

 
.

(ii) A dataset has a sample mean of x and a sample variance of s2 . Show that 
and  may be estimated by
2
2 2
ˆ x , ˆ x
s x s x
   
 
using the method of moments. 
(iii) State the circumstances under which the method of moments produce
inadmissible estimates of  and . [1]
%%%%%%%%%%%%%%%%%%%%%%%%%%%%%%%%%%%%%%%%%%%%%%%%%%%%%%%%%%%%%%%%%%%%%%%%%%%%%%%%%%%%%%%%%%%%%%%%%%
\end{enumerate}
%%%%%%%%%%%%%%%%%%%%%%%%%%%%%%%%%%%%%%%%%%%%%%%%%%%%%%%%%%%%%%%%%%%%%%%%%%%%%%%%%%%%%%%%%%%%%%%%%%
9 (i)
Y
2 4 6
1 0.2 0.0 0.2 0.4
X 2 0.0 0.2 0.0 0.2
3 0.2 0.0 0.2 0.4
0.4 0.2 0.4
E[X] = 0.4 × 1 + 0.2 × 2 + 0.4 × 3 = 2
E[Y] = 0.4 × 2 + 0.2 × 4 + 0.4 × 6 = 4
E[XY] = 1 × 2 × 0.2 + 1 × 4 × 0.0 + 1 × 6 × 0.2
+ 2 × 2 × 0.0 + 2 × 4 × 0.2 + 2 × 6 × 0.0
+ 3 × 2 × 0.2 + 3 × 4 × 0.0 + 3 × 6 × 0.2 = 8
E[XY] − E[X]E[Y] = 0 Therefore uncorrelated.
X and Y are not independent since
P(X = x and Y = y) ≠ P(X = x) P(Y = y)
e.g. x = 1 y = 2, 0.2 ≠ 0.4 × 0.4 = 0.16.
(ii) X and Y are independent if joint probability is:
Y
2 4 6
1 0.2 0.0 0.2 0.4
X 2 0.1 0.0 0.1 0.2
3 0.2 0.0 0.2 0.4
0.5 0.0 0.5
10 The mean number of claims per day is
{(32 × 1) + (17 × 2) + (2 × 3) + (0 × 4) + (1 × 5)}/100 = 0.77.
Use 0.77 as an estimate of the mean of the Poisson distribution. Thus
%%%%%%%%%%%%%%%%%%%%%%%%%%%%%%%%%%%%%%%%%%%%%%%%%%%%%%%%%%%%%%%%%%%%%%%%%%
Page 6
( )
!
e x P X x
x
−λλ
= = is estimated by
0.770.77 ( )
!
e x P X x
x
−
= = , x = 0, 1, 2, …
The expected frequencies are given by 100 × P(X = x).
No. of claims (x) 0 1 2 3 4 ≥5 Total
Obs. frequency (fi) 48 32 17 2 0 1 100
Exp. frequency (ei) 46.3 35.7 13.7 3.5 0.7 0.1 100.0
Categories x = 3, 4, and ≥5 are grouped together to ensure that all ei are greater than 1.
The expected frequency for ≥ 3 is 3.5 + 0.7 + 0.1 = 4.3; the corresponding observed
frequency is 3.
2 2 2 2
2 ( )2 / 1.7 3.7 3.3 1.3 1.63.
i i i 46.3 35.7 13.7 4.3 χ =Σ f − e e = + + + =
There are 2 d.f. [4 categories x = 0,1,2, and ≥3, and 1 parameter estimated from the
data.]
The probability value =
( ) 22
P χ >1.63 ≅ 1 – 0.557 = 0.443 from the Yellow Tables p164
There is insufficient evidence to suggest that the number of claims does not follow a
Poisson distribution (i.e. the model provides a good fit to the data).
An alternative solution (in this over-conservative approach some information is
thrown away unnecessarily - but it was awarded full marks):
Grouping categories x = 2, 3, 4 and ≥ 5 and using only 3 cells with observed
frequencies 48, 32, and 20 and expected frequencies 46.3, 35.7, and 18.0 gives χ2 =
0.668 on 1 degree of freedom. The probability value is 0.414. Same conclusion.
%%%%%%%%%%%%%%%%%%%%%%%%%%%%%%%%%%%%%%%%%%%%%%%%%%%%%%%%%%%%%%%%%%%%%%%%%%%%%%%%%%%%%%%%%%%%%%%%%%
11 (i) Total sum: 13046 + 12592 + 10965 + 12128 = 48731
Total sum of squares: 17322090 + 16116822 + 12208021 + 14929994
= 60576927
SST = 60576927 − 487312/40 = 1209168
%%%%%%%%%%%%%%%%%%%%%%%%%%%%%%%%%%%%%%%%%%%%%%%%%%%%%%%%%%%%%%%%%%%%%%%%%%
Page 7
SSB = (130462 + 125922 + 109652 + 121282)/10 − 487312/40
= 59607619 − 487312/40 = 239860
SSR = SST − SSB = 1209168 − 239860 = 969308
Source of variation df Sums of Squares Mean Squares
Between regions 3 239860 79953
Residual 36 969308 26925
Total 39 1209168
F = 79953/26925 = 2.97 on 3, 36 d.f.
Therefore, since the value of F3,36 (0.05) is 2.866, the observed F value (2.97)
exceeds it and so the null hypothesis that the population means are equal is
rejected at the 5% level of significance. However, as F3,36 (0.01) is 4.377, the
null hypothesis is not rejected at the 1% level.
(ii) Means:
A: y1. = 1304.6 B: y2. = 1259.2
C: y3. = 1096.5 D: y4. = 1212.8
Least significant difference, for each pair of regions, is (5% level):
1/2
t0.025,36 σˆ ( 1/10 +1/10) = 2.028 26925 (2/10)1/2 = 149
Differences between pairs of means:
y1. − y2. = 45.4 , y1. − y3. = 208.1 , y1. − y4. = 91.8
y2. − y3. =162.7 , y2. − y4. = 46.4 , y3. − y4. = −116.3
Region C Region D Region B Region A
y3. y4. y2. y1.
(Alternative answers which have the following conclusion are acceptable:
The population mean claim amount for region C appears to be less than the
population mean of region A and the population mean of region B. However,
the population mean for region C and the population mean for region D do not
appear to differ.)
%%%%%%%%%%%%%%%%%%%%%%%%%%%%%%%%%%%%%%%%%%%%%%%%%%%%%%%%%%%%%%%%%%%%%%%%%%
Page 8
12 (i) E[X ] = E[E(X |U)] = E[U] = α / λ
V[X ] = E[V (X |U)]+V[E(X |U)] = E[U]+V[U] = α / λ + α / λ2
(ii) Using the method of moments, α and λ may be estimated by solving the
equations
2
2 x and s α α α
= = +
λ λ λ
which gives
2
2 2
ˆ x and ˆ x .
s x s x
α = λ =
− −
(iii) If s2 ≤ x , then the method of moments produces inadmissible estimates as the
parameters α and λ must be positive and finite.
%%%%%%%%%%%%%%%%%%%%%%%%%%%%%%%%%%%%%%%%%%%%%%%%%%%%%%%%%%%%%%%%%%%%%%%%%%%%%%%%%%%%%%%%%%%%%%%%%%
\end{document}
