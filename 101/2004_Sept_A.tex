\documentclass[a4paper,12pt]{article}

%%%%%%%%%%%%%%%%%%%%%%%%%%%%%%%%%%%%%%%%%%%%%%%%%%%%%%%%%%%%%%%%%%%%%%%%%%%%%%%%%%%%%%%%%%%%%%%%%%%%%%%%%%%%%%%%%%%%%%%%%%%%%%%%%%%%%%%%%%%%%%%%%%%%%%%%%%%%%%%%%%%%%%%%%%%%%%%%%%%%%%%%%%%%%%%%%%%%%%%%%%%%%%%%%%%%%%%%%%%%%%%%%%%%%%%%%%%%%%%%%%%%%%%%%%%%

\usepackage{eurosym}
\usepackage{vmargin}
\usepackage{amsmath}
\usepackage{graphics}
\usepackage{epsfig}
\usepackage{enumerate}
\usepackage{multicol}
\usepackage{subfigure}
\usepackage{fancyhdr}
\usepackage{listings}
\usepackage{framed}
\usepackage{graphicx}
\usepackage{amsmath}
\usepackage{chngpage}

%\usepackage{bigints}
\usepackage{vmargin}

% left top textwidth textheight headheight

% headsep footheight footskip

\setmargins{2.0cm}{2.5cm}{16 cm}{22cm}{0.5cm}{0cm}{1cm}{1cm}

\renewcommand{\baselinestretch}{1.3}

\setcounter{MaxMatrixCols}{10}

\begin{document}

\begin{enumerate}
\item

%%%%%%%%%%%%%%%%%%%%%%%%%%%%%%%%%%%%%%%%%%%%%%%%%%%%%%%%%%%%%%%%%%%%%%%%%%%%%%%%%%%%%%%%%%%%%%%%%%%%%%
\begin{enumerate}
\item Claim sizes are normally distributed about a mean = £6,000 and with standard
deviation = £1000.
Calculate the probability that a claim is for more than £7,500, given that it is for more
than £6,000. [3]
\item Let X be a random variable which has a Poisson distribution with parameter .
(i) Write down the cumulant generating function CX (t). [1]
(ii) By differentiation of CX (t) show that the mean and variance of X are both
equal to . [2]
[Total 3]
\item A random sample of 25 observations from X ~ N( , 4) has sample mean x = 15.6.
Calculate a symmetrical 90% confidence interval for . [3]
\item The following test concerning the mean claim amount ( ) for a certain class of policy
H0: = £200 v. H1 : £200
is to be performed. A random sample of 50 claims is examined and yields a mean
amount of £207 and standard deviation £42.
Calculate the approximate probability-value for the test. 
\edn{enumerate}
%%%%%%%%%%%%%%%%%%%%%%%%%%%%%%%%%%%%%%%%%%%%%%%%%%%%%%%%%%%%%%%%%%%%%%%%%%%%%%%%%%%%%%%%%%%%%%%%%%%%%%
%%%%%%%%%%%%%%%%%%%%%%%%%%%%%%%%%%%%%%%%%%%%%%%%%%%%%%%%%%%%%%%%%%%%%%%%%%%%%%%%%%%%%%%%%%%%%%%%%%%%%%
\newpage
1 Let X be claim size in units of £1000: X ~ N(6,1)
P(X > 7.5|X>6) = P(X > 7.5 and X > 6) / P(X > 6) = P(X > 7.5) / P(X > 6)
= P(Z > 1.5) / P(Z > 0) where Z ~ N(0, 1)
= 0.0668 / 0.5 = 0.134


2 (i) ( ) ( t 1)https://www.overleaf.com/project/5c6fd45e40b4f453555876c8
CX t e
(ii) ( ) t (0)
CX t e CX (mean)
( ) t (0)
CX t e CX (variance)

3 CI is 15.6 {z ( / n)}
For a symmetrical 90% interval, z = 1.6449
i.e. 15.6 {1.6449 (2/5)} i.e. 15.6 0.66 i.e. 14.94 to 16.26
4 n = 50 is large, so Central Limit theorem allows the use of normality
P-value
207 200
2
42
50
P Z
= 2 P(Z > 1.18) = 2(1 0.88) = 0.24
Examiners Comment: Some candidates assumed that the claim amounts have a
normal distribution. This was not justifiable or necessary. What is required is the
approximate normality of the distribution of the sample mean, which is justified for
large samples by the central limit theorem.

\end{document}
