\documentclass[a4paper,12pt]{article}

%%%%%%%%%%%%%%%%%%%%%%%%%%%%%%%%%%%%%%%%%%%%%%%%%%%%%%%%%%%%%%%%%%%%%%%%%%%%%%%%%%%%%%%%%%%%%%%%%%%%%%%%%%%%%%%%%%%%%%%%%%%%%%%%%%%%%%%%%%%%%%%%%%%%%%%%%%%%%%%%%%%%%%%%%%%%%%%%%%%%%%%%%%%%%%%%%%%%%%%%%%%%%%%%%%%%%%%%%%%%%%%%%%%%%%%%%%%%%%%%%%%%%%%%%%%%

\usepackage{eurosym}
\usepackage{vmargin}
\usepackage{amsmath}
\usepackage{graphics}
\usepackage{epsfig}
\usepackage{enumerate}
\usepackage{multicol}
\usepackage{subfigure}
\usepackage{fancyhdr}
\usepackage{listings}
\usepackage{framed}
\usepackage{graphicx}
\usepackage{amsmath}
\usepackage{chngpage}

%\usepackage{bigints}
\usepackage{vmargin}

% left top textwidth textheight headheight

% headsep footheight footskip

\setmargins{2.0cm}{2.5cm}{16 cm}{22cm}{0.5cm}{0cm}{1cm}{1cm}

\renewcommand{\baselinestretch}{1.3}

\setcounter{MaxMatrixCols}{10}

\begin{document}

\begin{enumerate}
%%%%%%%%%%%%%%%%%%%%%%%%%%%%%%%%%%%%%%%%%%%%%%%%%%%%%%%%%%%%%%%%%%%%%%%%%%%%%%%%%%%%%%%%%%%%%%%%%%%%%%%%%%%%%%%%%%%%%%%%%%%%%%%%%%%%%%%%%%
\begin{enumerate}

\item  A magazine claims that 25% of its readers are students. A random sample of 200
readers is taken and is found to contain 42 students.
Calculate the probability of obtaining 42 or fewer student readers, assuming that the magazine’s claim is correct. [3]
\item  The percentage return on an investment of a particular type over a period of one year is to be modelled as a normal random variable X with mean  and variance 1. A
potential investor is interested in the chance that the return on such an investment will
exceed 9%.
A random sample of ten such returns have values
7.3, 8.9, 8.3, 6.2, 9.8, 7.7, 9.4 , 7.9, 9.1, and 7.4 .
Calculate the maximum likelihood estimate of  = P(X > 9). [3]
\end{enumerate}


%%%%%%%%%%%%%%%%%%%%%%%%%%%%%%%%%%%%%%%%%%%%%%%%%%%%%%%%%%%%%%%%%%%%%%%%%%%%%%%%%%%%%%%%%%%%%%%%%%%%%%%%%%%%%%%%%%%%%%%%%%%%%%%%%%%%%%%%%%
3 

\begin{itemize}
    \item Let X denote the number of magazine readers in the sample of 200 who are students.
\item Assuming the magazine’s claim is correct, X ~ binomial (200, 0.25).
\item Using the normal approximation to the binomial distribution,
X ~ N(50, 150/4) approximately:
P(X  42)  42.5 50
(150 / 4)
P Z

where Z ~ N(0, 1)
= P(Z < 1.225) = 0.110 (using continuity correction).
\end{itemize}


4 Data sum = 82; MLE of  is ˆ = x = 82 /10 = 8.2
 = P(X  9) = P(Z  9  ) where Z ~ N(0, 1)
 MLE of  = P(Z  9 ˆ ) = P(Z > 9 – 8.2) = P(Z > 0.8) = 0.212
%%%%%%%%%%%%%%%%%%%%%%%%%%%%%%%%%%%%%%%%%%%%%%%%%%%%%%%%%%%%%%%%%%%%%%%%%%%%%%%%%%%%%%%%%%%%%%%%%%%%%%%%%%%%%%%%%%%%%%%%%%%%%%%%%%%%%%%%%%

\end{document}
