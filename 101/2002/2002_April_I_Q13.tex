\documentclass[a4paper,12pt]{article}

%%%%%%%%%%%%%%%%%%%%%%%%%%%%%%%%%%%%%%%%%%%%%%%%%%%%%%%%%%%%%%%%%%%%%%%%%%%%%%%%%%%%%%%%%%%%%%%%%%%%%%%%%%%%%%%%%%%%%%%%%%%%%%%%%%%%%%%%%%%%%%%%%%%%%%%%%%%%%%%%%%%%%%%%%%%%%%%%%%%%%%%%%%%%%%%%%%%%%%%%%%%%%%%%%%%%%%%%%%%%%%%%%%%%%%%%%%%%%%%%%%%%%%%%%%%%

\usepackage{eurosym}
\usepackage{vmargin}
\usepackage{amsmath}
\usepackage{graphics}
\usepackage{epsfig}
\usepackage{enumerate}
\usepackage{multicol}
\usepackage{subfigure}
\usepackage{fancyhdr}
\usepackage{listings}
\usepackage{framed}
\usepackage{graphicx}
\usepackage{amsmath}
\usepackage{chngpage}

%\usepackage{bigints}
\usepackage{vmargin}

% left top textwidth textheight headheight

% headsep footheight footskip

\setmargins{2.0cm}{2.5cm}{16 cm}{22cm}{0.5cm}{0cm}{1cm}{1cm}

\renewcommand{\baselinestretch}{1.3}

\setcounter{MaxMatrixCols}{10}

\begin{document}

\begin{enumerate}

%%%%%%%%%%%%%%%%%%%%%%%%%%%%%%%%%%%%%%%%%%%%%%%%%%%%%%%%%%%%%%%%%%%%%%%%%%%%%%%%%%%%%%%%%%%%%%%%%%%%%%%%%%%%%%%%%%%%%%%%%%%%%%%%%%%%%%%%%%
\item  Six insurance companies were being compared with regard to premiums being
charged for house contents insurance for houses in a particular postcode region.
Independent random samples of five policies from each company are examined and
the premiums (in £) were recorded.

\begin{verbatim}
Company A B C D E F
151 152 175 149 123 145
168 141 155 148 132 131
128 129 162 137 142 155
167 120 186 138 161 172
134 115 148 169 152 141
Totals 748 657 826 741 710 744
% yij = 4, 426 ; yi2j = 661,796
\end{verbatim}


\begin{enumerate}[(a)]
\item Compute an ANOVA table for these data, and show that there are no
significant differences, at the 5\% level, between mean premiums being
charged by each company. 
\item State the assumptions required for the above analysis of variance, and, by
drawing a suitable diagram of these data, comment briefly on the validity of
these assumptions. 
\item Calculate a 95\% confidence interval for the underlying common standard
deviation % , using 2
SSR

as a pivotal quantity with a %  2  distribution. 
\item A colleague points out that company C has the largest mean premium of
£165.20 and that Company B has the smallest mean premium of £131.40 and
suggests performing a t-test to compare these two companies.
(a) Perform this t-test, using the estimate of variance from the ANOVA
table, and in particular show that there is a significant difference at the
1\% level.
(b) Your colleague states that there is therefore a significant difference
between the six companies.
Discuss the apparent contradiction with your conclusion and
explain the flaw in your colleague’s argument. 
\end{enumerate}

%%%%%%%%%%%%%%%%%%%%%%%%%%%%%%%%%%%%%%%%%%%%%%%%%%%%%%%%%%%%%%%%%%%%%%%%%%%%%%%%%%%%%%%%%%%%%%%%%%%%%%%%%%%%%%%%%%%%%%%%%%%%%%%%%%%%%%%%%%
13 
\begin{itemize}

\item SST =
44262 661796 = 8813.47
30 
SSB =
2
1 (7482 6572 8262 7412 7102 7442 ) 4426 = 3046.67
5 30

%  SSR = 8813.47  3046.67 = 5766.80
Source df SS MS F
Treatments 5 3046.7 609.3 2.54
Residual 24 5766.8 240.3
Total 29 8813.5
From tables F5,24(5%) = 2.621
Observed F < 2.621. Therefore not significant at 5% level.
\item Assumptions are that, for each company, such premiums are normally
distributed with the same variance.
              .   .           .         ..
  -------+---------+---------+---------+---------+---------
A
      .  .     .       .      .
  -------+---------+---------+---------+---------+---------
B
                            .   .    .        .      .
  -------+---------+---------+---------+---------+---------
C
                    ..      :             .
  -------+---------+---------+---------+---------+---------
D
           .     .      .     .     .
  -------+---------+---------+---------+---------+---------
E
                .      .  .     .           .
  -------+---------+---------+---------+---------+---------
F
       120       135       150       165       180       195
normality and equality of variance both seem reasonable.
%%%%%%%%%%%%%%%%%%%%%%%%%%%%%%%%%%%%%%%%%%%%%%%%%%%%%%%%%%%%%%%%%%%%%%%%%%%%
\item Here 2
2 30 6 24 SSR ~


% 2 P(12.40  SSR  39.36) = 0.95
% 
%  95% CI for 2 is 2
% 39.36 12.40
% SSR SSR
%   
Data gives SSR = 5766.8
%  146.51 < 2 < 465.06
and so 95% CI for  is 12.1 <  < 21.6
\item (a) Using ˆ 2 = 240.3 from the ANOVA
For comparing B and C: = 165.2 131.4 = 33.8 = 3.45
1 1 9.80 240.3( )
5 5
t 
% 
t24(0.5%) = 2.797 from tables.
Observed t > 2.797. Therefore significant at 1% level (two-sided).
(b) There is no contradiction.
It is wrong to pick out the largest and the smallest of a set of treatment
means, test for significance, and then draw conclusions about the set.
% Even if H0 : “i’s all equal” is true, the largest and smallest sample
means would, of course, differ.
\end{itemize}
\end{itemize}
\end{document}
