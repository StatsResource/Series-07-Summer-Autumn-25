\documentclass[a4paper,12pt]{article}

%%%%%%%%%%%%%%%%%%%%%%%%%%%%%%%%%%%%%%%%%%%%%%%%%%%%%%%%%%%%%%%%%%%%%%%%%%%%%%%%%%%%%%%%%%%%%%%%%%%%%%%%%%%%%%%%%%%%%%%%%%%%%%%%%%%%%%%%%%%%%%%%%%%%%%%%%%%%%%%%%%%%%%%%%%%%%%%%%%%%%%%%%%%%%%%%%%%%%%%%%%%%%%%%%%%%%%%%%%%%%%%%%%%%%%%%%%%%%%%%%%%%%%%%%%%%

\usepackage{eurosym}
\usepackage{vmargin}
\usepackage{amsmath}
\usepackage{graphics}
\usepackage{epsfig}
\usepackage{enumerate}
\usepackage{multicol}
\usepackage{subfigure}
\usepackage{fancyhdr}
\usepackage{listings}
\usepackage{framed}
\usepackage{graphicx}
\usepackage{amsmath}
\usepackage{chngpage}

%\usepackage{bigints}
\usepackage{vmargin}

% left top textwidth textheight headheight

% headsep footheight footskip

\setmargins{2.0cm}{2.5cm}{16 cm}{22cm}{0.5cm}{0cm}{1cm}{1cm}

\renewcommand{\baselinestretch}{1.3}

\setcounter{MaxMatrixCols}{10}

\begin{document}

\begin{enumerate}
%%%%%%%%%%%%%%%%%%%%%%%%%%%%%%%%%%%%%%%%%%%%%%%%%%%%%%%%%%%%%%%%%%
\item 12 A psychologist conducted an investigation into the effect of alcohol on reaction times
using 10 male and 10 female subjects. Each subject was given two tests on different
days, during which his/her reaction times were recorded.
\begin{itemize}
    \item Before each of the tests, the subject drank a glass of liquid. Some glasses contained a
fixed quantity of alcohol and others contained a liquid which had a similar colour and
taste but no alcohol.
\item Each subject drank one glass of each type. The order of
presentation was randomized, independently for each subject.
\item The data below give the reaction times, in units of 0.01 seconds.
\item Also given is the
difference between the reaction time with alcohol and the reaction time without
alcohol for each subject (reaction time with alcohol minus reaction time without).
\end{itemize}

\begin{verbatim}
    Males
Subject No. 1 2 3 4 5 6 7 8 9 10
With alcohol 45 51 35 43 51 54 51 49 44 52
Without alcohol 40 54 21 31 44 47 39 33 32 56
Difference 5 3 14 12 7 7 12 16 12 4
Females
Subject No. 1 2 3 4 5 6 7 8 9 10
With alcohol 47 54 58 48 60 46 55 74 56 49
Without alcohol 39 40 42 30 51 41 55 68 47 40
Difference 8 14 16 18 9 5 0 6 9 9
\end{verbatim}

\begin{enumerate}[(a)]
    \item (i) (a) Construct a 95\% confidence interval for the mean difference between
the reaction times with and without alcohol for the males, using the 10
difference values.
    \item  (b) Construct a similar 95\% confidence interval based on the female
difference values.
(c) Comment briefly on the two confidence intervals. 
    \item  (ii) (a) Perform a two-sample t-test to investigate whether the alcohol effect
differs between males and females.
    \item (b) Show that the variances in the male and female samples are not
significantly different at the 5\% level, and comment briefly with
reference to the test conducted in (ii)(a). 
\end{enumerate}


\newpage

12 (i) (a) Males: n1 = 10 x1= 7.8 s1 = 6.8605
95% CI for male data:
%%%%%%%%%%%%%%%%%%%%%%%%%%%%%%%%%%%%%%%
Page 8
x1 
 t9(0.025) 1
1
s
n
= 7.8 
 2.262 6.8605
10
= 7.8 
 4.907
= (2.89, 12.71)
(b) Females: n2 = 10 x2 = 9.4 s2 =5.37897
95% CI for female data:
2
2 9
2
x t (0.025) s
n

= 9.4 2.262 5.37897
10
 = 9.4 
 3.848
= (5.55, 13.25)

\begin{itemize}
    \item (c) Neither of the intervals include zero, and therefore there is evidence
that the alcohol has an effect on reaction times, i.e. it increases the
reaction time.
\item (ii) (a) Two sample t-test
\end{itemize}

1 2
2
1 2
1 1
p
t x x
s
n n


 
  
 	
2 2
2 1 1 2 2
1 2
( 1) ( 1)
p 2
s n s n s
n n
  

 
9(6.8605)2 9(5.37897)2
18

 = 37.9999
sp = 6.1644
2
10
7.8 9.4 1.6 0.58
6.1644 2.7568
t  
   
t18(0.25) = 0.688, and probability value is > 0.5.
There is no evidence to reject the null hypothesis that the means for
males and females are the same.
%%%%%%%%%%%%%%%%%%%%%%%%%%%%%%%%%%%%%%%%%%%%%%%%%%%%%%%%%%%%%%%%%%
We can conclude that alcohol has a similar effect.
(b)
2
1
2
2
s 1.626
s


\begin{itemize}
    \item The F9,9 distribution has upper 5\% critical point at 3.179.
    \item Our
observed value (1.626) is well within the main body of the distribution
and is not significant at the 10\% level of testing (and therefore not at
the 5\% level). 
\item There is no evidence to suggest that the variances
differ. 
\item The assumption of common variance was made when
conducting the test in (ii)(a).
\end{itemize}

\end{document}
