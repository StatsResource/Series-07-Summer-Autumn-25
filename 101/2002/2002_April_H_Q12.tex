\documentclass[a4paper,12pt]{article}

%%%%%%%%%%%%%%%%%%%%%%%%%%%%%%%%%%%%%%%%%%%%%%%%%%%%%%%%%%%%%%%%%%%%%%%%%%%%%%%%%%%%%%%%%%%%%%%%%%%%%%%%%%%%%%%%%%%%%%%%%%%%%%%%%%%%%%%%%%%%%%%%%%%%%%%%%%%%%%%%%%%%%%%%%%%%%%%%%%%%%%%%%%%%%%%%%%%%%%%%%%%%%%%%%%%%%%%%%%%%%%%%%%%%%%%%%%%%%%%%%%%%%%%%%%%%

\usepackage{eurosym}
\usepackage{vmargin}
\usepackage{amsmath}
\usepackage{graphics}
\usepackage{epsfig}
\usepackage{enumerate}
\usepackage{multicol}
\usepackage{subfigure}
\usepackage{fancyhdr}
\usepackage{listings}
\usepackage{framed}
\usepackage{graphicx}
\usepackage{amsmath}
\usepackage{chngpage}

%\usepackage{bigints}
\usepackage{vmargin}

% left top textwidth textheight headheight

% headsep footheight footskip

\setmargins{2.0cm}{2.5cm}{16 cm}{22cm}{0.5cm}{0cm}{1cm}{1cm}

\renewcommand{\baselinestretch}{1.3}

\setcounter{MaxMatrixCols}{10}

\begin{document}

\begin{enumerate}
%%%%%%%%%%%%%%%%%%%%%%%%%%%%%%%%%%%%%%%%%%%%%%%%%%%%%%%%%%%%%%%%%%%%%%%%%%%%%%%%%%%%%%%%%%%%%%%%%%%%%%%%%%%%%%%%%%%%%%%%%%%%%%%%%%%%%%%%%%
\item 12 A simple model for the movement of a stock price is such that, independently in each
time period, the stock either:
goes up with probability (1 )
4
% ;
stays the same with probability (5 2 )
8
%   ;
goes down with probability (1 )
8
% .
\begin{enumerate}[(a)]

\item Determine the range of admissible values of the parameter $\theta$ . 
\item (a) Calculate the probability that the stock goes down in one time period, in the case  = 0.1.
(b) Calculate the probability that the stock stays the same for two consecutive time periods, in the case  = 0 .
(c) Calculate the probability that, in four time periods, the stock goes up twice and stays the same twice, in the case  % =  0.2 . [4]
\item Data are collected for 80 consecutive time periods and yield the following
observed frequencies:

\begin{center}
\begin{tabular}{|c|c|c|c|}
change in stock & up & same &  down\\
no. of time periods & 24 & 35 & 21 \\
\end{tabular}
\end{center}

(a) (1) Write down an expression for L() , the likelihood of these
data, and show that
log L( ) 0 % 
%  
% 
reduces to the quadratic equation
% 51202  468 95  0
(2) Explain why one of the roots of this quadratic yields the
maximum likelihood estimate of  and hence determine this
estimate. 
    \item 
\end{enumerate}
(b) (1) Calculate the expected frequencies using the model with the
maximum likelihood estimate of .
(2) Hence perform a $\chi^2$ goodness of fit test of the model and state
your conclusion clearly. 
(c) Comment briefly on what additional information would be needed for
these data in order to investigate the validity of the assumption of
independence used in this model, and comment briefly on how the
validity might be checked. 


%%%%%%%%%%%%%%%%%%%%%%%%%%%%%%%%%%%%%%%%
12 
\begin{itemize}
\item 0 1 1 1 , 3

\begin{itemize}
\item ${\displaystyle 0 \leq \frac{1}{4} -\theta \leq 1 }$  ${\displaystyle  \theta \leq \frac{1}{4}, \theta \geq -\frac{3}{4} }$ -

\item ${\displaystyle 0 \leq \frac{5}{8} +2\theta \leq 1 }$  ${\displaystyle  \theta \leq \frac{3}{16}, \theta \geq -\frac{5}{16} }$ -

\item ${\displaystyle 0 \leq \frac{1}{8} -\theta \leq 1 }$  ${\displaystyle  \theta \leq \frac{1}{8}, \theta \geq -\frac{7}{8} }$ -

\end{itemize}
%%%%%%%%%%%%%%%%%%%%%%%%%%
combining these

\[ - \frac{5}{16} \leq \theta \leq \frac{1}{8}\]
\item
(a) 	 = 0.1 : P(down in one period) = 1 0.1= 0.025
8 
(b) 	 = 0 : P(same in two periods) = [P(same)]2 = (5)2 = 0.391
8
(c) % 	 = 0.2 : P(up) = 0.45, P(same) = 0.225
= 4! (0.45)2 (0.225)2 = 0.062
2!2!
p
\item
(a) (1) ( ) = (1 )24 (5 2 )35 (1 )21
4 8 8
% L     
log = 24log(1 ) 35log(5 2 ) 21log(1 )
4 8 8
% L     
log = 24 2(35) 21 1 5 2 1
4 8 8
 L

equate to zero % 
24(5 2 )(1 ) 70(1 )(1 ) 21(1 )(5 2 ) = 0
8 8 4 8 4 8
    \item 
\end{itemize}
%%%%%%%%%%%%%%%%%%%%%%%%%%%%%%%%%%%%%%%%%%%%%%%%%%%%%%%%%%%%%%%%%%%%%%%%%%%%%%%%%%%%%%%%%%%%%%%%%%%%%%%%%%%%%%%%%%%%%%%%%%%%%%%%%%%%%%%%%%
%  102402  936 190 = 0  51202  468 95 = 0
(2)
468 4682 4(5120)(95) 468 1471.2661 = =
2(5120) 10240


% = + 0.189 or 0.0980
+ 0.189 is inadmissible, 0.0980 is admissible
% MLE ˆ =  0.0980
(b) (1) with ˆ =  0.0980 , estimated probabilities are
P(up) = 0.3480, P(same) = 0.4290, P(down) = 0.2230
Multiply by 80 for expected frequencies:
up : 27.84, same : 34.32, down : 17.84
(2)
% O e (o e)2/e
24 27.84 0.530
35 34.32 0.013
21 17.84 0.560

2 = 1.103

% 2 = 1.103 on (3  1  1) = 1 d.f.
this is well below the 5% point for 2
% 1 from tables (3.841)
%  large P-value, and so no evidence against the model.
the model fits these data well.
(c) We need information on the order of the up/same/down’s;
and some method of investigating whether the up/same/down’s are distributed
randomly.

\end{document}
