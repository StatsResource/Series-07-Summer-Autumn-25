\documentclass[a4paper,12pt]{article}

%%%%%%%%%%%%%%%%%%%%%%%%%%%%%%%%%%%%%%%%%%%%%%%%%%%%%%%%%%%%%%%%%%%%%%%%%%%%%%%%%%%%%%%%%%%%%%%%%%%%%%%%%%%%%%%%%%%%%%%%%%%%%%%%%%%%%%%%%%%%%%%%%%%%%%%%%%%%%%%%%%%%%%%%%%%%%%%%%%%%%%%%%%%%%%%%%%%%%%%%%%%%%%%%%%%%%%%%%%%%%%%%%%%%%%%%%%%%%%%%%%%%%%%%%%%%

\usepackage{eurosym}
\usepackage{vmargin}
\usepackage{amsmath}
\usepackage{graphics}
\usepackage{epsfig}
\usepackage{enumerate}
\usepackage{multicol}
\usepackage{subfigure}
\usepackage{fancyhdr}
\usepackage{listings}
\usepackage{framed}
\usepackage{graphicx}
\usepackage{amsmath}
\usepackage{chngpage}

%\usepackage{bigints}
\usepackage{vmargin}

% left top textwidth textheight headheight

% headsep footheight footskip

\setmargins{2.0cm}{2.5cm}{16 cm}{22cm}{0.5cm}{0cm}{1cm}{1cm}

\renewcommand{\baselinestretch}{1.3}

\setcounter{MaxMatrixCols}{10}

\begin{document}

\begin{enumerate}
%%%%%%%%%%%%%%%%%%%%%%%%%%%%%%%%%%%%%%%%%%%%%%%%%%%%%%%%%%%%%%%%%%
\item 9 Thirty employees working in the call centre of a bank were chosen at random from
the workforce and agreed to assist in the assessment of two different training
programmes (A and B).
Ten of the selected employees were randomly assigned to each of the two training
programmes, while the other ten were to receive no additional training and act as
controls.
Shortly after the training began, one of the employees on training programme B and
two of the control group left the employment of the bank.
At the end of the training period each of the twenty-seven employees still involved
was observed at work over a period of time by performance assessors. Each
employee was then given a performance score (measuring a range of aspects of their
work) expressed as a mark out of a possible maximum of 100.
The results were as follows:
\begin{verbatim}
    Performance scores Totals
Control 55 74 64 62 37 78 50 44 — — 464
Training programme A 63 79 60 75 89 58 75 72 84 69 724
Training programme B 64 55 57 73 51 60 62 78 68 — 568

\end{verbatim}
xij
2 = 118,128 .
\begin{enumerate}[(a)]
\item Plot the data in a simple and informative way which displays both the “within
treatment” and “between treatment” variation. 
\item  Conduct an analysis of variance to investigate whether differences exist
among the three “treatments” and comment briefly on your findings. 
\end{enumerate}


\end{enumerate}

%%%%%%%%%%%%%%%%%%%%%%%%%%%%%%%%%%%%%%%%%%%%%%%%%%%%%%%%%%%%%%%%%%
\newpage
%%-- Solution to Question 9 
\begin{itemize}
\item (i)
^ indicates treatment means : helpful but not required for the marks
“Within treatment” variation is the spread within each set of points, “between
treatment” variation is the spread among the 3 treatment means.
\item (ii) SST = 118128 – 17562/27 = 3923.0
SSB = (4642/8 + 7242/10 + 5682/9) – 17562/27 = 971.67
 SSR = 3923.0 – 971.67 = 2951.3
\begin{verbatim}
Source of variation df SS MSS
Between treatments 2 971.67 485.84
Residual 24 2951.3 122.97
Total 26 3923.0
\end{verbatim}

\item Under H0 : no treatment effects F = 485.84/122.97 = 3.95 on 2,24 df
P-value is < 0.05, so we have some evidence against H0.
0 10 20 30 40 50 60 70 80 90 100
^
^
^
score
control
prog A
prog B
%%%%%%%%%%%%%%%%%%%%%%%%%%%%%%%%%%%%%%%
%%-- Page 6
\item We conclude that there is evidence of differences among the treatments. The
data suggest that training programme A gives a higher mean score than the
others.
\end{itemize}
\end{document}
