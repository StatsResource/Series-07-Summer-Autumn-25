\documentclass[a4paper,12pt]{article}

%%%%%%%%%%%%%%%%%%%%%%%%%%%%%%%%%%%%%%%%%%%%%%%%%%%%%%%%%%%%%%%%%%%%%%%%%%%%%%%%%%%%%%%%%%%%%%%%%%%%%%%%%%%%%%%%%%%%%%%%%%%%%%%%%%%%%%%%%%%%%%%%%%%%%%%%%%%%%%%%%%%%%%%%%%%%%%%%%%%%%%%%%%%%%%%%%%%%%%%%%%%%%%%%%%%%%%%%%%%%%%%%%%%%%%%%%%%%%%%%%%%%%%%%%%%%

\usepackage{eurosym}
\usepackage{vmargin}
\usepackage{amsmath}
\usepackage{graphics}
\usepackage{epsfig}
\usepackage{enumerate}
\usepackage{multicol}
\usepackage{subfigure}
\usepackage{fancyhdr}
\usepackage{listings}
\usepackage{framed}
\usepackage{graphicx}
\usepackage{amsmath}
\usepackage{chngpage}

%\usepackage{bigints}
\usepackage{vmargin}

% left top textwidth textheight headheight

% headsep footheight footskip

\setmargins{2.0cm}{2.5cm}{16 cm}{22cm}{0.5cm}{0cm}{1cm}{1cm}

\renewcommand{\baselinestretch}{1.3}

\setcounter{MaxMatrixCols}{10}

\begin{document} 

%%- Question 13 

The table below gives the frequency of coronary heart disease by age group. The
table also gives the age group midpoint (x) and
ˆ
= log 1 ˆ
y
  
     
, where ˆ denotes the
proportion in an age group with coronary heart disease.
Coronary Heart
Disease
\begin{verbatim}
   Age group x Yes No n y
20–29 25 1 9 10 2.19722
30–34 32.5 2 13 15 1.87180
35–39 37.5 3 9 12 1.09861
40–44 42.5 5 10 15 0.69315
45–49 47.5 6 7 13 0.15415
50–54 52.5 5 3 8 0.51083
55–59 57.5 13 4 17 1.17865
60–69 65 8 2 10 1.38629 
\end{verbatim}

x = 360; x2 =17437.5;y =  2.9392; y2 =13.615; xy =  9.0429
\begin{enumerate}[(a)]
\item (i) (a) Calculate an estimate of the probability of having coronary heart
disease under the assumption that the probability does not differ over
the age groups.
\item (b) Construct an 8 	 2 contingency table with marginal totals and conduct
a 2
 test for differences in the probability of having coronary heart
disease for the different age groups. [8]
\item (ii) Consider the regression model y =  x .

\begin{enumerate}
    \item (a) Draw a scatterplot of y against x, and comment on the appropriateness
of the suggested model.
    \item (b) Calculate the least squares fitted regression line of y on x.
    \item (c) Calculate a 99\% confidence interval for the slope parameter.
    \item (d) Interpret the result obtained in (i)(b) with reference to the confidence
interval obtained in (ii)(c). 
\end{enumerate}

\end{enumerate}
%%%%%%%%%%%%%%%%%%%%%%%%%%%%%%%%%%%%%%%%%%%%%%%%%%%%%%%%%%%%%%%%%%

\newpage
%%%%%%%%%%%%%%%%%%%%%%%%%%%%%%%%%%%%%%%%%%%%%%%%%%%%%%%%%%%%%%%%%%
13 
\begin{itemize}
    \item (i) ni = number in group i , ri = number with coronary heart disease in group i.
(a) ri = 43 ni =100
Estimate of the probability of having coronary heart disease is given by
ˆ = = 43 = 0.43.
100
i
i
r
n



(Assuming constant probability of having coronary heart disease over
age groups.)
\item (b)
Coronary heart
disease
Yes No Total
Age groups 20–29 1
(4.30)
9
(5.70)
10
30–34 2
(6.45)
13
(8.55)
15
35–39 3
(5.16)
9
(6.84)
12
40–44 5
(6.45)
10
(8.55)
15
45–49 6
(5.59)
7
(7.41)
13
50–54 5
(3.44)
3
(4.56)
8
55–59 13
(7.31)
4
(9.69)
17
60–69 8
(4.30)
2
(5.70)
10
Total 43 57 100
%%%%%%%%%%%%%%%%%%%%%%%%%%%%%%%%%%%%%%%%%%%%%%%%%%%%%%%%%%%%%%%%%%
\item The expected values assuming a constant probability of having coronary heart disease
are given in parentheses (= row total ˆ ).
2 2 2
2 = ( ) = (1 4.30) ... (2 5.7)
4.30 5.7
i i
i
f e
e
  
    = 26.6 on 7 d.f.
2
7 (0.01) 18.48.
\item Strongly reject null hypothesis of constant probability over the
different age groups. [Note: If one decides to combine cells to
safeguard against very low expected frequencies, one should combine
adjacent cells.]
\item (ii) (a)
Linear model seems appropriate, but extremes (x = 25 and x = 65) are
not as good as 32.5–57.5 age range.
\item (b) Sxx = 17437.5
3602
8  = 1237.5
Syy =
( 2.9392)2 13.615
8

 = 12.535
Sxy = 9.0429 (360)( 2.9392)
8

  = 123.22
Least squares estimates:
ˆ = = 123.22 = 0.09957
1237.5
xy
xx
S
S 
25 35 45 55 65
1
0
-1
-2
x
y
%%%%%%%%%%%%%%%%%%%%%%%%%%%%%%%%%%%%%%%
Page 11
ˆ = y ˆ x =  0.3674  0.09957(45) =  4.85
\item (c)
2
ˆ 2 = xy /( 2)
yy
xx
S
S n
S
 
    
 
 
(123.22)2 = 12.535 / 6
1237.5
 
    
 
= 0.04430
 ˆ = 0.210 on 6 d.f.
s.e.(ˆ
) =
ˆ 0.210
Sxx 1237.5

 = 0.0060
99% CI for 6 ˆ ˆ : (0.005) t   s.e. (ˆ
)
= 0.0996  3.707 (0.0060)
= 0.0996  0.0222 i.e. (0.0774, 0.1218)
\item (d) In (i)(b), the probability of having coronary heart disease was found to vary with age. The 99\% confidence interval for the slope parameter in the regression obtained in (ii)(c) also shows that the probability of having coronary heart disease depends (linearly) on age as zero is not within the interval.
\end{itemize}
\end[document}
