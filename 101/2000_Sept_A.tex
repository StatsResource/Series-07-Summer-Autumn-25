\documentclass[a4paper,12pt]{article}

%%%%%%%%%%%%%%%%%%%%%%%%%%%%%%%%%%%%%%%%%%%%%%%%%%%%%%%%%%%%%%%%%%%%%%%%%%%%%%%%%%%%%%%%%%%%%%%%%%%%%%%%%%%%%%%%%%%%%%%%%%%%%%%%%%%%%%%%%%%%%%%%%%%%%%%%%%%%%%%%%%%%%%%%%%%%%%%%%%%%%%%%%%%%%%%%%%%%%%%%%%%%%%%%%%%%%%%%%%%%%%%%%%%%%%%%%%%%%%%%%%%%%%%%%%%%

\usepackage{eurosym}
\usepackage{vmargin}
\usepackage{amsmath}
\usepackage{graphics}
\usepackage{epsfig}
\usepackage{enumerate}
\usepackage{multicol}
\usepackage{subfigure}
\usepackage{fancyhdr}
\usepackage{listings}
\usepackage{framed}
\usepackage{graphicx}
\usepackage{amsmath}
\usepackage{chngpage}

%\usepackage{bigints}
\usepackage{vmargin}

% left top textwidth textheight headheight

% headsep footheight footskip

\setmargins{2.0cm}{2.5cm}{16 cm}{22cm}{0.5cm}{0cm}{1cm}{1cm}

\renewcommand{\baselinestretch}{1.3}

\setcounter{MaxMatrixCols}{10}

\begin{document}
\begin{enumerate}
\item 1 A random sample of fifty claim amounts (£) arising in a particular section of an
insurance company's business are displayed below in a stem and leaf plot:
15
16
17
18
19
20
21
22
23
24
25
26
27
28
29
30
31
14678
0233368889
0000001233457888
3456779
0257
03
07
3
38
2
Stem unit = 100
Leaf unit = 10
The sum of the fifty amounts (before rounding) is £92780.
Calculate the mean and median claim amounts. 
%%%%%%%%%%%%%%%%%%%%%%%%%%%%%%%%%%%%%%%%%%%%%%%%%%%%%%%%%%%%%%%%%
\item Consider a random sample of 47 white-collar workers and a random sample of 24 blue-collar workers from the workforce of a large company. The mean salary for
the sample of white-collar workers is £28,470 and the standard deviation is £4,270; whereas the mean salary for the sample of blue-collar workers is £21,420
and the standard deviation is £3,020.
Calculate the mean and the standard deviation of the salaries in the combined
sample of 71 employees. 
3 The number of claims arising in a period of one month from a group of policies
can be modelled by a Poisson distribution with mean 24.
Determine the probability that fewer than 20 claims arise in a particular month.



%%%%%%%%%%%%%%%%%%%%%%%%%%%%%%%%%%%%%%%%%%%%%%%%%%%%%%%%%
\newpage
1 Mean = 92780/50 = £1855.60
25.5th value in order = (17.3 + 17.4)/2 = 17.35 so median amount = £1735
2 For simplicity change the units into thousands. The sum of the data is
Σx = 47 × 28.47 + 24 × 21.42 = 1,852.17.
So, the mean salary is 1852.17
71 = 26.087, or £26,087.
The sum of squares is given by
Σx2 = {46 × 4.272 + 47 × 28.472} + {23 × 3.022 + 24 × 21.422} = 50,155.5.
Thus, the standard deviation of the 71 values is
50155.5 1852.172 71
70
−
= 5.124, or £5,124.
%%%%%%%%%%%%%%%%%%%%%%%%%%%%%%%%%%%%%%%%%%%%%%%%%%%%%%%%%%%%%%%%%%%%%
3 Can get answer directly from Green tables:
\[P(X < 20) = P(X \leq 19) = 0.18026 = 0.18\]
OR: use normal approximation with continuity correction.
$X \sim  N(24, \sqrt{24}^2 )$

$P(X < 20) \rightarrow P(X < 19.5)$

\begin{eqnarray*}
P(X < 19.5) &=& P \left( Z < \frac{19.5-24}{\sqrt{24}} \right) \\ 
&=& P\left(Z \leq -0.92 \right) \\
&=&  1 − 0.82 \\
&=&  0.18.\\
\end{eqnarray*}
(Note: without continuity correction answer is 0.15)



%%%%%%%%%%%%%%%%%%%%%%%%%%%%%%%%%%%%%%%%%%%%%%%%%%%%%%%
\newpage


4 Suppose that a random sample of nine observations is taken from a normal
distribution with mean $\mu = 0$. Let X and S 2 denote the sample mean and
variance respectively.
Determine (to 2 decimal places) the probability that the value of X exceeds that of S, i.e. determine P( X > S) . 

4 $\mu = 0$, n = 9 so $X / (S / 3) \sim t_8$ so P( X > S) = P(t8 > 3) ,
which, from tables, is just less than 0.01
(actually 0.0085, but not needed for the marks)
%%%%%%%%%%%%%%%%%%%%%%%%%%%%%%%%%%%%%%%%%%%%%%%%%%%%%%%%%
Page 3
5 x
x
\pm 196
500
. i.e. $0.168 \pm (1.96 \times 0.01833) \mbox{ i.e. } 0.168 \pm 0.036$

\end{document}
