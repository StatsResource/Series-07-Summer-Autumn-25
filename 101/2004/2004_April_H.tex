\documentclass[a4paper,12pt]{article}

%%%%%%%%%%%%%%%%%%%%%%%%%%%%%%%%%%%%%%%%%%%%%%%%%%%%%%%%%%%%%%%%%%%%%%%%%%%%%%%%%%%%%%%%%%%%%%%%%%%%%%%%%%%%%%%%%%%%%%%%%%%%%%%%%%%%%%%%%%%%%%%%%%%%%%%%%%%%%%%%%%%%%%%%%%%%%%%%%%%%%%%%%%%%%%%%%%%%%%%%%%%%%%%%%%%%%%%%%%%%%%%%%%%%%%%%%%%%%%%%%%%%%%%%%%%%

\usepackage{eurosym}
\usepackage{vmargin}
\usepackage{amsmath}
\usepackage{graphics}
\usepackage{epsfig}
\usepackage{enumerate}
\usepackage{multicol}
\usepackage{subfigure}
\usepackage{fancyhdr}
\usepackage{listings}
\usepackage{framed}
\usepackage{graphicx}
\usepackage{amsmath}
\usepackage{chngpage}

%\usepackage{bigints}
\usepackage{vmargin}

% left top textwidth textheight headheight

% headsep footheight footskip

\setmargins{2.0cm}{2.5cm}{16 cm}{22cm}{0.5cm}{0cm}{1cm}{1cm}

\renewcommand{\baselinestretch}{1.3}

\setcounter{MaxMatrixCols}{10}

\begin{document}

12 For the estimation of a binomial probability p = P(success), a series of n independent
trials are performed and X represents the number of successes observed.
\begin{enumerate}[(a)]
    \item (i) Write down the likelihood function $L(p)$ and show that the maximum
likelihood estimator (MLE) of p is =
X
p
n
.

%%%%%%%%%%%%%%%%%%%%%%%%%%5
    \item (ii) (a) Determine the Cramer-Rao lower bound for the estimation of p.
(b) Show that the variance of the MLE is equal to the Cramer-Rao lower
bound.
    \item (c) Write down an approximate sampling distribution for p valid for
large n.
%%%%%%%%%%%%%%%%%%%%%%%%%%5
    \item (iii) In order to develop an approximate 95\% confidence interval for $p$ for large $n$,
the following pivotal quantity is to be used
(0,1)
(1 )
p p
N
p p
n
.
Assuming that this pivotal quantity is monotonic in p, show that rearrangement
of the inequality
1.96 1.96
(1 )
p p
p p
n
leads to a quadratic inequality in p, and hence determine an approximate 95%
confidence interval for p of the form pL ( p) p pU ( p) .

%%%%%%%%%%%%%%%%%%%%%%%%%%5
    \item (iv) A simpler and more widely used approximate confidence interval is obtained
by using the following pivotal quantity
(0,1)
(1 )
p p
N
p p
n
.
Determine the resulting approximate 95\% confidence interval using this. 
\end{enumerate}


(v) In two separate applications the following data were observed:
(a) 4 successes out of 10 trials
(b) 80 successes out of 200 trials
In each case calculate the two approximate confidence intervals from parts
(iii) and (iv) and comment briefly on your answers.
\end{enumerate}
%%%%%%%%%%%%%%%%%%%%%%%%%%%%%%%%%
\newpage

12 (i) %%%%%%%%%%%%%%%%%%%%%%%%%%%%%%%%%%%%%%%%%%%%%%%%%%%%%%%%%%%%%%%%5

%%- April 2004 Question 12

\[L ( p ) = {n \choose x} p^x (1-p)^{n-x}\]

\[ \log L ( p )  = \log {n \choose x} + x \log p + (n-x) \log(1-p)\]

\[ \frac{\partial}{\partial p} \left( \log L ( p ) \right)  =  0 +  \frac{x}{p} +  \frac{n-x}{1-p} \]


equate to zero. \[ \frac{x}{p} +  \frac{n-x}{1-p} = 0\]

i.e. 
\[\frac{x}{p} =  \frac{n-x}{1-p}

\[ x(1-p) = (n-x)p\]

\[p = \frac{x}{n}\]
clearly maximises $L(p)$
%%%%%%%%%%%%%%%%%%%%%%%%%%%
\begin{framed}

\[ \log( A \times B \times C) = \log(A) + \log(B) + \log(C) \]

\[\frac{d}{dx} \left( \log x \right) = \frac{1}{x} \]

\end{framed}
%%%%%%%%%%%%%%%%%%%%%%%%%%

%%%%%%%%%%%%%%%%%%%%%%%%%%%%%%%%%%%%%
Page 7
MLE is
X
p
n
(ii) 

\[ \frac{\partial^2}{\partial p^2} \left( \log L ( p ) \right)  =  -\frac{x}{p^2} -  \frac{n-x}{(1-p)^2} \]

\begin{eqnarray} 
E\left[\frac{\partial^2}{\partial p^2} \left( \log L ( p ) \right) \right] 
&=& -\frac{np}{p^2} -  \frac{n-np}{(1-p)^2} \\
&=& -\frac{n}{p} -  \frac{n}{1-p} \\
&=& - \frac{n}{p(1-p)}\\
\end{eqnarray*}


CRlb
n n
p p
(b) Var( p ) = 2
np(1 p) p(1 p)
CRlb
n n
(c)
(1 )
( , ) ( , )
p p
p N p CRlb N p
n
(iii) 1.96 1.96
(1 )
p p
p p
n
2
( ) 2
1.96
(1 )
p p
p p
n
2
2 2 1.96 2
p 2 pp p ( p p )
n
2 2
1.96 2 1.96 2
(1 ) p (2 p ) p p 0
n n
\begin{itemize}
    \item This is a quadratic and will be negative between its two roots.
So, pL , pU will be the two roots:
2 2 2
2 2
2
1.96 1.96 1.96
(2 ) (2 ) 4 (1 )
1.96
2(1 )
p p p
n n n
n
with pL from the " " sign, and pU from the "+" sign.
%%%%%%%%%%%%%%%%%%%%%%%%%%%%%%%%%%%%%
\item (iv) 1.96 1.96
(1 )
p p
p p
n
(1 ) (1 )
1.96 1.96
p p p p
p p p
n n
giving pL and pU .
\item (v) (a) x = 4, n = 10
quadratic from (iii) is 1.38416 p2 1.18416 p 0.16
roots give pL = 0.168 and pU = 0.687.
from (iv) pL = 0.096 and pU = 0.704.
quite a difference, especially in pL , but n = 10 is small.
\end{itemize}
(b) x = 80, n = 200
quadratic from (iii) is 1.019208 p2 0.819208 p 0.16
roots give pL = 0.335 and pU = 0.469.
from (iv) pL = 0.332 and pU = 0.468.
very similar ( and much narrower than (a)) with n = 200 being large.
In (i) many candidates wanted to write the likelihood as a product this is OK using
Bernoulli probabilities as the individual components, as in
1
1
( ) 1 i i
n
x x
i
L p p p
where xi = 1 or 0, but not as a product of binomial(n,p) components.
\end{document}
