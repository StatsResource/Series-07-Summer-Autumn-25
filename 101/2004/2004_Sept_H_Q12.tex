\documentclass[a4paper,12pt]{article}

%%%%%%%%%%%%%%%%%%%%%%%%%%%%%%%%%%%%%%%%%%%%%%%%%%%%%%%%%%%%%%%%%%%%%%%%%%%%%%%%%%%%%%%%%%%%%%%%%%%%%%%%%%%%%%%%%%%%%%%%%%%%%%%%%%%%%%%%%%%%%%%%%%%%%%%%%%%%%%%%%%%%%%%%%%%%%%%%%%%%%%%%%%%%%%%%%%%%%%%%%%%%%%%%%%%%%%%%%%%%%%%%%%%%%%%%%%%%%%%%%%%%%%%%%%%%

\usepackage{eurosym}
\usepackage{vmargin}
\usepackage{amsmath}
\usepackage{graphics}
\usepackage{epsfig}
\usepackage{enumerate}
\usepackage{multicol}
\usepackage{subfigure}
\usepackage{fancyhdr}
\usepackage{listings}
\usepackage{framed}
\usepackage{graphicx}
\usepackage{amsmath}
\usepackage{chngpage}

%\usepackage{bigints}
\usepackage{vmargin}

% left top textwidth textheight headheight

% headsep footheight footskip

\setmargins{2.0cm}{2.5cm}{16 cm}{22cm}{0.5cm}{0cm}{1cm}{1cm}

\renewcommand{\baselinestretch}{1.3}

\setcounter{MaxMatrixCols}{10}

\begin{document}

\begin{enumerate}
\item
12
The following data refer to an outbreak of botulism, a form of food poisoning that
may be fatal. Each subject is a person who contracted botulism in the outbreak. The
variables recorded are the subject s age in years, the time in hours between eating the
infected food and the first signs of illness (incubation period) and whether the subject
survived (denoted by survival category Y) or died (denoted by survival category N).
\begin{verbatim}
    Subject
Age ( x )
Incubation
period ( y )
Survival
1 2 3 4 5 6 7 8 9 10 11 12 13 14 15 16 17 18
29 39 44 37 42 17 38 43 51 30 32 59 33 31 32 32 36 50
13 46 43 34 20 20 18 72 19 36 48 44 21 32 86 48 28 16
N Y Y N N Y N Y N N N Y N N Y N Y N
Died: x = 405; y = 305; x 2 = 15517; y 2 = 10035
Survived: x = 270; y = 339; x 2 = 11396; y 2 = 19665
\end{verbatim}

\begin{enumerate}[(a)]
\item (i)
A scatterplot of incubation period against age is given below, in which
different symbols are used for subjects who died and for subjects who
survived.
A Plot of Incubation Period against Age
90
Died
Survived
80
70
60
50
40
30
20
10
20
30
40
50
60
Age
Comment briefly on any relationships between age and incubation period for
those subjects who died and for those subjects who survived.
\item  (ii)
Construct suitable dotplots to investigate any relationship between:
(a) age and survival, and
(b) incubation period and survival
and make a brief informal comparison of the died and survived groups based
on these dotplots.
\item (iii)
Construct 95% and 99% confidence intervals for the mean difference between
the incubation period for subjects who survived and subjects who died (i.e.
take the mean incubation period for subjects who survived minus the mean
incubation period for subjects who died).
Comment briefly on these confidence intervals.
\item  (iv)
%---------------------%
7
[6]
(a) Conduct a test to investigate whether the variances of the incubation
periods for subjects who died and subjects who survived are equal.
(b) Comment on the validity of the assumptions that are required for the
confidence intervals given in part (iii) to be appropriate.
\end{enumerate}


12

(i) There does not seem to be a clear relationship between age and incubation
period either for those subjects who died or for those subjects who survived.
(ii) (a)
Dotplots of age for died and survived subjects.
Died:
. ..:.
. .
.
..
-+---------+---------+---------+---------+---------+-----Age
Survived:
.
.
.
.
..
.
-+---------+---------+---------+---------+---------+-----Age
16.0
24.0
32.0
40.0
48.0
56.0
(b)
Dotplots of incubation period for died and survived subjects.
Died:
. ..:.
. ..
:
-----+---------+---------+---------+---------+---------+-IncubP
Survived:
.
.
: .
.
.
-----+---------+---------+---------+---------+---------+-IncubP
15
30
45
60
75
90
The dotplots suggest an association between survival and incubation period
(the people who survived tended to have longer incubation periods), but do not
suggest an association between survival and age.
(iii)
Survived: n 1
7 , y 1
Died: n 2 11 , y 2
339 / 7 48 . 429 , s 1
305 / 11
27 . 727 , s 2
3247 . 71 / 6 23 . 266
1578 . 18 / 10 12 . 563
Pooled variance and standard deviation:
s 2 p (n 1 - 1 )s 1 2 (n 2 - 1 )s 2 2
n 1 n 2 - 2
s p 17 . 37
6 ( 23 . 266 ) 2 10 ( 12 . 563 ) 2
16
301 . 633
95% confidence interval:
y 1
y 2 t 0.025, n 1
n 2 2 s p
48.429 27.727
Page 6
1
n 1
1
n 2
(2.120)(17.37)
1 1
7 11
%--------------------------------------%
Examiners Report
i.e. ( 2.9, 38.5) hours.
99% confidence interval:
y 1
y 2 t 0.005, n 1
n 2 2 s p
1
n 1
1
n 2
48.429 27.727 (2.921)(17.37)
1 1
7 11
20.702 24.531
i.e. ( 3.8, 45.2) hours.
\begin{itemize}
    \item The 95\% confidence interval does not contain zero, therefore a two-sided, two
sample t-test conducted at the 5\% significance level would conclude that there
is a difference between the means of the two populations.
\item However, as the 99\% confidence interval does contain zero, conducting a
more stringent 1\% level t-test would not reject the null hypothesis that the
population means are the same.
\end{itemize}

(iv)
(a)
Assuming that the variances of the two populations are equal,
S 1 2 / S 2 2 ~ F n 1 1, n 2 1 .
s 1 2 / s 2 2
23.266 2 /12.563 2
3.430
The value of the test statistic is below the 5% significance level critical
value of F 6,10 (2.5%) = 4.072. This indicates that there is insufficient
evidence to reject the null hypothesis that the two populations have
equal variances.
(b)
In part (iii) an assumption of normality of each sample was required.
The dotplots suggest that this assumption is valid.
Also, the assumption of equal variances of the two groups seems valid
from the result of the test conducted in part (iv)(a).

\end{document}
