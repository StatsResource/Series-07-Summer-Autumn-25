%-ACT 101 2000 April Q15

\documentclass[a4paper,12pt]{article}

%%%%%%%%%%%%%%%%%%%%%%%%%%%%%%%%%%%%%%%%%%%%%%%%%%%%%%%%%%%%%%%%%%%%%%%%%%%%%%%%%%%%%%%%%%%%%%%%%%%%%%%%%%%%%%%%%%%%%%%%%%%%%%%%%%%%%%%%%%%%%%%%%%%%%%%%%%%%%%%%%%%%%%%%%%%%%%%%%%%%%%%%%%%%%%%%%%%%%%%%%%%%%%%%%%%%%%%%%%%%%%%%%%%%%%%%%%%%%%%%%%%%%%%%%%%%

\usepackage{eurosym}
\usepackage{vmargin}
\usepackage{amsmath}
\usepackage{graphics}
\usepackage{epsfig}
\usepackage{enumerate}
\usepackage{multicol}
\usepackage{subfigure}
\usepackage{fancyhdr}
\usepackage{listings}
\usepackage{framed}
\usepackage{graphicx}
\usepackage{amsmath}
\usepackage{chngpage}

%\usepackage{bigints}
\usepackage{vmargin}

% left top textwidth textheight headheight

% headsep footheight footskip

\setmargins{2.0cm}{2.5cm}{16 cm}{22cm}{0.5cm}{0cm}{1cm}{1cm}

\renewcommand{\baselinestretch}{1.3}

\setcounter{MaxMatrixCols}{10}

\begin{document}
%%%%%%%%%%%%%%%%%%%%%%%%%%%%%%%%%%%%%%%%%%%%%%%%%%%%%%%%%%%%%%%%%%%%%%%%%%%%%%%%%%%%%
\newpage
\newpage

\item 13
For each of a group of policyholders the number of claims, Y, occurring in a period of
one year is modelled by the following modified Poisson random variable, which
incorporates a reluctance to claim:
there is a probability of that Y equals zero and a probability of (1
comes from a Poisson distribution with mean , so that
P ( Y = 0) =
(1
P ( Y = r ) = (1
) that Y
) P ( X = 0)
) P ( X = r ), r = 1, 2,3,
where X ~ Poisson( ).
\begin{enumerate}[(a)]
\item (i)
Show that the mean and variance of Y are given by
E ( Y ) (1
V ( Y ) (1
)
) (1
)
and comment briefly on these values in comparison to those for the
unmodified Poisson variable, X.
[5]
\item (ii) A random sample of such policyholders resulted in a distribution of numbers
of claims with sample mean y and sample standard deviation s. Use the
method of moments to determine estimators for and in terms of y and s 2 .
[4]
\item (iii) Data on the numbers of claims from a sample of 200 policyholders resulted in
the following frequency distribution.
number of claims:
frequency:
0
90
1
56
2
37
3
12
4
4
5
1
>5
0
(a) Calculate the mean and variance for this sample and hence calculate
the method of moments estimates of and .
(b) Using these estimates the expected frequencies under the fitted
modified Poisson model were calculated for y = 0, 1, 2, 3 and are given
in the table below.
y exp. freq.
0 88.9
1 59.2
2 34.0
3 13.1
Calculate the expected frequencies for y = 4, 5 and y > 5 and comment
briefly on the suitability of the model for these data.
\end{enumerate}

13
(i)
E(Y) = 0 . P(Y = 0) +
September 2004
r . P ( Y
Examiners Report
r )
r 1
= (1
)
r ) = (1
r . P ( X
)E(X) = (1
)
r 1
[Note: there is an alternative solution using conditional expectation]
r 2 . P ( Y
E(Y 2 ) = 0 +
r )
r 1
= (1
)
r 2 . P ( X r ) = (1
2 ) ) 2
)E(X 2 ) = (1
2
)(
r 1
V(Y) = (1
)(
= (1
)
(1 +
(1
2
= (1
)
{1 +
(1
)
}
)
E(Y) < E(X) as expected since there are more values equal to zero in the
adjusted distribution.
The original Poisson has V(X) = E(X). Here, with the extra zeros, we get
greater variability relative to the mean, as we see in the fact that V(Y) > E(Y).
(ii)
For method of moments we seek
y
s 2
(1
and
)
(1
) (1
First equation gives:
)
y
(1
)
Substituting into second equation gives:
Solving for
s 2
gives:
y
2
s 2
y
1
1
y
s
Substituting into expression for
Page 8
being solutions of
2
y
gives:
y 2
s 2
y
y
y
(1
)
%%%%%%%%%%%%%%%%%%%%%%%%%%%%%%%%%%%%%%%%%%%%%%%%%%%%%%%%%%%%%%%%%%%%%%%%
(iii)
(a)

n = 200, y = 187, y 2 = 401
187
200
y
0.935
401 187 2 / 200
1.1365
199
s 2
1.1365 0.935
0.935 2 1.1365 0.935
0.1873
0.935 2 1.1365 0.935
1.1505
0.935
(b)
P ( X
P ( Y
1.1505 4 e
4)
4!
1.1505
0.0231
4) 0.8127(0.0231) 0.0188
Exp. freq. = 200(0.0188) = 3.8
Similarly
P(X = 5) = 0.0053, P(Y = 5) = 0.0043,
and exp. freq. = 0.9
By subtraction exp. freq. for y > 5 is 200 - 199.9 = 0.1
obs.
exp.
90
88.9
56
59.2
37
34.0
12
13.1
4
3.8
1
0.9
0
0.1
which show good agreement with the observed frequencies so that the
model seems to fit the data well.

\end{itemize}
\end{document}
