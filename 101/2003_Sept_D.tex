\documentclass[a4paper,12pt]{article}

%%%%%%%%%%%%%%%%%%%%%%%%%%%%%%%%%%%%%%%%%%%%%%%%%%%%%%%%%%%%%%%%%%%%%%%%%%%%%%%%%%%%%%%%%%%%%%%%%%%%%%%%%%%%%%%%%%%%%%%%%%%%%%%%%%%%%%%%%%%%%%%%%%%%%%%%%%%%%%%%%%%%%%%%%%%%%%%%%%%%%%%%%%%%%%%%%%%%%%%%%%%%%%%%%%%%%%%%%%%%%%%%%%%%%%%%%%%%%%%%%%%%%%%%%%%%

\usepackage{eurosym}
\usepackage{vmargin}
\usepackage{amsmath}
\usepackage{graphics}
\usepackage{epsfig}
\usepackage{enumerate}
\usepackage{multicol}
\usepackage{subfigure}
\usepackage{fancyhdr}
\usepackage{listings}
\usepackage{framed}
\usepackage{graphicx}
\usepackage{amsmath}
\usepackage{chngpage}

%\usepackage{bigints}
\usepackage{vmargin}

% left top textwidth textheight headheight

% headsep footheight footskip

\setmargins{2.0cm}{2.5cm}{16 cm}{22cm}{0.5cm}{0cm}{1cm}{1cm}

\renewcommand{\baselinestretch}{1.3}

\setcounter{MaxMatrixCols}{10}

\begin{document}

\begin{enumerate}
\item
13 In an air pollution monitoring study undertaken in a residential area near an industrial
plant a sample of 20 sites is chosen and an observation at each site is made of the
content (in parts per million) of a particular contaminant. The data recorded are given
in the table below together with some summaries:
76 78 76 78 84 79 79 81 85 76
78 79 75 83 87 80 78 77 81 77

x = 1,587 
x2 = 126,131
(i) (a) Present these data graphically using a dotplot and comment briefly on
the shape of the distribution.
(b) Calculate a 95% confidence interval for the mean contaminant content
for the residential area from which the sites were selected.
(c) Comment briefly on the validity of this confidence interval in the light
of your answer to part (a). [8]
(ii) After some modifications by the operators of the industrial plant designed to
reduce the level of pollution due to this contaminant, another observation was
made at each of the same sites. The data recorded are given in the table below
with the sites in the same order as in the table above:
74 74 76 79 83 76 76 81 84 76
81 77 74 83 89 78 77 72 79 78
(a) Calculate the difference (before  after) in contaminant content for
each site, present these differences graphically and comment briefly on
the shape of the distribution.
(b) Perform an appropriate test to investigate whether the modification has
led to a reduction in the contaminant content.
(c) Comment briefly on the validity of this test in the light of your answer
to part (ii)(a). [9]
%%%%%%%%%%%%%%%%%%%%%%%%%%%%%%%%%%%%%%%%%%%%%%%%%%%%%%%%%%%%%%%%%%%%%%%%%%%%%%%%%%%%%%%%%%%%%%%%%%
\item Consider a linear regression model in which responses Yi are uncorrelated and have
expectations xi and common variance 2 (i 1, ,n)  , i.e. Yi is modelled as a linear
regression through the origin:
E(Yi | xi )  xi and V(Yi | xi )  2 (i 1, ,n)  .
(i) (a) Show that the least squares estimator of  is 2
1 1 1
ˆ / n n
i i i i i xY x
 
   .
(b) Derive the expectation and variance of 1 ˆ under the model. 
(ii) An alternative to the least squares estimator in this case is:
2
1 1
ˆ / /
n n
i i
i i
Y x Y x
 
    .
(a) Derive the expectation and variance of 2 ˆ under the model.
(b) Show that the variance of the estimator 2 ˆ is at least as large as that of
the least squares estimator 1 ˆ . 
(iii) Now consider an estimator 3 ˆ of  which is a linear function of the
responses, i.e. an estimator which has the form 3
1
ˆ ,
n
i i
i
a Y

  where a1, ,an 
are constants.
(a) Show that 3 ˆ is unbiased for  if 1 1 n
i i i a x

  , and that the variance
of 3 ˆ is 2 2
1
n
i i a

  .
(b) Show that the estimators 1 ˆ and 2 ˆ above may be expressed in the
form 3
1
ˆ
n
i i
i
a Y

  and hence verify that 1 ˆ and 2 ˆ satisfy the
condition for unbiasedness in (iii)(a).
(c) It can be shown that, subject to the condition 1 1 n
i i i a x

  , the
variance of 3 ˆ is minimised by setting
2
1
i .
i n
i i
a x
x



Comment on this result. [7]
%%%%%%%%%%%%%%%%%%%%%%%%%%%%%%%%%%%%%%%%%%%%%%%%%%%%%%%%%%%%%%%%%%%%%%%%%%%%%%%%%%%%%%%%%%%%%%%%%%
\item Let S(0) denote the price of a certain security, and let S(n) denote the price of the
security at the end of n successive weeks for n = 1, 2, 3, .... A model for the changes
in these prices is such that the price ratios ( )
( 1)
S n
S n
for n 1 are independent
identically distributed random variables with a lognormal distribution.
[Note: The random variable Y is lognormal with parameters  and 2 if log(Y) is
normal N(,2), that is, Y is lognormal if it can be expressed as Y  eX where
X ~ N(,2). The mean and variance of Y are given by E(Y) =
2
e 2

 and
V(Y) = e2 2 (e 2 1)  
 .]
(i) Using the above lognormal model with parameters  = 0.0125 and  = 0.055,
determine the probability that:
(a) the price of the security decreases over the next week
(b) the price of the security decreases over each of the next two weeks
(c) the price at the end of two weeks is greater than it is at present
(d) the price at the end of 20 weeks is less than it is at present
[11]
(ii) At the start of a period the price is £1,245 and it is then observed for 10 weeks.
The resulting 10 prices (£) and ratios are given in the following table:
Week Price Ratio(y)
0 1,245 -
1 1,230 0.988
2 1,280 1.041
3 1,392 1.088
4 1,431 1.028
5 1,428 0.998
6 1,439 1.008
7 1,346 0.935
8 1,265 0.940
9 1,513 1.196
10 1,468 0.970
y = 10.192 y2 = 10.441562
(a) Based on the specified model, explain why the 10 ratios constitute a
random sample from a lognormal distribution.
(b) Calculate the mean and the standard deviation for the random sample
of 10 ratios.
(c) Hence determine the method of moments estimates of the parameters 
and  for the lognormal model. 
\end{enumerate}
\newpage
%%%%%%%%%%%%%%%%%%%%%%%%%%%%%%%%%%%%%%%%%%%%%%%%%%%%%%%%%%%%%%%%%%%%%%%%%%%%%%%%%%%%%%%%%%%%%%%%%%

13 (i) (a)
. : .
. : : : : . : . . . .
---+---------+---------+---------+---------+---------+---ppm
75.0 77.5 80.0 82.5 85.0 87.5
Dotplot shows moderate positive skewness
(b)
2
2
126131 1587 1587 79.35, 20 10.66
20 19
x s
−
= = = =
95% confidence interval is
2
0.025,19 20
x ± t s
giving 79.35 2.093 10.66 79.35 1.53 (77.82,80.88)
20
± ⇒ ± ⇒
(c) This t confidence interval requires normality of the observations.
This may be doubtful in view of the skewness shown in part (a), but
the sample size of 20 is perhaps large enough to justify the validity due
to the robustness of the t analysis.
%%%%%%%%%%%%%%%%%%%%%%%%%%%%%%%%%%%%%%%%%%%%%%%%%%%%%%%%%%%%%%%%%%%%%%%%%%
Page 9
(ii) (a) Differences (before − after) are:
2 4 0 -1 1 3 3 0 1 0
-3 2 1 0 -2 2 1 5 2 -1
Dotplot of differences:
: : :
. . : : : : : . .
-+---------+---------+---------+---------+---------+-----diff
-3.0 -1.5 0.0 1.5 3.0 4.5
Seems quite symmetrical and normal
(b) Paired t test is appropriate.
Σd = 20 and Σd2 = 94
2
2
94 20 20 1.0 20 3.895
20 19
d s
−
= = = =
Observed 1.0 2.27
3.895
20
t= = on 19 d.f.
For one-sided test: 5% point = 1.729, 2.5% point = 2.093 and 1% point
= 2.539
P-value is approx. 0.020
So there is some evidence that the modifications have reduced the
contaminant content.
(c) This t analysis requires normality of the differences and this seems
reasonable from part (a).
14 (i) (a) The least squares estimate of β minimises
2 2 2 2
1 1 1 1 ( ) 2 . n n n n
i i i i i i i i i i q y x y xy x = = = = =Σ −β =Σ − βΣ +β Σ
Differentiating with respect toβ gives
%%%%%%%%%%%%%%%%%%%%%%%%%%%%%%%%%%%%%%%%%%%%%%%%%%%%%%%%%%%%%%%%%%%%%%%%%%
Page 10
2
1 1 2( ). n n
i i i i i
dq x x y
d = = = β −
β Σ Σ
Equating to zero gives the least squares estimator as
1
1 2
1
ˆ
n
i i i
n
i i
x Y
x
=
=
β = Σ
Σ
as required.
(b) Mean and variance of 1 ˆβ :
2
1 1 1
(ˆ ) ( | ) / n n
i i i i i i E x E Y x x = = β =Σ Σ
2
1 1 / n n
i i i i i x x x = = =Σ β Σ = β
2 2 22 2 2
1 1 1 1
(ˆ ) /( ) / . n n n
i i i i i i V x x x = = = β = σ Σ Σ = σ Σ
(ii) (a) The alternative estimator 2 1 1
ˆ / n n
i i i i Y x = = β =Σ Σ has expectation and
variance
2 1 1 1 1
(ˆ ) ( | ) / / , n n n n
i i i i i i i i i E EY x x x x = = = = β =Σ Σ =Σ β Σ = β
( )2
2 2 2
2 1
(ˆ ) / /( ). n
i i V n x nx = β = σ Σ = σ
(b) 2 1 V (βˆ ) ≥V (βˆ )
2 2 2 2
1 / / n
i i nx x = ⇔ σ ≥ σ Σ
2 2
1 0 n
i ix nx = ⇔Σ − ≥
2
1( ) 0 n
i ix x = ⇔Σ − ≥
∴ The variance of 2 ˆβ is at least as large as the variance of the least
squares estimator 1 ˆβ .
(iii) (a) 3
1 1 1 1
(ˆ ) ( | )
n n n n
i i i i i i i i i
i i i i
E E aY aEY x ax ax
= = = =
⎛ ⎞
β = ⎜⎜ ⎟⎟ = = β = β
⎝ ⎠
Σ Σ Σ Σ
%%%%%%%%%%%%%%%%%%%%%%%%%%%%%%%%%%%%%%%%%%%%%%%%%%%%%%%%%%%%%%%%%%%%%%%%%%
Page 11
∴E(βˆ3) = β , i.e. unbiased, if
1
1
n
i i
i
a x
=
Σ =
3
1
(ˆ )
n
i i
i
V V aY
=
⎛ ⎞
β = ⎜⎜ ⎟⎟
⎝ ⎠
Σ = 2 2
1
n
i
i
a
=
Σ σ
(b) 1
1
2
1
ˆ
n
i i
i
n
i
i
x Y
x
=
=
β =
Σ
Σ
=
1
,
n
i i
i
a Y
= Σ
where
2
1
i , 1, , .
i n
i
i
a x i n
x
=
= =
Σ
…
2
1
1 1 2 2
1 1
1
n
n n i
i i
i i n i n
i i
i i
i i
x
a x x x
x x
=
= =
= =
∴ = = =
Σ
Σ Σ
Σ Σ
i.e. the condition 1 1 n
i i i a x = Σ = is satisfied.
1
2
1
1
ˆ ,
n
i n
i
n i i
i
i
i
Y
a Y
x
=
=
=
β = =
Σ
Σ
Σ
where ai 1 , i 1, ,n.
nx
= = …
∴
1 1
1 1
n n
i i i
i i
a x x nx
= = nx nx
Σ =Σ = =
i.e. the condition 1 1 n
i i i a x = Σ = is satisfied.
(c) Among estimators of the form 1
n
i i i a Y = Σ , the minimum variance
unbiased estimator of β is
1
3 1
1 2
1
ˆ ˆ
n
n i i
i
i i n
i
i
i
x Y
a Y
x
=
=
=
β = = =β
Σ
Σ
Σ
i.e. the least squares estimator.
%%%%%%%%%%%%%%%%%%%%%%%%%%%%%%%%%%%%%%%%%%%%%%%%%%%%%%%%%%%%%%%%%%%%%%%%%%
Page 12
15 (i) (a) Let Y = ratio for one week, and Y = eX
P(decrease in one week) = P(Y < 1)
= P(eX < 1) = P(X < 0)
= P(Z < 0 0.0125
0.055
−
= − 0.227) = 1 − 0.5898 = 0.41
(b) P(decrease in next two weeks) = (0.41)2 = 0.17
(c) We require ( (2) 1) ( (2) . (1) 1)
(0) (1) (0)
P S P S S
S S S
> = >
= P(Y2 .Y1 > 1) = P(X2 + X1 > 0)
where X2, X1 are independent N(μ,σ2)
2
∴X2 + X1 ~ N(2μ,2σ )
( 0 2(0.0125) 0.321) 0.63
2(0.055)
P PZ −
∴ = > =− = from tables.
(d) Extending the method of part (c):
20
2
1
i ~ (20 , 20 )
i
X N
=
Σ μ σ
( 0 20(0.0125) 1.016) 0.155
20(0.055)
P PZ −
∴ = < =− =
(ii) (a) The ratios are independent and identically distributed lognormal r.v.’s.
This defines a random sample from a lognormal distribution.
(b) For the 10 observed ratios y1, . . . , y10:
Σy =10.192 ⇒ y =1.0192
Σy2 =10.441562 ⇒ s2 = 0.005986⇒ s = 0.0774
(c) For the method of moments:
solve the following equations for μ and σ2
%%%%%%%%%%%%%%%%%%%%%%%%%%%%%%%%%%%%%%%%%%%%%%%%%%%%%%%%%%%%%%%%%%%%%%%%%%

1 2
e 2 1.0192
μ+ σ
= (1)
2 2 2 e μ+σ (eσ −1) = 0.005986 (2)
2 2 (2) ÷ (1) ⇒eσ −1 = 0.0057625
∴σ2 = 0.005746 ∴σ = 0.0758
(1) log(1.0192) 1 2 0.0161
2
⇒ μ = − σ =
[Note: in MME candidates could use σˆ 2 =0.005388 with divisor n not
(n−1) to obtain σ = 0.0719 and μ = 0.0164 ]

\end{document}
