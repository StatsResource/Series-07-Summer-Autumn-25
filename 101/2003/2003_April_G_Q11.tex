\documentclass[a4paper,12pt]{article}

%%%%%%%%%%%%%%%%%%%%%%%%%%%%%%%%%%%%%%%%%%%%%%%%%%%%%%%%%%%%%%%%%%%%%%%%%%%%%%%%%%%%%%%%%%%%%%%%%%%%%%%%%%%%%%%%%%%%%%%%%%%%%%%%%%%%%%%%%%%%%%%%%%%%%%%%%%%%%%%%%%%%%%%%%%%%%%%%%%%%%%%%%%%%%%%%%%%%%%%%%%%%%%%%%%%%%%%%%%%%%%%%%%%%%%%%%%%%%%%%%%%%%%%%%%%%

\usepackage{eurosym}
\usepackage{vmargin}
\usepackage{amsmath}
\usepackage{graphics}
\usepackage{epsfig}
\usepackage{enumerate}
\usepackage{multicol}
\usepackage{subfigure}
\usepackage{fancyhdr}
\usepackage{listings}
\usepackage{framed}
\usepackage{graphicx}
\usepackage{amsmath}
\usepackage{chngpage}

%\usepackage{bigints}
\usepackage{vmargin}

% left top textwidth textheight headheight

% headsep footheight footskip

\setmargins{2.0cm}{2.5cm}{16 cm}{22cm}{0.5cm}{0cm}{1cm}{1cm}

\renewcommand{\baselinestretch}{1.3}

\setcounter{MaxMatrixCols}{10}

\begin{document} 11 Let S  X1  X2 ... XN (and S = 0 if N = 0) where the Xi ’s are independent and
identically distributed as exponential random variables with mean  and are also
independent of N, which has a Poisson distribution with mean .
\begin{enumerate}[(a)]
    \item i) Prove, from first principles (that is without quoting any general results for
compound distributions), that the moment generating function of S, MS(t), is
given by    ( ) = exp 1 1 1 MS t t        	
. 
\item (ii) Using the expression for the moment generating function of S in (i) (and
without quoting any general results for compound distributions), derive an
expression for the variance of S. 
\end{enumerate}

\newpage
%%%%%%%%%%%%%%%%%%%%%%%%%%%%%%%%%%%%%%%%%%%%%%%%%%%%%%%%%%%%%%%%%%%%%%%%%5555
11 (i) MS(t) = E[etS] = E[E(etS|N)]
E[etS|N=n] = E[exp{t(X1 + X2 +…+ Xn}|N=n]
= E[exp{t(X1 + X2 +…+ Xn}] (since the Xi’s are independent of N)
= 
 E[exp(tXi)] (since the Xi’s are iid)
= {MX(t)}n
 MS(t) = E[{MX(t)}N] = E[exp{NlogMX(t)}] = MN{logMX(t)}
Here MN(t) = exp{(et – 1)} and MX(t) = (1 	t)1 and so
     exp 1 1 1 MS t t       	 
%%%%%%%%%%%%%%%%%%%%%%%%%%%%%%%%%%%%%%%%%%%%%%%%%%%%%%%%%%%%%%%%%%%%%%%%%5555
(ii) MS′(t) = MS(t)  (1 	 t)2

%%- Subject 101 (Statistical Modelling)
%%--— April 2003 — Examiners’ Report
%%-- Page 6
MS′′(t) = MS(t)  22 (1 	 t)3 + MS′(t)  (1 	 t)2
So E[S] = MS′(0) = 
E[S2] = MS′′(0) = 22 + (  
22 + 22
V[S] = 22 + 22	22
22
Alternative solution using the cumulant generating function
Let CS(t) = logMS(t) = {(1 - t)-1 – 1}
CS (t) = (1 - t)-2 , CS(t) = 2(1 - t)-3
So E[S] = CS (0) =  and V[S] = CS(0) = 22
\end{document}
