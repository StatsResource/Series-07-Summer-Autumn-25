\documentclass[a4paper,12pt]{article}

%%%%%%%%%%%%%%%%%%%%%%%%%%%%%%%%%%%%%%%%%%%%%%%%%%%%%%%%%%%%%%%%%%%%%%%%%%%%%%%%%%%%%%%%%%%%%%%%%%%%%%%%%%%%%%%%%%%%%%%%%%%%%%%%%%%%%%%%%%%%%%%%%%%%%%%%%%%%%%%%%%%%%%%%%%%%%%%%%%%%%%%%%%%%%%%%%%%%%%%%%%%%%%%%%%%%%%%%%%%%%%%%%%%%%%%%%%%%%%%%%%%%%%%%%%%%

\usepackage{eurosym}
\usepackage{vmargin}
\usepackage{amsmath}
\usepackage{graphics}
\usepackage{epsfig}
\usepackage{enumerate}
\usepackage{multicol}
\usepackage{subfigure}
\usepackage{fancyhdr}
\usepackage{listings}
\usepackage{framed}
\usepackage{graphicx}
\usepackage{amsmath}
\usepackage{chngpage}

%\usepackage{bigints}
\usepackage{vmargin}

% left top textwidth textheight headheight

% headsep footheight footskip

\setmargins{2.0cm}{2.5cm}{16 cm}{22cm}{0.5cm}{0cm}{1cm}{1cm}

\renewcommand{\baselinestretch}{1.3}

\setcounter{MaxMatrixCols}{10}

\begin{document}

\begin{enumerate}
\item
13 In an air pollution monitoring study undertaken in a residential area near an industrial
plant a sample of 20 sites is chosen and an observation at each site is made of the
content (in parts per million) of a particular contaminant. The data recorded are given
in the table below together with some summaries:
\begin{verbatim}
    76 78 76 78 84 79 79 81 85 76
78 79 75 83 87 80 78 77 81 77

\end{verbatim}

x = 1,587 
x2 = 126,131
\begin{enumerate}
    \item (i) (a) Present these data graphically using a dotplot and comment briefly on
the shape of the distribution.
    \item  (b) Calculate a 95\% confidence interval for the mean contaminant content
for the residential area from which the sites were selected.
    \item (c) Comment briefly on the validity of this confidence interval in the light
of your answer to part (a). 
    \item  (ii) After some modifications by the operators of the industrial plant designed to
reduce the level of pollution due to this contaminant, another observation was
made at each of the same sites. The data recorded are given in the table below
with the sites in the same order as in the table above:
\begin{verbatim}
74 74 76 79 83 76 76 81 84 76
81 77 74 83 89 78 77 72 79 78
\end{verbatim}
\begin{enumerate}[(i)]
    \item (a) Calculate the difference (before  after) in contaminant content for
each site, present these differences graphically and comment briefly on
the shape of the distribution.
    \item (b) Perform an appropriate test to investigate whether the modification has
led to a reduction in the contaminant content.
    \item (c) Comment briefly on the validity of this test in the light of your answer
to part (ii)(a). 

\end{enumerate}
\end{enumerate}


\newpage
%%%%%%%%%%%%%%%%%%%%%%%%%%%%%%%%%%%%%%%%%%%%%%%%%%%%%%%%%%%%%%%%%%%%%%%%%%%%%%%%%%%%%%%%%%%%%%%%%%

13

\begin{itemize}
    \item (i) (a)
. : .
. : : : : . : . . . .
---+---------+---------+---------+---------+---------+---ppm
75.0 77.5 80.0 82.5 85.0 87.5
Dotplot shows moderate positive skewness
\item (b)
2
2
126131 1587 1587 79.35, 20 10.66
20 19
x s
−
= = = =
95% confidence interval is
2
0.025,19 20
x ± t s
giving 79.35 2.093 10.66 79.35 1.53 (77.82,80.88)
20
± ⇒ ± ⇒
\item (c) This t confidence interval requires normality of the observations.
This may be doubtful in view of the skewness shown in part (a), but
the sample size of 20 is perhaps large enough to justify the validity due
to the robustness of the t analysis.
%%%%%%%%%%%%%%%%%%%%%%%%%%%%%%%%%%%%%%%%%%%%%%%%%%%%%%%%%%%%%%%%%%%%%%%%%%


%%%%%%%%%%%%%%%%%%%%%%%%%%%%%%%%%%%%%%%%%%%%%%%%%%%%%%%%%%%%%%%%
\item (ii) (a) Differences (before − after) are:
2 4 0 -1 1 3 3 0 1 0
-3 2 1 0 -2 2 1 5 2 -1
\begin{verbatim}
Dotplot of differences:
: : :
. . : : : : : . .
-+---------+---------+---------+---------+---------+-----diff
-3.0 -1.5 0.0 1.5 3.0 4.5
\end{verbatim}

Seems quite symmetrical and normal


%%%%%%%%%%%%%%%%%%%%%%%%%%%%%%%%%%%%%%%%%%%%%%%%%%%%%%%%%%%%%%%%%%%%%%%%
% 2003 September Q12
\item (b)
Paired t test is appropriate.
$\sum d = 20$ and $\sum d^2 = 94$

$ \bar{d} = \frac{20}{20} = 1.0$

$ s^2 =  \frac{ 94 \; -\; \frac{(20)^2}{20} }{ 19 }  =   3.895 $

\[t_{TS} = \frac{1-0}{  \frac{3.895}{\sqrt{20} } } = 2.27 on 19 d.f. \]




\begin{itemize}
\item  For one-sided test: 5\% point = 1.729, 2.5\% point = 2.093 and 1\% point = 2.539
\item $p-$value is approx. 0.020
\item So there is some evidence that the modifications have reduced the contaminant content.
\end{itemize}

\item (c) This t analysis requires normality of the differences and this seems
reasonable from part (a).
\end{itemize}


\end{document}
