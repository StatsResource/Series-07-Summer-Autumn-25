\documentclass[a4paper,12pt]{article}

%%%%%%%%%%%%%%%%%%%%%%%%%%%%%%%%%%%%%%%%%%%%%%%%%%%%%%%%%%%%%%%%%%%%%%%%%%%%%%%%%%%%%%%%%%%%%%%%%%%%%%%%%%%%%%%%%%%%%%%%%%%%%%%%%%%%%%%%%%%%%%%%%%%%%%%%%%%%%%%%%%%%%%%%%%%%%%%%%%%%%%%%%%%%%%%%%%%%%%%%%%%%%%%%%%%%%%%%%%%%%%%%%%%%%%%%%%%%%%%%%%%%%%%%%%%%

\usepackage{eurosym}
\usepackage{vmargin}
\usepackage{amsmath}
\usepackage{graphics}
\usepackage{epsfig}
\usepackage{enumerate}
\usepackage{multicol}
\usepackage{subfigure}
\usepackage{fancyhdr}
\usepackage{listings}
\usepackage{framed}
\usepackage{graphicx}
\usepackage{amsmath}
\usepackage{chngpage}

%\usepackage{bigints}
\usepackage{vmargin}

% left top textwidth textheight headheight

% headsep footheight footskip

\setmargins{2.0cm}{2.5cm}{16 cm}{22cm}{0.5cm}{0cm}{1cm}{1cm}

\renewcommand{\baselinestretch}{1.3}

\setcounter{MaxMatrixCols}{10}

\begin{document}

\begin{enumerate}
\item
9 Suppose that the joint probability distribution of two random variables X and Y is
given by the following table:
Y
2 4 6
1 0.2 0.0 0.2
X 2 0.0 0.2 0.0
3 0.2 0.0 0.2
(i) Show that X and Y are uncorrelated, but are not independent. 
(ii) Leaving the probabilities in the first and third rows of the table the same,
change the entries in the second row so that X and Y are independent. 
%%%%%%%%%%%%%%%%%%%%%%%%%%%%%%%%%%%%%%%%%%%%%%%%%%%%%%%%%%%%%%%%%%%%%%%%%%%%%%%%%%%%%%%%%%%%%%%%%%
\item  The number of claims on a portfolio of policies was observed as follows:
Number of claims per day 0 1 2 3 4 5 Total
Frequency 48 32 17 2 0 1 100
Use a $\chi^2$ goodness-of-fit test to test the hypothesis that the number of claims each day
follows a Poisson distribution. [7]

%%%%%%%%%%%%%%%%%%%%%%%%%%%%%%%%%%%%%%%%%%%%%%%%%%%%%%%%%%%%%%%%%%%%%%%%%%%%%%%%%%%%%%%%%%%%%%%%%%
\end{enumerate}
%%%%%%%%%%%%%%%%%%%%%%%%%%%%%%%%%%%%%%%%%%%%%%%%%%%%%%%%%%%%%%%%%%%%%%%%%%%%%%%%%%%%%%%%%%%%%%%%%%
9 (i)
Y
2 4 6
1 0.2 0.0 0.2 0.4
X 2 0.0 0.2 0.0 0.2
3 0.2 0.0 0.2 0.4
0.4 0.2 0.4
\begin{itemize}
    \item $E[X] = 0.4 \times 1 + 0.2 \times 2 + 0.4 \times 3 = 2$
\item $E[Y] = 0.4 \times 2 + 0.2 \times 4 + 0.4 \times 6 = 4$
\end{itemize}
\begin{eqnarray*}
E[XY] &=& 1 \times 2 \times 0.2 + 1 \times 4 \times 0.0 + 1 \times 6 \times 0.2
+ \\ & & 2 \times 2 \times 0.0 + 2 \times 4 \times 0.2 + 2 \times 6 \times 0.0
+ \\ & &  3 \times 2 \times 0.2 + 3 \times 4 \times 0.0 + 3 \times 6 \times 0.2 \\
&=&  8\\
\end{eqnarray*}
%%%%%%%%%%%%%%%%%%%%%5

E[XY] − E[X]E[Y] = 0 Therefore uncorrelated.
X and Y are not independent since
P(X = x and Y = y) ≠ P(X = x) P(Y = y)
e.g. x = 1 y = 2, $0.2 \neq (0.4 \times 0.4) = 0.16$.
(ii) X and Y are independent if joint probability is:
Y
2 4 6
1 0.2 0.0 0.2 0.4
X 2 0.1 0.0 0.1 0.2
3 0.2 0.0 0.2 0.4
0.5 0.0 0.5
10 The mean number of claims per day is
\[\{(32 \times 1) + (17 \times 2) + (2 \times 3) + (0 \times 4) + (1 \times 5)\}/100 = 0.77.\]
Use 0.77 as an estimate of the mean of the Poisson distribution. Thus
%%%%%%%%%%%%%%%%%%%%%%%%%%%%%%%%%%%%%%%%%%%%%%%%%%%%%%%%%%%%%%%%%%%%%%%%%%
Page 6
( )
!
e x P X x
x
−λλ
= = is estimated by
0.770.77 ( )
!
e x P X x
x
−
= = , x = 0, 1, 2, …
The expected frequencies are given by $100 \times P(X = x)$.
\begin{verbatim}
  No. of claims (x) 0 1 2 3 4 ≥5 Total
Obs. frequency (fi) 48 32 17 2 0 1 100
Exp. frequency (ei) 46.3 35.7 13.7 3.5 0.7 0.1 100.0  
\end{verbatim}

Categories x = 3, 4, and ≥5 are grouped together to ensure that all ei are greater than 1.
The expected frequency for ≥ 3 is 3.5 + 0.7 + 0.1 = 4.3; the corresponding observed
frequency is 3.
2 2 2 2
2 ( )2 / 1.7 3.7 3.3 1.3 1.63.
i i i 46.3 35.7 13.7 4.3 χ =Σ f − e e = + + + =
There are 2 d.f. [4 categories x = 0,1,2, and ≥3, and 1 parameter estimated from the
data.]
The probability value =
( ) 22
P χ >1.63 ≅ 1 – 0.557 = 0.443 from the Yellow Tables p164
\begin{itemize}
    \item There is insufficient evidence to suggest that the number of claims does not follow a
Poisson distribution (i.e. the model provides a good fit to the data).
\item An alternative solution (in this over-conservative approach some information is
thrown away unnecessarily - but it was awarded full marks):
\item Grouping categories x = 2, 3, 4 and ≥ 5 and using only 3 cells with observed
frequencies 48, 32, and 20 and expected frequencies 46.3, 35.7, and 18.0 gives χ2 =
0.668 on 1 degree of freedom. 
\item The probability value is 0.414. Same conclusion.
\end{itemize}


\end{document}
