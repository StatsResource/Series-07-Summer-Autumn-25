\documentclass[a4paper,12pt]{article}

%%%%%%%%%%%%%%%%%%%%%%%%%%%%%%%%%%%%%%%%%%%%%%%%%%%%%%%%%%%%%%%%%%%%%%%%%%%%%%%%%%%%%%%%%%%%%%%%%%%%%%%%%%%%%%%%%%%%%%%%%%%%%%%%%%%%%%%%%%%%%%%%%%%%%%%%%%%%%%%%%%%%%%%%%%%%%%%%%%%%%%%%%%%%%%%%%%%%%%%%%%%%%%%%%%%%%%%%%%%%%%%%%%%%%%%%%%%%%%%%%%%%%%%%%%%%

\usepackage{eurosym}
\usepackage{vmargin}
\usepackage{amsmath}
\usepackage{graphics}
\usepackage{epsfig}
\usepackage{enumerate}
\usepackage{multicol}
\usepackage{subfigure}
\usepackage{fancyhdr}
\usepackage{listings}
\usepackage{framed}
\usepackage{graphicx}
\usepackage{amsmath}
\usepackage{chngpage}

%\usepackage{bigints}
\usepackage{vmargin}

% left top textwidth textheight headheight

% headsep footheight footskip

\setmargins{2.0cm}{2.5cm}{16 cm}{22cm}{0.5cm}{0cm}{1cm}{1cm}

\renewcommand{\baselinestretch}{1.3}

\setcounter{MaxMatrixCols}{10}

\begin{document}
%%%%%%%%%%%%%%%%%%%%%%%%%%%%%%%%%%%%%%%%%%%%%%%%%%%%%%%%%%%%%%%%%%%%%%%%%%%%%%%%%%%%%%%%%%%%%%%%%%
\item 7 Claims arise through time on a portfolio of policies one after another, at random, and
at a constant rate per week. Claim sizes are to be modelled as a N(\mu, \sigma^2) random
variable, independent of the times of occurrence and the accumulated numbers of
claims.
The moment generating function of S, the total size of all claims which occur in a
period of $k$ weeks (where $k$ is a positive integer), is to be used in a theoretical report
being written by two students.
\begin{itemize}
    \item One student (A) suggests that the moment generating function of $S$ is given by:
Suggestion A :
\[
M_S ( t ) = exp \left( k\;\lambda \left{ exp \left[ \mu\;t + s 2 t 2 \right] - 1 \right} \; \right)
\]

\item The other (B) disagrees and suggests that it is in fact given by:
Suggestion B :

\[M_S ( t ) = exp \left( k\;\lambda exp \left[ \mu\;t + s 2 t 2 \right] - 1 \right)  \]
\end{itemize}

One of the students is correct. Determine which one this is.

7
\begin{itemize}
    \item N is Poisson(k\lambda) with $MN(t) = exp[k\lambda{exp(t) – 1}]$.
\item S has a compound distribution with mgf \[MS(t) = MN{logMX(t)}\]
and \[MX(t) = exp(\mut + \sigma^2t2/2)]\.
\item So mgf of S is $MN(\mut + \sigma^2t2/2)$ and correct suggestion is A.
\item OR: by using the result quoted in the Formulae and Tables book
\item OR: we must have $MS(0) = 1$, so B is wrong.
\end{itemize}
%%%%%%%%%%%%%%%%%%%%%%%%%%%%%%%%%%%%%%%%%%%%%%%%%%%%%%%%%%%%%%%%%%%%%%%%%%%%%%%%%%%%%%%%%%%%%%%%%%
\newpage

 8 A student actuary is about to collect data from a batch of insurance proposal forms. It
is known that on average one in every fifty such forms contains incomplete
information, and the student wants to be 95\% sure of having at least 500 forms which
are complete.
Show that he must use a batch of at least 516 forms. (Use a suitable model and
appropriate tables for the distribution of the number of forms containing incomplete
information.) 

\medskip

8 Let X be the number of forms with incomplete information in a batch of n forms.
Then X ~ Poisson(0.02n) approximately
\begin{itemize}
    \item With n = 516, $X \sim P(10.32)$ and $P(X \leq 16) = 0.965$ approx, by linear interpolation
\item With n = 515, $X \sim P(10.3)$ and $P(X \leq 15) = 0.940$ approx, by linear interpolation
\item So he requires 516 forms
\end{itemize}


\end{document}
