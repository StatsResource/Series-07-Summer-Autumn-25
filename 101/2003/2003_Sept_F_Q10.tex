\documentclass[a4paper,12pt]{article}

%%%%%%%%%%%%%%%%%%%%%%%%%%%%%%%%%%%%%%%%%%%%%%%%%%%%%%%%%%%%%%%%%%%%%%%%%%%%%%%%%%%%%%%%%%%%%%%%%%%%%%%%%%%%%%%%%%%%%%%%%%%%%%%%%%%%%%%%%%%%%%%%%%%%%%%%%%%%%%%%%%%%%%%%%%%%%%%%%%%%%%%%%%%%%%%%%%%%%%%%%%%%%%%%%%%%%%%%%%%%%%%%%%%%%%%%%%%%%%%%%%%%%%%%%%%%

\usepackage{eurosym}
\usepackage{vmargin}
\usepackage{amsmath}
\usepackage{graphics}
\usepackage{epsfig}
\usepackage{enumerate}
\usepackage{multicol}
\usepackage{subfigure}
\usepackage{fancyhdr}
\usepackage{listings}
\usepackage{framed}
\usepackage{graphicx}
\usepackage{amsmath}
\usepackage{chngpage}

%\usepackage{bigints}
\usepackage{vmargin}

% left top textwidth textheight headheight

% headsep footheight footskip

\setmargins{2.0cm}{2.5cm}{16 cm}{22cm}{0.5cm}{0cm}{1cm}{1cm}

\renewcommand{\baselinestretch}{1.3}

\setcounter{MaxMatrixCols}{10}

\begin{document}
The following table contains 10 claim amounts for repair costs arising from a
particular type of storm damage to private houses, for each of four different postcode
regions:
Regions
A B C D
Claim amount (£) 961 1,507 1,303 1,022
1,263 1,349 959 997
1,304 1,521 1,297 1,335
1,532 1,134 1,051 1,216
1,294 1,293 1,163 1,277
1,605 993 993 1,135
1,308 1,126 978 1,273
1,393 1,140 891 1,244
1,255 1,305 1,177 1,105
1,131 1,224 1,153 1,524
Sums 13,046 12,592 10,965 12,128
Sums of squares 17,322,090 16,116,822 12,208,021 14,929,994
\begin{enumerate}
\item (i) Show that a one-way analysis of variance to compare the mean claim amounts
for the regions produces a significant result at the 5% level, but not at the 1%
level. 
\item (ii) Compare the mean claim amounts for the regions A, B, C, and D by using a
least significant difference approach with a significance level of 5%. 
\end{enumerate}

%%%%%%%%%%%%%%%%%%%%%%%%%%%%%%%%%%%%%%%%%%%%%%%%%%%%%

\newpage

11 (i) 
\begin{itemize}
    \item Total sum: 13046 + 12592 + 10965 + 12128 = 48731
    \item Total sum of squares: 17322090 + 16116822 + 12208021 + 14929994
= 60576927
    \item SST = 60576927 − 487312/40 = 1209168

    \item SSB = (130462 + 125922 + 109652 + 121282)/10 − 487312/40
= 59607619 − 487312/40 = 239860
    \item SSR = SST − SSB = 1209168 − 239860 = 969308
\end{itemize}
%%%%%%%%%%%%%%%%%%%%%%%%%%%%%%%%%%%%%%%%%%%%%%%%%%%%%%%%%%%%%%%%%%%%%%%%%%

\begin{verbatim}
  Source of variation df Sums of Squares Mean Squares
Between regions 3 239860 79953
Residual 36 969308 26925
Total 39 1209168
F = 79953/26925 = 2.97 on 3, 36 d.f.  
\end{verbatim}

\begin{itemize}
\item Therefore, since the value of F3,36 (0.05) is 2.866, the observed F value (2.97)
exceeds it and so the null hypothesis that the population means are equal is
rejected at the 5\% level of significance. However, as F3,36 (0.01) is 4.377, the
null hypothesis is not rejected at the 1% level.
\item  (ii) Means:
A: y1. = 1304.6 B: y2. = 1259.2
C: y3. = 1096.5 D: y4. = 1212.8
Least significant difference, for each pair of regions, is (5% level):
1/2
t0.025,36 σˆ ( 1/10 +1/10) = 2.028 26925 (2/10)1/2 = 149
\item Differences between pairs of means:
y1. − y2. = 45.4 , y1. − y3. = 208.1 , y1. − y4. = 91.8
y2. − y3. =162.7 , y2. − y4. = 46.4 , y3. − y4. = −116.3
Region C Region D Region B Region A
y3. y4. y2. y1.
\item  (Alternative answers which have the following conclusion are acceptable:
The population mean claim amount for region C appears to be less than the
population mean of region A and the population mean of region B.
\item However,
the population mean for region C and the population mean for region D do not
appear to differ.)
\end{itemize}
\end{document}
