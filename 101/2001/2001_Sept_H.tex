
\documentclass[a4paper,12pt]{article}

%%%%%%%%%%%%%%%%%%%%%%%%%%%%%%%%%%%%%%%%%%%%%%%%%%%%%%%%%%%%%%%%%%%%%%%%%%%%%%%%%%%%%%%%%%%%%%%%%%%%%%%%%%%%%%%%%%%%%%%%%%%%%%%%%%%%%%%%%%%%%%%%%%%%%%%%%%%%%%%%%%%%%%%%%%%%%%%%%%%%%%%%%%%%%%%%%%%%%%%%%%%%%%%%%%%%%%%%%%%%%%%%%%%%%%%%%%%%%%%%%%%%%%%%%%%%

\usepackage{eurosym}
\usepackage{vmargin}
\usepackage{amsmath}
\usepackage{graphics}
\usepackage{epsfig}
\usepackage{enumerate}
\usepackage{multicol}
\usepackage{subfigure}
\usepackage{fancyhdr}
\usepackage{listings}
\usepackage{framed}
\usepackage{graphicx}
\usepackage{amsmath}
\usepackage{chngpage}

%\usepackage{bigints}
\usepackage{vmargin}

% left top textwidth textheight headheight

% headsep footheight footskip

\setmargins{2.0cm}{2.5cm}{16 cm}{22cm}{0.5cm}{0cm}{1cm}{1cm}

\renewcommand{\baselinestretch}{1.3}

\setcounter{MaxMatrixCols}{10}

\begin{document}
\begin{enumerate}
%%%%%%%%%%%%%%%%%%%%%%%%%%%%%%%%%%%%%%%%%%%%%%%%%%%%%%%%%%%%%%%%%%%%%%%%%%%%%%%%%%%%%%%%%%%%%%%%%%%%%%%%%
\item 12 Twenty overweight executives take part in an experiment to compare the
effectiveness of two exercise methods, A (isometric), and B (isotonic). They are
allocated at random to the two methods, ten to isometric, ten to isotonic methods.
After several weeks, the reductions in abdomen measurements are recorded in
centimetres with the following results:
A (isometric method) 3.1 2.1 3.3 2.7 3.4 2.7 2.7 3.0 3.0 1.6
B (isotonic method) 4.5 4.1 2.7 2.2 4.7 2.2 3.6 3.0 3.3 3.4
\begin{enumerate}[(a)]
\item (a) Plot the data for the two exercise methods on a single diagram.
Comment on whether the response values for each exercise method
are well modelled by normal random variables.
(b) Perform a test to investigate whether the assumption of equal
variability for the responses for the two exercise methods is
reasonable.
(c) Perform a t-test to investigate whether these data support the
claim that the isotonic method is more effective than the other
method. [9]
\item (a) Determine a two-sided 95\% confidence interval for the difference in
the means for the two exercise methods.
(b) Assuming that the two sets of 10 measurements are taken from
normal populations with the same variance, determine a 95%
confidence interval for the common standard deviation. [7]
\end{enumerate}



%%%%%%%%%%%%%%%%%%%%%%%%%%%%%%%%%%%%%%%%%%%%%%%%%%%%
\end{enumerate}

%%%%%%%%%%%%%%%%%%%%%%%%%%%%%%%%%%%%%%%%%%%%%%%%%%%%%%%%%%%%%%%%%%%%%%%%%%%%%%%%%%%%%%%%%%%%%%%%%%%%%%%%%
13  (a) Plot for Isotonic-Isometric exercise methods:
Dotplot for Isotonic-Isometric
Normality seems OK for each data set.
Let XA , XB be reductions in measurements from the isometric and
isotonic methods, respectively.
A: 2 2 = 27.6, = 78.90 ; % xA xA xA = 2.76, sA = 0.3027, nA =10
B: 2 2 = 33.7, =120.53 ; % = 3.37, = 0.7734, 10 B B B B B x x x s n 
(b) 2 = 0.3027 ; 2 = 0.7734 A B s s
2 2 2 2
0 1 : = ; : A B A B H %   H   
1.6 2.6 3.6 4.6
Isotonic
Isometric

%%%%%%%%%%%%%%%%%%%%%%%%%%%%%%%%%%%%%%%%%%%%%%%%%%%%%%%%%%%%%%%%%%%%%%%%%%%%%%%%%%%%%%%%%%%%%%%%%%%%%%%%%
0.7734
= =2.56
0.3027
F on 9,9 d.f.
Upper 5\% point is 3.179, so p-value > 0.10.
Therefore do not reject H0.
(c) The pooled sample variance
%        2 2 2 = 1 1 / 2 p A A B B A B s n  s  n  s n  n 
% = {(9  0.3027) + (9  0.7734)}/18 = 0.538.
% The test statistic   2 1 1
= / B A
A B
t x x s
n n
%%%%%%%%%%%%%%%%
1 1
= (3.37 2.76) / 0.538
10 10
%%%%%%%%%%%%%%
= 1.86.
% H0 :  = 0; H1 :  > 0  : mean difference in reduction in
% abdomen measurements (B  A).
A one-sided test is appropriate.
There are 18 d.f. The upper 5% point of t18 is 1.734. Thus the
probability value is less than 0.05. There is sufficient evidence, at
the 5% level, to suggest that the isotonic method (B) is more
effective in reducing abdomen measurement.
\item (a) Two-sided 95% confidence interval for B  A :
%   2
18
1 1
(2.5%)
10 10 B A x x t s
2
3.37 2.76 2.101 0.538
10

0.61  0.69 = (0.08,1.3)
[Note that this just includes zero.]
(b)  
2
2
2 2 2 ~ A B
p
A B n n
S
n n     

Here nA + nB  2 = 18
95% confidence interval for 2 (common variance)

%%%%%%%%%%%%%%%%%%%%%%%%%%%%%%%%%%%%%%%%%%%%%%%%%%%%%%%%%%%%%%%%%%%%%%%%%%%%%%%%%%%%%%%%%%%%%%%%%%%%%%%%%
2 2
2 2
18 18
18 18
,
(0.025) (0.975)
%  s s 
%%%%%%%%%%%%%%%%%%%%%%%%%%%%%%%%%%%%%%%%%%%%%%5
18(0.538) 18(0.538)
,
31.53 8.231
%%%%%%%%%%%%%%%%%%%%%%%%%%%%%%%%%%%%%%%%%%%%%%%
Taking square-roots gives the 95% confidence interval for the
common standard deviation %  as  0.31, 1.18  = (0.55 ,1.08)
Some candidates used a % “paired samples” approach in part \item(a). This was quite
inappropriate.
\end{document}
