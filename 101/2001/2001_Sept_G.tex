\documentclass[a4paper,12pt]{article}

%%%%%%%%%%%%%%%%%%%%%%%%%%%%%%%%%%%%%%%%%%%%%%%%%%%%%%%%%%%%%%%%%%%%%%%%%%%%%%%%%%%%%%%%%%%%%%%%%%%%%%%%%%%%%%%%%%%%%%%%%%%%%%%%%%%%%%%%%%%%%%%%%%%%%%%%%%%%%%%%%%%%%%%%%%%%%%%%%%%%%%%%%%%%%%%%%%%%%%%%%%%%%%%%%%%%%%%%%%%%%%%%%%%%%%%%%%%%%%%%%%%%%%%%%%%%

\usepackage{eurosym}
\usepackage{vmargin}
\usepackage{amsmath}
\usepackage{graphics}
\usepackage{epsfig}
\usepackage{enumerate}
\usepackage{multicol}
\usepackage{subfigure}
\usepackage{fancyhdr}
\usepackage{listings}
\usepackage{framed}
\usepackage{graphicx}
\usepackage{amsmath}
\usepackage{chngpage}

%\usepackage{bigints}
\usepackage{vmargin}

% left top textwidth textheight headheight

% headsep footheight footskip

\setmargins{2.0cm}{2.5cm}{16 cm}{22cm}{0.5cm}{0cm}{1cm}{1cm}

\renewcommand{\baselinestretch}{1.3}

\setcounter{MaxMatrixCols}{10}

\begin{document}

%%%%%%%%%%%%%%%%%%%%%%%%%%%%%%%%%%%%%%%%%%%%%%%%%%%%%%%%%%%%%%%%%%%%%%%%%%%%%%%%%%%%%%%%%%%%%%%%%%%%%%%%%
\item 12 Claims are classified on inception into one of three categories, “simple”,
“standard” and “complex”.
\begin{itemize}
    \item Last year the percentage of all claims classified in
each of these categories was 18.4\%, 70.3\% and 11.3\% respectively.
\item A random sample of 120 of this year’s claims to date shows that the numbers
classified in each category are 15, 87 and 18 respectively.
\item Perform a goodness-of-fit test to investigate whether this year’s pattern to date
differs from that of last year, and state your conclusion. 
\end{itemize}
\end{enumerate}
%%%%%%%%%%%%%%%%%%%%%%%%%%%%%%%%%%%%%%%%%%%%%%%%%%%%%%
12 H0 : this year’s pattern is the same as last year’s v. H1 : not the same
\begin{itemize}
    \item Under H0 , the expected frequencies are:
% 120  0.184 ; 0.703 ; 0.113 = 22.08 ; 84.36 ; 13.56
% oi ei (o 
e)2/e
15 22.08 2.270
87 84.36 0.083
18 13.56 1.454
3.807 on 2 df
5% point from 22
 is 5.991. So cannot reject H0 at 5% level.
\item These data provide no evidence to suggest that this year’s pattern differs from
that of last year.
\item A few candidates worked with percentages of claims instead of numbers of claims
(when using the chi-squared goodness-of-fit statistic, one must work with observed
and expected frequencies). Such work received few if any marks.
\end{itemize}


\end{document}
