
\documentclass[a4paper,12pt]{article}

%%%%%%%%%%%%%%%%%%%%%%%%%%%%%%%%%%%%%%%%%%%%%%%%%%%%%%%%%%%%%%%%%%%%%%%%%%%%%%%%%%%%%%%%%%%%%%%%%%%%%%%%%%%%%%%%%%%%%%%%%%%%%%%%%%%%%%%%%%%%%%%%%%%%%%%%%%%%%%%%%%%%%%%%%%%%%%%%%%%%%%%%%%%%%%%%%%%%%%%%%%%%%%%%%%%%%%%%%%%%%%%%%%%%%%%%%%%%%%%%%%%%%%%%%%%%

\usepackage{eurosym}
\usepackage{vmargin}
\usepackage{amsmath}
\usepackage{graphics}
\usepackage{epsfig}
\usepackage{enumerate}
\usepackage{multicol}
\usepackage{subfigure}
\usepackage{fancyhdr}
\usepackage{listings}
\usepackage{framed}
\usepackage{graphicx}
\usepackage{amsmath}
\usepackage{chngpage}

%\usepackage{bigints}
\usepackage{vmargin}

% left top textwidth textheight headheight

% headsep footheight footskip

\setmargins{2.0cm}{2.5cm}{16 cm}{22cm}{0.5cm}{0cm}{1cm}{1cm}

\renewcommand{\baselinestretch}{1.3}

\setcounter{MaxMatrixCols}{10}

\begin{document}


%%%%%%%%%%%%%%%%%%%%%%%%%%%%%%%%%%%%%%%%%%%%%%%%%%%%%%%%%%%%%%%%%%%%%%%%%%%%%%%%%%%%%%%%%%%%%%%%%%%%%%%%%
\item 15 In a study into employee share ownership plans, data were obtained from ten
large insurance companies on the following two variables:
employee satisfaction with the plan (x);
employee commitment to the company (y).
For each company a random sample (of the same size) of employees completed
questionnaires in which satisfaction and commitment were recorded on a 1–10
scale, with 1 representing low satisfaction/commitment and 10 representing high
satisfaction/commitment. The resulting means provide each company’s
employees’ satisfaction and commitment score. These scores are given in the
following table:
Co. A B C D E F G H I J
x 5.05 4.12 5.38 4.17 3.81 4.47 5.41 4.88 4.64 5.19
y 5.36 4.59 5.42 4.35 4.03 5.34 5.64 4.89 4.52 5.88
% x = 47.12, x2 = 224.8554, y = 50.02, y2 = 253.5796, xy = 238.3676
\begin{enumerate}[(a)]
\item Draw a scatterplot of y against x and comment briefly on any relationship
between employee satisfaction and commitment. 
\item Calculate the fitted linear regression equation of y on x. 
\item Calculate the coefficient of determination R2 and relate its value to your
comment . 
\item Assuming the full normal model, calculate an estimate of the error
variance $\sigma^2$ and obtain a 95\% confidence interval for $\sigma^2$. 
\item Calculate a 95\% confidence interval for the true underlying slope
coefficient. 
\item For companies with an employees’ satisfaction score of 5.0, calculate an
estimate of the expected employees’ commitment score together with 95%
confidence limits. 
\end{enumerate}

%%%%%%%%%%%%%%%%%%%%%%%%%%%%%%%%%%%%%%%%%%%%%%%%%%%%
\end{enumerate}
15 \item see plot
there seems to be an increasing and linear relationship.
\item
47.122
= 224.8554 = 2.82596
10 xx S 
50.022
= 253.5796 = 3.37956
10 yy S 
(47.12)(50.02)
= 238.3676 = 2.67336
10 xy S 
% ˆ 2.67336 = = 0.946001
2.82596
% 
% ˆ 50.02 47.12 = (0.946001) = 0.544
10 10
%  
y = 0.544 + 0.9460x
4.0 4.5 5.0 5.5
4
5
6
x
y
Commitment v. Satisfaction

%%%%%%%%%%%%%%%%%%%%%%%%%%%%%%%%%%%%%%%%%%%%%%%%%%%%%%%%%%%%%%%%%%%%%%%%%%%%%%%%%%%%%%%%%%%%%%%%%%%%%%%%%
\begin{itemize}
    \item 

2
2 (2.67336)
=
(2.82596)(3.37956)
R = 0.748 or 74.8%
quite high, showing agreement with a linear relationship.
\item
2
2 1 2.67336
ˆ = (3.37956 ) = 0.1063
8 2.82596
 
For confidence interval use
2
2
2 2
( 2) ˆ
~ n
n
%%%%%%%%%%%%%%%%%%%%%%%%%%%%%%%%%%%%%%%
2 2
2 2
2 2
( 2) ˆ ( 2) ˆ
,
(0.025) (0.975) n n
n n
%%%%%%%%%%%%%%%%%%%%%%%%%%%%%%%%%%%%%%%%%%
8(0.1063) 8(0.1063)
= , = (0.0485,0.3902)
17.53 2.180
%%%%%%%%%%%%%%%%%%%%%%%%%%%%%%%%%%%%%%%%
\item 
= 0.9460
its standard error is
ˆ 2 0.1063
= = 0.1939
2.82596 xx S
\end{itemize}
%%%%%%%%%%%%%%%%%%%%%%%%%%%%%%%%%%
95\% confidence interval is 8

% ˆ  t (0.025)  s.e.
% = 0.9460  2.306(0.1939) = 0.946  0.447 or (0.499, 1.393)
% (vi) estimate is 0
% ˆ = ˆ  ˆ(5.0) = 0.544 + 0.9460(5.0) = 5.274
2
2
0
1 (5.0 )
. .( ˆ ) = ˆ ( )
xx
x
s e
n S

  
1 (5.0 4.712)2
= 0.1063( ) = 0.1173
10 2.82596
%%%%%%%%%%%%%%%
95\% confidence limits are  2.306(0.1173) =  0.270 or (5.004, 5.544)


%%%%%%%%%%%%%%%%%%%%%%%%%%%%%%%%

\newpage

\noindent{\bf Regression estimates}

\begin{eqnarray*}
	S_{XY} &=&
	\sum x_iy_i - \frac{\sum x_i\sum y_i}{n}\\
	S_{XX} &=&
	\sum x_i^2 - \frac{(\sum x_i)^2}{n}\\
	S_{YY} &=&
	\sum y_i^2 - \frac{(\sum y_i)^2}{n}\\
\end{eqnarray*}
{\bf Slope Estimate}
\begin{eqnarray*}
	b_1 = \frac{S_{XY}}{S_{XX}}
\end{eqnarray*}
{\bf Intercept Estimate}
\begin{eqnarray*}
	b_0 = \bar{y} -b_1\bar{x}
\end{eqnarray*}
{\bf Pearson's correlation coefficient}

\begin{eqnarray*}
	r = \frac{S_{XY}}{\sqrt{S_{XX} \times S_{YY}}}
\end{eqnarray*}
{\bf Standard error of the Slope}
\begin{eqnarray*}
	S.E.(b1) = \sqrt{\frac{s^2}{S_{XX}}}
\end{eqnarray*}

where $s^2 = \frac{SSE}{n-2}$
and SSE $= S_{YY} - b_1S_{XY}$

\end{document}
