\documentclass[a4paper,12pt]{article}

%%%%%%%%%%%%%%%%%%%%%%%%%%%%%%%%%%%%%%%%%%%%%%%%%%%%%%%%%%%%%%%%%%%%%%%%%%%%%%%%%%%%%%%%%%%%%%%%%%%%%%%%%%%%%%%%%%%%%%%%%%%%%%%%%%%%%%%%%%%%%%%%%%%%%%%%%%%%%%%%%%%%%%%%%%%%%%%%%%%%%%%%%%%%%%%%%%%%%%%%%%%%%%%%%%%%%%%%%%%%%%%%%%%%%%%%%%%%%%%%%%%%%%%%%%%%

\usepackage{eurosym}
\usepackage{vmargin}
\usepackage{amsmath}
\usepackage{graphics}
\usepackage{epsfig}
\usepackage{enumerate}
\usepackage{multicol}
\usepackage{subfigure}
\usepackage{fancyhdr}
\usepackage{listings}
\usepackage{framed}
\usepackage{graphicx}
\usepackage{amsmath}
\usepackage{chngpage}

%\usepackage{bigints}
\usepackage{vmargin}

% left top textwidth textheight headheight

% headsep footheight footskip

\setmargins{2.0cm}{2.5cm}{16 cm}{22cm}{0.5cm}{0cm}{1cm}{1cm}

\renewcommand{\baselinestretch}{1.3}

\setcounter{MaxMatrixCols}{10}

\begin{document}

%%%%%%%%%%%%%%%%%%%%%%%%%%%%%%%%%%%%%%%%%%%%%%%%%%%%%%%%%%%%%%%%%%%%%%%%%%%%%%%%%%%%%%%%%%%%%%%%%%%%%%%%%%%%%%%%%%%
\item 3 Suppose that the occurrence of events which give rise to claims in a portfolio of
motor policies can be modelled as follows: the events occur through time at
random, at rate $\mu$ per hour. Then the number of events which occur in a given
period of time has a Poisson distribution (you are given this).
Show that the time between two consecutive events occurring has an exponential
distribution with mean $1/\mu$ hours. 
%%%%%%%%%%%%%%%%%%%%%%%%%%%%%%%%%%%%%%%%%%%%%%%%%%%%%%%%%%%%%%%%%%%%%%%%%%%%%%%%%%%%%%%%%%%%%%%%%%%%%%%%%%%%%%%%%%%
\item 4 For a certain type of policy the probability that a policyholder will make a claim
in a year is 0.001. If a random sample of 10,000 policyholders is selected,
calculate an approximate value for the probability that not more than 5 will
make a claim next year. 

\newpage
%%%%%%%%%%%%%%%%%%%%%%%%%%%%%%%%%%%%%%%%%%%%%555
3 Let the number of events in a period of time of length t hours be Xt .
\begin{itemize}
\item Then $X_1 \sim Poisson(\mu)$ and $X_t \sim Poisson(\mu t)$.
\item Let the time between two consecutive events be T.
\item Then $P(T > t) = P(\mbox{no events in period of length t}) = P(Xt = 0) = exp(−\mu t)$.
\item So $P(T < t) = 1 − exp(−\mut)$, and so $f(t) = \mu exp(−\mu t) , t >0$
\item Hence T ~ exponential with mean $1/\mu$ hours.
\end{itemize}

%%%%%%%%%%%%%%%%%%%%%%%%%%%%%%%%%%%%%%%%%%%%%%%%%%%%%
4 Binomial (10,000, 0.001) approximated by Poisson with mean of 10.
\begin{itemize}
    \item Approximate probability of no more than 5 claims:
\item Binomial (10,000, 0.001) approximated by Poisson with mean of 10 ($\lambda = np$).
\item Approximate probability of no more than 5 claims:

\end{itemize}

%%% April 2001 - Question 4



\begin{eqnarray*}
P( X \leq 5) &=& \sum^{5}_{i=0} P(X=i) \\
& & \\
&=& \sum^{5}_{i=0} e^{\lambda=} \frac{\lambda^i }{(i)!} \\
& & \\
&=& e^{-10} \left( 1 + 10 + 
\frac{10^2}{2!} +  \frac{10^3}{3!} + 
\frac{10^4}{4!} +  \frac{10^5}{5!} \right) \\
& & \\
&=& 0.0671.\\
\end{eqnarray*}

[OR: 0.06709 using Green Book]
[OR: Use normal approximation with continuity correction]


\end{document}
