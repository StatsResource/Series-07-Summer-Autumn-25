\documentclass[a4paper,12pt]{article}

%%%%%%%%%%%%%%%%%%%%%%%%%%%%%%%%%%%%%%%%%%%%%%%%%%%%%%%%%%%%%%%%%%%%%%%%%%%%%%%%%%%%%%%%%%%%%%%%%%%%%%%%%%%%%%%%%%%%%%%%%%%%%%%%%%%%%%%%%%%%%%%%%%%%%%%%%%%%%%%%%%%%%%%%%%%%%%%%%%%%%%%%%%%%%%%%%%%%%%%%%%%%%%%%%%%%%%%%%%%%%%%%%%%%%%%%%%%%%%%%%%%%%%%%%%%%

\usepackage{eurosym}
\usepackage{vmargin}
\usepackage{amsmath}
\usepackage{graphics}
\usepackage{epsfig}
\usepackage{enumerate}
\usepackage{multicol}
\usepackage{subfigure}
\usepackage{fancyhdr}
\usepackage{listings}
\usepackage{framed}
\usepackage{graphicx}
\usepackage{amsmath}
\usepackage{chngpage}

%\usepackage{bigints}
\usepackage{vmargin}

% left top textwidth textheight headheight

% headsep footheight footskip

\setmargins{2.0cm}{2.5cm}{16 cm}{22cm}{0.5cm}{0cm}{1cm}{1cm}

\renewcommand{\baselinestretch}{1.3}

\setcounter{MaxMatrixCols}{10}

\begin{document}

\begin{enumerate}
%%%%%%%%%%%%%%%%%%%%%%%%%%%%%%%%%%%%%%%%%%%%%%%%%%%%%%%%%%%%%%%%%%%%%%%%%%%%%%%%%%%%%%%%%%%%%%%%%%%%%%%%%%%%%%%%%%%
\item 1 The following amounts are the sizes of claims (£) on house insurance policies for a
certain type of repair.
\begin{verbatim}
198 221 215 209 224 210 223 215 203 210
220 200 208 212 216
\end{verbatim}

Determine the lower quartile, median, upper quartile and interquartile range of
these claim amounts. 
%%%%%%%%%%%%%%%%%%%%%%%%%%%%%%%%%%%%%%%%%%%%%%%%%%%%%%%%%%%%%%%%%%%%%%%%%%%%%%%%%%%%%%%%%%%%%%%%%%%%%%%%%%%%%%%%%%%
\item 2 A certain medical test either gives a positive or negative result. The positive test
result is intended to indicate that a person has a particular (rare) disease, while
a negative test result is intended to indicate that they do not have the disease.
\begin{itemize}
    \item Suppose, however, that the test sometimes gives an incorrect result: 1 in 100 of
those who do not have the disease have positive test results, and 2 in 100 of those
having the disease have negative test results.
\end{itemize}

If 1 person in 1000 has the disease, calculate the probability that a person with a
positive test result has the disease. 

\end{enumerate}
\newpage

%%%%%%%%%%%%%%%%%%%%%%%%%%%%%%%%%%%%%%%%%%%%%%%%%%%%%%%%%%%%%%%%%%%%%%%%%%%%%%%%%%%%%%%%%%%%%%%%%%%%%%%%%%%%%%%%%%%%
1 Ordered data are:
\begin{verbatim}
 198 200 203 208 209 210 210 212 
 215 215 216 220 221 223 224   
\end{verbatim}

1
1
15 4
4
th
n = Q = observation = £208.25, median = 8th observation = £212
3
3 4 3 1 11 observation £219 £10.75 th Q = = Q −Q =
[OR: Using the alternative definitions of quartiles as 16/4th and 48/4th
observations gives Q1 = 208, Q3 = 220, Q3 − Q1 = 12.]
%%%%%%%%%%%%%%%%%%%%%%%%%%%%%%%%%%%%%%%%%%%%%%%%%%%5
\medskip 
2 Apply Bayes' theorem:
1
1000
p = (Probability of having disease)
P (has disease|positive result) =
98
100 98 0.089
98 1 1097 (1 )
100 100
p
p p
= =
+ −

\end{document}
