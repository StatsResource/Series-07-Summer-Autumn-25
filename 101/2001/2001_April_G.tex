\documentclass[a4paper,12pt]{article}

%%%%%%%%%%%%%%%%%%%%%%%%%%%%%%%%%%%%%%%%%%%%%%%%%%%%%%%%%%%%%%%%%%%%%%%%%%%%%%%%%%%%%%%%%%%%%%%%%%%%%%%%%%%%%%%%%%%%%%%%%%%%%%%%%%%%%%%%%%%%%%%%%%%%%%%%%%%%%%%%%%%%%%%%%%%%%%%%%%%%%%%%%%%%%%%%%%%%%%%%%%%%%%%%%%%%%%%%%%%%%%%%%%%%%%%%%%%%%%%%%%%%%%%%%%%%

\usepackage{eurosym}
\usepackage{vmargin}
\usepackage{amsmath}
\usepackage{graphics}
\usepackage{epsfig}
\usepackage{enumerate}
\usepackage{multicol}
\usepackage{subfigure}
\usepackage{fancyhdr}
\usepackage{listings}
\usepackage{framed}
\usepackage{graphicx}
\usepackage{amsmath}
\usepackage{chngpage}

%\usepackage{bigints}
\usepackage{vmargin}

% left top textwidth textheight headheight

% headsep footheight footskip

\setmargins{2.0cm}{2.5cm}{16 cm}{22cm}{0.5cm}{0cm}{1cm}{1cm}

\renewcommand{\baselinestretch}{1.3}

\setcounter{MaxMatrixCols}{10}

\begin{document}

%%%%%%%%%%%%%%%%%%%%%%%%%%%%%%%%%%%%%%%%%%%%%%%%%%%%%%%%%%%%%%%%%%%%%%%%%%%%%%%%%%%%%%%%%%%%%%%%%%%%%%%%%%%%%%%%%%%
\item 12 In 1998 a market research organisation conducted a survey in a large city. A
random sample of 400 households showed that 68 were such that at least one
person in the household was a member of a health/fitness club.
\begin{enumerate}[(a)]
\item  An estimate of the true proportion of households in the city with at least
one person being a member of a health/fitness club is therefore
68
400
= 0.17. Calculate 95\% confidence limits for the true proportion.
\item A similar survey was conducted in 1999 and a random sample of 400
households showed that 80 were such that at least one person in the
household was a member of a health/fitness club.
Perform a one-sided test to investigate whether the true proportion
increased from 1998 to 1999. In particular determine the p-value of the
test and state your conclusion. 
\end{enumerate}

%%%%%%%%%%%%%
12 (i) pˆ = 0.17
95\% confidence interval for the true proportion p is
ˆ (1 ˆ ) ˆ 1.96
400
p p
p ± −
(0.17)(0.83)
0.17 1.96 = 0.17 0.037 or (0.133,0.207)
400
 ± ±
%%%%%%%%%%%%%%%%%%%%%%%%%%%%%%%%%%%%%%%%%%%%%%%%%%%%
(ii) pˆ1 = 0.17 , 2 pˆ = 0.2
common
ˆ 68 80 = =0.185
800
p +
H0 : p2 = p1 v. H1 : p2 > p1
0.2 0.17 0.03
= = =1.09
1 1 0.0275
(0.185)(0.815)( )
400 400
z −
+
P-value = P(Z > 1.09) = 1 − 0.862 = 0.14
There is not sufficient evidence to conclude that there is an increase in the
true proportion. The observed difference could have occurred by chance.
%%%%%%%%%%%%%%%%%%%%%%%%%%%%%%%%%%%%%%%%%%%%%%%%%%%%
%%%%%%%%%%%%%%%%%%%%%%%%%%%%%%%%%%%%%%%%%%%%%%%%%%%%%%%%%%%%%%%%%%%%%%%%%%%%%%%%%%%%%%%%%%%%%%%%%%%%%%%%%%%
\end{document}
