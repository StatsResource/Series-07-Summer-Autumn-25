\documentclass[a4paper,12pt]{article}

%%%%%%%%%%%%%%%%%%%%%%%%%%%%%%%%%%%%%%%%%%%%%%%%%%%%%%%%%%%%%%%%%%%%%%%%%%%%%%%%%%%%%%%%%%%%%%%%%%%%%%%%%%%%%%%%%%%%%%%%%%%%%%%%%%%%%%%%%%%%%%%%%%%%%%%%%%%%%%%%%%%%%%%%%%%%%%%%%%%%%%%%%%%%%%%%%%%%%%%%%%%%%%%%%%%%%%%%%%%%%%%%%%%%%%%%%%%%%%%%%%%%%%%%%%%%

\usepackage{eurosym}
\usepackage{vmargin}
\usepackage{amsmath}
\usepackage{graphics}
\usepackage{epsfig}
\usepackage{enumerate}
\usepackage{multicol}
\usepackage{subfigure}
\usepackage{fancyhdr}
\usepackage{listings}
\usepackage{framed}
\usepackage{graphicx}
\usepackage{amsmath}
\usepackage{chngpage}

%\usepackage{bigints}
\usepackage{vmargin}

% left top textwidth textheight headheight

% headsep footheight footskip

\setmargins{2.0cm}{2.5cm}{16 cm}{22cm}{0.5cm}{0cm}{1cm}{1cm}

\renewcommand{\baselinestretch}{1.3}

\setcounter{MaxMatrixCols}{10}

\begin{document}
\item 
14 Consider a group of 1000 policyholders, all of the same age, and each of whose
lives is insured under one or more policies. The following frequency distribution
gives the number of claims per policyholder in 1999 for this group.
Number of claims per policyholder (i) 0 1 2 3 ≥4
Number of policyholders (fi ) 826 128 39 7 0
A statistician argues that an appropriate model for the distribution of X, the
number of claims per policyholder, is X ~ Poisson. Under this proposal, the
frequencies expected are as follows (you are not required to verify these):
Number of claims per policyholder 0 1 2 3 ≥4
Expected number of policyholders 796.9 180.9 20.5 1.6 0.1
A second statistician argues that a more appropriate model for the distribution of
X is given by:
\[P(X = x) = p(1 − p)x , x = 0, 1, 2,\ldots\]
%%%%%%%%%%%%%%%%%%%%%%%%%%%%%%%%%%%%%%%%%%%%%%%%%%%%%%%
\begin{enumerate}[(a)]
    \item (i) Without doing any further calculations, comment on the first statistician’s
proposed model for the data. 
Consider the second statistician's proposed model.
\item (ii) Verify that the mean of the distribution of X is (1 \;-\; p)/p and hence
calculate the method of moments estimate of p. 
(Note: this estimate is also the maximum likelihood estimate.)
\item (iii) Verify that the frequencies expected under the second statistician's
proposed model are as follows:
Number of claims per policyholder 0 1 2 3 ≥4
Expected number of policyholders 815.0 150.8 27.9 5.2 1.2

\item (iv) (a) Test the goodness-of-fit of the second statistician’s proposed model
to the data, quoting the p-value of your test statistic and your
conclusion.
(b) Assuming that you had been asked to test the goodness-of-fit at
the 1\% level”, state your conclusion.
\end{enumerate}

%%%%%%%%%%%%%%%%%%%%%%%%%%%%%%%%%%%%%%%%%%%%%%%%%%%%%%%%%%%%%%%%%%%%%%%%%%%%%%%%%%%%%%%%%%%%%%%%%%%%%%%%%%
\begin{itemize}
    \item 14 (i) The model greatly overestimates the number of policyholders who give
rise to exactly one claim, and greatly underestimates the number who give
rise to multiple claims.
\item (ii) E(X) = 1 × p(1−p) + 2 × p(1−p)2 + 3 × p(1−p)3 + …
= p(1−p)[1 + 2(1−p) + 3(1−p)2 + …] = p(1−p)[1 \;-\; (1−p)]−2 = (1−p)/p
Data mean is (128 + 78 + 21)/1000 = 0.227
so the method of moments estimate * of p is given by solving
(1−p)/p = 0.227, which gives estimate of p = 0.81500.
* it is also in fact the MLE
\item (iii) Estimates of probabilities for 0,1,2, and 3 claims are then 0.815,
0.815 × 0.185 = 0.15078, 0.815 × 0.1852 = 0.02789, and
0.815 × 0.1853 = 0.00516
So, for 4 or more claims, estimate is 1 \;-\; 0.99883 = 0.00117.
Hence expected frequencies are as follows:
\begin{tabular}{c|c|c|c|c|c|}
Number of claims per policyholder  & 0 & 1 & 2 & 3 & $\geq 4$ \\
Expected number of policyholders &  815.0 & 150.8 & 27.9 & 5.2&  1.2 \\ 
\end{tabular}

\item (iv) (a) Using 5 cells:
\[ χ2 = (826\;-\;815)2/815+(128\;-\;150.8)2/150.8+(39\;-\;27.9)2/27.9 +(7\;-\;5.2)2/5.2\]
+ 1.22/1.2 = 9.84 on 3 df
P-value is between 0.025 and 0.01
\item There is quite strong evidence against the null hypothesis (that the
second statistician’s model describes the data) and we conclude that
the model does not provide a good fit to the data.
\item [Alternative solution: Using 4 cells (for 0 , 1 , 2 , and ≥ 3) the χ2 value
is 8.07 on 2df. P-value is again between 0.025 and 0.01.]
\item (b) The P-value exceeds 0.01 so we cannot reject the hypothesis.
The hypothesis that the model describes the data stands.
\end{itemize}
\end{document}
