\documentclass[a4paper,12pt]{article}

%%%%%%%%%%%%%%%%%%%%%%%%%%%%%%%%%%%%%%%%%%%%%%%%%%%%%%%%%%%%%%%%%%%%%%%%%%%%%%%%%%%%%%%%%%%%%%%%%%%%%%%%%%%%%%%%%%%%%%%%%%%%%%%%%%%%%%%%%%%%%%%%%%%%%%%%%%%%%%%%%%%%%%%%%%%%%%%%%%%%%%%%%%%%%%%%%%%%%%%%%%%%%%%%%%%%%%%%%%%%%%%%%%%%%%%%%%%%%%%%%%%%%%%%%%%%

\usepackage{eurosym}
\usepackage{vmargin}
\usepackage{amsmath}
\usepackage{graphics}
\usepackage{epsfig}
\usepackage{enumerate}
\usepackage{multicol}
\usepackage{subfigure}
\usepackage{fancyhdr}
\usepackage{listings}
\usepackage{framed}
\usepackage{graphicx}
\usepackage{amsmath}
\usepackage{chngpage}

%\usepackage{bigints}
\usepackage{vmargin}

% left top textwidth textheight headheight

% headsep footheight footskip

\setmargins{2.0cm}{2.5cm}{16 cm}{22cm}{0.5cm}{0cm}{1cm}{1cm}

\renewcommand{\baselinestretch}{1.3}

\setcounter{MaxMatrixCols}{10}

\begin{document}
%%-Question 7 
In a correlation analysis based on a random sample of 10 values from a bivariate
normal distribution, a t-test of
$H0 : \rho = 0 v. H1 : \rho > 0$
results in a probability-value of 0.025.
Calculate the value of the sample correlation coefficient. 
\newpage


7 

%%%%%%% Question 7

The $t-$test is based on
\[ t_{TS} = \frac{r\sqrt{n-2}}{\sqrt{1-r^2}} \sim t_{n-2}\]

here $n − 2 = 8$
If $p-$value for this one-sided test is 0.025 then observed t = 2.306 (t0.025,8 from
Green tables)


\[ t_{TS} = 2.306 = \frac{r\sqrt{8}}{\sqrt{1-r^2}} \sim t_{n-2}\]

Also  $r > 0$

Squaring both sides, and algebraically re-arranging

\[5.318 (1 − r^2) = 8r^2\]
Therefore

\[r^2 = \frac{5.318}{13.318}=  0.3993\]

\[r = 0.632\]




An alternative approach is to use Fisher’s transformation of r. The
statistic is
\[ t_{TS} = \frac{1}{2} \log \frac{1+r}{1-r}, \]

which, under $H_0$, has, approximately, a $N(0,1/7)$ distribution.
This approach gives $r = 0.630$ .
%%%%%%%%%%%%%%%%%%%%%%%%%%%%%%%%%%%%%%%%%
\newpage Claims on a certain class of policy are classified as being of two types, I and II.
Past experience has shown that:
25\% of claims are of type I and 75\% are of type II;
\begin{itemize}
    \item Type I claim amounts have mean £500 and standard deviation £100;
\item Type II claim amounts have mean £300 and standard deviation £70.
\end{itemize}

Calculate the mean and the standard deviation of the claim amounts on this
class of policy. 
%%%%%%%%%%%%%%%%%%%%%%%%%%%%%%%%%%%%%%%%%
\newpage 
%%- Question 8 

Let Y = amount
\begin{itemize}
    \item Let X = 1, 2 for types I, II
Therefore P(X = 1) = 0.25, P(X = 2) = 0.75
\item $E(Y|X = 1) = 500$, $Var(Y|X = 1) = 100^2$
all given
\item $E(Y|X = 2) = 300, Var(Y|X = 2) = 70^2$
\item \begin{eqnarray*}
E(Y) &=& E(E(Y|X)) \\ &=& 500(0.25) + 300(0.75)\\
&=& 125 + 225 \\ &=& \$350\\
\end{eqnarray*}
\end{itemize}

%%%%%%%%%%%%%%%%%%%%%%%%%%%%%%%%%%%%%%%%%%%%%%%%%%%%%%%%%
%%--- Page 4
\begin{itemize}
    \item $V(Y) = E(V(Y|X)) + V(E(Y|X))$
\item \begin{eqnarray*}
E(V(Y|X)) &=& 100^2(0.25) + 70^2(0.75)\\
&=& 2500 + 3675\\ &=& 6175\\
\end{eqnarray*}
%%%%%%%%%%%%%%%%%%%%%%%%%%%%%%%%%%%%%%%%%%%%
\item \begin{eqnarray*}V(E(Y|X)) &=& 500^2(0.25) + 300^2(0.75) − 350^2\\
&=& 62500 + 67500 − 122500 \\ &=& 7500\\
\end{eqnarray*}
Therefore $V(Y) = 6175 + 7500 = 13675$
Therefore $s.d.(Y) = \$116.9$
\item OR: $V(E(Y|X)) = 0.25(500 − 350)^2 + 0.75(300 − 350)^2 = 7500$
alternative method for $V(Y)$:
\item E(Y2|X = 1) = 1002 + 5002 = 260000
\[E(Y2|X=2) = 70^2 + 300^2 = 94900\]
Therefore $E(Y^2) = 0.25(260000) + 0.75(94900)
= 136175$
\item Therefore V(Y) = 136175 − 3502 = 13675
Therefore $s.d.(Y) = 116.9$
\end{itemize}

Q8 Comment: Very few candidates seemed to be aware of the result \[V(Y) = E[V(Y|X)]
+ V[E(Y|X)] \]and how and when it should be used.

\end{document}
