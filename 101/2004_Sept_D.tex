\documentclass[a4paper,12pt]{article}

%%%%%%%%%%%%%%%%%%%%%%%%%%%%%%%%%%%%%%%%%%%%%%%%%%%%%%%%%%%%%%%%%%%%%%%%%%%%%%%%%%%%%%%%%%%%%%%%%%%%%%%%%%%%%%%%%%%%%%%%%%%%%%%%%%%%%%%%%%%%%%%%%%%%%%%%%%%%%%%%%%%%%%%%%%%%%%%%%%%%%%%%%%%%%%%%%%%%%%%%%%%%%%%%%%%%%%%%%%%%%%%%%%%%%%%%%%%%%%%%%%%%%%%%%%%%

\usepackage{eurosym}
\usepackage{vmargin}
\usepackage{amsmath}
\usepackage{graphics}
\usepackage{epsfig}
\usepackage{enumerate}
\usepackage{multicol}
\usepackage{subfigure}
\usepackage{fancyhdr}
\usepackage{listings}
\usepackage{framed}
\usepackage{graphicx}
\usepackage{amsmath}
\usepackage{chngpage}

%\usepackage{bigints}
\usepackage{vmargin}

% left top textwidth textheight headheight

% headsep footheight footskip

\setmargins{2.0cm}{2.5cm}{16 cm}{22cm}{0.5cm}{0cm}{1cm}{1cm}

\renewcommand{\baselinestretch}{1.3}

\setcounter{MaxMatrixCols}{10}

\begin{document}

\begin{enumerate}
\item
12
The following data refer to an outbreak of botulism, a form of food poisoning that
may be fatal. Each subject is a person who contracted botulism in the outbreak. The
variables recorded are the subject s age in years, the time in hours between eating the
infected food and the first signs of illness (incubation period) and whether the subject
survived (denoted by survival category Y) or died (denoted by survival category N).
\begin{verbatim}
    Subject
Age ( x )
Incubation
period ( y )
Survival
1 2 3 4 5 6 7 8 9 10 11 12 13 14 15 16 17 18
29 39 44 37 42 17 38 43 51 30 32 59 33 31 32 32 36 50
13 46 43 34 20 20 18 72 19 36 48 44 21 32 86 48 28 16
N Y Y N N Y N Y N N N Y N N Y N Y N
Died: x = 405; y = 305; x 2 = 15517; y 2 = 10035
Survived: x = 270; y = 339; x 2 = 11396; y 2 = 19665
\end{verbatim}

\begin{enumerate}[(a)]
\item (i)
A scatterplot of incubation period against age is given below, in which
different symbols are used for subjects who died and for subjects who
survived.
A Plot of Incubation Period against Age
90
Died
Survived
80
70
60
50
40
30
20
10
20
30
40
50
60
Age
Comment briefly on any relationships between age and incubation period for
those subjects who died and for those subjects who survived.
\item  (ii)
Construct suitable dotplots to investigate any relationship between:
(a) age and survival, and
(b) incubation period and survival
and make a brief informal comparison of the died and survived groups based
on these dotplots.
\item (iii)
Construct 95% and 99% confidence intervals for the mean difference between
the incubation period for subjects who survived and subjects who died (i.e.
take the mean incubation period for subjects who survived minus the mean
incubation period for subjects who died).
Comment briefly on these confidence intervals.
\item  (iv)
%---------------------%
7
[6]
(a) Conduct a test to investigate whether the variances of the incubation
periods for subjects who died and subjects who survived are equal.
(b) Comment on the validity of the assumptions that are required for the
confidence intervals given in part (iii) to be appropriate.
\end{enumerate}
\newpage

\item 13
For each of a group of policyholders the number of claims, Y, occurring in a period of
one year is modelled by the following modified Poisson random variable, which
incorporates a reluctance to claim:
there is a probability of that Y equals zero and a probability of (1
comes from a Poisson distribution with mean , so that
P ( Y = 0) =
(1
P ( Y = r ) = (1
) that Y
) P ( X = 0)
) P ( X = r ), r = 1, 2,3,
where X ~ Poisson( ).
\begin{enumerate}[(a)]
\item (i)
Show that the mean and variance of Y are given by
E ( Y ) (1
V ( Y ) (1
)
) (1
)
and comment briefly on these values in comparison to those for the
unmodified Poisson variable, X.
[5]
\item (ii) A random sample of such policyholders resulted in a distribution of numbers
of claims with sample mean y and sample standard deviation s. Use the
method of moments to determine estimators for and in terms of y and s 2 .
[4]
\item (iii) Data on the numbers of claims from a sample of 200 policyholders resulted in
the following frequency distribution.
number of claims:
frequency:
0
90
1
56
2
37
3
12
4
4
5
1
>5
0
(a) Calculate the mean and variance for this sample and hence calculate
the method of moments estimates of and .
(b) Using these estimates the expected frequencies under the fitted
modified Poisson model were calculated for y = 0, 1, 2, 3 and are given
in the table below.
y exp. freq.
0 88.9
1 59.2
2 34.0
3 13.1
Calculate the expected frequencies for y = 4, 5 and y > 5 and comment
briefly on the suitability of the model for these data.
\end{enumerate}
%===================================%
\item 14
Consider the following data, which comprise four groups of claim sizes (y), each
comprising four observations. In scenario I, information is also given on the sum
assured under the policy concerned the sum assured is the same for all four
policies in a group. In scenario II, we regard the policies in the different groups as
having been issued by four different companies
the policies in a group are all
issued by the same company.
All monetary amounts are in units of £10,000. Summaries of the claim sizes in each
group are given in a second table.
Group
Claim sizes y
I: Sum assured x
II: Company
1
0.11 0.46
0.71 1.45
1
A
2
0.52 1.43
1.84 2.47
2
B
3
1.48 2.05
2.38 3.31
3
C
4
1.52 2.36
2.95 4.08
4
D
Summaries of claim sizes:
Group
y
y 2
(i)
1
2.73
2.8303
2
6.26
11.8018
3
9.22
23.0134
4
10.91
33.2289
In scenario I, suppose we adopt the linear regression model
Y i =
+ x i + e i
where Y i is the i th claim size and x i is the corresponding sum assured,
i = 1, , 16.
\begin{enumerate}[(i)]
\item calculate the total sum of squares and its partition into the regression
(model) sum of squares and the residual (error) sum of squares.
\item (b) Fit the model and calculate the fitted values for the first claim size of
group 1 (namely 0.11) and the last claim size of group 4 (namely 4.08).
\item (c) Consider a test of the hypothesis H 0 : = 0 against a two-sided
alternative. By performing appropriate calculations, assess the strength
of the evidence against this no linear relationship hypothesis.
\end{enumerate}
(ii)
In scenario II, suppose we adopt the analysis of variance model
Y ij =
+
i
+ e ij
where Y ij is the j th claim size for company i and
j = 1, 2, 3, 4 and i = A, B, C, D.
is the i th company effect,
(a) Calculate the partition of the total sum of squares into the between
companies (model) sum of squares and the within companies
(residual/error) sum of squares.
(b) Fit the model.
(c) Calculate the fitted values for the first claim size of group 1 and the
last claim size of group 4.
(d) Consider a test of the hypothesis H 0 : i 0 , i = A, B, C, D against a
general alternative. By performing appropriate calculations, assess the
strength of the evidence against this no company effects hypothesis.
\end{enumerate}
\newpage


%=============================================================%


12

(i) There does not seem to be a clear relationship between age and incubation
period either for those subjects who died or for those subjects who survived.
(ii) (a)
Dotplots of age for died and survived subjects.
Died:
. ..:.
. .
.
..
-+---------+---------+---------+---------+---------+-----Age
Survived:
.
.
.
.
..
.
-+---------+---------+---------+---------+---------+-----Age
16.0
24.0
32.0
40.0
48.0
56.0
(b)
Dotplots of incubation period for died and survived subjects.
Died:
. ..:.
. ..
:
-----+---------+---------+---------+---------+---------+-IncubP
Survived:
.
.
: .
.
.
-----+---------+---------+---------+---------+---------+-IncubP
15
30
45
60
75
90
The dotplots suggest an association between survival and incubation period
(the people who survived tended to have longer incubation periods), but do not
suggest an association between survival and age.
(iii)
Survived: n 1
7 , y 1
Died: n 2 11 , y 2
339 / 7 48 . 429 , s 1
305 / 11
27 . 727 , s 2
3247 . 71 / 6 23 . 266
1578 . 18 / 10 12 . 563
Pooled variance and standard deviation:
s 2 p (n 1 - 1 )s 1 2 (n 2 - 1 )s 2 2
n 1 n 2 - 2
s p 17 . 37
6 ( 23 . 266 ) 2 10 ( 12 . 563 ) 2
16
301 . 633
95% confidence interval:
y 1
y 2 t 0.025, n 1
n 2 2 s p
48.429 27.727
Page 6
1
n 1
1
n 2
(2.120)(17.37)
1 1
7 11
%--------------------------------------%
Examiners Report
i.e. ( 2.9, 38.5) hours.
99% confidence interval:
y 1
y 2 t 0.005, n 1
n 2 2 s p
1
n 1
1
n 2
48.429 27.727 (2.921)(17.37)
1 1
7 11
20.702 24.531
i.e. ( 3.8, 45.2) hours.
\begin{itemize}
    \item The 95\% confidence interval does not contain zero, therefore a two-sided, two
sample t-test conducted at the 5\% significance level would conclude that there
is a difference between the means of the two populations.
\item However, as the 99\% confidence interval does contain zero, conducting a
more stringent 1\% level t-test would not reject the null hypothesis that the
population means are the same.
\end{itemize}

(iv)
(a)
Assuming that the variances of the two populations are equal,
S 1 2 / S 2 2 ~ F n 1 1, n 2 1 .
s 1 2 / s 2 2
23.266 2 /12.563 2
3.430
The value of the test statistic is below the 5% significance level critical
value of F 6,10 (2.5%) = 4.072. This indicates that there is insufficient
evidence to reject the null hypothesis that the two populations have
equal variances.
(b)
In part (iii) an assumption of normality of each sample was required.
The dotplots suggest that this assumption is valid.
Also, the assumption of equal variances of the two groups seems valid
from the result of the test conducted in part (iv)(a).
%%%%%%%%%%%%%%%%%%%%%%%%%%%%%%%%%%%%%%%%%%%%%%%%%%%%%%%%%%%%%%%%%%%%%%%%%%%%%%%%%%%%%
\newpage

13
(i)
E(Y) = 0 . P(Y = 0) +
September 2004
r . P ( Y
Examiners Report
r )
r 1
= (1
)
r ) = (1
r . P ( X
)E(X) = (1
)
r 1
[Note: there is an alternative solution using conditional expectation]
r 2 . P ( Y
E(Y 2 ) = 0 +
r )
r 1
= (1
)
r 2 . P ( X r ) = (1
2 ) ) 2
)E(X 2 ) = (1
2
)(
r 1
V(Y) = (1
)(
= (1
)
(1 +
(1
2
= (1
)
{1 +
(1
)
}
)
E(Y) < E(X) as expected since there are more values equal to zero in the
adjusted distribution.
The original Poisson has V(X) = E(X). Here, with the extra zeros, we get
greater variability relative to the mean, as we see in the fact that V(Y) > E(Y).
(ii)
For method of moments we seek
y
s 2
(1
and
)
(1
) (1
First equation gives:
)
y
(1
)
Substituting into second equation gives:
Solving for
s 2
gives:
y
2
s 2
y
1
1
y
s
Substituting into expression for
Page 8
being solutions of
2
y
gives:
y 2
s 2
y
y
y
(1
)
%%%%%%%%%%%%%%%%%%%%%%%%%%%%%%%%%%%%%%%%%%%%%%%%%%%%%%%%%%%%%%%%%%%%%%%%
(iii)
(a)

n = 200, y = 187, y 2 = 401
187
200
y
0.935
401 187 2 / 200
1.1365
199
s 2
1.1365 0.935
0.935 2 1.1365 0.935
0.1873
0.935 2 1.1365 0.935
1.1505
0.935
(b)
P ( X
P ( Y
1.1505 4 e
4)
4!
1.1505
0.0231
4) 0.8127(0.0231) 0.0188
Exp. freq. = 200(0.0188) = 3.8
Similarly
P(X = 5) = 0.0053, P(Y = 5) = 0.0043,
and exp. freq. = 0.9
By subtraction exp. freq. for y > 5 is 200 - 199.9 = 0.1
obs.
exp.
90
88.9
56
59.2
37
34.0
12
13.1
4
3.8
1
0.9
0
0.1
which show good agreement with the observed frequencies so that the
model seems to fit the data well.
14
(i)
(a)
(b)
y = 29.12, y 2 = 70.8744
SS TOT = 70.8744 29.12 2 /16 = 17.8760
x = 4 10 = 40, x 2 = 4 30 = 120
S xx = 120 40 2 /16 = 20
xy = 1 2.73 + 2 6.26 + 3 9.22 + 4 10.91 = 86.55
S xy = 86.55 40 29.12/16 = 13.75
Regression sum of squares SS REG = 13.75 2 /20 = 9.4531
Residual sum of squares SS RES = 17.8760 9.4531 = 8.4229
13.75 / 20 0.6875
29.12 /16 0.6875 (40 /16) 0.1012
Fitted model is y 0.1012 0.6875 x
y = 0.11, x = 1 fitted value = 0.7887
Page 9Subject 101 (Statistical Modelling)
y = 4.08, x = 4
(c)
s . e .
%------------------------------------------------------%
fitted value = 2.8512
8.4229 /14
20
Under H 0 , P
September 2004
0.5
0.6875
0.1734
P t 14
0.6875 / 0.1734
P t 14
3.965
which is very much lower than 0.005, so P-value of test statistic is very
much lower than 0.01.
We have strong evidence against the no linear relationship
hypothesis (p << 0.01)
(ii)
\begin{itemize}
    \item (a)
SS TOT = 17.8760
Between companies sum of squares
SS B = (2.73 2 + 6.26 2 + 9.22 2 + 10.91 2 )/4
29.12 2 /16 = 9.6709
Residual sum of squares SS RES = 17.8760
    \item (b)
1
3
9.6709 = 8.205
29.12 /16 1.82
2.73 / 4 1.82
1.1375 , 2 6.26 / 4 1.82
0.255
9.22 / 4 1.82 0.485 , 4 10.91/ 4 1.82 0.9075
    \item (c) y = 0.11, company A
y = 4.08, company D
fitted value = 2.73/4 = 0.6825
fitted value = 10.91/4 = 2.7275
    \item (d) Observed F statistic is (9.6709/3) / (8.2051/12) = 4.715 on 3,12 df
P-value of test statistic is lower than 0.05 (but higher than 0.01)
\end{itemize}

We have some evidence against the no company effects hypothesis
(0.01 < p < 0.05)
\end{document}
