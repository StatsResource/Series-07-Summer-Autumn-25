\documentclass[a4paper,12pt]{article}

%%%%%%%%%%%%%%%%%%%%%%%%%%%%%%%%%%%%%%%%%%%%%%%%%%%%%%%%%%%%%%%%%%%%%%%%%%%%%%%%%%%%%%%%%%%%%%%%%%%%%%%%%%%%%%%%%%%%%%%%%%%%%%%%%%%%%%%%%%%%%%%%%%%%%%%%%%%%%%%%%%%%%%%%%%%%%%%%%%%%%%%%%%%%%%%%%%%%%%%%%%%%%%%%%%%%%%%%%%%%%%%%%%%%%%%%%%%%%%%%%%%%%%%%%%%%

\usepackage{eurosym}
\usepackage{vmargin}
\usepackage{amsmath}
\usepackage{graphics}
\usepackage{epsfig}
\usepackage{enumerate}
\usepackage{multicol}
\usepackage{subfigure}
\usepackage{fancyhdr}
\usepackage{listings}
\usepackage{framed}
\usepackage{graphicx}
\usepackage{amsmath}
\usepackage{chngpage}

%\usepackage{bigints}
\usepackage{vmargin}

% left top textwidth textheight headheight

% headsep footheight footskip

\setmargins{2.0cm}{2.5cm}{16 cm}{22cm}{0.5cm}{0cm}{1cm}{1cm}

\renewcommand{\baselinestretch}{1.3}

\setcounter{MaxMatrixCols}{10}

\begin{document}
% 18 September 2000 (pm)
% Subject 101 — Statistical Modelling
% Faculty of Actuaries Institute of Actuaries
% EXAMINATIONS
%%%%%%%%%%%%%%%%%%%%%%%%%%%%%%%%%%%%%%%%%%%%%%%%%%%%%%%%%%%%%%%%%%%%%%%%
\begin{enumerate}
\item 5 The number of claims which arise under a policy of a particular type in a year is
to be modelled as a Poisson($\lambda$) random variable. A random sample of 500 such policies gave rise to a total of 84 claims in 1999.
Calculate a 95\% confidence interval for $\lambda$. 
%%%%%%%%%%%%%%%%%%%%%%%%%%%%%%%
\item 6 Suppose that the linear regression model
\[Y = \alpha + \beta x + e\]
is fitted to data $\{(y_i , x_i) : i = 1, 2, \ldots , n\}$, where y is the salary (£) of a company
manager and x (years) is the number of years of relevant experience of that
manager.
State the units of measurement (if any) of
\begin{enumerate}
    \item  $\hat{\alpha}$ , the estimate of $\alpha$ ,
\item $\hat{\beta}$ , the estimate of $\beta$ ,
\item $R^2$ , the coefficient of determination of the fit. 
\end{enumerate}

\newpage

\item 7 In a correlation analysis based on a random sample of 10 values from a bivariate
normal distribution, a t-test of
$H0 : \rho = 0 v. H1 : \rho > 0$
results in a probability-value of 0.025.
Calculate the value of the sample correlation coefficient. 
\newpage
\item  Claims on a certain class of policy are classified as being of two types, I and II.
Past experience has shown that:
25\% of claims are of type I and 75\% are of type II;
\begin{itemize}
    \item Type I claim amounts have mean £500 and standard deviation £100;
\item Type II claim amounts have mean £300 and standard deviation £70.
\end{itemize}

Calculate the mean and the standard deviation of the claim amounts on this
class of policy. [6]
\end{enumerate}
%%%%%%%%%%%%%%%%%%%%%%%%%%%%%%%%%%%%%%%%%%%%%%%%%%%%%%%%%%%%%%%%%%%%%%%%%%%%%%%%%%55
6 (a) £
(b) £/year i.e. £ × year−1
(c) no units


\newpage
7 

%%%%%%% Question 7

The $t-$test is based on
\[ t_{TS} = \frac{r\sqrt{n-2}}{\sqrt{1-r^2}} \sim t_{n-2}\]

here $n − 2 = 8$
If $p-$value for this one-sided test is 0.025 then observed t = 2.306 (t0.025,8 from
Green tables)


\[ t_{TS} = 2.306 = \frac{r\sqrt{8}}{\sqrt{1-r^2}} \sim t_{n-2}\]

Also  $r > 0$

Squaring both sides, and algebraically re-arranging

\[5.318 (1 − r^2) = 8r^2\]
Therefore

\[r^2 = \frac{5.318}{13.318}=  0.3993\]

\[r = 0.632\]




An alternative approach is to use Fisher’s transformation of r. The
statistic is
\[ t_{TS} = \frac{1}{2} \log \frac{1+r}{1-r}, \]

which, under $H_0$, has, approximately, a N(0,1/7) distribution.
This approach gives r = 0.630 .
%%%%%%%%%%%%%%%%%%%%%%%%%%%%%%%%%%%%%%%%%

%%%%%%%%%%%%%%%%%%%%%%%%%%%%%%%%%%%%%%%%%%%%%%%%%%%%%%%%%%%%%%%%%%%%%%%%%%%%%%%%%%%%

8 Let Y = amount
\begin{itemize}
    \item Let X = 1, 2 for types I, II
∴ P(X = 1) = 0.25, P(X = 2) = 0.75
\item E(YX = 1) = 500, Var(YX = 1) = 1002
all given
\item E(YX = 2) = 300, Var(YX = 2) = 702
\item E(Y) = E(E(YX)) = 500(0.25) + 300(0.75)
= 125 + 225 = £350

\end{itemize}

%%%%%%%%%%%%%%%%%%%%%%%%%%%%%%%%%%%%%%%%%%%%%%%%%%%%%%%%%
Page 4
\begin{itemize}
    \item V(Y) = E(V(YX)) + V(E(YX))
\item E(V(YX)) = 1002(0.25) + 702(0.75)
= 2500 + 3675 = 6175
\item V(E(YX)) = 5002(0.25) + 3002(0.75) − 3502
= 62500 + 67500 − 122500 = 7500
∴ V(Y) = 6175 + 7500 = 13675
∴ s.d.(Y) = £116.9
\item OR: V(E(YX)) = 0.25(500 − 350)2 + 0.75(300 − 350)2 = 7500
alternative method for V(Y):
\item E(Y2X = 1) = 1002 + 5002 = 260000
E(Y2X=2) = 702 + 3002 = 94900
∴ E(Y2) = 0.25(260000) + 0.75(94900)
= 136175
\item ∴ V(Y) = 136175 − 3502 = 13675
∴ s.d.(Y) = 116.9
\end{itemize}

Q8 Comment: Very few candidates seemed to be aware of the result V(Y) = E[V(Y|X)]
+ V[E(Y|X)] and how and when it should be used.


%%%%%%% Question 9

C(t) = \log M(t) = −\alpha \log(1− \theta t)
C^{\prime}(t) = \alpha\theta (1 − \theta t)^{−1} , 
C^{\prime}^{\prime}(t) = \alpha\theta 2(1 − \theta t)^{−2} , C^{\prime}^{\prime}^{\prime}(t) = 2\alpha\theta 3(1 − \theta t)^{−3}
∴ 

\begin{itemize}
\item $\kappa_2 = C^{\prime}^{\prime}(0) = \alpha\theta 2$ , 
\item $\kappa_3 = C^{\prime}^{\prime}^{\prime}(0) = 2\alpha\theta^3$ 
\end{itemize}
%-------------%

so coefficient is
\begin{eqnarray*}
\frac{\kappa_3 }{ (\kappa_2)^{3/2} } &=& \frac{2\alpha\theta 3 }{ (\alpha\theta^2)^{3/2} }\\ 
&=& \frac{2}{\sqrt{\alpha}}
\end{eqnarray*}

\newpage

13 ( ) i E Y⋅ = μ + τi and E(Y⋅⋅ ) = μ
( ) i E Y⋅ E(μˆ ) = E(Y⋅⋅ ) = μ ∴ unbiased
(ˆ ) i E τ = ( ) i E Y⋅ −Y⋅⋅ = μ + τi − μ = τi ∴ unbiased
%%%%%%%%%%%%%%%%%%%%%%%%%%%%%%%%%%%%%%%%%%%%%%%%%%%%%%%%%

Q13 Comment: This material was unfamiliar to most candidates.

\end{document}
