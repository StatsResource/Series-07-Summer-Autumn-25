\documentclass[a4paper,12pt]{article}

%%%%%%%%%%%%%%%%%%%%%%%%%%%%%%%%%%%%%%%%%%%%%%%%%%%%%%%%%%%%%%%%%%%%%%%%%%%%%%%%%%%%%%%%%%%%%%%%%%%%%%%%%%%%%%%%%%%%%%%%%%%%%%%%%%%%%%%%%%%%%%%%%%%%%%%%%%%%%%%%%%%%%%%%%%%%%%%%%%%%%%%%%%%%%%%%%%%%%%%%%%%%%%%%%%%%%%%%%%%%%%%%%%%%%%%%%%%%%%%%%%%%%%%%%%%%

\usepackage{eurosym}
\usepackage{vmargin}
\usepackage{amsmath}
\usepackage{graphics}
\usepackage{epsfig}
\usepackage{enumerate}
\usepackage{multicol}
\usepackage{subfigure}
\usepackage{fancyhdr}
\usepackage{listings}
\usepackage{framed}
\usepackage{graphicx}
\usepackage{amsmath}
\usepackage{chngpage}

%\usepackage{bigints}
\usepackage{vmargin}

% left top textwidth textheight headheight

% headsep footheight footskip

\setmargins{2.0cm}{2.5cm}{16 cm}{22cm}{0.5cm}{0cm}{1cm}{1cm}

\renewcommand{\baselinestretch}{1.3}

\setcounter{MaxMatrixCols}{10}

\begin{document}
%% EXAMINATIONS
%% 17 April 2000 (pm)
%% Subject 101 — Statistical Modelling
%%%%%%%%%%%%%%%%%%%%%%%%%%%%%%%%%%%%%%%%%%%%%%%%%%%%%%%%%%%%%%%%%%%%%
\begin{enumerate}
%%%%%%%%%%%%%%%%%%%%%%%%%%%%%%%%%%%%%%%%%%%%%%%%%%%%%%%
\item  An insurance company’s records suggest that experienced drivers (those aged
over 21) submit claims at a rate of 0.1 per year, and inexperienced drivers (those 21 years old or younger) submit claims at a rate of 0.15 per year. 
\begin{itemize}
\item A driver can submit more than one claim a year. The company has 40 experienced and 20 inexperienced drivers insured with it.
\item The number of claims for each driver can be modelled by a Poisson distribution, and claims are independent of each other.
\end{itemize}
 Calculate the
probability the company will receive three or fewer claims in a year. 

%%%%%%%%%%%%%%%%%%%%%%%%%%%%%%%%%%%%%%%%%%%%%%%%%%%%%%%

\begin{itemize}
\item  As claims are independent the number of claims by inexperienced drivers will
follow a Poisson distribution with mean $20 \times 0.15 = 3$, and the number of claims
made by experienced drivers will follow a Poisson distribution with mean
$40 \times 0.1 = 4$. Again using the independence assumption, the total number of
claims, $X$, is Poisson with mean 7.
\item Thus, 
\begin{eqnarray*}
P(X \leq 3) &=& \sum^{3}_{i=0} e^{-7} \frac{7^i}{i!}\\
& & \\
&=& P(X \leq 0) + P(X \leq 1) + P(X \leq 2) + P(X \leq 3)\\
&=& \left(e^{-7} \frac{7^0}{0!} \right) + \left(e^{-7} \frac{7^1}{1!}\right) + \\
& & \left(e^{-7} \frac{7^2}{2!}\right) + \left(e^{-7} \frac{7^3}{3!}\right)\\
& & \\
&=& 0.082.
\end{eqnarray*}
\item (The answer can also be taken directly from the Green Book, which gives
0.08177.)
\end{itemize}

\newpage

\item The number of claims which arise under a policy of a particular type in a year
is to be modelled as a Poisson(l) random variable. A random sample of 600
such policies gave rise to a total of 72 claims in 1999.
Determine the approximate probability value of this result in a test of
H0 : l = 0.14 v H1 : l < 0.14 .
\item 6 
Total number of claims $X \sim \mbox{ Poisson}(600\lambda)$

Under H0: $X \sim \mbox{ Poisson}(84) \sim N(84,84)$ approximately

\begin{eqnarray*}
Prob. value &=& P(X \leq 72) \\
&=& P(Z < \frac{(72.5 - 84)}{\sqrt{84}} \\
&=& P(Z < -1.255) \\ 
&=& 0.105\\
\end{eqnarray*}
\item 7 ( ) ( , ) ( | ) ( ) ( | ) ( ) Y Y E X xf x y dxdy xf x y dx f y dy E X Y y f y dy

\item Suppose that X and Y are continuous random variables.
Prove that E(X ) E(X|Y y)fY( y)dy .
%%%%%
Solution: 
\begin{eqnarray*}
E(X)&=&  \int^{infty}_{-\infty} \int^{infty}_{-\infty} x f(x,y) dx dy\\
&=&  \int^{infty}_{-\infty} \left( \int^{infty}_{-\infty} x f(x|y) dx \right) f_Y(y) dy\\

&=&  \int^{infty}_{-\infty} E(X|Y=y) f(x|y) dx dy\\
\end{eqnarray*}
%%%%%%%%%%%%%%%%%%%%%%%%%%%%%%%%%%%%%%%%%%%%%%%%%%%%%%%
\newpage
\item 8 A device contains an electronic component which has a lifetime modelled by a
distribution with mean 3.6 hours and standard deviation 2.6 hours. On failure
a new component is automatically and instantaneously inserted as a
replacement.
Consider the operation of the device with 100 such components used one after
the other. Determine the approximate probability that the resulting total
lifetime of the device will be greater than 400 hours. 

\end{enumerate}
%%%%%%%%%%%%%%%%%%%%%%%%%%%%
8
$T = \sum^{100}_{i=1} X_i$ has mean 100(3.6) = 360 hours
and s.d. 100(2.6) = 26 hours.
Central limit theorem 
 T is approximately normal as n is large.
Therefore
\begin{eqnarray*}
P(T > 400) &=& P \left(Z > \frac{400-360}{26} \right) \\
&=& P \left(Z >1.54 \right) \\
&=& 1 - 0.93822 \\
&=& 0.062
\end{eqnarray*}



\end{document}
