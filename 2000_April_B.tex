\documentclass[a4paper,12pt]{article}

%%%%%%%%%%%%%%%%%%%%%%%%%%%%%%%%%%%%%%%%%%%%%%%%%%%%%%%%%%%%%%%%%%%%%%%%%%%%%%%%%%%%%%%%%%%%%%%%%%%%%%%%%%%%%%%%%%%%%%%%%%%%%%%%%%%%%%%%%%%%%%%%%%%%%%%%%%%%%%%%%%%%%%%%%%%%%%%%%%%%%%%%%%%%%%%%%%%%%%%%%%%%%%%%%%%%%%%%%%%%%%%%%%%%%%%%%%%%%%%%%%%%%%%%%%%%

\usepackage{eurosym}
\usepackage{vmargin}
\usepackage{amsmath}
\usepackage{graphics}
\usepackage{epsfig}
\usepackage{enumerate}
\usepackage{multicol}
\usepackage{subfigure}
\usepackage{fancyhdr}
\usepackage{listings}
\usepackage{framed}
\usepackage{graphicx}
\usepackage{amsmath}
\usepackage{chngpage}

%\usepackage{bigints}
\usepackage{vmargin}

% left top textwidth textheight headheight

% headsep footheight footskip

\setmargins{2.0cm}{2.5cm}{16 cm}{22cm}{0.5cm}{0cm}{1cm}{1cm}

\renewcommand{\baselinestretch}{1.3}

\setcounter{MaxMatrixCols}{10}

\begin{document}
%% EXAMINATIONS
%% 17 April 2000 (pm)
%% Subject 101 — Statistical Modelling
%%%%%%%%%%%%%%%%%%%%%%%%%%%%%%%%%%%%%%%%%%%%%%%%%%%%%%%%%%%%%%%%%%%%%
\begin{enumerate}
    \item 

\item  As claims are independent the number of claims by inexperienced drivers will
follow a Poisson distribution with mean 20  0.15 = 3, and the number of claims
made by experienced drivers will follow a Poisson distribution with mean
40  0.1 = 4. Again using the independence assumption, the total number of
claims, X, is Poisson with mean 7.
Thus, P(X 3) =
3
7
0
7
!
i
i
e
i


 = 0.082.
(The answer can also be taken directly from the Green Book, which gives
0.08177.)
\item 6 Total number of claims X ~ Poisson(600)
Under H0: X ~ Poisson(84) ~ N(84,84) approximately
Prob. value = P(X  72) = P[Z < (72.5  84)/	84] = P(Z < 1.255) = 0.105
\item 7 ( ) ( , ) ( | ) ( ) ( | ) ( ) Y Y E X xf x y dxdy xf x y dx f y dy E X Y y f y dy

\item Suppose that X and Y are continuous random variables.
Prove that E(X ) E(X|Y y)fY( y)dy .
¥
-¥
= ò = 
%%%%%%%%%%%%%%%%%%%%%%%%%%%%%%%%%%%%%%%%%%%%%%%%%%%%%%%
\newpage
\item 8 A device contains an electronic component which has a lifetime modelled by a
distribution with mean 3.6 hours and standard deviation 2.6 hours. On failure
a new component is automatically and instantaneously inserted as a
replacement.
Consider the operation of the device with 100 such components used one after
the other. Determine the approximate probability that the resulting total
lifetime of the device will be greater than 400 hours. 

\end{enumerate}
%%%%%%%%%%%%%%%%%%%%%%%%%%%%
8 T =
100
i1
 Xi has mean 100(3.6) = 360 hours
and s.d. 100(2.6) = 26 hours.
Central limit theorem 
 T is approximately normal as n is large.
 P(T > 400) 

400 360
1.54
26
P Z
  
   
 	
= 1  0.93822 = 0.062


\end{document}
