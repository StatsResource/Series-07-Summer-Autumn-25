
\documentclass[a4paper,12pt]{article}

%%%%%%%%%%%%%%%%%%%%%%%%%%%%%%%%%%%%%%%%%%%%%%%%%%%%%%%%%%%%%%%%%%%%%%%%%%%%%%%%%%%%%%%%%%%%%%%%%%%%%%%%%%%%%%%%%%%%%%%%%%%%%%%%%%%%%%%%%%%%%%%%%%%%%%%%%%%%%%%%%%%%%%%%%%%%%%%%%%%%%%%%%%%%%%%%%%%%%%%%%%%%%%%%%%%%%%%%%%%%%%%%%%%%%%%%%%%%%%%%%%%%%%%%%%%%

\usepackage{eurosym}
\usepackage{vmargin}
\usepackage{amsmath}
\usepackage{graphics}
\usepackage{epsfig}
\usepackage{enumerate}
\usepackage{multicol}
\usepackage{subfigure}
\usepackage{fancyhdr}
\usepackage{listings}
\usepackage{framed}
\usepackage{graphicx}
\usepackage{amsmath}
\usepackage{chngpage}

%\usepackage{bigints}
\usepackage{vmargin}

% left top textwidth textheight headheight

% headsep footheight footskip

\setmargins{2.0cm}{2.5cm}{16 cm}{22cm}{0.5cm}{0cm}{1cm}{1cm}

\renewcommand{\baselinestretch}{1.3}

\setcounter{MaxMatrixCols}{10}

\begin{document}

10
Let X denote a random variable with a continuous uniform (0, 1000) distribution.
Define a random variable Y as the minimum of X and 800.
(i)
Show that the conditional distribution of X given X < 800 is a continuous
uniform (0, 800) distribution.
[2]
(ii) Verify (giving clear reasons) that the expectation of the random variable Y is
480.
[3]
(iii) Suppose that Y 1 , , Y n are independent and identically distributed, each with
the same distribution as Y.
In the case that n is large, determine the approximate distribution of
1 n
Y =
Y i , stating its expectation. (You are not required to derive or state
n i 1
the variance of Y .)
[1]
(iv)
CT3 S2005
Comment on the comparison of the conditional expectation of X given X < 800
with the expectation of Y.
[2]
[Total 8]
411
Consider the following simple model for the number of claims, N, which occur in a
year on a policy:
0
0.55
n
P(N = n)
1
0.25
2
0.15
3
0.05
(a) Explain how you would simulate an observation of N using a number r, an
observation of a random variable which is uniformly distributed on (0, 1).
(b) Illustrate your method described in (i) by simulating three observations of N
using the following random numbers between 0 and 1:
0.6221, 0.1472, 0.9862.
[4]
12
A certain type of insurance policy has a claim rate of per year and the cover ceases
and the policy expires after the first claim. Accordingly the duration of a policy is
modelled by an exponential distribution with density function e x : 0 x
.
A company has data on (m + n) policies which have expired and which may be
assumed to be independent. Of these, m policies had duration less than 5 years and n
policies had duration greater than or equal to 5 years.
(i)
An investigator makes note of the actual durations, x 1 , , x n , of the latter
group of n policies, but ignores the former group without even noting the
value of m.
(a)
Explain why the x i s come from a truncated exponential distribution
with density function
f ( x ) = k . e
, 5
x
e 5 .
and show that k
(b)
x
Write down the likelihood for the data from the point of view of this
investigator and hence show that the maximum likelihood estimate
(MLE) of is given by
n
n
.
x i 5 n
i 1
(c)
The data yield the values: n = 10 and x i = 71. Calculate this
investigator s MLE of .
[8]
CT3 S2005
5
PLEASE TURN OVER(ii)
A second investigator ignores the actual policy durations and simply notes the
values of m and n.
(a)
Write down the likelihood for this information and hence show that the
resulting MLE of is given by
=
(b)
1
m n
log
.
5
n
The same data as in part (i) yield the values: m = 120 and n = 10.
Calculate this investigator s MLE of .
[5]
(iii)
The two investigators decide to pool their data, and so have the information
that there are m policies with duration less than 5 years, and n policies with
actual durations x 1 , ... , x n .
(a)
Explain why the likelihood for this joint information is given by
L ( ) = (1 e
5
) m .
n
e
x i
i 1
and determine an equation, the solution of which will lead to the MLE
of .
(b)
Given that this leads to an MLE of
comparison of the three MLE s.
equal to 0.508, comment on the
[5]
[Total 18]

%%%%%%%%%%%%%%%%%%%%%%%%%%%%%%%%%%%%%%%%%%%%%%%%%%%%%%%%%%%%%%%%%%%%%%%%%%%%%%%%
X ~ U(0,1000) , Y = min(X,800)
(i)
P ( X
x | X
800)
P ( X
x and X 800)
P ( X 800)
P ( X x )
for 0 < x < 800
800 /1000
x /1000
800 /1000
x
for 0
800
x 800
so the conditional distribution is U(0,800)
[other reasonable arguments were given credit, e.g. the conditional
distribution is simply a scaled version of the original uniform distribution on a
restricted range .]
(ii)
E[Y] = E[X|X < 800] P(X < 800) + 800 P(X
400
= 480
Page 4
800
1000
800
200
1000
800)Subject CT3 (Probability and Mathematical Statistics Core Technical)
September 2005
Examiners Report
(iii) Y is approximately normal with expectation 480 by Central Limit Theorem
(iv) E[X | X < 800] = 400 whereas E[Y] = 480.
The higher value for E[Y] results from 20% of the Y values being 800 (and
80% being between 0 and 800).
11
(a)
For 0 r < 0.55
0.55 r < 0.8
0.8 r < 0.95
0.95 r 1
n = 0
n = 1
n = 2
n = 3
[OR any equivalent allocation which reflects the probabilities of the 4 values
of N.]
12
(b) 0.6221
0.1472
0.9862
(i) (a)
n = 1
n = 0
n = 3
The x i s are known to be such that x i
is a scaled form of e
x
5 , therefore have density which
for 5 < x < .
The scaling constant k is such that
k . e
x
dx 1
5
k [ e
x
] 5
1
k . e
5
1
k
e 5
[Note: this can be argued in other ways; e.g. by referring to a
conditional density and dividing by P(X > 5)]
Page 5Subject CT3 (Probability and Mathematical Statistics Core Technical)
n
(b)
x i
e 5 e
L ( )
n 5 n
e
September 2005
Examiners Report
x i
e
i 1
log L ( )
n log
d
log L ( )
d
5 n
n
x i
5 n
x i
n
equate to zero for MLE
n
x i 5 n
i 1
[OR It could be noted that X 5 ~ exp( ) and that the MLE is
therefore the reciprocal of the mean of the data ( x i 5) giving the
required answer]
(ii)
(c) n = 10, x i = 71
(a) L ( ) (1 e
log L ( )
5
10
71 50
) m ( e
m log(1 e
d
5 me
log L ( )
d
1 e
5
5
5
n
m n
) n
) n log( e
5
)
5
5
equate to zero for MLE
e
0.476
5 n
e
5
1 e
5
n
m
1
m n
log(
)
5
n
[OR Reason via the MLE for a binomial p = P(X > 5) such that
n
p
and p e 5 ]
m n
(b)
Page 6
m = 120, n = 10
0.513
%%%%%%%%%%%%%%%%%%%%%%%%%%%%%%%%%%%%%%%%%%%%%%%%%%5
(iii)
(a)
(1 e
5
n
e

) m is the likelihood of observing m policies with duration < 5
x i
is the likelihood of observing the actual durations x 1 ,
, x n
i 1
and independence leads to the product of these
log L ( )
m log(1 e
d
5 me
log L ( )
d
1 e
5
5
) n log
n
5
x i
x i
equate to zero and the solution gives the MLE.
(b)
All three are re-assuringly close.
The pooled estimate is between the first two (as expected, but it is
closer to 0.513).

\end{document}
