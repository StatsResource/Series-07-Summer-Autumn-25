
\documentclass[a4paper,12pt]{article}

%%%%%%%%%%%%%%%%%%%%%%%%%%%%%%%%%%%%%%%%%%%%%%%%%%%%%%%%%%%%%%%%%%%%%%%%%%%%%%%%%%%%%%%%%%%%%%%%%%%%%%%%%%%%%%%%%%%%%%%%%%%%%%%%%%%%%%%%%%%%%%%%%%%%%%%%%%%%%%%%%%%%%%%%%%%%%%%%%%%%%%%%%%%%%%%%%%%%%%%%%%%%%%%%%%%%%%%%%%%%%%%%%%%%%%%%%%%%%%%%%%%%%%%%%%%%

\usepackage{eurosym}
\usepackage{vmargin}
\usepackage{amsmath}
\usepackage{graphics}
\usepackage{epsfig}
\usepackage{enumerate}
\usepackage{multicol}
\usepackage{subfigure}
\usepackage{fancyhdr}
\usepackage{listings}
\usepackage{framed}
\usepackage{graphicx}
\usepackage{amsmath}
\usepackage{chngpage}

%\usepackage{bigints}
\usepackage{vmargin}

% left top textwidth textheight headheight

% headsep footheight footskip

\setmargins{2.0cm}{2.5cm}{16 cm}{22cm}{0.5cm}{0cm}{1cm}{1cm}

\renewcommand{\baselinestretch}{1.3}

\setcounter{MaxMatrixCols}{10}

\begin{document}
\begin{enumerate}

\item 6
A large portfolio of policies is such that a proportion p (0 < p < 1) incurred claims
during the last calendar year. An investigator examines a randomly selected group of
25 policies from the portfolio.

\begin{enumerate}[(a)]
\item Use a Poisson approximation to the binomial distribution to calculate an
approximate value for the probability that there are at most 4 policies with
claims in the two cases where (a) p = 0.1 and (b) p = 0.2.
\item Comment briefly on the above approximations, given that the exact values of
the probabilities in part (i), using the binomial distribution, are 0.9020 and
0.4207 respectively.
\end{enumerate}

One variable of interest, T, in the description of a physical process can be modelled as
T = XY where X and Y are random variables such that X ~ N(200, 100) and Y depends
on X in such a way that Y|X = x ~ N(x, 1).
Simulate two observations of T, using the following pairs of random numbers
(observations of a uniform (0, 1) random variable), explaining your method and
calculations clearly:
Random numbers
0.5714 , 0.8238
0.3192 , 0.6844
[6]
\item 7
Let (X 1 , X 2 , , X n ) be a random sample from a uniform distribution on the interval
( , ), where is an unknown positive number.
A particular sample of size 5 gives values 0.87, 0.43, 0.12, 0.92, and 0.58.
\begin{enumerate}
\item (i) Draw a rough graph of the likelihood function L( ) against
\item (ii) State the value of the maximum likelihood estimate of . for this sample.
\end{enumerate}
\end{enumerate}
%%%%%%%%%%%%%%%%%%%%%%%%%%%%%%%%%%%%%%%%%%%%%%%%%%%%%%%%%%%%%%%%%%%%%%%%%%%




7
Solving P(Z < z) = 0.5714 z = 0.180
Solving P(Z < z) = 0.8238 z = 0.930
t = 201.80 202.73 = 40911
Solving P(Z < z) = 0.3192 z = 0.470
Solving P(Z < z) = 0.6844 z = 0.480
t = 195.3 195.78 = 38236
(i)
L ( )
n
1
2
c
1
x = 200 + 10(0.180) = 201.80
y = 201.80 + 0.930 = 202.73
x = 200 + 10( 0.470) = 195.3
y = 195.3 + 0.480 = 195.78
n
for
< x i < , i = 1, 2,
, n and L( ) = 0 otherwise
So, as increases from zero, L( ) is zero until it reaches the largest
observation in absolute value i.e. max |x i |, i = 1, 2, , n. For the data given,
this value is 0.92.
It is then a decreasing function . Hence the graph is as below:
6
0
0
(ii)
0.92
The maximum value of L( ) is attained at the largest absolute value of the
data. The ML estimate of is 0.92.
%%%%%%%%%%%%%%%%%%%%%%%%%%%%%%%%%%%%%%%%%%%%%%%%%%%%%%%%%%%%%%%%%%%%%%%%%%%%%%%%
\end{document}
