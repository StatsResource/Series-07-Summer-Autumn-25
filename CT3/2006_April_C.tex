
\documentclass[a4paper,12pt]{article}

%%%%%%%%%%%%%%%%%%%%%%%%%%%%%%%%%%%%%%%%%%%%%%%%%%%%%%%%%%%%%%%%%%%%%%%%%%%%%%%%%%%%%%%%%%%%%%%%%%%%%%%%%%%%%%%%%%%%%%%%%%%%%%%%%%%%%%%%%%%%%%%%%%%%%%%%%%%%%%%%%%%%%%%%%%%%%%%%%%%%%%%%%%%%%%%%%%%%%%%%%%%%%%%%%%%%%%%%%%%%%%%%%%%%%%%%%%%%%%%%%%%%%%%%%%%%

\usepackage{eurosym}
\usepackage{vmargin}
\usepackage{amsmath}
\usepackage{graphics}
\usepackage{epsfig}
\usepackage{enumerate}
\usepackage{multicol}
\usepackage{subfigure}
\usepackage{fancyhdr}
\usepackage{listings}
\usepackage{framed}
\usepackage{graphicx}
\usepackage{amsmath}
\usepackage{chngpage}

%\usepackage{bigints}
\usepackage{vmargin}

% left top textwidth textheight headheight

% headsep footheight footskip

\setmargins{2.0cm}{2.5cm}{16 cm}{22cm}{0.5cm}{0cm}{1cm}{1cm}

\renewcommand{\baselinestretch}{1.3}

\setcounter{MaxMatrixCols}{10}

\begin{document}
\begin{enumerate}

PLEASE TURN OVER8
The events that lead to potential claims on a policy arise as a Poisson process at a rate
of 0.8 per year. However the policy is limited such that only the first three claims in
any one year are paid.
(i) Determine the probabilities of 0, 1, 2 and 3 claims being paid in a particular
year.
[2]
(ii) The amounts (in units of £100) for the claims paid follow a gamma
distribution with parameters = 2 and = 1.
Calculate the expectation of the sum of the amounts for the claims paid in a
particular year.
[3]
(iii)
9
Calculate the expectation of the sum of the amounts for the claims paid in a
particular year, given that there is at least one claim paid in the year.
[2]
[Total 7]
The total claim amount on a portfolio, S, is modelled as having a compound
distribution
S = X 1 + X 2 +
+ X N
where N is the number of claims and has a Poisson distribution with mean , X i is the
amount of the i th claim, and the X i s are independent and identically distributed and
independent of N. Let M X (t) denote the moment generating function of X i .
(i)
Show, using a conditional expectation argument, that the cumulant generating
function of S, C S (t), is given by
C S (t) =
M X (t) 1}.
Note: You may quote the moment generating function of a Poisson random
variable from the book of Formulae and Tables.
[4]
(ii)
Calculate the variance of S in the case where
variance 10.
CT3 A2006 4
= 20 and X has mean 20 and
[2]
[Total 6]10
A marketing consultant was commissioned to conduct a questionnaire survey of the
clients of a financial company. The total number of respondents was 650, of whom
220 had investments above a specified threshold.
(i)
Each respondent who had investments above the threshold was asked about
the percentage of these investments that was held in the form of a certain type
of trust. The respondents answered by ticking appropriate boxes and the
results led to the following frequency distribution.
percentage
frequency
(ii)
< 10
22
10 25
76
25 50
73
> 50
49
(a) Present these data graphically using a carefully drawn histogram.
(b) Calculate the mean percentage for the full set of 220 such respondents,
assuming that the frequencies in each category are uniformly spread
over the corresponding range.
[5]
Calculate a 95% confidence interval for the percentage of such investors who
would have investments above the threshold.
[4]
The same respondents with investments referred to in part (i) were also asked to
specify their satisfaction with the current return received from their full portfolio of
investment. This was in the form of a four-point qualitative scale: very satisfied, quite
satisfied, a little disappointed, very disappointed. The following two-way table of
frequencies was obtained.
<10
percentage in type of trust
10 25
25 50
>50
1
8
10
3
very satisfied
quite satisfied
a little disappointed
very disappointed
6
29
37
4
7
36
28
2
6
27
15
1
In order to investigate whether there is any relationship between the percentage in
such trusts and satisfaction with current return, a 2 test is to be performed.
(iii) Calculate the expected frequencies for the above table under an appropriate
hypothesis (which should be stated) and comment on why it would be
inappropriate to carry out a 2 test directly with these data.
[3]
(iv) Combining the very satisfied and quite satisfied categories together and
the a little disappointed and very disappointed categories together results
in the following reduced two-way table.
<10
satisfied
disappointed
9
13
percentage in type of trust
10 25
25 50
>50
35
41
43
30
33
16
Perform the required 2 test at the 5% level using this reduced table and
comment on your conclusion.
[7]
[Total 19]
CT3 A2006 5
PLEASE TURN OVER


%%%%%%%%%%%%%%%%%%%%%%%%%%%%%%%%%%%%%%%%%%%%%%%%%%%%%%%%%%%%%%%%%%%%%%%%%%%%%%%%%%%%% Solutions

8
\begin{enumerate}
\item (i)
April 2006
Examiners Report
By subtraction using entries in tables for Poisson(0.8), the probabilities for the
Poisson distribution for 0, 1, 2 and 3 are: [or by evaluation]
0.44933, 0.35946, 0.14379 and (1 0.95258) = 0.04742
\item (ii)
Let N = number of claims paid and let X 1 ,
S = X i is the sum of the amounts.
, X n be the claim amounts then
E[S] = E[N]E[X]
Here E[N] = 1(0.35946) + 2(0.14379) + 3(0.04742) = 0.7893
and E[X] = 2/1 = 2 from gamma(2,1)
So E[S] = (0.7893)(2) = 1.5786 = £157.86
\item (iii)
Given that N > 0, divide the probabilities in part (i) by (1 0.44933) =
0.55067 to give the probabilities for 1, 2 and 3 claims paid as:
0.6528, 0.2611 and 0.0861
E[N] = 1(0.6528) + 2(0.2611) + 3(0.0861) = 1.4333
So E[S] = (1.4333)(2) = 2.8666 = £286.66
\end{enumerate}
%%%%%%%%%%%%%%%%%%%%%%%%%%%%%%%%%%%%%%%%%%%%%%%%%%%%%%%%%%%%%%%%%%%%%%%%%%%%5
9
\begin{enumerate}
\item (i)
M S (t) = E[e tS ] = E[E[e tS |N]]
Now E[e tS |N = n] = E[exp(tX 1 +
+ tX n )] = E[exp(tX i )] = {M X (t)} n
M S (t) = E[{M X (t)} N ] = E[exp{NlogM X (t)}] = M N {logM X (t)}
= exp[ M X (t) 1}] since N ~ Poisson( )
C S (t) = logM S (t) = M X (t) 1}
\item (ii)
V[S] = C S (0) =
M X (0)} = E[X 2 ] = 20(10 + 20 2 ) = 8200
OR V[S] = E[N]V[X] + V[N]{E[X]} 2 = 20 10 + 20 20 2 = 8200
\end{enumerate}
%%%%%%%%%%%%%%%%%%%%%%%%%%%%%%%%%%%%%%%%%%%%%%%%%%%%%%%%%%%%%%%%%%%%%%%%%%%%%
10
(i)
(a)
April 2006
Examiners Report
The key feature of the histogram is that the areas of the four rectangles
should be proportional to the frequencies.
See histogram below.
(b)
Mean is calculated from the following frequency distribution:
5
22
x
f
f = 220,
(ii)
17.5
76
fx = 7852.5
Estimated proportion is p
220
650
37.5
73
7852.5
220
x
0.338
75
49
35.7%
(or 33.8%)
95% confidence interval for underlying proportion is
p 1.96
p (1 p )
650
0.338 1.96
0.338(0.662)
650
as a percentage: 33.8%
0.338 0.036
3.6% or (30.2%, 37.4%)
Page 7Subject CT3 (Probability and Mathematical Statistics Core Technical)
(iii)
Examiners Report
Under the null hypothesis of no association between percentage in type of trust
and satisfaction with current return, expected frequencies are
2.0
10.0
9.0
1.0
22
6.9
34.5
31.1
3.5
76
6.6
33.2
29.9
3.3
73
six are less than 5 which would invalidate a
(iv)
April 2006
4.5
22.3
20.0
2.2
49
2
20
100
90
10
220
test
expected frequencies (e) are
12.000
10.000
41.455
34.545 39.818
33.182 26.727
22.273
6.455
+6.455 +3.182
3.182 +6.273
6.273
1.005
1.206 0.254
0.305 1.472
1.767
table of residuals (o e) is
3.000
+3.000
table of contributions to
2
is
0.750
0.900
giving
2
= 7.659 on 3 d.f.
2
3 (5%)
7.815
must accept the null hypothesis that there is no
relationship between percentage in type of trust and satisfaction with current
return.
However this decision to accept is marginal at the 5% level and there is some
evidence, but not strong, to suggest that satisfaction improves as the
percentage increases.

\end{document}
