
\documentclass[a4paper,12pt]{article}

%%%%%%%%%%%%%%%%%%%%%%%%%%%%%%%%%%%%%%%%%%%%%%%%%%%%%%%%%%%%%%%%%%%%%%%%%%%%%%%%%%%%%%%%%%%%%%%%%%%%%%%%%%%%%%%%%%%%%%%%%%%%%%%%%%%%%%%%%%%%%%%%%%%%%%%%%%%%%%%%%%%%%%%%%%%%%%%%%%%%%%%%%%%%%%%%%%%%%%%%%%%%%%%%%%%%%%%%%%%%%%%%%%%%%%%%%%%%%%%%%%%%%%%%%%%%

\usepackage{eurosym}
\usepackage{vmargin}
\usepackage{amsmath}
\usepackage{graphics}
\usepackage{epsfig}
\usepackage{enumerate}
\usepackage{multicol}
\usepackage{subfigure}
\usepackage{fancyhdr}
\usepackage{listings}
\usepackage{framed}
\usepackage{graphicx}
\usepackage{amsmath}
\usepackage{chngpage}

%\usepackage{bigints}
\usepackage{vmargin}

% left top textwidth textheight headheight

% headsep footheight footskip

\setmargins{2.0cm}{2.5cm}{16 cm}{22cm}{0.5cm}{0cm}{1cm}{1cm}

\renewcommand{\baselinestretch}{1.3}

\setcounter{MaxMatrixCols}{10}

\begin{document}
\begin{enumerate}
%%%%%%%%%%%%%%%%%%%%%
11
Twenty insects were used in an experiment to examine the effect on their activity
level, y, of 3 standard preparations of a chemical. The insects were randomly
assigned, 4 to receive each of the preparations and 8 to remain untreated as controls.
Their activity levels were metered from vibrations in a test chamber and were as
follows:
Activity levels (y)
Totals
Control
Preparation A
Preparation B
Preparation C 43
73
84
46
40
55
63
91
For these data y = 1, 203 ,
65
61
51
84
51
65
72
71
33
39
54
62
387
254
270
292
y 2 = 77, 249 .
(i) Conduct an analysis of variance test to establish whether the data indicate
significant differences amongst the results for the four treatments.
[7]
(ii) (a)
Complete the following table of residuals for the data and analysis in
part (i) above:
Control
Preparation A
Preparation B
Preparation C
(iii)
?
9.5
16.5
?
?
8.5
?
?
16.6
2.5
16.5
?
2.6
1.5
4.5
2
15.4
9.4
5.6 13.6
(b) Make a rough plot of the residuals against the treatment means.
(c) State the assumptions underlying the analysis of variance test
conducted in part (i).
(d) Comment on how well the data conform to these assumptions in the
light of the residual plot.
[8]
It is suggested that any differences can be explained in terms of a difference
between controls on the one hand and treated groups on the other.
Comment on any evidence for this and state how you would formally test for
this effect (but do not carry out the test).

%%%%%%%%%%%%%%%%%%%%%%%%%%%%%%%%%%%%%%%%%%%%%%%%%%%%%%%%%%%%%%%%%%%%%%%%%%%%%%%%%%%%%%%%%%%%%%%%%%%%%%%%%%%5
5
PLEASE TURN OVER12
(i)
A random variable Y has a Poisson distribution with parameter but there is a
restriction that zero counts cannot occur. The distribution of Y in this case is
referred to as the zero-truncated Poisson distribution.
(a)
Show that the probability function of Y is given by
y
p ( y ) =
(ii)
e
y !(1 e )
( y = 1, 2, ).
(b) Show that E [ Y ] = /(1 e ).
[4]
(a) Let y 1 , , y n denote a random sample from the zero-truncated Poisson
distribution.
Show that the maximum likelihood estimate of
by the solution to the following equation:
y
e
may be determined
= 0,
1 e
and deduce that the maximum likelihood estimate is the same as the
method of moments estimate.
(b)
(iii)
Obtain an expression for the Cramer-Rao lower bound (CRlb) for the
variance of an unbiased estimator of .
[9]
The following table gives the numbers of occupants in 2,423 cars observed on
a road junction during a certain time period on a weekday morning.
Number of occupants
Frequency of cars
1
1,486
2
694
3
195
4
37
5
10
6
1
The above data were modelled by a zero-truncated Poisson distribution as
given in (i).
The maximum likelihood estimate of is = 0.8925 and the Cramer-Rao
lower bound on variance at = 0.8925 is 5.711574 10 4 (you do not need to
verify these results.)
%===================================================================================%
(a) Obtain the expected frequencies for the fitted model, and use a 2
goodness-of-fit test to show that the model is appropriate for the data.
(b) Calculate an approximate 95% confidence interval for and hence
calculate a 95% confidence interval for the mean of the zero-truncated
Poisson distribution.
[9]
[Total 22]
613
As part of an investigation into health service funding a working party was concerned
with the issue of whether mortality rates could be used to predict sickness rates. Data
on standardised mortality rates and standardised sickness rates were collected for a
sample of 10 regions and are shown in the table below:
Region Mortality rate m (per 10,000) Sickness rate s (per 1,000)
1
2
3
4
5
6
7
8
9
10 125.2
119.3
125.3
111.7
117.3
100.7
108.8
102.0
104.7
121.1 206.8
213.8
197.2
200.6
189.1
183.6
181.2
168.2
165.2
228.5
Data summaries:
m = 1136.1,
(i)
m 2 = 129,853.03, s = 1934.2,
s 2 = 377,700.62, ms = 221,022.58
Calculate the correlation coefficient between the mortality rates and the
sickness rates and determine the probability-value for testing whether the
underlying correlation coefficient is zero against the alternative that it is
positive.
[4]
(ii) Noting the issue under investigation, draw an appropriate scatterplot for these
data and comment on the relationship between the two rates.
[3]
(iii) Determine the fitted linear regression of sickness rate on mortality rate and test
whether the underlying slope coefficient can be considered to be as large
as 2.0.
[5]
(iv) For a region with mortality rate 115.0, estimate the expected sickness rate and
calculate 95% confidence limits for this expected rate.
[4]
[Total 16]
END OF PAPER
CT3 A2005

\end{document}
7
