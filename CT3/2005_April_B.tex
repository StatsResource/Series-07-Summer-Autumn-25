
\documentclass[a4paper,12pt]{article}

%%%%%%%%%%%%%%%%%%%%%%%%%%%%%%%%%%%%%%%%%%%%%%%%%%%%%%%%%%%%%%%%%%%%%%%%%%%%%%%%%%%%%%%%%%%%%%%%%%%%%%%%%%%%%%%%%%%%%%%%%%%%%%%%%%%%%%%%%%%%%%%%%%%%%%%%%%%%%%%%%%%%%%%%%%%%%%%%%%%%%%%%%%%%%%%%%%%%%%%%%%%%%%%%%%%%%%%%%%%%%%%%%%%%%%%%%%%%%%%%%%%%%%%%%%%%

\usepackage{eurosym}
\usepackage{vmargin}
\usepackage{amsmath}
\usepackage{graphics}
\usepackage{epsfig}
\usepackage{enumerate}
\usepackage{multicol}
\usepackage{subfigure}
\usepackage{fancyhdr}
\usepackage{listings}
\usepackage{framed}
\usepackage{graphicx}
\usepackage{amsmath}
\usepackage{chngpage}

%\usepackage{bigints}
\usepackage{vmargin}

% left top textwidth textheight headheight

% headsep footheight footskip

\setmargins{2.0cm}{2.5cm}{16 cm}{22cm}{0.5cm}{0cm}{1cm}{1cm}

\renewcommand{\baselinestretch}{1.3}

\setcounter{MaxMatrixCols}{10}

\begin{document}
\begin{enumerate}
%%%%%%%%%%%%%%%%%%%%%
%%%%%%%%%%%%%%%%%%%%%%%%%%%%%%%%%%%%%%%%%%%%%%%%%%%%%%%%%%%%%%%%%%%%%%%%%%%%%%%%%%%%%%%%%%%%%%%%%%%%%%%%%%%
5
[3]
[3]
Consider a random sample of size 21 taken from a normal distribution with mean
= 25 and variance 2 = 4. Let the sample variance be denoted S 2 .
State the distribution of the statistic 5S 2 and hence find the variance of the statistic S 2 .
[3]
%%%%%%%%%%%%%%%%%%%%%%%%%%%%%%%%%%%%%%%%%%%%%%%%%%%%%%%%%%%%%%%%%%%%%%%%%%%%%%%%%%%%%%%%%%%%%%%%%%%%%%%%%%%

6
In a survey conducted by a mail order company a random sample of 200 customers
yielded 172 who indicated that they were highly satisfied with the delivery time of
their orders.
Calculate an approximate 95% confidence interval for the proportion of the
company s customers who are highly satisfied with delivery times.
%%%%%%%%%%%%%%%%%%%%%%%%%%%%%%%%%%%%%%%%%%%%%%%%%%%%%%%%%%%%%%%%%%%%%%%%%%%%%%%%%%%%%%%%%%%%%%%%%%%%%%%%%%%
7
[3]
For a group of policies the total number of claims arising in a year is modelled as a
Poisson variable with mean 10. Each claim amount, in units of £100, is independently
modelled as a gamma variable with parameters = 4 and = 1/5.
Calculate the mean and standard deviation of the total claim amount.


%%%%%%%%%%%%%%%%%%%%%%%%%%%%%%%%%%%%%%%%%%%%%%%%%%%%%%%%%%%%%%%%%%%%%%%%%%%%%%%%%%%%%%%%%%%%%
5
n 1 S 2
5 S 2 ~
2
x
n
p
172
200
7
2
20 =
40, so V[S 2 ] = 40/25 = 1.6
0.86
95% CI is p 1.96
0.86
Examiners Report
2
20
V[5S 2 ] = variance of
6
p (1 p )
n
1.96(0.0245)
0.86
0.048 or (0.812, 0.908)
Let S = X 1 + X 2 + . . . + X N be the total claim amount.
Note that E[N] = V[N] = 10, E[X] = 4/(1/5) = 20, V[X] = 4/(1/5) 2 = 100
E[S] = E[N]E[X] = 10(20) = 200
mean of total claim amount = £20,000.
V[S] = E[N]V[X] + V[N]{E[X]} 2
=10(100) + 10(20) 2 = 5000
s.d.[S] = 70.71
8
(i)
s.d. of the total claim amount = £7,071
\end{document}
